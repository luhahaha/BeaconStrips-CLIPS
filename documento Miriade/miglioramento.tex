\section{Idee per il miglioramento del prodotto finale}
\label{miglioramento}
%Qui metteremo le idee da proporre a Miriade per migliorare il prodotto, come l'interfaccia grafica per l'amministratore
Durante la progettazione del software il gruppo \AUTORE\ ha pensato ad alcune modifiche che, per mancanza di tempo, ha deciso di non applicare; l'implementazione di queste cambiamenti per� porterebbe ad un miglioramento della qualit� e dei servizi offerti dal prodotto. Di seguito saranno proposte queste modifiche, insieme ad una spiegazione dei miglioramenti che apporterebbero.
	\subsection{Interfaccia grafica per l'amministratore}
		Una componente che risulterebbe molto utile � l'interfaccia grafica per l'amministratore con cui inserire i dati da salvare nel server. I vantaggi principali sono due: si evitano i problemi di una formattazione errata dei dati da aggiungere, che possono presentarsi se i dati vengono inseriti direttamente a mano, e permette anche a responsabili che non conoscono gli oggetti JSON, il linguaggio SQL o la struttura del server di poter aggiungere dati senza preoccuparsi di provocare danni al sistema. Lo svantaggio � l'elevata mole di lavoro che richiede, in quanto sarebbe necessario creare un'interfaccia grafica per ogni tipo di dato da inserire, e in pi� richiederebbe una progettazione a s� per ottimizzarne l'usabilit�.
	\label{libreria_beacon}
	\subsection{Utilizzo di librerie generiche per la lettura dei beacon}
		Le librerie Kontakt sono semplici da implementare e funzionano bene, hanno per� il difetto di non poter interagire con tutti i tipi di beacon esistenti sul mercato. L'utilizzo di altre librerie, come ad esempio Android Beacon Library, dovrebbe permettere di risolvere questo problema.
	\label{gestione_errori}
	\subsection{Migliorare la gestione degli errori}
		Esiste gi� una gestione degli errori, ma � ancora grezza e migliorabile. In molti casi ad esempio viene mostrato un messaggio fisso, mentre sarebbe pi� utile visualizzare una delle spiegazioni contenute nella classe dell'errore, chiamata ServerError; in questo modo sarebbe pi� facile diversificare i messaggi mostrati a seconda del tipo di errore. Possono poi presentarsi due tipi di avvisi: il primo non permette di proseguire con l'operazione, ad esempio quando il percorso scelto non contiene delle stazioni perch� non � ancora pronto, mentre il secondo permette comunque di andare avanti, ad esempio quando il percorso non contiene il titolo o un altro dato di secondaria importanza. Una buona gestione degli errori pu� aiutare ad individuare gli errori in tempi pi� brevi, e inoltre migliora la user experience, perch� � meglio comunicare all'utente che c'� qualcosa di errato piuttosto di proseguire l'operazione inutilmente.
	\label{tutorial}
	\subsection{Aggiungere un tutorial introduttivo da mostrare al primo utilizzo}
		L'applicazione � semplice da utilizzare, ma alcuni passaggi potrebbero non essere immediati per chi la utilizza per la prima volta. Ad esempio � bene spiegare all'utente che la registrazione del profilo non � obbligatoria, o anche mostrargli dove pu� trovare una descrizione dell'applicazione. Per risolvere questo problema pu� essere utile mostrare un tutorial introduttivo al primo utilizzo, mentre per quelli successivi viene disabilitato ed eventualmente riabilitato dall'utente se lo desidera; in questo modo egli pu� comprendere pi� velocemente come funziona l'applicazione.
	\subsection{Aggiungere la schermata per mostrare i risultati delle singole prove}
		L'utente pu� vedere i risultati che ha ottenuto tramite un bottone dentro la GUI del profilo. Pu� essere utile e interessante permettergli di visualizzare anche i risultati delle singole prove per ogni percorso salvato, ad esempio trasformando la cella di ogni esito in un bottone cliccabile, da cui poi � possibile aprire una GUI simile in cui sono contenuti i risultati delle prove di quel percorso. Questi dati poi vengono gi� ricevuti quando si richiedono gli esiti dell'utente, quindi si tratterebbe solo di aggiungere un'Activity e di collegarla al resto dell'applicazione.
	\subsection{Aggiungere l'impossibilit� di iniziare il percorso di un edificio distante dall'utente}
		Allo stato attuale l'utente pu� cominciare qualunque percorso anche se non si trova vicino all'edificio; ovviamente quando cerca il primo beacon non riesce a trovarlo. Per evitare questo inconveniente basterebbe sostituire il pulsante ??Inizia percorso?? con un messaggio in cui si avvisa l'utilizzatore che per iniziare il percorso deve recarsi in quell'edificio.
	\subsection{}