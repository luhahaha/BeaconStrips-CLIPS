\section{Idee per il miglioramento del prodotto finale}
\label{miglioramento}
%Qui metteremo le idee da proporre a Miriade per migliorare il prodotto, come l'interfaccia grafica per l'amministratore
Durante la progettazione del software il gruppo \AUTORE\ ha pensato ad alcune modifiche che, per mancanza di tempo, ha deciso di non applicare; l'implementazione di queste cambiamenti però porterebbe ad un miglioramento della qualità e dei servizi offerti dal prodotto. Di seguito saranno proposte queste modifiche, insieme ad una spiegazione dei miglioramenti che apporterebbero.
	\subsection{Interfaccia grafica per l'amministratore}
		Una componente che risulterebbe molto utile è l'interfaccia grafica per l'amministratore con cui inserire i dati da salvare nel server. I vantaggi principali sono due: si evitano i problemi di una formattazione errata dei dati da aggiungere, che possono presentarsi se i dati vengono inseriti direttamente a mano, e permette anche a responsabili che non conoscono gli oggetti JSON, il linguaggio SQL o la struttura del server di poter aggiungere dati senza preoccuparsi di provocare danni al sistema. Lo svantaggio è l'elevata mole di lavoro che richiede, in quanto sarebbe necessario creare un'interfaccia grafica per ogni tipo di dato da inserire, e in più richiederebbe una progettazione a sè per ottimizzarne l'usabilità.
	\label{libreria_beacon}
	\subsection{Utilizzo di librerie generiche per la lettura dei beacon}
		Le librerie Kontakt sono semplici da implementare e funzionano bene, hanno però il difetto di non poter interagire con tutti i tipi di beacon esistenti sul mercato. L'utilizzo di altre librerie, come ad esempio Android Beacon Library, dovrebbe permettere di risolvere questo problema.
	\label{gestione_errori}
	\subsection{Migliorare la gestione degli errori}
		Esiste già una gestione degli errori, ma è ancora grezza e migliorabile. In molti casi ad esempio viene mostrato un messaggio fisso, mentre sarebbe più utile visualizzare una delle spiegazioni contenute nella classe dell'errore, chiamata ServerError; in questo modo sarebbe più facile diversificare i messaggi mostrati a seconda del tipo di errore. Possono poi presentarsi due tipi di avvisi: il primo non permette di proseguire con l'operazione, ad esempio quando il percorso scelto non contiene delle stazioni perché non è ancora pronto, mentre il secondo permette comunque di andare avanti, ad esempio quando il percorso non contiene il titolo o un altro dato di secondaria importanza. Una buona gestione degli errori può aiutare ad individuare gli errori in tempi più brevi, e inoltre migliora la user experience, perché è meglio comunicare all'utente che c'è qualcosa di errato piuttosto di proseguire l'operazione inutilmente.
	\label{tutorial}
	\subsection{Aggiungere un tutorial introduttivo da mostrare al primo utilizzo}
		L'applicazione è semplice da utilizzare, ma alcuni passaggi potrebbero non essere immediati per chi la utilizza per la prima volta. Ad esempio è bene spiegare all'utente che la registrazione del profilo non è obbligatoria, o anche mostrargli dove può trovare una descrizione dell'applicazione. Per risolvere questo problema può essere utile mostrare un tutorial introduttivo al primo utilizzo, mentre per quelli successivi viene disabilitato ed eventualmente riabilitato dall'utente se lo desidera; in questo modo egli può comprendere più velocemente come funziona l'applicazione.
	\subsection{Aggiungere la schermata per mostrare i risultati delle singole prove}
		L'utente può vedere i risultati che ha ottenuto tramite un bottone dentro la GUI del profilo. Può essere utile e interessante permettergli di visualizzare anche i risultati delle singole prove per ogni percorso salvato, ad esempio trasformando la cella di ogni esito in un bottone cliccabile, da cui poi è possibile aprire una GUI simile in cui sono contenuti i risultati delle prove di quel percorso. Questi dati poi vengono già ricevuti quando si richiedono gli esiti dell'utente, quindi si tratterebbe solo di aggiungere un'Activity e di collegarla al resto dell'applicazione.
	\subsection{Aggiungere l'impossibilità di iniziare il percorso di un edificio distante dall'utente}
		Allo stato attuale l'utente può cominciare qualunque percorso anche se non si trova vicino all'edificio; ovviamente quando cerca il primo beacon non riesce a trovarlo. Per evitare questo inconveniente basterebbe sostituire il pulsante ??Inizia percorso?? con un messaggio in cui si avvisa l'utilizzatore che per iniziare il percorso deve recarsi in quell'edificio.
	\subsection{Aggiungere la possibilità di cambiare solo lo username}
		L'utente può già cambiare lo username e la password oppure solo la password, ma non può cambiare solo lo username. L'impedimento in realtà è dovuto soltanto alla GUI, perché il server prevede già la possibilità di effettuare la modifica dei dati solo con lo username e la vecchia password.
	\subsection{Aggiungere un'immagine dinamica alla proximity}
		La proximity per ora presenta solo un testo, che comunque dovrebbe migliorare la user experience. Verrebbe passato anche un valore percentuale, inzialmente pensato per poterlo rappresentare come una barretta o delle tacchette. In questo modo si dovrebbe dare un'idea all'utente di quanto si sta avvicinando o meno.
	\subsection{Migliorare la grafica dell'applicazione}
		Avere una bella grafica porta l'utente ad apprezzare di più l'applicazione, quindi sarà più disposto a provarla. Inoltre aiuta a convincere il possibile acquirente, dato che la grafica migliore aumenta
	\subsection{Lasso di tempo per poter far vedere i suggerimenti delle domande}
		Al momento attuale le domande contengono un pulsante per poter vedere un suggerimento, in modo da aiutare l'utente ad individuare la risposta corretta. Sarebbe più utile riuscire a far vedere quel pulsante dopo un certo lasso di tempo, così da poter inserire dei suggerimenti meno generici. Così facendo si aiuta l'utente a proseguire quando non riesce ad andare avanti. Sarebbe utile poi combinarlo con un algoritmo che calcola il punteggio anche in base al tempo impiegato dall'utente, in questo modo a lui non conviene aspettare di visualizzare il suggerimento se non è sicuro della risposta da dare.
	\subsection{Stazioni bonus}
		Un'aggiunta interessante potrebbe essere l'aggiunta di stazioni bonus, per assegnare dei punti bonus all'utente che riesce a completare la prova o semplicemente per mostrare un messaggio. In questo modo chi utilizza questa applicazione è portato a visitare anche i posti che non c'entrano con la tappa successiva, e in più è un incentivo a chi piace scoprire queste sorprese.
