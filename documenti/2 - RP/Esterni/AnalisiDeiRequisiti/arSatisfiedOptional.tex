\begin{tabella}{!{\VRule}l!{\VRule}X[1,m,c]!{\VRule}r!{\VRule}}\intestazionethreecol{Requisito}{Descrizione}{Soddisfatto}\Req{R2F3.1.1.4} & Il dispositivo può sfruttare dei beacon aggiuntivi ed estranei al percorso per segnalare all'utente la distanza tra lui e la stazione. & {\color{reqNonSoddisfatto} Non soddisfatto}\\ 
\Req{R2F3.1.1.4.1} & La distanza viene espressa con dei valori approssimativi (ad esempio delle tacche) per aiutare l'utente a capire se si sta allontanando o avvicinando alla stazione e quanto è distante approssimativamente. & {\color{reqNonSoddisfatto} Non soddisfatto}\\ 
\Req{R2F4.3.2} & Quando l'utente clicca sul pulsante della segnalazione si apre una schermata dove l'utente seleziona delle caratteristiche per determinare l'errore avuto. & {\color{reqNonSoddisfatto} Non soddisfatto}\\ 
\Req{R2F4.3.2.1} & L'utente seleziona il tipo di errore avuto. & {\color{reqNonSoddisfatto} Non soddisfatto}\\ 
\Req{R2F4.3.2.2} & L'utente può selezionare l'eventuale stazione dove si è verificato l'errore. & {\color{reqNonSoddisfatto} Non soddisfatto}\\ 
\Req{R2F4.3.2.3} & La conferma della schermata invia un'email all'indirizzo destinato alla segnalazione errori avente un testo predefinito a seconda dell'errore segnalato. & {\color{reqNonSoddisfatto} Non soddisfatto}\\ 
\Req{R2F5.1.1.7} & Il dispositivo può visualizzare la posizione attuale del risultato ottenuto. & {\color{reqNonSoddisfatto} Non soddisfatto}\\ 
\Req{R2F6.1.1.2} & Se la ricerca ha esito negativo l'app suggerisce all'utente di aumentare il raggio inserito. & {\color{reqNonSoddisfatto} Non soddisfatto}\\ 
\Req{R2F6.3.1.4} & L'utente può visualizzare una breve descrizione sull'edificio e sul tipo di percorsi presenti. & {\color{reqNonSoddisfatto} Non soddisfatto}\\ 
\hiderowcolors
\caption{Riepilogo requisiti opzionali soddisfatti}
\end{tabella}