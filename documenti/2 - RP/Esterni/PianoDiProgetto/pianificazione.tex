\section{Pianificazione} 
	\subsection{Introduzione}
	A fronte dell'analisi dei rischi e della scadenza delle revisioni di avanzamento vi saranno tre periodi durante lo svolgimento del \gl{progetto}: uno di \textbf{analisi}, uno di \textbf{progettazione e codifica} ed uno di \textbf{incremento e validazione}.
	Per rendere più controllabile lo sviluppo del \gl{progetto} si è deciso di dividere il lavoro in sei periodi specifici, i quali vengono riportati nella seguente tabella con le relative date di inizio e di fine.
		
		\begin{tabella}{!{\VRule}c!{\VRule}c!{\VRule}c!{\VRule}c!{\VRule}} %date non standard \gl{ISO}
				
			\intestazionethreecol{Periodo}{Data di inizio}{Data di fine}
			
			Periodo di analisi e management & 2016-03-01 & 2016-04-18  \\
			Periodo di consolidamento dei requisiti & 2016-04-19 & 2016-05-06  \\
			Periodo di progettazione architetturale & 2016-05-07 & 2016-06-17 \\
			Periodo di progettazione di dettaglio e codifica & 2016-06-18 & 2016-08-24 \\
			Periodo di codifica dei requisiti desiderabili e opzionali & 2016-08-25 & 2016-08-30 \\
			Periodo di validazione e collaudo & 2016-08-31 & 2016-09-12 \\ 
			
			\hiderowcolors
			\caption{Periodi di sviluppo con relative abbreviazioni e date di inizio e fine.}
			
		\end{tabella}
		
	Ogni periodo contiene diverse attività che verranno riportate e descritte in un elenco puntato. \\ Successivamente nei diagrammi di \gl{Gantt} si potrà notare come le attività siano state suddivise temporalmente. In questi saranno inoltre presenti delle \gl{milestones} che indicheranno i giorni in cui dovranno essere consegnati i documenti in entrata alle revisioni, quelli in cui si svolgeranno le revisioni di avanzamento ed, eventualmente, quelli in cui vi saranno incontri con il Proponente. 
	
	\subsection{Periodo di analisi e management (AM)}
	\begin{center}
		\textbf{Data di inizio}: 2016-03-01 \\
		\textbf{Data di fine}: 2016-04-18 \\
	\end{center}

	Questo periodo inizia con la formazione del gruppo e termina il giorno della Revisione dei Requisiti. \\
	I processi principali di questo periodo sono: 
	\begin{itemize}
		\item \textbf{Management}
			\att
			\begin{itemize} 
				\item \textbf{Individuazione strumenti da utilizzare}: il gruppo deve trovare degli strumenti che aiutino ad automatizzare e rendere più facile lo sviluppo del \gl{progetto}.
			\end{itemize}
		\item \textbf{Documentazione}: viene creata la documentazione da consegnare in ingresso alla RR.
		\att
		\begin{itemize}
			\item \textbf{Norme di Progetto}: viene steso il documento \NPdocRR{} in cui saranno elencate e descritte le norme da seguire durante tutto lo svolgimento del \gl{progetto} indipendentemente dal \gl{capitolato} scelto; 
			\item \textbf{Piano di Progetto}: viene steso il documento \PPdocRR{} per pianificare dettagliatamente i tempi e i costi del \gl{progetto};
			\item \textbf{Studio di Fattibilità}: viene steso il documento \SFdocRR{} che riporta l'analisi che ha portato il gruppo a scegliere il \gl{capitolato} C2;
			\item \textbf{Analisi dei Requisiti}: viene steso il documento \ARdocRR{} in cui viene svolta un'analisi molto più approfondita di quella svolta in \SFdocRR. Vengono elencati e descritti i casi d'uso e i requisiti del \gl{prodotto} che si andrà a sviluppare;
			\item \textbf{Piano di Qualifica}: viene steso il documento \PQdocRR{} che riporta quali obiettivi di qualità si è prefissato il gruppo;
			\item \textbf{Glossario}: viene steso il \GldocRR{} il quale riporta la descrizione dei termini presenti nei vari documenti che potrebbero causare ambiguità nel lettore.
		\end{itemize}
	\end{itemize}
	
		
		\subsubsection{Diagramma di \gl{Gantt} delle attività}
		% \gantt{img/gantt/A}{Diagramma di \gl{Gantt} delle attività - Periodo A}
		
		\begin{figure}[!h]
			\centering
			\includegraphics[height=12cm, width=15cm]{img/gantt/A} 
			\caption{Diagramma di \gl{Gantt} delle attività - Periodo di analisi e management}
		\end{figure}
		
	\subsection{Periodo di consolidamento dei requisiti (CR)}
	\begin{center}
		\textbf{Data di inizio}: 2016-04-19 \\
		\textbf{Data di fine}: 2016-04-28 \\
	\end{center}
	Questo periodo inizia al termine del periodo di analisi e management, ovvero dopo la Revisione dei Requisiti, e termina con un incontro con il \gl{proponente}. \\ 
	Il processo principale di questo periodo è:
	\begin{itemize}
		\item \textbf{Documentazione}
		\att
		\begin{itemize}
			\item \textbf{Miglioramento di tutti i documenti}: seguendo le indicazioni del Committente verranno attuate le modifiche necessarie a migliorare tutti i documenti stesi nel periodo di analisi e management;
			\item \textbf{Analisi dei Requisiti}: Questo documento oltre ad essere corretto verrà anche arricchito con nuovi requisiti.
		\end{itemize}
	\end{itemize}
		
		
		\subsubsection{Diagramma di \gl{Gantt} delle attività}
		\gantt{img/gantt/AD}{Diagramma di \gl{Gantt} delle attività - Periodo di consolidamento dei requisiti}
		
	\subsection{Periodo di progettazione architetturale (PA)}
	\begin{center}
		\textbf{Data di inizio}: 2016-04-29 \\
		\textbf{Data di fine}: 2016-06-17 \\
	\end{center}
	Questo periodo inizia subito dopo il termine del periodo di consolidamento dei requisiti e termina con la data della Revisione di Progettazione. \\
	Il processo principale di questo periodo è:
		\begin{itemize}
			\item \textbf{Documentazione}
			\att
			\begin{itemize}
				\item \textbf{Specifica Tecnica}: viene creato il documento \STdocRP{} che conterrà le scelte progettuali decise dai progettisti; 
				\item \textbf{Norme di Progetto}: viene incrementato questo documento in modo da normare anche la stesura del documento \STdocRP;
				\item \textbf{Piano di Progetto}: viene aggiunto il consuntivo del periodo contenente i periodi di consolidamento dei requisiti e di progettazione architetturale ed il preventivo a finire. Vengono inoltre riportati i rischi che si sono verificati nei periodi precedenti;
				\item \textbf{Piano di Qualifica}: viene aggiunta la parte di pianificazione dei test;
				\item \textbf{Glossario}: viene incrementato con i nuovi termini presenti nella \STdocRP, nel \PQdocRP{} e nelle \NPdocRP.
			\end{itemize}
		\end{itemize}
		\subsubsection{Diagramma di \gl{Gantt} delle attività}
		\gantt{img/gantt/PA}{Diagramma di \gl{Gantt} delle attività - Periodo di progettazione architetturale}
		
		
		
	\subsection{Periodo di progettazione di dettaglio e codifica (PDC)}
	\begin{center}
		\textbf{Data di inizio}: 2016-06-18 \\
		\textbf{Data di fine}: 2016-08-24 \\
	\end{center}
	Questo periodo inizia subito dopo la fine del periodo di progettazione architetturale, ovvero dopo la Revisione di Progettazione, e termina con la data della Revisione di Qualifica. \\
	I processi principali di questo periodo sono: 
		\begin{itemize}
			\item \textbf{Documentazione} 
			\att
			\begin{itemize} 
				\item \textbf{Definizione di \gl{Prodotto}}: viene steso il documento \DPdocRQ{} il quale definisce la struttura e la relazione tra le componenti del \gl{prodotto}. È basato sul documento \STdocRQ;
				\item \textbf{Manuale utente}: viene redatta la versione preliminare del \MUdocRQ{} il quale fornirà agli utenti le indicazioni per l'utilizzo del \gl{prodotto};
				\item \textbf{Incremento altri documenti}: come nel periodo precedente anche in questa vi sarà il miglioramento dei documenti che necessitano tale trattamento.
			\end{itemize}
			\item \textbf{Sviluppo}
			\att
			\begin{itemize}
				\item \textbf{Codifica}: avviene la scrittura del codice dei requisiti obbligatori del \gl{prodotto};
				\item \textbf{Verifica}: per verificare l'efficacia del codice \gl{prodotto} nell'attività di codifica vengono eseguiti i test di unità e di integrazione e ne vengono osservati i risultati. 
			\end{itemize}
		\end{itemize}
		\subsubsection{Diagramma di \gl{Gantt} delle attività}
		% \gantt{img/gantt/PDC}{Diagramma di \gl{Gantt} delle attività - Periodo PDC}
		
		\begin{figure}[!h]
			\centering
			\includegraphics[height=12cm, width=15cm]{img/gantt/PDC} 
			\caption{Diagramma di \gl{Gantt} delle attività - Periodo di progettazione di dettaglio e codifica}
		\end{figure}
		
	\subsection{Periodo di codifica dei requisiti desiderabili e opzionali (CDO)}
	\begin{center}
		\textbf{Data di inizio}: 2016-08-25 \\
		\textbf{Data di fine}: 2016-08-30 \\
	\end{center}
	Questo periodo inizia subito dopo la fine del periodo di progettazione di dettaglio e codifica, ovvero dopo la Revisione di Qualifica, e termina sei giorni dopo. \\
	I processi principali di questo periodo sono: 
		\begin{itemize}
			\item \textbf{Documentazione}:
			\att
			\begin{itemize}
				\item \textbf{Correzioni e aggiornamenti}: Verranno corretti e aggiornati tutti i documenti che lo necessitano. 
			\end{itemize}
			\item \textbf{Sviluppo}:
			\att
			\begin{itemize}
				\item \textbf{Codifica}: avviene la scrittura del codice dei requisiti desiderabili e opzionali del \gl{prodotto};
				\item \textbf{Verifica}: per verificare l'efficacia del codice \gl{prodotto} nell'attività di codifica vengono eseguiti i test di unità e di integrazione e ne vengono osservati i risultati. 
			\end{itemize}
		\end{itemize}
		\subsubsection{Diagramma di \gl{Gantt} delle attività}
		% \gantt{img/gantt/RD}{Diagramma di \gl{Gantt} delle attività - Periodo RD}
		
		\begin{figure}[!h]
			\centering
			\includegraphics[height=12cm, width=15cm]{img/gantt/RD} 
			\caption{Diagramma di \gl{Gantt} delle attività - Periodo di codifica dei requisiti desiderabili e opzionali}
		\end{figure}
		
	\subsection{Periodo di validazione e collaudo (VC)}
	\begin{center}
		\textbf{Data di inizio}: 2016-08-31 \\
		\textbf{Data di fine}: 2016-09-12 \\
	\end{center}
	Questo periodo inizia subito dopo la fine del periodo di codifica dei requisiti desiderabili e opzionali e termina con la data della Revisione di Accettazione. \\
	I processi principali di questo periodo sono:
		\begin{itemize}
			\item \textbf{Documentazione}
			\att
			\begin{itemize}
				\item \textbf{Correzioni e aggiornamenti}: Verranno corretti e aggiornati tutti i documenti che lo necessitano. Si otterrà la versione finale della documentazione. 
			\end{itemize}
			\item \textbf{Sviluppo}
			\att
				\begin{itemize}
					\item \textbf{Test}: vengono eseguiti i test di sistema previsti e ne vengono osservati e monitorati i risultati. 
				\end{itemize}
			\item \textbf{Verifica e validazione}
			\att
			\begin{itemize}
				\item \textbf{Collaudo}: il \gl{prodotto} viene collaudato sulle funzionalità previste;
				\item \textbf{Verifica}: tramite tracciamento si verifica di aver soddisfatto i requisiti presenti nel documento \ARdocRA. Si verificheranno inoltre i canoni di qualità previsti nel \PQdocRA;
				\item \textbf{Validazione}: una volta svolte tutte le verifiche il \gl{prodotto} può considerarsi validato.
			\end{itemize}
		\end{itemize}
		\subsubsection{Diagramma di \gl{Gantt} delle attività}
		% \gantt{img/gantt/V}{Diagramma di \gl{Gantt} delle attività - Periodo V}
		
		\begin{figure}[!h]
			\centering
			\includegraphics[height=11cm, width=15cm]{img/gantt/V} 
			\caption{Diagramma di \gl{Gantt} delle attività - Periodo di validazione e collaudo}
		\end{figure}
