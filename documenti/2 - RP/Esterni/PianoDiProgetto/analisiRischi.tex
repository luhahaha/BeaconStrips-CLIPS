\section{Analisi dei rischi} 
\label{analisiRischi}
	È stata attuata una profonda analisi dei rischi. In questo modo saremo pronti ad affrontarli in caso si presentassero.
	Ogni rischio è stato analizzato seguendo questa scaletta:
	\begin{enumerate}
		\item \textbf{Identificazione}: individuazione dei possibili rischi che si potranno riscontrare durante lo sviluppo del \gl{progetto}.
		\item \textbf{Analisi}: verrà analizzata la probabilità che i rischi si verifichino e come questi potrebbero influire sul lavoro;
		\item \textbf{Pianificazione di controllo}: verranno delineati i metodi grazie ai quali si cercherà di evitare che il rischio si verifichi;
		\item \textbf{Tecniche di mitigazione}: verranno delineati i metodi grazie ai quali verranno mitigati i rischi, nel caso si presentassero.
	\end{enumerate}
	Per ogni rischio verranno riportate le seguenti informazioni:
	\begin{itemize}
		\item \textbf{descrizione};
		\item \textbf{metodi di identificazione};
		\item \textbf{possibilità che si verifichi};
		\item \textbf{pericolosità};
		\item \textbf{conseguenze};
		\item \textbf{contromisure};
		\item \textbf{riscontro effettivo}.
	\end{itemize}
	
	\subsection{Livello tecnologico}
		\subsubsection{Uso di tecnologie e strumenti}
			\begin{itemize}
				\item \textbf{Descrizione}: alcune tecnologie e alcuni strumenti che verranno utilizzati sono sconosciuti ad alcuni membri del gruppo, altri sono sconosciuti a tutti i membri del gruppo; 
				\item \textbf{Metodi di identificazione}: ogni componente del gruppo sarà consapevole delle proprie conoscenze e dei propri limiti in fase di apprendimento;
				\item \textbf{Possibilità che si verifichi}: alta;
				\item \textbf{Pericolosità}: alta;
				\item \textbf{Conseguenze}: rallentamento generale nell'avanzamento del \gl{progetto};
				\item \textbf{Contromisure}:  per evitare che il rischio si presenti ognuno si occuperà di studiare la tecnologia o lo strumento che ha intenzione di usare. Qualora un membro riscontrasse difficoltà con una tecnologia o uno strumento dovrà chiedere aiuto al \RES{} o ad uno degli Amministratori i quali gli forniranno quanto richiesto in forma scritta o verbale;
				\item \textbf{Riscontro effettivo}:
				\begin{itemize}
					\item \textbf{Periodo di Analisi e Management:} i membri del team hanno studiato e testato gli strumenti da utilizzare e non si è riscontrato alcun rallentamento significativo alla progettazione;
					\item \textbf{Periodo di Analisi di Dettaglio:} i membri del team hanno studiato e testato gli strumenti da utilizzare e non si è riscontrato alcun rallentamento significativo alla progettazione;
					\item \textbf{Periodo di Progettazione Architetturale:} i membri del team hanno studiato e testato gli strumenti da utilizzare e non si è riscontrato alcun rallentamento significativo alla progettazione;
				\end{itemize}
			\end{itemize}	
		
		\subsubsection{Danneggiamento strumentazione \gl{hardware}}
			\begin{itemize}
				\item \textbf{Descrizione}: è possibile che i personal computer o altri strumenti in uso dal team subiscano danneggiamenti accidentali.
				\item \textbf{Metodi di identificazione}: ogni membro del gruppo dovrà essere consapevole del funzionamento o meno della strumentazione che possiede e$/$o che ha in uso;
				\item \textbf{Possibilità che si verifichi}: bassa;
				\item \textbf{Pericolosità}: alta;
				\item \textbf{Conseguenze}: rallentamento del lavoro che il proprietario$/$fruitore dello strumento danneggiato dovrebbe svolgere;
				\item \textbf{Contromisure}: non è possibile prevedere danneggiamenti \gl{hardware}, ma per evitare perdite di lavoro ogni componente del gruppo al termine di una sessione di lavoro si occuperà di fare una \gl{commit} sulla \gl{repository}. Nel caso si verifichino danni \gl{hardware} il proprietario$/$fruitore dovrà, se possibile, preoccuparsi di aggiustare lo strumento danneggiato o di procurarne uno sostitutivo. Se lo considererà necessario potrà chiedere aiuto ad uno degli Amministratori; 
				\item \textbf{Riscontro effettivo}:
				\begin{itemize}
					\item \textbf{Periodo di Analisi e Management:} non c'è stato alcun guasto di strumentazione hardware all'interno del team;
					\item \textbf{Periodo di Analisi di Dettaglio:} non c'è stato alcun guasto di strumentazione hardware all'interno del team;
					\item \textbf{Periodo di Progettazione Architetturale:} non c'è stato alcun guasto di strumentazione hardware all'interno del team;
				\end{itemize}
			\end{itemize}
		
		\subsubsection{Problemi \gl{software} strumenti utilizzati}
		\begin{itemize}
			\item \textbf{Descrizione}: è possibile che gli strumenti scelti per agevolare i processi abbiano problemi di varia natura;
			\item \textbf{Metodi di identificazione}: chi utilizza uno strumento che sembra causare problemi lo farà presente ad un \AM{} che si occuperà di verificare la reale esistenza del problema;
			\item \textbf{Possibilità che si verifichi}: media;
			\item \textbf{Pericolosità}: alta;
			\item \textbf{Conseguenze}: forte rallentamento del lavoro;
			\item \textbf{Contromisure}: non è possibile evitare a priori che si verifichino problemi con il SW. Nel caso questi problemi si presentassero gli Amministratori dovranno estinguere il problema se il SW che causa problemi è stato creato dal team, altrimenti provvederà a trovare uno strumento alternativo che faccia un lavoro migliore di quello che causa problemi;
			\item \textbf{Riscontro effettivo}:
			\begin{itemize}
				\item \textbf{Periodo di Analisi e Managment:} si sono riscontrate difficoltà nell'installazione del software Trender all'interno del server dato in dotazione da \PROPONENTE, le quali sono state successivamente risolte contattando il produttore e intervenendo sul codice sorgente. Non si sono riscontrati ritardi significativi nell'avanzamento della progettazione;
				\item \textbf{Periodo di Analisi di Dettaglio:} non si sono riscontrati problemi di questo tipo;
				\item \textbf{Periodo di Progettazione Architetturale:} non si sono riscontrati problemi di questo tipo;
			\end{itemize}
		\end{itemize}
		
		
	\subsection{Livello personale}
	\label{sez2.2}
		\subsubsection{Problemi personali dei componenti}
		\begin{itemize}
			\item \textbf{Descrizione}: ogni membro del gruppo ha impegni relativi alla propria vita privata che potrebbero incidere sulla pianificazione delle attività;
			\item \textbf{Metodi di identificazione}: il \RES{} verrà informato tempestivamente se si presenteranno impegni personali non precedentemente comunicati;
			\item \textbf{Possibilità che si verifichi}: media;
			\item \textbf{Pericolosità}: media;
			\item \textbf{Conseguenze}: rallentamento del lavoro individuale o, in casi più gravi, rallentamento del lavoro dell'intero gruppo;
			\item \textbf{Contromisure}: ogni membro del gruppo dichiarerà all'inizio del \gl{progetto} i propri impegni personali al \RES{} tenendo conto anche dei possibili impegni extra che riesce a prevedere. Nel caso in cui un impegno personale rallenti un membro del gruppo per molto tempo il \RES{} sposterà in avanti le scadenze prefissate, se questo sarà possibile. In alternativa il \RES{} si occuperà di incaricare un altro membro del gruppo a svolgere il lavoro di chi non può farlo; 
			\item \textbf{Riscontro effettivo}:
			\begin{itemize}
				\item \textbf{Periodo di Analisi e Management:} non si sono riscontrati problemi di questo tipo;
				\item \textbf{Periodo di Analisi di Dettaglio:} Tommaso è stato impegnato in un lavoro extra-curricolare che lo ha sottratto a qualche ora di attività di \AN; grazie alla sua segnalazione si è deciso di assegnare le ore mancanti ad altri componenti del team provvedendo a farle recuperare a Tommaso nei periodi successivi.
				\item \textbf{Periodo di Progettazione Architetturale:} l'impegno lavorativo extra-curricolare di Tommaso è proseguito anche in questa fase. Data la minore disponibilità di Tommaso si è deciso di:
					\begin{itemize}
						\item cambiare il ruolo di \RES {}da Tommaso a Matteo; Tommaso sarà quindi responsabile nella prossima fase, in quanto l'impegno lavorativo aggiuntivo non sarà più presente;
						\item cambiare il ruolo di quarto \PRJ {} da Tommaso a Viviana;
						\item assegnare a Tommaso i ruoli di \VER{} e \AN.
					\end{itemize}
			\end{itemize}
		\end{itemize}
		
		
		\subsubsection{Problemi tra i componenti}
		\begin{itemize}
			\item \textbf{Descrizione}: il gruppo è formato da sei persone ed è possibile che in certi momenti del \gl{progetto} sorgano disaccordi e discussioni;
			\item \textbf{Metodi di identificazione}: chi ha problemi con uno o più membri del gruppo deve comunicarlo tempestivamente al \RES{};
			\item \textbf{Possibilità che si verifichi}: bassa;
			\item \textbf{Pericolosità}: media;
			\item \textbf{Conseguenze}: svogliatezza nello svolgere i compiti assegnati, ritardo nel portare a termine tali compiti;
			\item \textbf{Contromisure}: ogni membro del gruppo si impegnerà ad essere disponibile a discussioni costruttive e cercherà di non generare litigi insensati. Qualora questi ultimi sorgessero il \RES{} si occuperà di placare gli animi.
			\item \textbf{Riscontro effettivo}:
			\begin{itemize}
				\item \textbf{Periodo di Analisi e Management:} non si sono riscontrati problemi di questo tipo; 
				\item \textbf{Periodo di Analisi di Dettaglio:} non si sono riscontrati problemi di questo tipo;
				\item \textbf{Periodo di Progettazione Architetturale:} non si sono riscontrati problemi di questo tipo.
			\end{itemize}
		\end{itemize}
	
	\subsection{Livello organizzativo}
		\subsubsection{Errori nella valutazione dei tempi e dei costi}
		\begin{itemize}
			\item \textbf{Descrizione}: è possibile che durante la stesura della pianificazione venga commesso qualche errore riguardo i tempi e i costi dovuto all'inesperienza;
			\item \textbf{Metodi di identificazione}: qualora qualcuno si accorgesse di discrepanze rispetto alla pianificazione deve prontamente notificarlo al \RES{}. Quest'ultimo dovrà assicurarsi, in ogni caso, che lo svolgimento delle attività prosegua secondo i piani; 
			\item \textbf{Possibilità che si verifichi}: alta;
			\item \textbf{Pericolosità}: alta;
			\item \textbf{Conseguenze}: rallentamento nell'ultimazione delle attività pianificate;
			\item \textbf{Contromisure}: ogni membro del gruppo si impegnerà per rispettare le consegne assegnategli. Colui che assegna i \gl{tasks} ai membri del gruppo dovrà tenere conto di possibili ritardi e calcolare che questi non compromettano il lavoro del resto del gruppo; 
			\item \textbf{Riscontro effettivo}:
			\begin{itemize}
				 \item \textbf{Periodo di Analisi e Managment} il team non era a conoscenza che la consegna dell'offerta tecnico-economica fosse anticipata (1 Aprile), rispetto alla data di consegna della documentazione per la RP (11 Aprile). Insieme ai team che si trovavano nella nostra stessa condizione si è deciso di segnalare, tramite una mail comune, il problema al Committente, il quale ha posticipato la data (7 Aprile); questo evento ha prodotto un'accelerazione nell'ultimazione dei documenti, non creando però gravi conseguenze;
				 \item \textbf{Periodo di Analisi di Dettaglio}: sono stati necessari più giorni di quelli inizialmente pianificati per sistemare soddisfacentemente il lavoro di analisi dei requisiti. Conseguentemente si è dovuto accorciare il periodo di Progettazione Architetturale;
				 \item \textbf{Periodo di Progettazione Architetturale:}  %da aggiungere con consuntivo
			\end{itemize}
		\end{itemize}
		
		
		\subsubsection{Problemi di comprensione dei requisiti}
		\begin{itemize}
			\item \textbf{Descrizione}: è possibile che durante l'analisi dei requisiti alcuni aspetti vengano compresi in modo incompleto o, addirittura, errato. Questo sempre a causa dell'inesperienza del gruppo;
			\item \textbf{Metodi di identificazione}: se sorgeranno dubbi riguardo i requisiti ci si accorderà con il Proponente sulla strada giusta da seguire;
			\item \textbf{Possibilità che si verifichi}: media;
			\item \textbf{Pericolosità}: bassa;
			\item \textbf{Conseguenze}: rallentamento del lavoro, possibilità di mancanza di requisiti fondamentali. 
			\item \textbf{Contromisure}: non sono state identificate contromisure efficaci se non il prestare particolare attenzione a quanto detto dal Proponente durante le riunioni sostenute;
			\item \textbf{Riscontro effettivo}:
			\begin{itemize}
				\item \textbf{Periodo di Analisi e Managment}: si è discusso di alcuni dubbi sui requisiti del progetto con il Proponente durante gli incontri effettuati facilitando la parte iniziale dell'analisi dei requisiti;
				\item \textbf{Periodo di Analisi di Dettaglio:} non si sono verificati ulteriori problemi di questo tipo;
				\item \textbf{Periodo di Progettazione Architetturale:} non si sono verificati ulteriori problemi di questo tipo.
			\end{itemize}
		\end{itemize}
