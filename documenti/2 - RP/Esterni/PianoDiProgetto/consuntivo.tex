\section{Consuntivo e preventivo a finire}
\label{consuntivo}
A fronte delle spese sostenute nei vari periodi, verranno illustrate le differenze tra il preventivo e ciò che effettivamente è stato speso. \\
Infine verrà riportato il bilancio totale contenente la somma delle spese rispetto al preventivo e il preventivo a finire.

\subsection{Analisi e management}

\subsubsection{Consuntivo}
Verranno per ogni ruolo le ore e le spese effettivamente sostenute durante il periodo di analisi e management (ore non rendicontate). % manca una parola tra "ruolo" e "le ore" mi sa

\begin{tabella}{!{\VRule}c!{\VRule}c!{\VRule}c!{\VRule}}
		
	\intestazionethreecol{Ruolo}{Ore}{Costo}
		
	Responsabile& 12(-2) & \euro360,00(-\euro60,00)\\
	Amministratore & 36(-4) & \euro720,00(-\euro80,00)\\
	Analista & 50(+3) & \euro1250,00(+\euro75,00) \\
	Progettista & - & - \\
	Programmatore & - & -\\
	Verificatore & 37(+3) & \euro555,00(+\euro45,00) \\
	\hline
	\textbf{Totale consuntivo} & 135 & \euro2865,00\\
	\textbf{Totale preventivo} & 135 & \euro2885,00\\
	\textbf{Totale (differenza)} & - & -\euro20,00\\
		
	\hiderowcolors
	\caption{Ore non rendicontate - differenza preventivo/consuntivo periodo di analisi e management}
		
\end{tabella}
	
\subsubsection{Conclusioni}
Il team ha impiegato tre ore in più in attività da \AN, tre in più in attività di verifica, 4 in meno in attività di amministrazione e 2 in meno in attività da \RES. Si è quindi riusciti a risparmiare \textbf{\euro20,00} che non potranno essere riutilizzate poichè sono ore non rendicontate.
% da \AN -> di analisi
% 4 -> quattro
% 2 -> due
%  Si è quindi riusciti a risparmiare -> si sono quindi risparmiati

\newpage

\subsection{Analisi di dettaglio}

\subsubsection{Consuntivo}
Verranno indicate, per ogni ruolo, le ore e le spese effettivamente sostenute durante il periodo di analisi di dettaglio (ore rendicontate).

\begin{tabella}{!{\VRule}c!{\VRule}c!{\VRule}c!{\VRule}}
	
	\intestazionethreecol{Ruolo}{Ore}{Costo}
	
	Responsabile& 5 & \euro150,00\\
	Amministratore & 8(-2) & \euro160,00(-\euro40,00)\\
	Analista & 30(+4) & \euro750,00(+\euro100,00) \\
	Progettista & - & - \\
	Programmatore & - & -\\
	Verificatore & 17(-1) & \euro255,00(-\euro15,00) \\
	\hline
	\textbf{Totale consuntivo} & 61 & \euro1350,00\\
	\textbf{Totale preventivo} & 60 & \euro1315,00\\
	\textbf{Totale (differenza)} & +1 & +\euro45,00\\
	
	\hiderowcolors
	\caption{Ore rendicontate - differenza preventivo/consuntivo periodo di analisi di dettaglio}

\end{tabella}

\subsubsection{Conclusioni}
Il team ha impiegato quattro ore in più in attività di analisi, due in meno in attività di amministrazione e una in meno in attività di verifica. È stata utilizzando un' ora in più rispetto a quelle preventivate e sono stati utilizzati \textbf{\euro45,00} in più rispetto a quanto preventivato.

\subsubsection{Preventivo a finire}
Il consuntivo supera il preventivo quindi il bilancio è in negativo di \textbf{\euro45,00} e il team si impegnerà nel periodo di progettazione architetturale ad ammortizzare i costi.\\
% grazie au due periodi abbiamo capito che blablabla (tipo)
Grazie ai due periodi passati si cercherà di risparmiare, se possibile, nelle ore di \AM{} e \RES{} vista la facilità a gestire le attività di gestione e amministrazione del progetto.
La distribuzione delle ore è stata modificata per limitare i rischi illustrati nella \hyperref[sez2.2]{sezione 2.2}; infatti si è deciso di scambiare i ruoli di Tommaso e Matteo nei periodi di progettazione architetturale e progettazione di dettaglio. Queste modifiche non hanno apportato cambiamenti nel conteggio totale delle ore di ciascun componente nè del preventivo totale.