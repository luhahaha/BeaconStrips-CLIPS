\section{Consuntivo e preventivo a finire}
\label{consuntivo}
A fronte delle spese sostenute nei vari periodi, verranno illustrate le differenze tra il preventivo e ciò che effettivamente è stato speso. \\
Infine verrà riportato il bilancio totale contenente la somma delle spese rispetto al preventivo e il preventivo a finire.

\subsection{Analisi e Management}

\subsubsection{Consuntivo parziale}
Verranno indicate, per ogni ruolo, le ore e le spese effettivamente sostenute durante il periodo di Analisi e Management (ore non rendicontate). 

\begin{tabella}{!{\VRule}c!{\VRule}c!{\VRule}c!{\VRule}}
		
	\intestazionethreecol{Ruolo}{Ore}{Costo}
		
	Responsabile& 12(-2) & \euro360,00(-\euro60,00)\\
	Amministratore & 36(-4) & \euro720,00(-\euro80,00)\\
	Analista & 50(+3) & \euro1250,00(+\euro75,00) \\
	Progettista & - & - \\
	Programmatore & - & -\\
	Verificatore & 37(+3) & \euro555,00(+\euro45,00) \\
	\hline
	\textbf{Totale consuntivo} & 135 & \euro2865,00\\
	\textbf{Totale preventivo} & 135 & \euro2885,00\\
	\textbf{Totale (differenza)} & - & -\euro20,00\\
		
	\hiderowcolors
	\caption{Ore non rendicontate - differenza preventivo/consuntivo periodo di Analisi e Management}
		
\end{tabella}
	
\subsubsection{Conclusioni}
Il team ha impiegato tre ore in più in attività di analisi, tre in più in attività di verifica, quattro in meno in attività di amministrazione e due in meno in attività da \RES.
Poichè il periodo di Analisi e Management non fa parte delle ore rendicontate i \textbf{\euro20,00} risparmiati non potranno essere riutilizzati nei periodi seguenti.

\newpage

\subsection{Analisi di Dettaglio}

\subsubsection{Consuntivo parziale}
Verranno indicate, per ogni ruolo, le ore e le spese effettivamente sostenute durante il periodo di Analisi di Dettaglio (ore rendicontate).

\begin{tabella}{!{\VRule}c!{\VRule}c!{\VRule}c!{\VRule}}
	
	\intestazionethreecol{Ruolo}{Ore}{Costo}
	
	Responsabile& 5 & \euro150,00\\
	Amministratore & 8(-2) & \euro160,00(-\euro40,00)\\
	Analista & 30(+4) & \euro750,00(+\euro100,00) \\
	Progettista & - & - \\
	Programmatore & - & -\\
	Verificatore & 17(-1) & \euro255,00(-\euro15,00) \\
	\hline
	\textbf{Totale consuntivo} & 61 & \euro1350,00\\
	\textbf{Totale preventivo} & 60 & \euro1315,00\\
	\textbf{Totale (differenza)} & +1 & +\euro45,00\\
	
	\hiderowcolors
	\caption{Ore rendicontate - differenza preventivo/consuntivo periodo di Analisi di Dettaglio}

\end{tabella}

\subsubsection{Conclusioni}
Il team ha impiegato quattro ore in più in attività di analisi, due in meno in attività di amministrazione e una in meno in attività di verifica. È stata utilizzata un' ora in più rispetto a quelle preventivate e sono stati utilizzati \textbf{\euro45,00} in più rispetto a quanto preventivato. \\
Il consuntivo supera il preventivo quindi il bilancio è in negativo di \textbf{\euro45,00} e il team si impegnerà nel periodo di Progettazione Architetturale ad ammortizzare i costi.\\
Grazie ai due periodi passati abbiamo riscontrato una notevole facilità nel gestire le attività di amministrazione e gestione del progetto per questo si cercherà di risparmiare, se possibile, nelle ore di \AM{} e \RES{}. A tal proposito è stato modificato il preventivo del periodo di Progettazione Architetturale come segue:

\begin{itemize}
	\item per far fronte all'ora in più utilizzata nel periodo di Analisi e Management, è stata tolta un ora di \RES{} al periodo di Progettazione Architetturale;
	\item per far fronte ai \textbf{\euro45,00} spesi in più nel periodo di analisi e di management, sono state tolte un' ora di \AM{} e un'ora di \AN, mentre sono state aggiunte due ore di \VER.
\end{itemize}

Grazie a questi cambiamenti il preventivo riesce ad ammortizzare i \textbf{\euro45,00} in eccesso. Il preventivo rimane quindi \textbf{11581\euro} come precedentemente stabilito.\\

È stata inoltre modificata la distribuzione delle ore per limitare i rischi illustrati nella \hyperref[sez2.2]{sezione 2.2}; infatti si è deciso di scambiare i ruoli di Tommaso e Matteo nei periodi di Progettazione Architetturale e Progettazione di Dettaglio. Queste modifiche non hanno apportato cambiamenti nel conteggio totale delle ore di ciascun componente nè del preventivo totale.

\subsection{Progettazione Architetturale}

\subsubsection{Consuntivo parziale}
Verranno indicate, per ogni ruolo, le ore e le spese effettivamente sostenute durante il periodo di Progettazione Architetturale (ore rendicontate).

\begin{tabella}{!{\VRule}c!{\VRule}c!{\VRule}c!{\VRule}}
	
	\intestazionethreecol{Ruolo}{Ore}{Costo}
	
	Responsabile& 9 & \euro270,00\\
	Amministratore & 9(-1) & \euro180,00(-\euro20,00)\\
	Analista & 19(-3) & \euro475,00(-\euro75,00) \\
	Progettista & 91(+4) &  \euro2002,00(+\euro88,00) \\
	Programmatore & - & -\\
	Verificatore & 77 & \euro1155,00 \\
	\hline
	\textbf{Totale consuntivo} & 60 & \euro4075,00\\
	\textbf{Totale preventivo} & 60 & \euro4082,00\\
	\textbf{Totale (differenza)} & 0 & -\euro7,00\\
	
	\hiderowcolors
	\caption{Ore rendicontate - differenza preventivo/consuntivo periodo di Progettazione Architetturale}
	
\end{tabella}

\subsubsection{Conclusioni}
Il team ha impiegato quattro ore in più in attività di progettazione, tre in meno in attività di analisi e una in meno in attività di amministrazione. Non sono state utilizzate ore in più rispetto a quanto preventivato, ma sono stati risparmiati \euro7,00 grazie al cambio delle ore mostrato sopra.

\subsubsection{Preventivo a finire}
Grazie alle modifiche del preventivo effettuate al termine del periodo di Analisi di Dettaglio, si è riusciti ad ammortizzare i costi in eccesso e a risparmiare, infatti ora il preventivo supera il consuntivo: il bilancio quindi è in positivo di \euro7,00.
Il team potrà utilizzare questi \euro7,00 nei periodi successivi nel caso fosse necessario un maggiore dispendio di ore.