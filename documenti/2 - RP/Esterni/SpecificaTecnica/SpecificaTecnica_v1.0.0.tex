\documentclass[a4paper,titlepage]{article}

\makeatletter
\def\input@path{{../../../template/}{./img}}
\makeatother

\usepackage{Comandi}
\usepackage{Riferimenti}
\usepackage{Stile}

\usepackage{eurosym}
\usepackage{comment}
\usepackage{hyperref}

% -- COMANDI PER LE COMPONENTI -- %
\newenvironment{componente}[1]
{
	\subsubsection{Componente \texttt{#1}}
	\label{#1}
	Informazioni sul package
	\begin{itemize}
}{
	\end{itemize}
}
\newcommand\compDescrizione[1] {
	\item \textbf{Descrizione}: [#1]
}
\newcommand\compPadre[1] {
	\item \textbf{Padre}: [#1]
}
\newcommand\compUtilizzo[1] {
	\item \textbf{Utilizzo}: [#1]
}
\newenvironment{compPackageContenuti}[1] {
	\item \textbf{Package contenuti}: [#1]
	\begin{itemize}
} {
	\end{itemize}
}
\newenvironment{compClassi} [1] {
	\item \textbf{Classi}: [#1]
	\begin{itemize}
} {
	\end{itemize}
}
\newenvironment{classe} [1] {
	\texttt{#1}
	\begin{itemize}
} {
	\end{itemize}
}
\newcommand\classeDescrizione [1] { \item \textbf{Descrizione}: #1 }
\newcommand\classeUtilizzo [1] { \item \textbf{Utilizzo}: #1 }
\newenvironment{classeAttributi} [1] {
	\item \textbf{Attributi}: #1
	\begin{itemize}
} {
	\end{itemize}
}
\newcommand\classeAttributo[3] { \item \texttt{#1: #2} \\ #3}
\newenvironment{classeMetodi} [1] {
	\item \textbf{Metodi}: #1
	\begin{itemize}
} {
	\end{itemize}
}
\newcommand\classeMetodo[4]{
	\item \texttt{#1(#2) : #3} \\ #4
}
\newenvironment{classeMetodoArgomenti} {
	\item \textbf{Argomenti}:
	\begin{itemize}
}{
	\end{itemize}
}
\newcommand\classeMetodoArgomento[3]{\item \texttt{#1 : #2} \\ #3 }
\newenvironment{classeRelazioni} {
	\item \textbf{Relazioni con altre classi}:
	\begin{itemize}
} {
	\end{itemize}
}
\newcommand\classeRelazione[3]{\item \textt{#1::#2}: #3}

\def\NOME{Specifica Tecnica}
\def\VERSIONE{1.0.0}
\def\DATA{2016-06-xx}
\def\REDATTORE{Viviana Alessio \\ & Matteo Franco \\ & Andrea Grandene \\ & Luca Soldera}
\def\VERIFICATORE{Enrico Bellio \\ & Tommaso Panozzo}
\def\RESPONSABILE{Matteo Franco}
\def\USO{Esterno}
\def\DESTINATARI{\COMMITTENTE \\ & \CARDIN \\ & \PROPONENTE}
\def\SOMMARIO{Descrizione dell'architettura ad alto livello per il \gl{progetto} \PROGETTO.}

\begin{document}


\maketitle

\begin{diario}
	\modifica{Viviana Alessio}{\PRJ}{Stesura sezione "\hyperref[attivita]{Diagrammi delle attività}"}{2016-05-25}{0.1.3}
	\modifica{Viviana Alessio}{\PRJ}{Stesura appendice "\hyperref[pattern]{Design pattern utilizzati}"}{2016-05-21}{0.1.2}
	\modifica{Viviana Alessio}{\PRJ}{Stesura sezione "\hyperref[tecnologie]{Tecnologie}"}{2016-05-20}{0.1.1}
	\modifica{Viviana Alessio}{\PRJ}{Stesura sezione "\hyperref[introduzione]{Introduzione}"}{2016-05-12}{0.1.0}
	\modifica{Viviana Alessio}{\PRJ}{Stesura indice}{2016-05-10}{0.0.2}
	\modifica{Matteo Franco}{\PRJ}{Creazione documento}{2016-05-09}{0.0.1}
\end{diario}

\newpage
\tableofcontents
\newpage
\listoftables
\newpage
\listoffigures
\newpage

\section{Introduzione}
	\subsection{Scopo del documento} 
	Questo documento ha lo scopo di spiegare dettagliatamente le strategie secondo cui il gruppo \gl{Beacon Strips} intende condurre il progetto didattico.
	\subsection{Riferimenti}
	\subsection{Ciclo di vita}
	\subsection{Scadenze}
	

\section{Tecnologie}
\label{tecnologie}
Per lo sviluppo del progetto abbiamo la necessità di utlizzare delle tecnologie specifiche che sono state scelte dopo attenta analisi.
In questa sezione del documento vengono riportate quelle che verranno utilizzate principalmente.

\subsection{Java}

	Java è virtualmente alla base di qualsiasi tipo di applicazione in rete ed è lo standard globale per lo sviluppo e la distribuzione di applicazioni incorporate e per sistemi portatili, giochi, contenuto basato su Web e software aziendale. Con oltre 9 milioni di sviluppatori in tutto il mondo, Java consente di sviluppare, distribuire e utilizzare applicazioni e servizi in modo efficiente. \\

	\begin{itemize}
		\item \textbf{Vantaggi:}
			\begin{itemize}
				\item è un linguaggio molto diffuso;
				\item è un linguaggio libero e gratuito; %la licenza è GNU GPL ma penso di non specificarla perché altrimenti dovrei definirne le caratteristiche
				\item è un linguaggio multipiattaforma, ovvero dallo stesso codice si ottiene lo stesso programma, a prescindere dal sistema operativo usato e dalle caratteristiche specifiche dell'elaboratore;
				\item contiene già di base molti design pattern e strutture varie, semplificando la loro implementazione;
				\item sono disponibili numerose librerie utilizzabili con questo linguaggio, di conseguenza sono presenti molte funzioni già definite e validate;
				\item è disponibile online una documentazione molto dettagliata;
				\item utilizza una sintassi molto simile al C++, un linguaggio che tutti i componenti del \gl{team}\ conoscono;
				\item rispetto al C++ è più semplice perché gestisce automaticamente molte funzionalità, come ad esempio la memoria e quindi l'allocazione e la deallocazione degli oggetti;
			\end{itemize}
		\item \textbf{Svantaggi:}
			\begin{itemize}
				\item essendo un linguaggio interpretato è più lento in fase di esecuzione rispetto ad uno nativo, sebbene questo svantaggio fosse più importante in passato, perché ormai la potenza che hanno raggiunto gli elaboratori è di gran lunga maggiore di quella necessaria per superare questo problema, di fatto al giorno d'oggi questo svantaggio si presenta solo con applicazioni molto pesanti, come i programmi grafici, o in supporti hardware più limitati, come ad esempio gli smartphone e i tablet;
				\item richiede un interprete, e quindi un'ulteriore applicazione, comunque già presente in Android visto che tutte le sue applicazioni sono scritte in Java.
			\end{itemize}
	\end{itemize}
	\subsubsection{Utilizzo}
	Java viene utilizzato come linguaggio di programmazione per creare applicazioni mobile per il sistema operativo Android per questo verrà utilizzato dal team.

\subsection{Android}

	Android è il sistema operativo più diffuso al mondo, usato per smartphone, tablet e altri dispositivi, come Android TV e Google Glass, sviluppati da Google, la proprietaria del sistema operativo. Strutturalmente deriva da Linux, ma usa Java per le applicazioni che interagiscono con l'utente. In pratica il linguaggio di programmazione principale previsto è Java, più supportato e ricco di funzionalità già fornite di base, mentre in genere si usano C e C++ per ottenere prestazioni migliori, sebbene siano presenti meno librerie rispetto al primo linguaggio. Esiste comunque la possibilità di usare entrambi per la stessa applicazione, ad esempio utilizzando Java per la struttura principale e C++ per le parti più pesanti da eseguire. Il \gl{team} userà solo Java, dato che il programma sviluppato dovrebbe essere abbastanza leggero da eseguire. \\
	Android inoltre fornisce delle proprie librerie, che, insieme alla struttura del sistema operativo, caratterizzano la stesura del codice dell'applicazione. Ad esempio ogni programma viene eseguito in una macchina virtuale propria, così la sua esecuzione non può modificare né il sistema operativo né le altre applicazioni, mentre la comunicazione con gli altri programmi può avvenire soltanto tramite dei pattern specifici. La struttura del programma sfrutta altri pattern già definiti, come le Activity, che caratterizzano lo sviluppo dell'applicazione. Android quindi presenta vantaggi e svantaggi differenti rispetto a Java.

	\begin{itemize}
		\item \textbf{Vantaggi:}
			\begin{itemize}
				\item è il sistema operativo più usato al mondo;
				\item è libero e gratuito;
				\item è disponibile online una documentazione molto dettagliata sulle librerie di Android;
				\item per sviluppare un'applicazione Android bisogna usare dei pattern già definiti, semplificando il codice da scrivere e lo studio della loro applicazione;
				\item dato che Android è stato pensato per i dispositivi mobile la connessione ad Internet, il GPS e le altre interazioni con sistemi di comunicazione esterni all'applicazione sono automatici, quindi il programmatore non deve conoscere come siano implementatoi ma soltanto come interagirci tramite il codice;
				\item l'utilizzo di altri linguaggi come complemento a Java, ad esempio l'XML per le GUI, semplifica il lavoro da svolgere e lo rende più intuitivo;
				\item dato che il programma viene eseguito in una macchina virtuale non può in alcun modo danneggiare né il sistema operativo né le altre applicazioni;
				\item Android non viene usato solo per dispositivi mobile, di conseguenza è possibile adattare o creare un programma per gli altri prodotti Google che usano questo sistema operativo;
				\item sono presenti numerose librerie, di cui molte sono ufficiali di Android.
			\end{itemize}
		\item \textbf{Svantaggi:}
			\begin{itemize}
				\item dato che Android è open source ogni azienda che lo usa tende a personalizzarlo, quindi la stessa applicazione può dare risultati leggermente diversi cambiando dispositivo, di conseguenza la programmazione potrebbe dover variare per rispondere a queste differenze;
				\item l'utilizzo di linguaggi complementari, come l'XML e il C/C++, può richiedere una maggiore conoscenza per poterli sfruttare, aumentando quindi il carico di lavoro necessario per poter scrivere l'applicazione.
			\end{itemize}
	\end{itemize}
	\subsubsection{Utilizzo}
	Il team ha deciso di utilizzare Android come linguaggio per la creazione dell'applicazione del progetto, applicazione che costituisce la parte Client del progetto.

\subsection{XML}

	XML è un metalinguaggio per la definizione di linguaggi di markup, ovvero un linguaggio marcatore basato su un meccanismo sintattico che consente di definire e controllare il significato degli elementi contenuti in un documento o in un testo.

	\begin{itemize}
		\item \textbf{Vantaggi:}
			\begin{itemize}
				\item è un linguaggio molto diffuso;
				\item è un linguaggio libero e gratuito;
				\item è disponibile online una documentazione molto dettagliata;
				\item permette di definire dei propri tag e di fissare il loro contenuto, evitando così problemi di interpretazione dei termini, anche se nel caso della nostra applicazione questo vantaggio è irrilevante, visto che siamo noi ad usare dei tag già definiti da Google;
				\item i file XML vengono gestiti da quasi tutti i linguaggi di programmazione, permettendo così una completa integrazione con le applicazioni.
			\end{itemize}
		\item \textbf{Svantaggi:}
			\begin{itemize}
				\item l'utilizzo degli schemi per fissare il tipo di contenuto dei tag può rendere il controllo dei contenuti molto pesante;
				\item anche il file XML stesso può aumentare parecchio le dimensioni, arrivando addirittura ad un incremento esponenziale, sebbene esistano delle tecniche di compressione che permettono di ridurre notevolmente la grandezza del file;
				\item il file XML è poco significativo finché non viene usato un programma per interpretare i tag previsti e di conseguenza eseguire delle azioni ben precise.
			\end{itemize}
	\end{itemize}
	\subsubsection{Utilizzo}
	In particolare nello sviluppo di applicazioni Android la parte grafica viene scritta proprio attraverso file XML, quindi questo linguaggio è essenziale per lo sviluppo del nostro progetto.


\subsection{JSON}
	JSON (JavaScript Object Notation) è un semplice formato per lo scambio di dati. È scritto in JavaScript, un linguaggio molto usato nel mondo del Web che ha permesso a JSON di diffondersi rapidamente. Viene usato al posto di XML, anche se quest'ultimo nasce come linguaggio di markup e non specificatamente per lo scambio di dati. Per le persone è facile da leggere e scrivere, mentre per le macchine risulta facile da generare e analizzarne la sintassi.

	\begin{itemize}
		\item \textbf{Vantaggi:}
			\begin{itemize}
				\item è un linguaggio molto diffuso;
				\item è un linguaggio libero e gratuito;
				\item è disponibile online una documentazione molto dettagliata;
				\item è conciso e facile da leggere;
				\item è leggero, garantendo delle prestazioni migliori per le richieste e le risposte tra client e server;
				\item è supportato da tutti i browser che implementano Javascript, dato che è scritto con questo linguaggio;
				\item permette di distinguere i tipi di dato inviati, ad esempio può specificare se viene inviato un booleano, un numero intero o una stringa;
				\item molti linguaggi di programmazione possono gestire JSON e le relative comunicazione tra client e server tramite delle librerie apposite.
			\end{itemize}
		\item \textbf{Svantaggi:}
			\begin{itemize}
				\item è un linguaggio efficiente solo per lo scambio di dati tra client e server, quando serve in locale esistono sistemi più efficienti;
				\item rispetto a XML ha funzionalità più limitate, anche se generalmente le caratteristiche mancanti non sono necessarie per svolgere il lavoro previsto;
				\item non è validato, quindi se il mittente omette campi o sbaglia il formato dei dati, client diversi potrebbero interpretare diversamente il JSON.
			\end{itemize}
	\end{itemize}
	\subsubsection{Utilizzo}
	Nel progetto il team utilizzerà questo formato per lo scambio dei dati tra Server e Client.

\subsection{JavaScript}

	Javascript è un linguaggio di scripting orientato agli oggetti e agli eventi, utilizzato soprattutto nella programmazione Web. Un linguaggio di scripting è un tipo di linguaggio che viene integrato all'interno di un altro programma, nel caso del Web ad esempio l'interprete Javascript viene ospitato dal browser, che esegue il codice delle pagine quando esse vengono chiamate.

	\begin{itemize}
		\item \textbf{Vantaggi:}
			\begin{itemize}
				\item è un linguaggio molto diffuso;
				\item è un linguaggio libero e gratuito;
				\item è disponibile online una documentazione molto dettagliata;
				\item dato che il codice viene eseguito in locale dal client il trasferimento dei dati risulta leggero e al server basta eseguire poche istruzioni;
				\item è semplice e meno restrittivo di altri linguaggi famosi, come C++ e Java, perché ad esempio le variabili non sono tipizzate;
				\item molti linguaggi di programmazione integrano e gestiscono Javascript tramite delle librerie apposite.
			\end{itemize}
		\item \textbf{Svantaggi:}
			\begin{itemize}
				\item non è un linguaggio sicuro, perché ad esempio permette di eseguire azioni malevole con un semplice script, anche se al giorno d'oggi questo problema è stato risolto in parte dai browser, usando accorgimenti come il blocco dei pop-up;
				\item essendo un linguaggio meno restrittivo la stesura del codice Javascript è più esposto agli errori di programmazione, ad esempio il cast automatico di variabili può provocare dei comportamenti inaspettati dello script in certe condizioni;
				\item nonostante esista uno standard universale di Javascript, le istruzioni implementate dai browser possono essere leggermente diverse a seconda del tipo o addirittura della versione del browser stesso.
			\end{itemize}
	\end{itemize}
	\subsubsection{Utilizzo}
	Nell'applicazione sviluppata dal \gl{team}\ questo linguaggio viene usato per impostare le risposte del server da inviare al client, utilizzando JSON per il trasferimenti dei dati e per effettuare le query per la gestione dei dati del database.

\subsection{SQL}
	SQL (Structured Query Language) è un linguaggio standardizzato per database basati sul modello relazionale (RDBMS) progettato per:
	\begin{itemize}
		\item 	creare e modificare schemi di database;
		\item 	inserire, modificare e gestire dati memorizzati;
		\item 	interrogare i dati memorizzati;
		\item 	creare e gestire strumenti di controllo ed accesso ai dati.
	\end{itemize}
	È costruito per essere semplice, sia in scrittura sia in lettura, e poco verboso, e per permettere automaticamente una gestione ottimale dei dati. \\

	\begin{itemize}
		\item \textbf{Vantaggi:}
			\begin{itemize}
				\item è un linguaggio molto diffuso;
				\item è un linguaggio libero e gratuito;
				\item è disponibile online una documentazione molto dettagliata;
				\item è semplice da utilizzare per eseguire qualsiasi delle sue operazioni;
				\item è efficiente, tutte le azioni che si possono eseguire sono leggere e veloci;
				\item molti linguaggi di programmazione permettono l'interrogazione dei database SQL tramite delle apposite librerie;
				\item è un linguaggio molto restrittivo, garantendo così un'alta solidità del database.
			\end{itemize}
		\item \textbf{Svantaggi:}
			\begin{itemize}
				\item essendo un linguaggio molto restrittivo SQL non permette di eseguire alcune operazioni avanzate, e inoltre un cambiamento, anche piccolo, del database può risultare più difficile del previsto, ad esempio perché crea temporaneamente un'incongruenza delle chiavi primarie;
				\item il database SQL non può essere partizionato e suddiviso in più parti, perché risulterebbe molto difficile o addirittura impossibile effettuare i controlli di integrità del database, di conseguenza diventa problematico ingrandirlo quando è necessario memorizzare moltissimi dati.
			\end{itemize}
	\end{itemize}
	\subsubsection{Utilizzo}
	Nell'applicazione SQL verrà utilizzato per la creazione del data base e per la gestione, tramite query, dei dati.

\subsection{Librerie}
\subsubsection{Kontakt.io Android Proximity SDK}
	Kontakt.io Android Proximity SDK è una libreria per Android che supportai formati iBeacon e Eddystone.\\
	Fornisce gli strumenti per la gestione e la creazione di applicazioni che utilizzino i beacon kontakt.io

	\begin{itemize}
		\item \textbf{Vantaggi:}
			\begin{itemize}
				\item semplici da utilizzare;
				\item sviluppata da un'azienda giovane che garantisce molti aggiornamenti;.
			\end{itemize}
		\item \textbf{Svantaggi:}
			\begin{itemize}
				\item utilizza due formati proprietari;
				\item supportano solo i beacon kontakt.io.
			\end{itemize}
	\end{itemize}
	\paragraph{Utilizzo}
	Nel nostro progetto l'interazione coi beacon verrà utilizzata durante lo svolgimento dei percorsi, per questo la libreria Kontakt verrà utilizzata dalle classi all'interno del package \texttt{pathprogress}, componente che controlla l'avanzamento dell'utente nel percorso che sta svolgendo.

\subsubsection{Volley}
Volley è una libreria Android che permette e facilita la comunicazione tra client e server tramite HTTP. Effettua automaticamente le chiamate, asincrone o sincrone, al server, in base alle impostazioni fornite dal programmatore. È nata per sopperire alla mancanza di librerie simili in Java, dove le uniche classi presenti per effettuare chiamate REST sono obsolete e non esenti da errori. Il risultato ottenuto è rappresentato da una libreria molto veloce, leggera e soprattutto con un'eccellente gestione della memoria.

	\begin{itemize}
		\item \textbf{Vantaggi:}
			\begin{itemize}
				\item è una libreria libera e gratuita;
				\item è una libreria ufficiale di Android, quindi il suo funzionamento e il suo mantenimento vengono garantiti;
				\item sono disponibili online delle spiegazioni su come farla funzionare ed usare;
				\item imposta e gestisce automaticamente le chiamate da effettuare al server, quindi il programmatore deve solo fornire i dati necessari;
				\item le chiamate che effettua possono essere già asincrone, quindi il programmatore non deve preoccuparsi di gestire i thread;
				\item permette di inviare vari tipi di oggetti, tra cui quelli scritti in JSON;
				\item gestisce molto bene la memoria in modo automatico, è veloce e leggera.
			\end{itemize}
		\item \textbf{Svantaggi:}
			\begin{itemize}
				\item le spiegazioni online a volte sono scarse o poco chiare;
				\item Volley è adatto per le piccole comunicazioni con il server, ma quando è richiesto, ad esempio, di fare un upload abbastanza grande, la libreria mostra delle lacune, come la mancanza di una barra di avanzamento. Nella nostra applicazione non dovrebbero esserci problemi del genere, dato che i dati scambiati sono abbastanza piccoli e le chiamate risultano essere semplici;
				\item non è la libreria che garantisce il minor tempo di attesa per inviare e ricevere una risposta, altre come Retrofit sono più performanti.
			\end{itemize}
	\end{itemize}
	\paragraph{Utilizzo}
	Nel nostro progetto la libreria Volley è stata utilizzata per effettuare le chiamate dal Client al Server, infatti sono necessarie delle richieste per ottenere i dati dal database e per aggiornarli qual'ora un utente svolga dei percorsi e desideri salvarne il risultato.
	Nello specifico tutte le chiamate verranno effettuate dalle classi all'interno del package \texttt{urlrequest}.


	\subsubsection{Travis CI}
	Travis CI è un servizio di \gl{Continuous Integration} distribuita usato per compilare e testare le API server del progetto, ospitate da \gl{GitHub}. La configurazione avviene attraverso un file \texttt{.travis.yaml} nella root directory del progetto che descrive le istruzioni da eseguire per compilare e testare il prodotto.

		\begin{itemize}
			\item \textbf{Vantaggi:}
				\begin{itemize}
					\item è una servizio gratuito per le repository pubbliche;
					\item ad ogni push di codice su GitHub l'intero progetto viene compilato e i test vengono eseguiti evidenziando subito se qualche commit ha introdotto errori o regressioni;
					\item è ben integrato con \gl{Slack} cui invia le notifiche sul successo o fallimento dei test;
					\item l'integrazione richiede solamente la stesura di un file \textit{YAML} di piccole dimensioni.
				\end{itemize}
			\item \textbf{Svantaggi:}
				\begin{itemize}
					\item il linguaggio \textit{YAML} per la definizione delle istruzioni non è conosciuto da alcun componente;
					\item il linguaggio \textit{YAML} è sì conciso ma allo stesso tempo può essere complesso.
				\end{itemize}
		\end{itemize}
		\paragraph{Utilizzo}
		Nel nostro progetto, il file di configurazione di \textit{Travis CI} si occupa di:
		\begin{itemize}
			\item installare \textit{node} nella macchina ospite;
			\item installare le dipendenze del back-end;
			\item installare \textit{mySQL};
			\item creare la struttura del database;
			\item popolare il database con dei dati di test;
			\item avviare il server del backend;
			\item eseguire i test;
			\item notificare su Slack il successo o il fallimento delle operazioni precedenti.
		\end{itemize}






		\subsubsection{Mocha}
		\textit{Mocha} è un framework di test per \gl{Node.js}. Attraverso \textit{Mocha} è facile eseguire in serie dei test asincroni (caratteristica indispensabile per testare il backend) e si generano degli accurati report che aiutano ad identificare gli errori.

			\begin{itemize}
				\item \textbf{Vantaggi:}
					\begin{itemize}
						\item è un framework molto diffuso per il test di progetti \textit{JavaScript} e \textit{Node.js} in particolare;
						\item riduce notevolmente il codice necessario ad eseguire test asincroni;
						\item non richiede l'integrazione di alcun pacchetto nel codice in produzione, solamente nelle classi che si occupano del test;
						\item è perfettamente integrato con le Promise di JavaScript ampiamente utilizzate nel codice.
					\end{itemize}
				\item \textbf{Svantaggi:}
					\begin{itemize}
						\item è necessario leggere la documentazione presente nel sito ufficiale per comprenderne il funzionamento.
					\end{itemize}
			\end{itemize}
			\paragraph{Utilizzo}
			Nel nostro progetto \textit{Mocha} viene usato per i test del backend.


			\subsubsection{npm}
			npm è un package manager per \textit{JavaScript}, utile per gestire le dipendenze e creare un pacchetto facilmente installabile



\section{Descrizione architettura} 
\label{architettura}
	In questa sezione verrà descritta lo schema generale dell'architettura, nella sezione successiva % aggiungere label
	verranno riportate tutti i packages e le classi dettagliatamente.
	 Per i diagrammi delle classi e di attività è stato utilizzato il formalismo UML 2.0. \\
	 L'architettura descritta in questo documento è ad alto livello. Classi, sottoclassi e attributi verranno descritti più dettagliatamente nel periodo in cui attueremo la Progettazione di dettaglio. \\
	 Si procederà nella descrizione dell'architettura con un approccio top-down. Saranno descritte prima quindi le parti più generali per poi andare sempre più nello specifico. Si procederà quindi prima con la descizione dei packages e delle componenti, per poi passare alle singole classi. Successivamente verranno descritti i design pattern utilizzati.
	
	\subsection{Architettura generale}
	L'architettura generale dell'applicazione segue il modello client - server. \\
	In particolare si utilizzerà lo stile architetturale REST (representational state transfer) per coordinare compomenti, connettori e dati attraverso un sistema ipermediale distribuito dove l'attenzione è data al ruolo delle componenti.
	
	\begin{itemize}
		\item \textbf{Vantaggi}: REST offre una interfaccia che separa il client dal server. Questa separazione 
	\end{itemize}
		
		rest - A uniform interface separates clients from servers. This separation of concerns means that, for example, clients are not concerned with data storage, which remains internal to each server, so that the portability of client code is improved. Servers are not concerned with the user interface or user state, so that servers can be simpler and more scalable. Servers and clients may also be replaced and developed independently, as long as the interface between them is not altered.

\componente{CLIPS}
\compDescrizione{package generale contenente il prodotto del progetto}
\begin{compPackageContenuti}
	\item CLIPS::client: componente globale per il front end del prodotto che utilizza il design pattern \gl{MVC}. Si occupa di fornire un'interfaccia grafica dell'applicazione e di interagire con il lato server.
	\item CLIPS::server: componente globale per il back end del prodotto
	\end{compPackageContenuti}
	\end{componente}
	\componente{CLIPS::client}
	\compDescrizione{componente globale per il front end del prodotto che utilizza il design pattern \gl{MVC}. Si occupa di fornire un'interfaccia grafica dell'applicazione e di interagire con il lato server.}
	\compPadre{CLIPS}
	\begin{compPackageContenuti}
		\item CLIPS::client::authentication: componente che si occupa di gestire l'autenticazione dell'utente
		\end{compPackageContenuti}
		\end{componente}
		\componente{CLIPS::client::authentication}
		\compDescrizione{componente che si occupa di gestire l'autenticazione dell'utente}
		\compPadre{client}
		\begin{compClassi}
			\begin{classe}{CLIPS::client::authentication::ForgotPasswordView}
				\classeDescrizione{classe che si occupa della visualizzazione della schermata per la richiesta di una nuova password}
				\classeUtilizzo{consente all'utente di inserire la mail per ricevere una nuova password}
				\end{classe}\begin{classe}{CLIPS::client::authentication::LoggedUser}
				\classeDescrizione{classe che si occupa di memorizzare in locale i dati dell'utente loggato}
				\classeUtilizzo{permette il salvataggio in locale dei dati di un utente loggato}
				\end{classe}\begin{classe}{CLIPS::client::authentication::LoginView}
				\classeDescrizione{classe che si occupa della visualizzazione della schermata per il login}
				\classeUtilizzo{consente all'utente di inserire i propri dati per effettuare il login}
				\end{classe}\begin{classe}{CLIPS::client::authentication::RegistrationView}
				\classeDescrizione{classe che si occupa della visualizzazione della schermata per la registrazione}
				\classeUtilizzo{consente all'utente di inserire i propri dati per effettuare la registrazione}
				\end{classe}\begin{classe}{CLIPS::client::authentication::UpdateUserInfoView}
				\classeDescrizione{classe che si occupa della visualizzazione della schermata per il cambio delle credenziali}
				\classeUtilizzo{consente all'utente di inserire i nuovi dati per cambiare le sue credenziali}
				\end{classe}\end{compClassi}
				\end{componente}
				\componente{CLIPS::server}
				\compDescrizione{componente globale per il back end del prodotto}
				\compPadre{CLIPS}
				\end{componente}
				\componente{gamelogic}
				\compDescrizione{componente che si occupa di tutte le componenti riguardanti il gioco.}
				\begin{compPackageContenuti}
					\item gamelogic::buildings: componete che gestisce le informazioni e le interazioni dell'utente con gli edifici abilitati
					\item gamelogic::paths: componente che gestisce i dati dei percorsi giocabili dall'utente
					\end{compPackageContenuti}
					\end{componente}
					\componente{gamelogic::buildings}
					\compDescrizione{componete che gestisce le informazioni e le interazioni dell'utente con gli edifici abilitati}
					\compPadre{gamelogic}
					\end{componente}
					\componente{gamelogic::paths}
					\compDescrizione{componente che gestisce i dati dei percorsi giocabili dall'utente}
					\compPadre{gamelogic}
					\begin{compClassi}
						\begin{classe}{gamelogic::paths::Path}
							\classeDescrizione{classe che si occupa di salvare in locale i dati riguardanti un percorso}
							\classeUtilizzo{permette di salvare i dati di un percorso in locale}
							\end{classe}\begin{classe}{gamelogic::paths::PathInfo}
							\classeDescrizione{classe che si occupa di salvare in locale le informazioni generali di un percorso}
							\classeUtilizzo{consente di salvare in locale le informazioni generali di un percorso}
							\end{classe}\begin{classe}{gamelogic::paths::PathView}
							\classeDescrizione{classe che si occupa della visualizzazione della schermata riguardante un percorso}
							\classeUtilizzo{consente all'utente di visualizzare le informazioni riguardanti un percorso e se l'utente si trova nell'edificio del percorso consente di iniziarlo}
							\end{classe}\end{compClassi}
							\end{componente}
							\componente{games}
							\compDescrizione{componente che gestisce le prove che l'utente deve completare all'interno di un percorso}
							\begin{compClassi}
								\begin{classe}{games::MultipleChoiceQuiz}
									\classeDescrizione{classe per il quiz a risposta multipla}
									\classeUtilizzo{si occupa di fornire un'interfaccia per il quiz a risposta multipla}
									\begin{classeAttributi}
										\classeAttributo{answerButtons}{void}{una lista di buttons per visualizzare le possibili risposte}
										\end{classeAttributi}
										\begin{classeMetodi}
											\classeMetodo{buttonPressed}{atIndex}{void}{segnala al controller il button premuto}
											\begin{classeMetodoArgomenti}
												\classeMetodoArgomento{atIndex}{int}{indica l'indice della risposta selezionata}
												\end{classeMetodoArgomenti}
												\end{classeMetodi}
												\end{classe}\begin{classe}{games::QuizResultView}
												\classeDescrizione{classe per la visualizzazione del risultato di un quiz}
												\classeUtilizzo{fornisce all'utente un'interfaccia affinché visualizzi il risultato del quiz}
												\begin{classeAttributi}
													\classeAttributo{continueButton}{void}{button per chiudere la schermata e continuare il percorso}
													\classeAttributo{feedbackLabel}{string}{mostra la frase di successo/fallimento del quiz}
													\end{classeAttributi}
													\begin{classeMetodi}
														\classeMetodo{continueButtonPressed}{}{void}{notifica il controller che il button per continuare è stato premuto}
														\classeMetodo{showFailureResult}{correctAnswer}{void}{mostra la risposta corretta se il quiz è stato fallito}
														\begin{classeMetodoArgomenti}
															\classeMetodoArgomento{correctAnswer}{string}{indica la risposta corretta}
															\end{classeMetodoArgomenti}
															\classeMetodo{showSuccessfulResult}{score}{void}{mostra il risultato ottenuto se il quiz è stato superato}
															\begin{classeMetodoArgomenti}
																\classeMetodoArgomento{score}{int}{indica il punteggio ottenuto}
																\end{classeMetodoArgomenti}
																\end{classeMetodi}
																\end{classe}\begin{classe}{games::QuizView}
																\classeDescrizione{classe base per i quiz}
																\classeUtilizzo{fornisce una base per i vari tipi di test da istanziare}
																\begin{classeAttributi}
																	\classeAttributo{questionLabel}{string}{rappresenta la domanda da porre nel quiz}
																	\end{classeAttributi}
																	\end{classe}\begin{classe}{games::StrangersQuiz}
																	\classeDescrizione{classe per il quiz strana coppia}
																	\classeUtilizzo{si occupa di fornire un'interfaccia per la prova strana coppia}
																	\begin{classeAttributi}
																		\classeAttributo{answerCheckBox}{void}{fornisce una lista di risposte da selezionare tramite checkbox}
																		\classeAttributo{confirmButton}{void}{button grafico per confermare la selezione delle risposte}
																		\end{classeAttributi}
																		\begin{classeMetodi}
																			\classeMetodo{confirmButtonPressed}{}{void}{notifica il controller che è stato premuto il tasto conferma e quindi si può procedere alla valutazione delle risposte}
																			\end{classeMetodi}
																			\end{classe}\begin{classe}{games::TestResultView}
																			\classeDescrizione{classe che fornisce una base per la visualizzazione del risultato della prova}
																			\classeUtilizzo{permette all'utente di visualizzare il risultato della prova}
																			\begin{classeMetodi}
																				\classeMetodo{showResult()}{}{void}{restituisce la view con il risultato}
																				\end{classeMetodi}
																				\end{classe}\begin{classe}{games::TestView}
																				\classeDescrizione{questa classe fornisce una base dalla quale è possibile creare vari tipi di giochi }
																				\classeUtilizzo{viene utilizzata per visualizzare un'interfaccia di gioco all'utente}
																				\begin{classeMetodi}
																					\classeMetodo{showTest}{}{TestView}{restituisce l'interfaccia grafica del test}
																					\end{classeMetodi}
																					\end{classe}\begin{classe}{games::TrueFalseQuiz}
																					\classeDescrizione{classe per il quiz vero/falso}
																					\classeUtilizzo{si occupa di fornire un'interfaccia per la prova di tipo vero/falso}
																					\begin{classeAttributi}
																						\classeAttributo{falseButton}{void}{button grafico per rispondere falso al quiz}
																						\classeAttributo{trueButton}{void}{button grafico per rispondere vero al quiz}
																						\end{classeAttributi}
																						\begin{classeMetodi}
																							\classeMetodo{falseButtonPressed}{}{void}{questo metodo si occupa di notificare al controller che è stato premuto falseButton}
																							\classeMetodo{trueButtonPressed}{}{void}{questo metodo si occupa di notificare al controller che è stato premuto trueButton}
																							\end{classeMetodi}
																							\end{classe}\end{compClassi}
																							\end{componente}
																							


% package
% \include{package}

% classi
% \include{classi}

\section{Diagrammi delle attività} 
I diagrammi delle attività realizzati mostrano in dettaglio i flussi di azioni che un utente può svolgere all'interno dell'applicazione. \\
Alcuni diagrammi contengono azioni che sono a loro volta state descritte dettagliatamente in altri diagrammi di attività. Per migliorare la comprensibilità queste azioni sono state differenziate dalle altre inserendovi uno sfondo colorato.
\subsection{Avvio}
\subsection{Registrazione}
\subsection{Login}
\subsection{Modifica credenzali}
\subsection{Percorsi}
\subsection{Ricerca edifici}

\include{stimeFattibilita}

% tracciamento
% \include{tracciamento}

\appendix
\newcommand{\utilizzo}{\item \textbf{Utilizzo nel progetto}}

\section{Design pattern utilizzati}
\label{pattern}
\subsection{Pattern architetturali}
	\subsubsection{Model View Presenter}
	
		\begin{figure}[!h]
			\centering
			\includegraphics[scale=0.5]{img/mvp}  
			\caption{Struttura del pattern MVP}
		\end{figure}
		
		\begin{itemize}	
			\item \textbf{Descrizione} \\ Model View Presenter è un design pattern architetturale simile al Model View Controller nel quale il presenter è posizionato tra model e view. Permette di dividere l'architettura del sistema che si intende sviluppare in tre blocchi:
			\begin{itemize}
				\item \textbf{il model} che contiene le classi con i metodi di accesso ai dati;
				\item \textbf{la view}, completamente pasiva, contiene le classi che permettono all'utente di visualizzare i dati e segnala tramite gli eventi le interazioni dell'utente;
				\item \textbf{il presenter} che si occupa di fare da tramite tra vista e modello, ovvero riceve i comandi dell'utente attraverso la vista e va a cambiare lo stato del modello di conseguenza, successivamente aggiorna la vista.				
			\end{itemize}
			
			\item \textbf{Vantaggi} \\
			Questo pattern architetturale permette il riuso del codice in quanto molte parti di lavoro sono indipendenti. Ad esempio il modello creato potrà essere utilizzato con diverse viste. \\ Grazie all'indipendenza di alcune parti è semplice dividere il lavoro tra più componenti di un team e sarà quindi più facile anche la manutenzione.
			\utilizzo \\
			Nel nostro progetto il pattern MVP è usato nella parte client dell'architettura, nello specifico il model è rappresentato dalle classi del database locale, ciè quelle del package \textbf{Data}, le view sono rappresentate dai \textbf{file XML} di Android che vengono utilizzati per la gestione del layout mentre il presenter è costituito dalle classi \textbf{activity} e dalle classi all'interno del package \textbf{Datamanager}.
		\end{itemize}
		
		\newpage
		
\subsection{Pattern creazionali}
	\subsubsection{Abstract Factory}
	
	\begin{figure}[!h]
		\centering
		\includegraphics[scale=0.2]{img/abstract_factory}  
		\caption{Struttura del pattern Abstract Factory}
	\end{figure}
	
		\begin{itemize}
			\item \textbf{Descrizione}\\ 
			Il design pattern Abstract Factory fornisce un'interfaccia per creare famiglie di prodotti senza specificare classi concrete. Ogni famiglia di prodotti ha una classe base astratta da cui derivano delle classi concrete. Queste classi concrete sono istanziate dalle classi concrete della factory corrispondenti.
			% immagine di esempio
			
			\item \textbf{Vantaggi}\\ 
			Questo design pattern offre vantaggi quando si vogliono modellare famiglie di prodotti che potranno essere ampliate nel futuro. Le modifiche necessarie ad aggiungere nuovi elementi alle famiglie saranno essenzialmente due, aggiungere una classe in ogni famiglia di prodotti ed un solo metodo nella factory.
			\utilizzo \\ 
			Nel nostro progetto abbiamo utilizzato questo design pattern per modellare i tipi di quiz che vengono proposti agli utenti. Abbiamo infatti deciso di utilizzare solo due famiglie di quiz, quindi vi è la necessità di avere modo in futuro di ampliare le tipologie di quiz disponibili in modo semplice ed efficace.
			
			\begin{figure}[!h]
				\centering
				\includegraphics[scale=0.4]{img/our_abstract_factory}  
				\caption{Utilizzo di Abstract Factory nel progetto}
			\end{figure}
			
		\end{itemize}
	 \newpage
	 
	\subsubsection{Singleton}
		\begin{figure}[!h]
			\centering
			\includegraphics[scale=0.4]{img/singleton}  
			\caption{Struttura del pattern Singleton}
		\end{figure}
		
		\begin{itemize}
			\item \textbf{Descrizione}\\ 
			 Assicura l'esistenza di un'unica istanza di una classe e permette di avere un punto di accesso globale a questa.
			 Per rendere possibile ciò si mette il costruttore protetto o privato e si crea un metodo statico, chiamato factory, che fornisce l’accesso all'unica copia dell’oggetto (contiene un puntatore all’unica istanza). \\
			 È stata valutata come alternativa la \gl{Dependency Injection}, un altro \gl{design pattern}. Essa prevede la costruzione di una classe le cui dipendenze vengono ricevute dall'esterno. Di base non ha le stesse proprietà del \gl{Singleton}, ma può acquisirle tramite degli appositi accorgimenti, come l'utilizzo di una \gl{Factory}\ per la costruzione delle dipendenze. Questo \gl{design pattern}\ ha come vantaggi una maggiore facilità nel testare la classe, perché si possono creare delle dipendenze false da inviare alla \gl{Dependency Injection}, e la separazione tra il comportamento della componente dalla risoluzione delle sue dipendenze; al contrario il \gl{Singleton}\ rende privato tutto quello che le occorre per costruire la classe, mantenendolo quindi al proprio interno. Lo svantaggio invece risiede nella maggiore complessità di costruzione, in quanto la \gl{Dependency Injection}\ richiede l'utilizzo di almeno due classi con una forte dipendenza tra loro, anziché un'unica classe come il \gl{Singleton}. Quest'ultimo è stato scelto alla fine perché nel nostro caso le dipendenze sono poche e il risultato prodotto è facilmente verificabile tramite degli appositi metodi, perciò lo svantaggio della \gl{Dependency Injection}\ peserebbe molto di più rispetto a vantaggi.
			 \item \textbf{Vantaggi}\\ 
			 Se, al contrario di quanto avviene utilizzando Singleton, venisse reso visibile il costruttore della classe non si potrebbe garantire che esista un solo esemplare della classe. Un altro modo di procedere potrebbe essere quello di dichiarare una variabile globale, ma in questo modo si ``ruberebbe'' un nome al namespace globale.
			 Sigleton permette inoltre di dichiarare sottoclassi.			 
			 \utilizzo \\ 
			 Nel nostro progetto abbiamo creato un singleton per ``LoginManager'' che contiene i dati relativi a i dati di login dell'utente che sta utilizzando l'applicazione.
			 
			 	\begin{figure}[!h]
			 		\centering
			 		\includegraphics[scale=0.4]{img/package/png/client--loginmanager}  
			 		\caption{Utilizzo di Abstract Factory nel progetto}
			 	\end{figure}
			 
		\end{itemize}
		
		
		


\end{document}
