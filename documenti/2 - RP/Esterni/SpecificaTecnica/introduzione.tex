\section{Introduzione}
\subsection{Scopo del documento} 
Questo documento ha lo scopo di spiegare dettagliatamente le strategie secondo cui il gruppo \AUTORE{} intende condurre il \gl{progetto} didattico. 
\subsection{Scopo del \gl{prodotto}}
\SCOPO
\subsection{Glossario}
\GLOSSARIO
\subsection{Riferimenti}
\subsubsection{Normativi}
\begin{itemize}
	\item \textbf{Capitolato d'appalto C2 - CLIPS:} Communication \& Localisation with Indoor Positioning Systems. \\
	\url{http://www.math.unipd.it/~tullio/IS-1/2015/Progetto/C2.pdf}
	\item \textbf{Norme di Progetto} \\ \NPdoc
	\item \textbf{Analisi dei Requisiti} \\ \ARdoc
\end{itemize}	

\subsubsection{Informativi}
\begin{itemize}
	\item \textbf{Software Engineering (10th edition}) \\
	Ian Sommerville \\
	Pearson Education | Addison-Wesley
	\item \textbf{Guide to the Software Engineering Body of Knowledge}
	IEEE Computer Society. Software Engineering Coordinating Committee
	\item \textbf{Slides del \COMMITTENTE} \\ riguardo la  \href{http://www.math.unipd.it/~tullio/IS-1/2015/Dispense/L07.pdf}{progettazione}
	\item \textbf{Slides del \CARDIN} \\ riguardo UML \\ 	
	\href{http://www.math.unipd.it/~tullio/IS-1/2015/Dispense/E03.pdf}{diagrammi delle classi} \\
	\href{http://www.math.unipd.it/~tullio/IS-1/2015/Dispense/E04.pdf}{diagrammi dei package} \\
	\href{http://www.math.unipd.it/~tullio/IS-1/2015/Dispense/E06.pdf}{diagrammi di attività} \\
	\href{http://www.math.unipd.it/~tullio/IS-1/2015/Dispense/E05.pdf}{diagrammi di sequenza} \\
	riguardo i design pattern \\
	\href{http://www.math.unipd.it/~tullio/IS-1/2015/Dispense/E07.pdf}{design pattern strutturali} \\
	\href{http://www.math.unipd.it/~tullio/IS-1/2015/Dispense/E08.pdf}{design pattern creazionali} \\ 
	\href{http://www.math.unipd.it/~tullio/IS-1/2015/Dispense/E09.pdf}{design pattern comportamentali}
	
	% aggiungere successivamente tutti i link alle documentazioni delle tecnologie e blablabla
	
\end{itemize}
