\section{Esito delle revisioni}
\label{sec:B}
	Durante lo sviluppo del progetto ci saranno quattro revisioni a cui sottoporsi. Il \gl{committente} segnalerà gli errori riscontrati fornendo una valutazione generica dell'andamento del progetto ed una dettagliata per ogni documento. Si elencano di seguito le modifiche apportate in seguito alle revisioni.
	\subsection{Revisione dei Requisiti}
	\label{sec:B.1}
		\begin{itemize}
			\item \textbf{Studio di fattibilità:}\ sono stati corretti gli acronimi scritti in maniera errata.
			\item \textbf{Norme di progetto:}\ nel documento sono state aggiunte le sezioni che erano state impropriamente inserite nel documento \PQdocRR.
			\item \textbf{Analisi dei Requisiti:}\ sono stati modificati numerosi casi d'uso, cambiando ad esempio la descrizione, lo scenario principale e il padre. Inoltre sono stati aggiunti parecchi casi d'uso, mentre altri sono stati eliminati. Sono stati modificati anche numerosi requisiti sia come conseguenza delle modifiche dei casi d'uso sia per le segnalazioni ricevute. Sono stati aggiunti tanti requisiti e ne sono stati eliminati alcuni.
			\item \textbf{Piano di Progetto:}\ sono state aggiunte le sezioni sul'Analisi dinamica dei rischi e sono stati cambiati i nomi dei periodi.
			\item \textbf{Piano di Qualifica:}\ il documento è stato completamente rivisto a seguito delle segnalazioni. Sono state aggiunte altre metriche, soprattutto per la verifica dei processi, ed è stata ampliata l'appendice con i risultati della verifica.
			%se anche altri documenti vengono modificati, tipo il Glossario, bisogna aggiungere qua cosa è stato modificato
		\end{itemize}