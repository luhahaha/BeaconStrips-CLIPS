\section{Pianificazione dei test}
	\subsection{Descrizione dei test}
		Vengono ora indicati i test di validazione, di sistema e di integrazione previsti. I test di unità saranno inseriti in un momento successivo. \\
		Poiché i test saranno applicati in uno stadio di lavoro successivo a quello attuale, lo stato dei singoli è indicato come \textbf{N.I.}: non implementati. \\
		Di ogni test verranno indicati la tipologia ed altri parametri come specificato dalla seguente sintassi:
		\begin{itemize}
			\item per i test di unità: \textbf{TU[Codice Test]};
			\item per i test di integrazione: \textbf{TU[Identificativo del componente]};
			\item per i test di sistema: \textbf{TS[Tipo Requisito][Codice Requisito]};
			\item per i test di validazione: \textbf{[Tipo Requisito][Codice Requisito]}.
		\end{itemize}
		In particolare:
		\begin{itemize}
			\item \textbf{Codice Requisito}: è il codice gerarchico univoco di ogni vincolo espresso in numero (esempio: 1.3.2);
			\item \textbf{Identificativo del componente}: corrisponde al componente i cui elementi sono integrati;
			\item \textbf{Tipo Requisito}: può assumere solo uno fra i seguenti valori:
			\begin{itemize}
				\item F: funzionale;
				\item Q: di qualità;
				\item P: prestazionale;
				\item V: di vincolo.
			\end{itemize}
		\end{itemize}
	\subsection{Test di validazione}
		I test di validazione servono per accertarsi che il prodotto realizzato sia conforme alle attese del \PROPONENTE. \\
		Per ognuno vengono indicati i passi necessari all'utente per testare i requisiti associati. Il tracciamento tra i test di validazione e i requisiti correlati viene riportato nel documento \ARdoc.
	\subsection{Test di sistema}
		I test di sistema servono per accertarsi che il comportamento dinamico del sistema rispetti i requisiti software individuati e descritti nel documento \ARdoc.
	\subsection{Test di integrazione}
		I test di integrazione servono per verificare che tutti i diversi componenti del sistema comunichino correttamente tra loro, e che vi sia all'interno del software il flusso di dati atteso. \\
		Verrà utilizzata una strategia di integrazione incrementale per poter sviluppare e verificare più componenti in parallelo. Questo metodo permette di dare priorità ai test relativi alle componenti che vengono ritenute più importanti.