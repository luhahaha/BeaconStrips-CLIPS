% !TEX encoding = UTF-8 Unicode
% !TEX TS-program = pdflatex
% !TEX spellcheck = it-IT

\section{Gestione amministrativa della revisione}
	\label{sec:4}
	\subsection{Comunicazione e risoluzione delle anomalie}
		\label{sec:4.1}
		Un'anomalia corrisponde a:
		\begin{itemize}
			\item\ un errore ortografico;
			\item\ la violazione delle norme tipografiche del documento;
			\item\ l'uscita dal \gl{range} di accettazione degli indici di misurazione, descritti nella \hyperref[sec:3.7]{sottosezione 3.7};
			\item\ un'incongruenza del \gl{prodotto} rispetto a determinate funzionalità. Tali funzionalità sono state indicate nel documento \ARdoc;
			\item\ un'incongruenza del codice con il design del \gl{prodotto}.
		\end{itemize}
		Nel caso in cui un \VER\ individui un'anomalia, dovrà aprire un \gl{ticket} seguendo la procedura indicata nelle \NPdoc.