
\lettera{L} 

\parola{\LaTeX}{\gl{Linguaggio di markup} usato per la preparazione di testi. Si basa sul principio WYSIWYM (What You See Is What You Mean), contrapposto al WYSIWYG (What You See Is What You Get) tipico dei più comuni programmi di videoscrittura. Permette di generare un file in formato .pdf dai file di \LaTeX{} tramite un apposito compilatore. Maggiori informazioni al sito \url{http://www.latex-project.org}.}

\parola{Libreria}{Un insieme di funzioni o strutture dati costruite per essere collegate facilmente ad un programma software, attraverso un opportuno collegamento statico o dinamico.}

\parola{Linguaggio di markup}{In generale un linguaggio di markup, o linguaggio a marcatori, è un insieme di regole che descrivono i meccanismi di rappresentazione di un testo che, utilizzando convenzioni standardizzate, sono utilizzabili su più supporti. La tecnica di composizione di un testo con l'uso di marcatori (o espressioni codificate) richiede quindi una serie di convenzioni, ovvero di un linguaggio a marcatori di documenti (riferimento: \url{https://it.wikipedia.org/wiki/Linguaggio_di_markup}).}

\parola{Linguaggio di scripting}{Un linguaggio di scripting, in informatica, è un linguaggio di programmazione interpretato destinato in genere a compiti di automazione del sistema operativo (batch) o delle applicazioni (macro), o a essere usato all'interno delle pagine web (riferimento: \url{https://it.wikipedia.org/wiki/Linguaggio_di_scripting}).}

\parola{Linguaggio funzionale}{Tipo di linguaggio che utilizza la programmazione funzionale, in cui il flusso di esecuzione del programma assume la forma di una serie di valutazioni di funzioni matematiche.}

\parola{Linguaggio orientato agli oggetti}{Tipo di linguaggio che utilizza la programmazione orientata agli oggetti, la quale permette di definire dei tipi di dato chiamati oggetti, in grado di interagire gli uni con gli altri attraverso lo scambio di messaggi.}

\parola{Loopback}{Loopback è un \gl{framework} \gl{open-source} di \gl{Node.js} }