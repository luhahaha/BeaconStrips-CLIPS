
\lettera{J} 

\parola{Java}{In informatica Java è un linguaggio di programmazione orientato agli oggetti a tipizzazione statica, specificatamente progettato per essere il più possibile indipendente dalla piattaforma di esecuzione (riferimento: \url{https://it.wikipedia.org/wiki/Java_(linguaggio_di_programmazione)}).}

\parola{JavaScript}{\gl{Linguaggio di scripting} orientato agli oggetti comunemente usato nei siti web. Fu originariamente sviluppato da Brendan Eich della Netscape Communications con il nome di Mocha e successivamente di LiveScript. In seguito è stato rinominato JavaScript ed è stato formalizzato con una sintassi più vicina a quella del linguaggio \gl{Java} di Oracle.}

\parola{JSON}{JSON, acronimo di JavaScript Object Notation, è un formato adatto all'interscambio di dati fra applicazioni client-server (riferimento: \url{https://it.wikipedia.org/wiki/JavaScript_Object_Notation}).}

\parola{JVM}{La macchina virtuale \gl{Java} è il componente della piattaforma Java che esegue i programmi tradotti in bytecode dopo una prima compilazione (riferimento: \url{https://it.wikipedia.org/wiki/Macchina_virtuale_Java}).}