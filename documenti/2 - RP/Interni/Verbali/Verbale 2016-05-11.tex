% !TEX encoding = UTF-8 Unicode
% !TEX TS-program = pdflatex
% !TEX spellcheck = it-IT
\documentclass[a4paper,titlepage]{article}

\usepackage[utf8x]{inputenc}

\makeatletter
\def\input@path{{../../../template/}}
\makeatother

\usepackage{Comandi}
\usepackage{Riferimenti}
\usepackage{Stile}

\def\NOME{Verbale del Giorno 2016-05-11}
\def\VERSIONE{1.0.0}
\def\DATA{2016-05-17}
\def\REDATTORE{Andrea Grendene}
\def\VERIFICATORE{Luca Soldera}
\def\RESPONSABILE{Matteo Franco}
\def\USO{Interno}
\def\DESTINATARI{\COMMITTENTE \\ & \CARDIN \\ & \PROPONENTE}
\def\SOMMARIO{In questa riunione si è deciso quale sistema operativo utilizzare per l'applicazione e come organizzarsi per cominciare la progettazione dell'architettura software.}


\begin{document}

\maketitle

\begin{diario}
	\modifica{Andrea Grendene}{\PRJ}{Attuazione delle modifiche segnalate durante la verifica}{2016-05-17}{0.1.1}
	\modifica{Luca Soldera}{\VER}{Verifica del verbale}{2016-05-15}{0.1.0}
	\modifica{Andrea Grendene}{\PRJ}{Stese le Sezioni 1, 2 e 3}{2016-05-12}{0.0.1}
\end{diario}

\newpage
\tableofcontents

\newpage
\section{Informazioni}
\label{sec:Informazioni}

\begin{itemize}
 \item \textbf{Luogo}: LuF1, Plesso Paolotti - Via G. B. Belzoni 7, 35121, Padova (PD);
 \item \textbf{Data}: 2016-05-11;
 \item \textbf{Ora}: 9:30;
 \item \textbf{Durata}: 2 ore;
 \item \textbf{Partecipanti}: Viviana Alessio, Luca Soldera, Matteo Franco, Andrea Grendene, Enrico Bellio.
\end{itemize}

\section{Ordine del giorno}
\label{sec:Ordine del giorno}
Di seguito sono riportati i punti affrontati durante la riunione.

\begin{enumerate}
	\item Decidere per quale sistema operativo tra Android e iOS creare l'applicazione.
	\item Decidere come organizzarsi per la progettazione, ovvero come dividere i progettisti, cosa assegnare ad ogni eventuale gruppo e come procedere per fare la progettazione.
	\item Capire quali linguaggi serviranno per l'applicazione.
	\item Decidere quali stili architetturali utilizzare per l'applicazione.
\end{enumerate}

\section{Decisioni}
Vengono riportate ora le decisioni prese durante la riunione. \\
Per rendere più facile il tracciamento e il riferimento dentro e fuori questo documento ad ogni decisione viene assegnato un codice identificativo secondo la seguente codifica:
\begin{center}
D[Codice Identificativo]
\end{center}
Vengono riportate anche le fonti che hanno portato a queste decisioni, sia interne che esterne. Per fonti interne si intendono le domande a cui è stata trovata risposta durante la riunione.

\begin{tabella}{!{\VRule}c!{\VRule}X[l,b,l]!{\VRule}c!{\VRule}}
	\intestazionethreecol{Decisione}{Descrizione}{Fonti}
		D1 & Il sistema operativo scelto è Android. Le motivazioni di questa decisione riguardano sia questo progetto sia l'utilità della conoscenza di questa tecnologia nel futuro; più precisamente sono:
		\begin{itemize}
		\item\ la grande richiesta di programmatori Android;
		\item\ la grossa disponibilità di dispositivi con questo sistema operativo che possono essere facilmente usati per la fase di testing;
		\item\ il numero elevato di persone con una certa esperienza di Android che i membri del gruppo conoscono;
		\item\ l'utilità per i membri del gruppo anche per affrontare alcuni esami universitari, dato che questo linguaggio di programmazione deriva da Java.
		\end{itemize}
		& Punto 1 \\
		D2 & I 4 progettisti si divideranno in due gruppi da 2 persone, il primo sarà composto da Andrea Grendene e Matteo Franco mentre il secondo sarà formato da Viviana Alessio e Luca Soldera & Punto 2 \\
		D3 & Il primo gruppo affronterà i requisiti da 1 a 3, inclusi tutti i relativi sottocasi, mentre il secondo gruppo si occuperà dei requisiti rimanenti & Punto 2 \\
		D4 & La progettazione sarà top-down, ovvero partirà dagli stili architetturali dell'applicazione, affronterà poi la suddivisione dei package e infine arriverà a progettare le classi e i collegamenti tra loro & Punto 2 \\
		D5 & I linguaggi che sicuramente useremo sono Java e XML. Non saranno gli unici che verranno utilizzati ma la loro individuazione completa verrà delegata ad una riunione successiva & Punto 3 \\
		D6 & I possibili stili architetturali che potrebbero essere adottati sono l'MVC o il Three Tier. Non è ancora stata presa una decisione precisa perché è necessario un ulteriore confronto all'interno del gruppo, visto che entrambi i metodi possono essere validi & Punto 4 \\
	\hiderowcolors
	\caption{Tabella delle decisioni prese}
\end{tabella}

\end{document}