% !TEX encoding = UTF-8 Unicode
% !TEX TS-program = pdflatex
% !TEX spellcheck = it-IT
\documentclass[a4paper,titlepage]{article}

\usepackage[utf8x]{inputenc}

\makeatletter
\def\input@path{{../../../template/}}
\makeatother

\usepackage{Comandi}
\usepackage{Riferimenti}
\usepackage{Stile}

\def\NOME{Verbale del Giorno 2016-05-05}
\def\VERSIONE{1.0.0}
\def\DATA{\today}
\def\REDATTORE{Andrea Grendene}
\def\VERIFICATORE{Luca Soldera}
\def\RESPONSABILE{Luca Soldera}
\def\USO{Interno}
\def\DESTINATARI{\COMMITTENTE \\ & \CARDIN \\ & \PROPONENTE}
\def\SOMMARIO{Breve descrizione della riunione interna del 2016-05-05.}


\begin{document}

\maketitle

\begin{diario}
	\modifica{Andrea Grendene}{\PRJ}{Stese le Sezioni 1, 2 e 3}{2016-05-05}{0.0.1}
\end{diario}

\newpage
\tableofcontents

\newpage
\section{Informazioni}
\label{sec:Informazioni}

\begin{itemize}
 \item \textbf{Luogo}: LabTA, Torre Archimede - Via Trieste 63, 35121, Padova (PD);
 \item \textbf{Data}: 2016-05-05;
 \item \textbf{Ora}: 10:30;
 \item \textbf{Durata}: 90 minuti;
 \item \textbf{Partecipanti}: Viviana Alessio, Luca Soldera, Matteo Franco, Andrea Grendene, Enrico Bellio.
\end{itemize}

\section{Ordine del giorno}
\label{sec:Ordine del giorno}
Di seguito sono riportati i punti affrontati durante la riunione.

\begin{enumerate}
	\item Assegnare ad un membro del gruppo la stesura dei verbali delle riunioni interne ed esterne.
	\item Assegnare ad un membro del gruppo la verifica dei verbali scritti.
	\item Assegnare ad un membro del gruppo la modifica e l'aggiunta dei diagrammi dei casi d'uso necessarie a seguito dell'esito della Revisione dei Requisiti e delle ultime modifiche effettuate ai casi d'uso stessi.
	\item Assegnare ad un membro del gruppo la verifica dei diagrammi dei casi d'uso modificati o aggiunti.
	\item Assegnare ad un membro del gruppo la verifica del \PQdoc.
	\item Assegnare ad un membro del gruppo la verifica del \PPdoc.
	\item Decidere come procedere per la modifica dell'UC5 e dei relativi sottocasi d'uso.
\end{enumerate}

\section{Decisioni}
A seguito di questo incontro vengono riportate le decisioni prese riguardo agli argomenti discussi. \\
Per facilitare il loro tracciamento all'esterno di questo documento ad ogni decisione viene assegnato un codice identificativo secondo la seguente codifica:
\begin{center}
D[Codice Identificativo]
\end{center}
Vengono inoltre riportate le fonti, esterne e interne, che hanno portato a prendere una determinata decisione. Per fonti interne si intendono le domande presentate durante questo incontro.

\begin{tabella}{!{\VRule}c!{\VRule}X[l,b,l]!{\VRule}c!{\VRule}}
	\intestazionethreecol{Decisione}{Descrizione}{Fonti}
		D1 & La stesura dei verbali interni ed esterni è stata assegnata a Andrea Grendene. Per la stesura dei verbali esterni è stato deciso di consultare dei verbali stesi da un altro gruppo il cui esito è stato ottimo & Punto 1 \\
		D2 & La verifica dei verbali interni ed esterni è stata assegnata a Luca Soldera & Punto 2 \\
		D3 & L'aggiunta e la modifica dei diagrammi dei casi d'uso sono state assegnate a Tommaso Panozzo & Punto 3 \\
		D4 & La verifica dei diagrammi dei casi d'uso è stata assegnata a Matteo Franco & Punto 4 \\
		D5 & La verifica del \PQdoc\ è stata assegnata a Enrico Bellio & Punto 5 \\
		D6 & La verifica del \PPdoc\ è stata assegnata a Enrico Bellio & Punto 6 \\
		D7 & È stata migliorata la descrizione relativa al caso d'uso dell'attivazione del GPS & Punto 7 \\
		D8 & È stato aggiunto il caso d'uso generale ‘‘Contatta edificio’’ & Punto 7 \\
		D9 & R1Q4 diventa un requisito funzionale (principale o no? Distribuito o unico? Ai posteri/verificatori l'ardua sentenza) & Punto 7 \\
		D10 & Sono state corrette tutte le precondizioni errate & Punto 7 \\
	\hiderowcolors
	\caption{Tabella delle decisioni prese}
\end{tabella}

\end{document}