% !TEX encoding = UTF-8 Unicode
% !TEX TS-program = pdflatex
% !TEX spellcheck = it-IT
\documentclass[a4paper,titlepage]{article}

\usepackage[utf8x]{inputenc}

\makeatletter
\def\input@path{{../../../template/}}
\makeatother

\usepackage{Comandi}
\usepackage{Riferimenti}
\usepackage{Stile}

\def\NOME{Verbale del Giorno 2016-05-05}
\def\VERSIONE{1.0.0}
\def\DATA{\today}
\def\REDATTORE{Andrea Grendene}
\def\VERIFICATORE{Luca Soldera}
\def\RESPONSABILE{Luca Soldera}
\def\USO{Interno}
\def\DESTINATARI{\COMMITTENTE \\ & \CARDIN \\ & \PROPONENTE}
\def\SOMMARIO{Breve descrizione della riunione interna del 2016-05-05.}


\begin{document}

\maketitle

\begin{diario}
	\modifica{Andrea Grendene}{\PRJ}{Attuazione delle modifiche segnalate durante la verifica}{2016-05-17}{0.1.1}
	\modifica{Luca Soldera}{\VER}{Verifica del verbale}{2016-05-15}{0.1.0} 
	\modifica{Andrea Grendene}{\PRJ}{Stese le Sezioni 1, 2 e 3}{2016-05-06}{0.0.1}
\end{diario}

\newpage
\tableofcontents

\newpage
\section{Informazioni}
\label{sec:Informazioni}

\begin{itemize}
 \item \textbf{Luogo}: LabTA, Torre Archimede - Via Trieste 63, 35121, Padova (PD);
 \item \textbf{Data}: 2016-05-05;
 \item \textbf{Ora}: 10:30;
 \item \textbf{Durata}: 90 minuti;
 \item \textbf{Partecipanti}: Viviana Alessio, Luca Soldera, Matteo Franco, Andrea Grendene, Enrico Bellio.
\end{itemize}

\section{Ordine del giorno}
\label{sec:Ordine del giorno}
Di seguito sono riportati i punti affrontati durante la riunione.

\begin{enumerate}
	\item Assegnare ad un membro del gruppo la stesura dei verbali delle riunioni interne ed esterne.
	\item Assegnare ad un membro del gruppo la verifica dei verbali scritti.
	\item Assegnare ad un membro del gruppo la modifica e l'aggiunta dei diagrammi dei casi d'uso necessarie a seguito dell'esito della Revisione dei Requisiti e delle ultime modifiche effettuate ai casi d'uso stessi.
	\item Assegnare ad un membro del gruppo la verifica dei diagrammi dei casi d'uso modificati o aggiunti.
	\item Assegnare ad un membro del gruppo la verifica del \PQdoc.
	\item Assegnare ad un membro del gruppo la verifica del \PPdoc.
	\item Discutere delle riunioni esterne effettuate in precedenza con il \COMMITTENTE\ e con il \CARDIN\ e del fatto che erano presenti pochi membri del gruppo.
	\item Discutere delle problematiche varie che ogni membro del gruppo ha o ha avuto con gli impegni assegnati o con altri membri.
\end{enumerate}

\section{Decisioni}
Vengono riportate ora le decisioni prese durante la riunione. \\
Per rendere più facile il tracciamento e il riferimento dentro e fuori questo documento ad ogni decisione viene assegnato un codice identificativo secondo la seguente codifica:
\begin{center}
D[Codice Identificativo]
\end{center}
Vengono riportate anche le fonti che hanno portato a queste decisioni, sia interne che esterne. Per fonti interne si intendono le domande a cui è stata trovata risposta durante la riunione.

\begin{tabella}{!{\VRule}c!{\VRule}X[l,b,l]!{\VRule}c!{\VRule}}
	\intestazionethreecol{Decisione}{Descrizione}{Fonti}
		D1 & La stesura dei verbali interni ed esterni è stata assegnata a Andrea Grendene. Per la stesura dei verbali esterni è stato deciso di consultare dei verbali stesi da un altro gruppo il cui esito è stato ottimo & Punto 1 \\
		D2 & La verifica dei verbali interni ed esterni è stata assegnata a Luca Soldera & Punto 2 \\
		D3 & L'aggiunta e la modifica dei diagrammi dei casi d'uso sono state assegnate a Tommaso Panozzo & Punto 3 \\
		D4 & La verifica dei diagrammi dei casi d'uso è stata assegnata a Matteo Franco & Punto 4 \\
		D5 & La verifica del \PQdoc\ è stata assegnata a Enrico Bellio & Punto 5 \\
		D6 & La verifica del \PPdoc\ è stata assegnata a Enrico Bellio & Punto 6 \\
		D7 & È stata migliorata la descrizione relativa al caso d'uso dell'attivazione del GPS & Punto 7 \\
		D8 & È stato aggiunto il caso d'uso generale ‘‘Contatta edificio’’ & Punto 7 \\
		D9 & R1Q4 diventa un requisito funzionale (principale o no? Distribuito o unico? Ai posteri/verificatori l'ardua sentenza) & Punto 7 \\
		D10 & Sono state corrette tutte le precondizioni errate & Punto 7 \\
		D11 & È stato deciso che ogni membro del gruppo deve cercare di essere presente alle prossime riunioni, anche se questo comporta saltare delle lezioni o spostare degli impegni & Punto 8 \\
	\hiderowcolors
	\caption{Tabella delle decisioni prese}
\end{tabella}
Per il Punto 9 sono state fatte soprattutto raccomandazioni, che quindi non hanno portato ancora a decisioni precise ma potrebbero richiederne in futuro.

\end{document}