\section{Ambiente di lavoro}
	\subsection{Sistema operativo}
	Non ci sono restrizioni riguardo l'utilizzo di un sistema operativo: ogni membro del team può utilizzare quindi il sistema operativo che preferisce; in ogni caso, è tenuto a rendere il materiale prodotto \textbf{completamente} compatibile su tutte le piattaforme.
	\subsection{Condivisione}
	Tutti i file e documenti non soggetti a controllo di versione verranno condivisi su Slack e \gl{Google Drive}. \\
	È possibile anche condividere link o manuali utili per la formazione del team.
	\subsection{Documentazione}
	\subsubsection{Stesura documentazione}
	Per redigere la documentazione si è scelto di utilizzare il linguaggio di markup \LaTeX in quanto: \\
	\begin{itemize} 
	\item permette di produrre documenti di alta qualità rispetto a \gl{word processor} tradizionali; 
	\item è possibile creare dei template comuni per i documenti;
	\item permette una separazione tra contenuto e formattazione;
	\item è estendibile ed altamente personalizzabile tramite l'utilizzo di pacchetti specifici;
	\item è multipiattaforma, essendo un file sorgente di \LaTeX una semplice codifica \gl{ASCII};
	\end{itemize}
	Come editor di file \LaTeX è consigliato l'uso di \textit{TeXstudio}, anch'esso multipiattaforma.
	\subsection{Verifica documentazione}
		\subsubsection{Controllo ortografico automatico}
		