\section{Ambiente di lavoro}
	\subsection{Sistema operativo}
	Non ci sono restrizioni riguardo l'utilizzo di un sistema operativo: ogni membro del team può utilizzare quindi il sistema operativo che preferisce; in ogni caso, è tenuto a rendere il materiale prodotto \textbf{completamente} compatibile su tutte le piattaforme.
	\subsection{Condivisione}
	Il servizio da utilizzare per i documenti informali è \gl{Google Drive}.
	All'interno di questo repository vanno caricati documenti che:
	\begin{itemize}
		\item non necessitano di controllo di versione;
		\item necessitano di essere modificati contemporaneamente da più persone; infatti Google Drive permette la modifica di un documento in contemporanea: in questo modo non ci saranno conflitti.
	\end{itemize}
	È possibile anche condividere link o manuali utili per la formazione del team. Si può installare Google Drive sul proprio PC: i file così potranno essere disponibili anche offline e sincronizzare il contenuto della cartella ad ogni aggiornamento.
	\subsection{Gestione}
		\begin{itemize}
			\item \textbf{Ticket}: per la gestione dei ticket è stato scelto di utilizzare \gl{Asana}; per la gestione dei ticket, si rimanda alla sezione %vedi sezione ticketing
			\item \textbf{Requisiti e casi d'uso}: per il tracciamento dei requisiti si utilizzerà \gl{Trender}. Per quanto riguarda la rappresentazione dei casi d'uso l'editor consigliato è \gl{Star UML}; %vedi sezione requisiti 
			\item \textbf{Scadenze}: \gl{Google Calenar} verrà utlizzato per segnare le scadenze del gruppo; si riceverà una notifica su Slack trenta minuti prima dell'evento;
			\item \textbf{Progetto}: per la realizzazione dei \gl{grafici di Gantt} è stato impiegato il programma \gl{GanttProject}.
		\end{itemize}
	\subsection{Documentazione}
		\subsubsection{Stesura documentazione}
		Per redigere la documentazione si è scelto di utilizzare il linguaggio di markup \LaTeX in quanto: \\
		\begin{itemize} 
			\item permette di produrre documenti di alta qualità rispetto a \gl{word processor} tradizionali; 
			\item è possibile creare dei template comuni per i documenti;
			\item permette una separazione tra contenuto e formattazione;
			\item è estendibile ed altamente personalizzabile tramite l'utilizzo di pacchetti specifici;
			\item è multipiattaforma, essendo un file sorgente di \LaTeX una semplice codifica \gl{ASCII};
		\end{itemize}
		Come editor di file \LaTeX è consigliato l'uso di \textit{TeXstudio}, anch'esso multipiattaforma.
		\subsubsection{Verifica documentazione}
			In TeXstudio è presente un correttore ortografico automatico che permette la correzione in tempo reale. I file da utilizzare sono presenti nella cartella condivisa in Google Drive all'indirizzo \file{/dizionari}.
	\subsection{Script}
	Su Github è presente una cartella \texttt{/script} dove sono presenti degli script utilizzabili dal gruppo per semplificare alcune operazioni.