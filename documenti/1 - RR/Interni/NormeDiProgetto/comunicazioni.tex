\subsection{Comunicazioni}
	\subsubsection{Comunicazioni interne}
	Per le comunicazioni interne si è deciso di utilizzare Slack. Sono stati creati diversi canali per suddividere il lavoro. \\
	Inoltre, è possibile utilizzare strumenti di messaggistica istantanea, quali \gl{Telegram}, per comunicazioni interne non inerenti al progetto. \\
	In caso di videoconferenza invece verrà utilizzato \gl{Skype}	
	\subsubsection{Comunicazioni esterne}
	Per le comunicazioni esterne è stata creata la casella di posta elettronica: \\
	\highlight{\href{mailto:beaconstrips.swe@gmail.com}{beaconstrips.swe@gmail.com}} \\
	Questo indirizzo deve essere l'unico usato per comunicare con le componenti esterne al team ed è controllato unicamente dal \gl{Responsabile}. Il Responsabile è tenuto poi ad informare i membri del team riguardo le comunicazioni avvenute con l'esterno.

\subsection{Riunioni}
	\subsubsection{Organizzazione}
	Il gruppo si ritroverà ad organizzare una riunione con frequenza almeno quindicinale. \\
	Il Responsabile si occuperà di convocare le riunioni generali, dove tutti i membri del team sono convocati, avvisando i componenti con almeno due giorni di preavviso. \\
	In caso di necessità, un componente del team può richiedere la convocazione di una riunione: tale richiesta deve essere inoltrata al Responsabile, che deciderà se accettarla o respingerla. \\
	Inoltre sono possibili riunioni tra specifici membri: in questo caso sono tenuti ad informare il resto del team tramite un verbale, nel caso siano state prese decisioni rilevanti. \\
	