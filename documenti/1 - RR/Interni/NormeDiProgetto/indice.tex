\documentclass[a4paper,titlepage]{article}

\makeatletter
\def\input@path{{../../../template/}}
\makeatother

\usepackage{Comandi}
\usepackage{Stile}
\usepackage{Riferimenti}



\date{ }
 
 \def\NOME{Norme di Progetto}
 \def\VERSIONE{0.1}
 \def\DATA{\today}
 \def\REDATTORE{Matteo Franco\\ & Luca Soldera}
 \def\VERIFICATORE{Nome Cognome}
 \def\RESPONSABILE{Viviana Alessio}
 \def\USO{Interno}
 \def\DESTINATARI{\COMMITTENTE \\ & \CARDIN \\ & \PROPONENTE}
 \def\SOMMARIO{Prova}
 
\begin{document}
\maketitle
 
 \tableofcontents
\newpage

\section{Introduzione}
	\subsection{Scopo del documento}
	Questo documento definisce le norme da rispettare all'interno del gruppo Beacon Strips durante lo sviluppo del progetto CLIPS. \\
	Tutti i membri del team sono tenuti a visionare il seguente documento e a rispettare le norme, per migliorare l'efficienza ed avere uniformit� nei prodotti sviluppati. \\
	In particolare si tratteranno:
	\begin{itemize}
		\item le interazioni tra i membri del team;
		\item le interazioni del team con componenti esterne;
		\item le modalit� di stesura dei documenti;
		\item la gestione del repository;
		\item le modalit� di lavoro durante le varie fasi del progetto;
		\item l'ambiente di lavoro utilizzato.
	\end{itemize}
	
	\subsection{Scopo del Prodotto}
	Lo scopo del prodotto � uno studio di fattibilit� riguardo all'utilizzo dei beacon
	\subsection{Glossario}
	
	\subsection{Riferimenti}

\section{Comunicazioni}
	\subsection{Comunicazioni interne}
	Per le comunicazioni interne si � scelto di utilizzare due strumenti:
	\begin{itemize}
		\item \gl{Telegram}: viene utilizzato per le comunicazioni interne informali
		\item Slack: viene utilizzato per comunicazioni interne formali, quali avvisi riguardo le riunioni e la gestione dei vari strumenti di sviluppo. 
	\end{itemize}
	\subsection{Comunicazioni esterne}
	Per le comunicazioni esterne � stata creata l'e-mail \\
	beaconstrips.swe@gmail.com \\
	
\section{Repository}
	\subsection{Struttura del repository}
	\subsection{}
%	\subsection{} da aggiungere altri punti

%\section{Codice}
%	\subsection{Intestazione}
%	\subsection{Responsabilit� del codice}
%	\subsection{Versionamento}

\section{Documenti}
	\subsection{Template}
	\subsection{Struttura del documento}
	\subsection{Frontespizio}
	\subsection{Formattazione}
	\subsection{Norme tipografiche}
	\subsection{Versionamento}
	\subsection{Ciclo di vita}

\section{Analisi dei requisiti}
	\subsection{Studio di Fattibilit� e Analisi dei Rischi}
	\subsection{Analisi dei requisiti}
		\subsubsection{Classificazione dei requisiti}
		\subsubsection{Tracciamento}

\section{Sviluppo}
	\subsection{Sistema di sviluppo}
	\subsection{Ticket}
		\subsubsection{Creazione di un ticket}
		\subsubsection{Aggiornamento di un ticket}
		\subsubsection{Chiusura di un ticket}
		\subsubsection{Gestione delle anomalie}

\section{Ambiente di lavoro}
	\subsection{Sistema operativo}
	\subsection{Condivisione}
	\subsection{Documentazione}
	\subsection{Verifica documentazione}
	\subsection{Versionamento}



\end{document}