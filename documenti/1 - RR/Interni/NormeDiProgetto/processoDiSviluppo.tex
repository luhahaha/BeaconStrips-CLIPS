\subsubsection{Attività}
		\paragraph{Analisi dei requisiti}
		Sarà compito degli analisti redigere l'analisi dei requisiti in seguito a riunioni interne col team e allo studio dei capitolati d'appalto.
			\subparagraph{Studio di Fattibilità e Analisi dei Rischi}
			Lo studio di fattibilità sarà il primo documento dell'analisi dei requisiti che analizzerà i seguenti punti per ogni capitolato, con maggior interesse per quello scelto dal team:
			\begin{itemize}
				\item \textbf{scopo del progetto}: analisi delle richieste del capitolato;
				\item \textbf{studio del dominio}: valutazione delle tecnologie e conoscenze richieste rispetto l'attuale livello del team;
				\item \textbf{analisi dei rischi}: ricerca dei rischi e delle criticità per ciascun capitolato.
			\end{itemize}
			\subparagraph{Analisi dei requisiti}
			Il secondo documento che gli analisti andranno a redigere sarà l'analisi dei requisiti che produrrà dei requisiti semplici a partire dalle informazioni raccolte tramite lo studio del capitolato e riunioni esterne con il proponente.
			Per rendere più precisa e veloce la stesura dei requisiti è stato utilizzato il software \gl{Trender}.
	\subsubsection{Norme}
		\paragraph{Classificazione  dei requisiti}
		I requisiti saranno rappresentati secondo la seguente codifica:
		\begin{center}
			R[importanza][tipo][identificativo]
		\end{center}
		\begin{itemize}
			\item \textbf{Importanza}:indica se il requisito è:
			\begin{enumerate}
				\item \textbf{0}: obbligatorio;
				\item \textbf{1}: desiderabile;
				\item \textbf{2}: opzionale.
			\end{enumerate}
			\item \textbf{Tipo}: indica se è di tipo:
			\begin{enumerate}
				\item \textbf{F}: funzionale;
				\item \textbf{Q}: di qualità;
				\item \textbf{P}: prestazionale;
				\item \textbf{V}: di vincolo.
			\end{enumerate}
			\item \textbf{Identificativo}: è il codice univoco e gerarchico che automaticamente il software assegna al requisito(esempio: 4.2.1);
			\item \textbf{Descrizione}: una breve descrizione del requisito.
			\item \textbf{Fonte}:la fonte da cui deriva il requisito;
		\end{itemize}
		\paragraph{Classificazione dei casi d'uso}
		I casi d'uso saranno rappresentati secondo la seguente codifica:
		\begin{center}
			UC[identificativo]
		\end{center}
		\begin{itemize}
			\item \textbf{identificativo}: codice univoco e gerarchico che automaticamente il software assegna al caso d'uso (esempio: 3.4.1);
		\end{itemize}
		inoltre i casi d'uso saranno caratterizzati da:
		\begin{itemize}
			\item \textbf{tipo}: se non specificato è di tipo standard altrimenti viene scelto fra:
			\begin{enumerate}
				\item \textbf{estensione}
				\item \textbf{inclusione}
				\item \textbf{generalizzazione}
			\end{enumerate}
			\item \textbf{titolo} breve del caso d'uso;
			\item \textbf{descrizione} del caso d'uso;
			\item \textbf{precondizione} del caso d'uso;
			\item \textbf{postcondizione} del caso d'uso 
			\item \textbf{scenario principale} degli eventi;
			\item \textbf{scenario secondario} eventuale;
			\item \textbf{attori}: lista degli attori; coinvolti;
		\end{itemize}
\subsubsection{Strumenti}
	%aggiungere Trender