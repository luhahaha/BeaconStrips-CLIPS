\section{Repository}
Per la gestione condivisa e per il controllo delle versioni dei file il \gl{team} ha deciso di utilizzare una repository fornita dal servizio web \gl{GitHub}.
	\subsection{Struttura del repository}
	La respository si divide in due cartelle principali:
	\begin{itemize}
		\item \textbf{Codice}: contenente il codice del progetto, con struttura ancora da definire;
		\item \textbf{Documenti}: contenente:
		\begin{enumerate}
			\item \textbf{template}: con al suo interno i file \file{.sty} e una cartella \textbf{img} contenente le immagini;
			\item una cartella per ogni \textbf{revisione di avanzamento}, il cui nome è definito dal numero e dalla sigla della revisione, separata da un trattino. Ciascuna di esse è suddivisa in:
			\begin{enumerate}
				\item \textbf{interni}: contenente una cartella per ogni documento interno, con nome, in notazione \gl{CamelCase}, uguale a quello del documento(senza numero di versione);
				\item \textbf{esterni}: contenente una cartella per ogni documento esterno, con nome, in notazione \gl{CamelCase}, uguale a quello del documento(senza numero di versione);
			\end{enumerate}
		\end{enumerate}
	\end{itemize}
		\subsubsection{Tipi di file e .gitignore}
		All'interno delle cartelle dei documenti saranno presenti solamente i file \file{.tex}, i restanti file ausiliari prodotti da \LaTeX{} sono stati aggiunti a .gitignore e quindi vengono ignorati e resi invisibili da \gl{Git}.
	\subsection{Norme sulla commit}
	Per rendere effettive le modifiche applicate ai file della repository locale è necessario eseguire una \textbf{commit}.
	I membri del team sono tenuti ad effettuare una commit ogni qual volta venga aggiunta, rimossa o aggiornata una funzionalità oppure una sezione, aggiungendo i file interessati tramite il commando:
	\begin{center}
		\file{git add nome\_del\_file.est}
	\end{center}
	ed eseguendo la commit tramite il commando:
	\begin{center}
		\file{git commit -m ``messaggio''}
	\end{center}
	con \file{messaggio} che segua le regole della sezione 3.2.1.
	Infine per rendere effettive le modifiche nel repository in remoto eseguire:
	\begin{center}
		\file{git push}
	\end{center}
	\subsubsection{Messaggi delle commit}
	Ogni qual volta venga eseguita una \file{commit} è obbligatorio inserire un breve messaggio con la spiegazione
	delle modifiche apportate; il messaggio deve iniziare con:
	\begin{itemize}
		\item \textbf{Add:} se vengono aggiunte funzionalità o sezioni;
		\item \textbf{Remove:} se vengono tolte funzionalità o sezioni;
		\item \textbf{Update:} se vendono aggiornate funzionalità o sezioni;
		\item \textbf{Fix:} se viene sistemato un bug noto;
	\end{itemize}
	deve seguire una descrizione sintetica in lingua italiana delle modifiche apportate.
%	\subsection{} da aggiungere altri punti