\section{Documenti}
	\subsection{Template}
		È presente una cartella nel \gl{repository} in \file{documenti/template}. \\
		La cartella contiene i seguenti files:
		\begin{itemize}
			\item \textit{Comandi.sty}: contiene i comandi personalizzati (\gl);
			\item \textit{Riferimenti.sty}: %contiene i riferimenti, tipo "Variabile globale"
			\item \textit{Stile.sty}: contiene gli stili grafici da applicare al template;
			\item \textit{TemplateDoc.tex}: è il template da utilizzare per realizzare i documenti.
			\item \textit{Glossario.sty}: %aggiungere descrizione
		\end{itemize}
		È presente inoltre una cartella \textit{img} dove è presente il logo del team e dove potranno essere caricate tutte le altre immagini necessarie al template. \\
		Ogni documento dovrà essere realizzato utilizzando il \gl{template} presente nella cartella.
	\subsection{Struttura del documento}
		\subsubsection{Frontespizio}
		La prima pagina di ogni documento dovrà contenere le seguenti informazioni:
		\begin{itemize}
			\item nome del gruppo;
			\item nome del progetto;
			\item nome del documento e la sua versione;
			\item sommario;
			\item data di redazione;
			\item nome e cognome dei redattori del documento;
			\item nome e cognome dei verificatori del documento;
			\item nome e cognome del responsabile per l'approvazione del documento;
			\item destinazione d'uso del documento;
			\item lista di distribuzione del documento.
		\end{itemize}
		\subsubsection{Diario delle modifiche}
		La seconda pagina di ogni documento dovrà contenere il diario delle modifiche.\\
		Il diario consiste in una tabella ordinata in modo decrescente secondo la data di modifica e, conseguentemente, al numero di versione. È particolarmente utile per tenere traccia delle varie modifiche effettuate nel documento. \\
		Ogni riga del diario conterrà:
		\begin{itemize}
			\item numero di versione;
			\item breve riepilogo delle modifiche effettuate;
			\item autore delle modifiche;
			\item ruolo ricoperto dall'autore;
			\item data di modifica.
		\end{itemize}		
		\subsubsection{Indici}
		In ogni documento è presente un indice delle sezioni, il quale fornisce una visione macroscopica della struttura del documento.
		%Indice delle tabelle e delle figure
		\subsubsection{Intestazione e piè di pagina}
		L'intestazione di ogni pagina contiene i seguenti elementi:
		\begin{itemize}
			\item numero della sezione;
			\item titolo della sezione.
		\end{itemize}
		A piè di pagina invece si trovano:
		\begin{itemize}
			\item nome del documento e numero di versione, allineato a sinistra;
			\item nome del team, allineato al centro;
			\item pagina X di Y, dove X è la pagina corrente e Y è il numero di pagine totali, allineato a destra.
		\end{itemize}
		\subsubsection{Formattazione}
		
	
	\subsection{Norme tipografiche}
	In questa sezione vengono definite le norme riguardanti l'ortografia e la tipografia, al fine di avere uno stile uniforme per tutti i documenti prodotti.
		\subsubsection{Stile del testo}
		\begin{itemize}
			\item \textbf{Grassetto}: il grassetto va utilizzato nei seguenti casi:
			\begin{itemize}
				\item titoli;
				\item elenchi puntati: può essere utilizzato il grassetto nel caso sia necessario evidenziare il concetto da sviluppare.
			\end{itemize}
			\item Corsivo: il corsivo va utilizzato:
			\begin{itemize}
				\item quando si parla di documenti;
				\item per indicare il percorso di un file o una cartella;
				\item 
			\end{itemize}
			\item \LaTeX: ogni occorrenza di \LaTeX{} va scritta con il comando \textbackslash LaTeX
			\item Maiuscolo: è possibile utilizzare lo stile maiuscolo solo per gli acronimi;
			\item \gl{Monospace}: le porzioni di testo scritte in monospace definiscono:
			\begin{itemize}
				\item frammenti di codice;
				\item comandi;
				\item URL;
			\end{itemize} 
			%\item Glossario: le parole che hanno un riferimento nel glossario sono in corsivo e hanno una 'g' a pedice
		\end{itemize}
		\subsubsection{Composizione del testo}
		\subsubsection{Formati}
		\begin{itemize}
			\item \textbf{Date}: per le date va utilizzata la notazione definita dallo standard \gl{ISO} 8601:2004:
			\highlight{\textit{AAAA -- MM -- GG}}
			dove:
			\begin{itemize}
				\item AAAA: rappresenta l'anno utilizzando quattro cifre;
				\item MM: rappresenta il mese utilizzando due cifre;
				\item GG: rappresenta il giorno utilizzando quattro cifre;
			\end{itemize}
			\item \textbf{Sigle}: le sigle dei documenti vanno utilizzate solo nei diagrammi o nelle tabelle, con lo scopo di risparmiare spazio. Tali sigle sono:
			\begin{itemize}
				\item \textbf{AdR} per \textit{Analisi dei Requisiti}
				\item \textbf{Gl} per \textit{Glossario}
				\item \textbf{NdP} per \textit{Norme di Progetto}
				\item \textbf{PdP} per \textit{Piano di Progetto}
				\item \textbf{PdQ} per \textit{Piano di Qualifica}
				\item \textbf{SdF} per \textit{Studio di Fattibilità}
				\item \textbf{ST} per \textit{Specifica Tecnica}
			\end{itemize}	
		\end{itemize}
	\subsection{Versionamento}
	\subsection{Ciclo di vita}
