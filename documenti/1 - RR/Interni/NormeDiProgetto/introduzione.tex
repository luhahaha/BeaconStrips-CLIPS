\section{Introduzione}
	\subsection{Scopo del documento}
	Questo documento intende definire le norme da rispettare all'interno del gruppo Beacon Strips durante lo sviluppo del progetto CLIPS. \\
	Tutti i membri del team sono tenuti a visionare il documento e a rispettare le norme. Tali norme permettono di ottenere uniformità nei documenti sviluppati,
	migliorare l'efficienza del lavoro svolto e ridurre il numero di errori. \\
	In particolare si tratteranno:
	\begin{itemize}
		\item le interazioni tra i membri del team;
		\item le interazioni del team con componenti esterne;
		\item le modalità di stesura dei documenti;
		\item la gestione del \gl{repository};
		\item le modalità di lavoro durante le varie fasi del progetto;
		\item l'ambiente di lavoro utilizzato.
	\end{itemize}
	
	\subsection{Scopo del prodotto}
	Lo scopo del prodotto è la ricerca di nuovi scenari riguardanti la navigazione tramite l'utilizzo dei \gl{beacon}. Inoltre verrà realizzata un'applicazione per \gl{smartphones} per interfacciarsi con i beacon.
	\subsection{Glossario}
	Al fine di evitare ambiguità di linguaggio e rendere i documenti il più comprensivi possibile, si allega il \textit {Glossario v1.0.0}, che contiene tutti i termini 
	con un significato particolare. \\
	Una parola presente nel glossario è identificata dall'essere scritta in corsivo e da una 'g'  minuscola in pedice. %corsiva?
	\subsection{Riferimenti}
		\subsubsection{Riferimenti normativi}
		\NORMATIVI
	
