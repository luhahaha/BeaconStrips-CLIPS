\section{Sviluppo}
	\subsection{Sistema di sviluppo}
	Per tracciare e suddividere il lavoro si è deciso di utilizzare il software web di ticketing \gl{Asana}.Questo software permette di assegnare dei \gl{ticket} a ciascun membro del gruppo.
	\subsection{Ticket}
	I ticket avranno la seguente struttura:
	\begin{itemize}
		\item \textbf{titolo}: deve rappresentare in modo chiaro e conciso il task;
		\item \textbf{scadenza}: data entro cui l'assegnatario deve concludere il task;
		\item \textbf{descrizione}: se necessario,deve chiarire con una breve descrizione il task;
		\item \textbf{assegnatario}: colui che dovrà completare il task;
		\item \textbf{tag di stato}: deve contenere il giusto tag che rappresenta lo stato attuale del task; 
		\item \textbf{subtasks}: il task può essere suddiviso in task piu piccoli.
	\end{itemize}
		\subsubsection{Lista dei tag}
		Ogni ticket dovrà sempre contenere un tag che rappresenti lo stato attuale di avanzamento:
		\begin{itemize}
			\item \textbf{Assegnato}:il ticket è stato assegnato ad un membro del team;
			\item \textbf{accettato}: l'assegnatario ha preso in carico il ticket;
			\item \textbf{Rifiutato}: l'assegnatario ha rifiutato il ticket ed ha inserito la motivazione in un commento;
			\item \textbf{inCorso}: l'assegnatario sta svolgendo il task;
			\item \textbf{inSospeso}: l'assegnatario specifica della motivazione per cui ha messo in attesa il ticket;
			\item \textbf{Completato}: l'assegnatario ha concluso il task;
			\item \textbf{Verificato}:il ticket è stato verificato.
		\end{itemize}
		\subsubsection{Creazione di un ticket}
		La creazione e l'assegnazione dei ticket sarà delegata al Responsabile che seguirà la seguente procedura:
		\begin{itemize}
			\item assegna il \textbf{titolo} del ticket;
			\item assegna la data di \textbf{scadenza} del ticket;
			\item se necessario descrive il task più approfonditamente tramite la \textbf{descrizione};
			\item assegna il ticket ad un \textbf{assegnatario};
			\item imposta il \textbf{tag di stato} ad \textbf{Assegnato}.
		\end{itemize}
		\subsubsection{Aggiornamento di un ticket}
		Ogni membro del gruppo avrà una lista di ticket a lui assegnati, su di essi dovrà modificare il tag di stato ogni qual volta uno di questi cambi di stato come descritto dalla sezione 8.2.1. Nel caso in cui il ticket venga \textbf{rifiutato} o venga messo \textbf{in sospeso} è necessario scrivere una motivazione attraverso un commento.
		\subsubsection{Verifica di un ticket}
		Ogni qual volta un ticket sarà contrassegnato come \textbf{Completato} il Responsabile procederà nel seguente modo: 
		\begin{itemize}
			\item crea un ticket di verifica;
			\item lo assegna ad un \textbf{verificatore};
			\item non appena il ticket viene contrassegnato come \textbf{completato} imposta il tag del ticket oggetto di verifica a \textbf{Verificato}.
		\end{itemize}
		\paragraph{Notifiche delle anomalie}
		Il \textbf{validatore} qualora riscontrasse delle anomalie procederà secondo la seguente procedura:
		\begin{itemize}
			\item crea un nuovo subtask del ticket di verifica per ogni anomalia riscontrata;
			\item assegna un titolo breve e preciso ad ogni subtask;
			\item se necessario aggiunge un commento che descriva l'anomalia riscontrata;
			\item assegna i subtasks al redattore del documento.
		\end{itemize}
		