\documentclass[a4paper,titlepage]{article}

\makeatletter
\def\input@path{{../../../template/}}
\makeatother

\usepackage{Comandi}
\usepackage{Riferimenti}
\usepackage{Stile}

\def\NOME{Verbale del Giorno 2016-03-07}
\def\VERSIONE{1.0.0}
\def\DATA{\today}
\def\REDATTORE{Luca Soldera}
\def\VERIFICATORE{Matteo Franco}
\def\RESPONSABILE{Viviana Alessio}
\def\USO{Esterno}
\def\DESTINATARI{\COMMITTENTE \\ & \CARDIN \\ & \PROPONENTE}
\def\SOMMARIO{Breve descrizione della riunione.}


\begin{document}

\maketitle

\newpage
\tableofcontents

\newpage
\section{Informazioni}
\label{sec:Informazioni}

\begin{itemize}
  \item \textbf{Luogo}: Torre Archimede - Via Trieste 63, 35121, Padova (PD);
  \item \textbf{Data}: 2016-03-07;
  \item \textbf{Ora}: 09:30;
  \item \textbf{Durata}: 2 ore;
  \item \textbf{Partecipanti}: Viviana Alessio, Luca Soldera, Matteo Franco, Andrea Grendene, Enrico Bellio, Tommaso Panozzo.
\end{itemize}

\newpage
\section{Premessa}

La riunione è stata indetta a scopo conoscitivo poiché il team essendosi appena formato non si era ancora incontrato.

\section{Ordine del giorno}
\label{sec:OrdineDelGiorno}
%qui si potrebbe aggiungere gli argomenti che si intendono trattare sono:
%attenzione: in caso va modificato anche negli altri e nel template; poi va messa la minuscola nell'elenco puntato
\begin{enumerate}
  \item Presentazione membri del gruppo: impegni accademici, impegni extra-accademici, skills;
  \item discussione strumenti di gestione progetto tra cui ticketing, tracking dei requisiti,comunicazioni tra membri e con esterni,strumento per il versionamento; % qui si potrebbe usare itemize secondo me
  \item discussione generale sulle norme di progetto.
\end{enumerate}

\newpage
\section{Verbale}
\label{sec:Verbale}

\begin{enumerate}
	\item C'è stata una rapida presentazione tra i membri del gruppo ed ognuno ha illustrato i propri impegni e abilità personali.\\
	\textbf{Impegni}:
	
	\begin{tabella}{X[2,m,c]!{\VRule}X[2,m,c]!{\VRule}X[2,m,c]!{\VRule}X[2,m,c]!{\VRule}X[1.9,m,c]!{\VRule}X[2.1,m,c]!{\VRule}}
		\intestazionesixcol{Andrea}{Viviana}{Luca}{Tommaso}{Enrico}{Matteo}
		\cellacaporiga{lunedì\\13:30-15:15,\\martedì\\13:30-15:15,\\lunedì 	sera,\\algoritmi}&\cellacaporiga{martedì\\pomeriggio,\\sabato \\mattina}&\cellacaporiga{venerdì\\pomeriggio,\\sabato e \\domenica\\mattina,\\algoritmi}&\cellacaporiga{sabato\\pomeriggio,\\algoritmi}&nessuno&\cellacaporiga{mercoledì\\dalle 16:30,\\qualche\\giovedì\\pomeriggio}
	\end{tabella}

    \textbf{Skills e preferenze}:
    \begin{tabella}{X[2,m,c]!{\VRule}X[2.1,m,c]!{\VRule}X[2,m,c]!{\VRule}X[2.1,m,c]!{\VRule}X[1.8,m,c]!{\VRule}X[2,m,c]!{\VRule}}
    	\intestazionesixcol{Andrea}{Viviana}{Luca}{Tommaso}{Enrico}{Matteo}
    	accademiche&\cellacaporiga{accademiche,\\ \LaTeX}&accademiche&\cellacaporiga{accademiche,\\matematica,\\iOS,\\C++,\\conoscenze\\sui beacon}&\cellacaporiga{Java,\\HTML,\\JavaScript,\\C++}&accademiche
    \end{tabella}
    \item Si è discusso su quali strumenti utilizzare ed è stato deciso di utilizzare:
    \begin{itemize}
	  	\item \textbf{Ticketing}: è stato scelto Asana;
	  	\item \textbf{Tracking dei requisiti}: si è deciso di valutare tra Tracy e Trender;
	  	\item \textbf{Comunicazioni}: per quelle interne Slack e Telegram mentre per quelle formali tramite la mail che è stata creata: \file{beaconstrips.swe@gmail.com};
	  	\item \textbf{Versionamento}: è stato scelto di utilizzare GitHub;
	  	\item \textbf{Redazione Documenti}: è stato scelto \LaTeX;
	  	\item \textbf{Schemi UML}: è stato scelto StarUML;
	\end{itemize}
	\item Sulle norme di progetto è stato discusso della gestione dei ticket e del loro stato ma non è stata ancora decisa alcuna regola.
\end{enumerate}

\subsection{Altri argomenti}
\label{sub:AltriArgomenti}

Sono state inoltre toccati i seguenti punti:

\begin{itemize}
  \item divisione di compiti a breve termine:
  \begin{enumerate} %io userei itemize qui, perché l'ordine degli elementi non è rilevante come per gli ordini del giorno
  	%i nomi dei documenti vanno in corsivo
  	\item \textbf{Andrea}: redazione bozza del \file{Piano di Qualifica}
  	\item \textbf{Viviana}: redazione bozza del \file{Piano di Progetto}
  	\item \textbf{Luca}: ricerca e prova del software di analisi dei requisiti;
  	\item \textbf{Tommaso}: redazione template \LaTeX;
  	\item \textbf{Enrico}: redazione \file{Studio di fattibilità} per la parte "altri capitolati";
  	\item \textbf{Matteo}: redazione indice delle \file{Norme di progetto}.
  \end{enumerate}
  \item è stato deciso che tutti i membri devono informarsi sulla tecnologia dei beacon e ideare qualche proposta da fare al Proponente; %va chiesto se proponente è stato messo con la maiuscola altrove
\end{itemize}

\end{document}