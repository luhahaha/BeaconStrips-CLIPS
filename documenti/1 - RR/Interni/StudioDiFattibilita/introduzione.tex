\section{Introduzione}
\subsection{Scopo del documento}

Questo documento ha l'obiettivo di mettere in evidenza i ragionamenti e le motivazioni che hanno portato alla scelta del \gl{progetto} CLIPS.\\
È presente l'analisi di tutti e sei i capitolati proposti con particolare attenzione ai casi d'uso e alle tecnologie utilizzabili per ognuno.

\subsection{Glossario}
\GLOSSARIO

\subsection{Riferimenti}
\subsubsection{Normativi}
Per le norme di \gl{progetto} riferirsi al documento \NPdoc
\subsubsection{Informativi}
\begin{itemize}
	\item Capitolato C1 - Actorbase: \url{http://www.math.unipd.it/~tullio/IS-1/2015/Progetto/C1.pdf}
	\item Capitolato C2 - CLIPS: \url{http://www.math.unipd.it/~tullio/IS-1/2015/Progetto/C2.pdf}
	\item Capitolato C3 - Internet of things: \url{http://www.math.unipd.it/~tullio/IS-1/2015/Progetto/C3.pdf}
	\item Capitolato C4 - MaaS: \url{http://www.math.unipd.it/~tullio/IS-1/2015/Progetto/C4.pdf}
	\item Capitolato C5 - Quizzipedia: \url{http://www.math.unipd.it/~tullio/IS-1/2015/Progetto/C5.pdf}
	\item Capitolato C6 - Sintesi vocale su dispositivi mobili: \url{http://www.math.unipd.it/~tullio/IS-1/2015/Progetto/C6.pdf}
\end{itemize}