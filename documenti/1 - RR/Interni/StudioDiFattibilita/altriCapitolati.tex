\section{Altri Capitolati}

\subsection{Capitolato C1 - Actorbase}
\subsubsection{Scopo del progetto}

Il progetto Actorbase consiste nella progettazione di un \gl{database non relazionale} che utilizzi il modello ad attori grazie all'uso delle seguenti tecnologie:
\begin{itemize}
	\item La libreria \gl{Akka} per l'implementazione del modello ad attori su \gl{JVM};
	\item Java o \gl{Scala} come linguaggi di programmazione.
\end{itemize}
Inoltre è prevista l'implementazione di un \gl{DSL} per poter interagire con il database da riga di comando.

\subsubsection{Osservazioni}
Poiché un progetto in cui viene utilizzato il modello ad attori è già stato affrontato per il progetto del corso di Programmazione Concorrente e Distribuita, il gruppo ha deciso di
non intraprendere lo sviluppo di questo capitolato.


\subsection{Capitolato C3 - Internet of things}
\subsubsection{Scopo del progetto}

Il progetto Internet of things consiste, citando il capitolato, nella creazione di un un algoritmo predittivo in grado analizzare i dati provenienti da “oggetti”, inseriti
in diversi contesti, e fornire delle previsioni su possibili guasti, interazioni con nuovi utenti ed identificare dei pattern di comportamento degli utenti per prevedere le azioni degli
stessi su altri oggetti o altri contesti.

L'applicativo software dovrà essere composto in tre parti:
\begin{itemize}
	\item Una console web amministrativa per la definizione di regole di apprendimento a seconda del contesto e tipo di dati;
	\item Una console web di amministrazione per le singole aziende;
	\item Dei servizi web restful \gl{JSON} interrogabili.
\end{itemize}
La piattaforma dovrà inoltre permettere la comunicazione tramite i protocolli \gl{HTTP}/\gl{HTTPS} standard e il protocollo \gl{MQTT}.


Le tecnologie consigliate sono le seguenti:
\begin{itemize}
	\item \gl{MongoDB} e/o \gl{OrientDB} per il database;
	\item \gl{Amazon Web Services} per l'infrastruttura;
	\item \gl{JAVA} e/o \gl{Scala} come linguaggi di programmazione;
	\item \gl{Play Framework} come framework di sviluppo;
	\item \gl{HTML5}, \gl{CSS3}, \gl{Javascript} e il \gl{framework Bootstrap} di Twitter per l'interfaccia web.
\end{itemize}

\subsubsection{Osservazioni}
La progettazione di un algoritmo predittivo è un argomento che interessa ai membri del gruppo ma la complessità dell'argomento e la mancanza delle conoscenze
richieste per lo svolgimento del progetto hanno portato all'esclusione del capitolato da parte del gruppo.



\subsection{Capitolato C4 - MaaS}
\subsubsection{Scopo del progetto}
Il progetto MaaS consiste nella realizzazione di una piattaforma per rendere facilmente accessibile i dati contenuti in un database a coloro che non possiedono conoscenze in ambito informatico (es.: uomini d'affari).
L'applicazione dovrà essere accessibile tramite un servizio web per le compagnie che ne usufruiranno e sfruttare \gl{MaaP} per la rappresentazione grafica dei dati, inoltre dovrà estenderlo con le seguenti funzioni:

\begin{itemize}
	\item \textbf{\gl{SaaS}:} deve essere disponibile come unica istanza disponibile a più gruppi di persone, dedicando a ciascun gruppo una propria area di lavoro;
	\item \textbf{\gl{DSL}:} deve essere possibile modificare online le definizioni del \gl{DSL}, inoltre dovrebbero anche essere rese disponibili delle azioni predefinite (es.: esporta il \gl{csv} del documento)
	e la \gl{dashboard}.
\end{itemize}

I  requisiti tecnologici sono i seguenti:
\begin{itemize}
	\item \gl{Node.js} per il backend, per la precisione deve supportare la versione \gl{LTS} \gl{Argon};
	\item \gl{MongoDB} con versione non inferiore alla 3 come database;
	\item Il \gl{framework Loopback} per la gestione del sistema;
	\item Rendere disponibile il servizio su \gl{Heroku};
	\item Utilizzare \gl{github} o \gl{bitbucket} per il versionamento.
\end{itemize}

\subsubsection{Osservazioni}
La carenza delle conoscenze necessarie per sviluppare il progetto ha portato il gruppo a decidere di scartare il capitolato data la grande quantità
di tempo necessaria per colmare le lacune.


\subsection{Capitolato C5 - Quizzipedia}
\subsubsection{Scopo del progetto}

Il progetto Quizzipedia consiste nella progettazione di un sistema composto da:
\begin{itemize}
	\item Un archivio di domande;
	\item Un sistema di test che somministra all'utente una serie di domande relative all'argomento scelto.
\end{itemize}
Le domande devono essere raccolte attraverso uno specifico linguaggio chiamato QML (Quiz Markup Language).

I requisisti minimi da soddisfare sono i seguenti:
\begin{itemize}
	\item Archiviare i quiz in un server e suddividerli per argomento;
	\item Tradurre le domande archiviate da QML a HTML;
	\item Il QML deve poter gestire risposte vero/falso, a scelta multipla, testi ed immagini;
	\item Archiviare questionari contenenti le domande archiviate nel server;
	\item Proporre questionari preconfezionati;
	\item Valutare le risposte date dall'utente.
\end{itemize}

Il sistema dovrà essere utilizzato con tecnologie web quali:
\begin{itemize}
	\item \gl{JAVA} e server \gl{Tomcat} oppure \gl{Javascript} e server \gl{Node.js} per la parte server;
	\item \gl{HTML5}, \gl{CSS} e \gl{Javascript} per il client che dovrà essere eseguibile in un browser.
\end{itemize}
La parte destinata ai creatori di domande e quiz dovrà essere utilizzabile su PC mentre la parte destinata agli esaminandi
dovrà funzionare con qualunque dispositivo.

\subsubsection{Osservazioni}
I membri del gruppo si sono trovati interessati allo sviluppo dell'applicazione visto che le conoscenze necessarie per lo sviluppo rientrano
nelle conoscenze possedute dai membri stessi. Sfortunatamente non è stato possibile scegliere il capitolato in quanto non più disponibile al momento
della creazione del gruppo.

\subsection{Capitolato C6 - Sintesi vocale su dispositivi mobili}
\subsubsection{Scopo del progetto}
Il progetto consiste nella realizzazione di un'applicazione che aggiunga nuove funzioni su smartphone e/o tablet per la sintesi vocale.
L'applicazione deve usare il motore di sintesi \gl{Flexible and Adaptive Text-To-Speechmcat} e deve rispettare i seguenti requisiti obbligatori:

\begin{itemize}
	\item Gestire i problemi causati dall'utilizzo di un servizio remoto (es.: gestire il caso in cui non si è in grado di accedere ad internet);
	\item Implementare un'interfaccia grafica per la configurazione dei servizi \gl{TTS}.
\end{itemize}

I requisiti opzionali sono:

\begin{itemize}
	\item Supporto multipiattaforma;
	\item Utilizzo e integrazione di servizi aggiuntivi (es.: l'integrazione del servizio di personalizzazione
	della voce nell'applicazione o l'utilizzo di risorse esterne per ottenere contenuti).
\end{itemize}

Per quanto riguarda le tecnologie da utilizzare, l'unico vincolo è quello di utilizzare il motore di sintesi Flexible and Adaptive Text-to-Speech.

\subsubsection{Osservazioni}
Il gruppo non ha riscontrato alcun interesse nello sviluppo di applicazioni riguardanti il \gl{TTS}.
