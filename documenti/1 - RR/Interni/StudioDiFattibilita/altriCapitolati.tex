\section{Altri Capitolati}

\subsection{Capitolato C1 - Actorbase}
\subsubsection{Scopo del progetto}

Il progetto Actorbase consiste nella progettazione di un database non relazionale che utilizzi il modello ad attori grazie all'uso delle seguenti tecnologie:
\begin{itemize}
	\item La libreria Akka per l'implementazione del modello ad attori su JVM;
	\item Java o Scala come linguaggi di programmazione.
\end{itemize}
Inoltre è prevista l'implementazione di un DSL per poter interagire con il database da riga di comando.

\subsubsection{Osservazioni}
Poichè un progetto con il modello ad attori è già stato affrontato per il progetto del corso di Programmazione Concorrente e Distribuita, il gruppo ha deciso di
non intraprendere lo sviluppo di questo capitolato.


\subsection{Capitolato C3 - Internet of things}
\subsubsection{Scopo del progetto}

Il progetto Internet of things consiste, citando il capitolato, nella creazione di un un algoritmo predittivo in grado analizzare i dati provenienti da “oggetti”, inseriti
in diversi contesti, e fornire delle previsioni su possibili guasti, interazioni con nuovi utenti ed identificare dei pattern di comportamento degli utenti per prevedere le azioni degli
stessi su altri oggetti o altri contesti.

L'applicativo software dovrà essere composto in tre parti:
\begin{itemize}
	\item Una console web amministrativa per la definizione di regole di apprendimento a seconda del contesto e tipo di dati;
	\item Una console web di amministrazione per le singole aziende;
	\item Dei servizi web restful JSON interrogabili.
\end{itemize}
La piattaforma dovrà inoltre permettere la comunicazione tramite i protocolli HTTP/HTTPS standard e il protocollo MQTT.


Le tecnologie consigliate sono le seguenti:
\begin{itemize}
	\item MongoDB e/o OrientDB per il database;
	\item Amazon Web Services per l'infrastruttura;
	\item Java e/o Scala come linguaggi di programmazione;
	\item Play Framework come framework di sviluppo;
	\item HTML5, CSS3, Javascript e il framework Bootstrap di Twitter per l'interfaccia web.
\end{itemize}

\subsubsection{Osservazioni}
La progettazione di un algoritmo predittivo è un argomento che interessa ai membri del gruppo ma la complessità dell'argomento e la mancanza delle conoscenze
richieste per lo svolgimento del progetto hanno portato all'esclusione del capitolato da parte del gruppo.



\subsection{Capitolato C4 - MaaS}
\subsubsection{Scopo del progetto}
Il progetto MaaS prevede la realizzazione di un servizio web che renda disponibile direttamente attraverso internet la piattaforma MaaP alle varie compagnie
che vogliono utilizzare tale piattaforma. Il servizio deve estendere MaaP in due modi:
\begin{itemize}
	\item \textbf{SaaS:} deve essere disponibile come unica istanza disponibile a più gruppi di persone, dedicando a ciascun gruppo una propria area di lavoro;
	\item \textbf{DSL:} deve essere possibile modificare online le definizioni del DSL, inoltre dovrebbero anche essere progettate delle azioni predefinite (es.: esporta il csv del documento).
	e la dashboard
\end{itemize}

I  requisiti tecnologici sono i seguenti:
\begin{itemize}
	\item Node.js per il backend, per la precisione deve supportare la versione LTS Argon;
	\item MongoDB con versione non inferiore alla 3 come database;
	\item Il framework loopback per la gestione del sistema;
	\item Rendere disponibile il servizio su Heroku;
	\item Utilizzare github o bitbucket per il versionamento.
\end{itemize}

\subsubsection{Osservazioni}
La carenza delle conoscenze necessarie per sviluppare il progetto ha portato il gruppo a decidere di scartare il capitolato data la grande quantità
di tempo necessaria per colmare le lacune.


\subsection{Capitolato C5 - Quizzipedia}
\subsubsection{Scopo del progetto}

Il progetto Quizzipedia consiste nella progettazione di un sistema composto da:
\begin{itemize}
	\item Un archivio di domande;
	\item Un sistema di test che somministra all'utente una serie di domande relative all'argomento scelto.
\end{itemize}
Le domande devono essere raccolte attraverso uno specifico linguaggio chiamato QML (Quiz Markup Language).

I requisisti minimi da soddisfare sono i seguenti:
\begin{itemize}
	\item Archiviare i quiz in un server e suddividerli per argomento;
	\item Tradurre le domande archiviate da QML a HTML;
	\item Il QML deve poter gestire risposte vero/falso, a scelta multipla, testi ed immagini;
	\item Archiviare questionari contenenti le domande archiviate nel server;
	\item Proporre questionari preconfezionati;
	\item Valutare le risposte date dall'utente.
\end{itemize}

Il sistema dovrà essere utilizzato con tecnologie web quali:
\begin{itemize}
	\item Java e server Tomcat oppure Javascript e server Node.js per la parte server;
	\item HTML5, CSS e Javascript per il client che dovrà essere eseguibile in un browser.
\end{itemize}
La parte destinata ai creatori di domande e quiz dovrà essere utilizzabile su PC mentre la parte destinata agli esaminandi
dovrà funzionare con qualunque dispositivo.

\subsubsection{Osservazioni}
I membri del gruppo si sono trovati interessati allo sviluppo dell'applicazione visto che le conoscenze necessarie per lo sviluppo rientrano
nelle conoscenze possedute dai membri stessi. Sfortunatamente non è stato possibile scegliere il capitolato in quanto non più disponibile al momento
della creazione del gruppo.

\subsection{Capitolato C6 - Sintesi vocali su dispositivi mobili}
\subsubsection{Scopo del progetto}
Il progetto consiste nella realizzazione di un'applicazione che aggiunga nuove funzioni su smartphone e/o tablet per la sintesi vocale.
L'applicazione deve usare il motore di sintesi Flexible and Adaptive Text-To-Speech e deve rispettare i seguenti requisiti obbligatori:

\begin{itemize}
	\item Gestire i problemi causati dall'utilizzo di un servizio remoto (es.: gestire il caso in cui non si è in grado di accedere ad internet);
	\item Implementare un'interfaccia grafica per la configurazione dei servizi TTS.
\end{itemize}

I requisiti opzionali sono:

\begin{itemize}
	\item Supporto multipiattaforma;
	\item Utilizzo e integrazione di servizi aggiuntivi (es.: l'integrazione del servizio di personalizzazione
	della voce nell'applicazione o l'utilizzo di risorse esterne per ottenere contenuti).
\end{itemize}

Per quanto riguarda le tecnologie da utilizzare, l'unico vincolo è quello di utilizzare il motore di sintesi Flexible and Adaptive Text-to-Speech.

\subsubsection{Osservazioni}
Il gruppo non ha riscontrato alcun interesse nello sviluppo di applicazioni riguardanti il TTS.
