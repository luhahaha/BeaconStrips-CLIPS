\section{Altri Capitolati}

\subsection{Capitolato C1 - Actorbase}
\subsubsection{Scopo del progetto}

Il progetto Actorbase consiste nella progettazione di un database non relazionale ad attori utilizzando le seguenti teconologie:
\begin{itemize}
	\item La libreria Akka per l'implementazione del modello ad attori su JVM
	\item Java o Scala come linguaggi di programmazione
\end{itemize}
Inoltre è prevista l'implementazione di un DSL per poter interagire con il database da riga di comando.

\subsubsection{Osservazioni}



\subsection{Capitolato C3 - Internet of things}
\subsubsection{Scopo del progetto}

Il progetto Internet of things consiste, citando il capitolato, nella creazione di un un algoritmo predittivo in grado analizzare i dati provenienti da “oggetti”, inseriti
in diversi contesti, e fornire delle previsioni su possibili guasti, interazioni con nuovi utenti ed identificare dei pattern di comportamento degli utenti per prevedere le azioni degli
stessi su altri oggetti o altri contesti.

L'applicativo software dovrà essere composto in tre parti:
\begin{itemize}
	\item Una console web amministrativa per la definizione di regole di apprendimento a seconda del contesto e tipo di dati
	\item Una console web di amministrazione per le singole aziende
	\item Dei servizi web restful JSON interrogabili
\end{itemize}
La piattaforma dovrà inoltre permettere la comunicazione tramite i protocolli HTTP/HTTPS standard e il protocollo MQTT.


Le tecnologie consigliate sono le seguenti:
\begin{itemize}
	\item MongoDB e/o OrientDB per il database
	\item Amazon Web Services per l'infrastruttura
	\item Java e/o Scala come linguaggi di programmazione
	\item Play Framework come framework di sviluppo
	\item HTML5, CSS3, Javascript e il framework Bootstrap di Twitter per l'interfaccia web
\end{itemize}

\subsubsection{Osservazioni}

\subsection{Capitolato C4 - MaaS}
\subsubsection{Scopo del progetto}

\subsubsection{Osservazioni}


\subsection{Capitolato C5 - Quizzipedia}
\subsubsection{Scopo del progetto}

Il progetto Quizzipedia consiste nella progettazione di un sistema composto da:
\begin{itemize}
	\item Un archivio di domande
	\item Un sistema di test che somministra all'utente una serie di domande relative all'argomento scelto
\end{itemize}
Le domande devono essere raccolte attraverso uno specifico linguaggio chiamato QML (Quiz Markup Language).

I requisisti minimi da soddisfare sono i seguenti:
\begin{itemize}
	\item Archiviare i quiz in un server e suddividerli per argomento
	\item Tradurre le domande archiviate da QML a HTML
	\item Il QML deve poter gestire risposte vero/falso, a scelta multipla, testi ed immagini
	\item Archiviare questionari contenenti le domande archiviate nel server
	\item Proporre questionari preconfezionati
	\item Valutare le risposte date dall'utente
\end{itemize}

Il sistema dovrà essere utilizzato con tecnologie web quali:
\begin{itemize}
	\item Java e server Tomcat oppure Javascript e server Node.js per la parte server
	\item HTML5, CSS e Javascript per il client che dovrà essere eseguibile in un browser
\end{itemize}
La parte destinata ai creatori di domande e quiz dovrà essere utilizzabile su PC mentre la parte destinata agli esaminandi
dovrà funzionare con qualunque dispositivo.

\subsubsection{Osservazioni}

\subsection{Capitolato C6 - MIVOQ}
\subsubsection{Scopo del progetto}

\subsubsection{Osservazioni}

