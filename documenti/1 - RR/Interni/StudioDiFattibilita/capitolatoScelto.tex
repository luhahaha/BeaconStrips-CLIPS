\section{Capitolato scelto: C2 - CLIPS}

\subsection{Descrizione}

Il progetto CLIPS consiste nel ricercare nuovi scenari per l'implementazione della navigazione indoor e in particolare 
un nuovo metodo di navigazione alternativo al GPS che utilizzi la tecnologia BLE e un dispositivo mobile.
Alcuni esempi di applicazione sono i seguenti:
\begin{itemize}
	\item Interrelazione con altri dispositivi e macchinari robotici (per esempio la programmazione di un apparecchio pilota per diversamente abili);
	\item Trasmissione di contenuti attraverso i beacon, con sviluppo di un progetti di interazione e comunicazione (per esempio broadcast all'interno di un campus universitario);
	\item Utilizzo dei beacon nel social gaming (per esempio la caccia al tesoro).
\end{itemize}

\subsection{Studio del Dominio}

Come si evince dalla descrizione, il dominio del progetto è molto ampio, in quanto la tecnologia BLE può essere applicata in una moltitudine
di casi molto diversi l'uno dall'altro. L'obiettivo principale è quello di trovare, se possibile, un nuovo metodo di localizzazione per la navigazione indoor e secondariamente
un nuovo tipo di utilizzo.


\subsubsection{Dominio Applicativo}
Il problema principale affrontato dal capitolato è quello della navigazione indoor. I metodi di localizzazione di un dispositivo in una zona limitata (es.: una stanza di un qualsiasi edificio)
risultano essere molto approssimativi, infatti la maggioranza dei beacon viene utilizzata per fornire dei contenuti agli utenti che si trovano nel raggio d'azione del beacon stesso, senza sapere 
la posizione esatta.

\subsubsection{Dominio Tecnologico}
Le principali conoscenze tecnologiche richieste sono:

\begin{itemize}
	\item \textbf{JAVA/Objective C:} questi sono i linguaggi di programmazioni necessari per sviluppare nativamente un'applicazione per Android e iOS. Un'alternativa è un framework come Phonegap che 
	permette di programmare in HTML + CSS + Javascript per sviluppare un'applicazione multipiattaforma;
	\item \textbf{Beacon:} vista la natura del progetto è necessario essere a conoscenza di come i beacon interagiscono con i dispositivi a loro collegati;
	\item \textbf{Database:} qualsiasi sia l'ambiente di applicazione scelto, risulta necessario utilizzare un database per il salvataggio dei dati;
	\item \textbf{Comunicazione tra database e beacon:} è necessario conoscere dei protocolli di trasferimento dati (es: HTTP) per gestire la comunicazione tra il database e i beacon.
\end{itemize}

In aggiunta potrebbe essere necessario dover realizzare un portale web quindi in tal caso la conoscenza dei linguaggi HTML, CSS, Javascript e PHP risulta molto utile.


\subsubsection{Conclusioni}

Aspetti positivi:

\begin{itemize}
	\item L'ampiezza del dominio applicativo consente di scegliere un'applicazione in cui il gruppo si trova a proprio agio a lavorare;
	\item Le conoscenze necessarie allo sviluppo del progetto rientrano per la maggior parte nelle conoscenze necessarie per affrontare alcuni dei corsi
	del percorso di laurea triennale.
\end{itemize}

Aspetti negativi:

\begin{itemize}
	\item L'utilizzo dei beacon per la navigazione indoor potrebbe risultare fallimentare vista la quantità di ostacoli che potrebbero causare problemi con 
	la ricezione del segnale (es.: tipo di materiale delle pareti, persone,ecc.).
\end{itemize}