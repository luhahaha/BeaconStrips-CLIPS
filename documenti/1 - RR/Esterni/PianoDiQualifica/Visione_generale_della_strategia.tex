% !TEX encoding = UTF-8 Unicode
% !TEX TS-program = pdflatex
% !TEX spellcheck = it-IT

\section{Visione generale della strategia}
	\subsection{Procedure di controllo di qualità di processo}
		Per garantire la qualità dei processi e quindi un loro miglioramento continuo verrà usato il principio \gl{PDCA}. Di conseguenza migliorerà la qualità del \gl{prodotto}. \\
		Per avere il controllo dei processi, e di conseguenza qualità, è necessario che:
		\begin{itemize}
			\item\ i processi siano pianificati dettagliatamente;
			\item\ vi sia un controllo sul lavoro di ogni membro del \gl{team};
			\item\ nella pianificazione siano ripartite chiaramente le risorse.
		\end{itemize}
		L'attuazione di questi punto è approfondita nel \PPdoc. \\
		Analizzando la qualità del prodotto si controlla anche la qualità dei processi. Un prodotto scadente indica che i processi sono migliorabili. \\
		Inoltre si possono usare delle metriche per quantificare la qualità dei processi, tali metriche sono descritte nella sezione \hyperref[sec:MisureGenerale]{3.9 Misure e metriche}.
	\subsection{Procedure di controllo di qualità di prodotto}
		Il controllo di qualità del prodotto verrà garantito da:
		\begin{itemize}
			\item \textbf{quality assurance:} è l'insieme di attività realizzate per raggiungere gli obiettivi di qualità. Queste attività prevedono l'attuazione di tecniche di analisi statica e dinamica, descritte nella sezione \hyperref[sec:TecnicheAnalisi]{3.8 Tecniche di analisi}.
			\item \textbf{verifica:} è un processo che determina se l'output di una fase è consistente, corretto e completo. Per tutta la durata del progetto verranno svolte attività di verifica, i cui risultati sono e saranno riportati nell'Appendice \hyperref[sec:A]{A}.
			\item \textbf{validazione:} è il processo di conferma oggettiva del soddisfacimento dei requisiti, attuato per ogni fase del gl{progetto}.
		\end{itemize}
	\subsection{Organizzazione}
		L'organizzazione della strategia di verifica si basa sull'attuazione di verifiche per ogni processo attuato. Queste verifiche controllano sia la qualità del processo stesso sia la qualità del prodotto ottenuto. Grazie al diario delle modifiche sarà possibile eseguire una verifica solo sui cambiamenti effettuati. \\
		A causa della diversa natura dei risultati ottenuti da ogni fase del processo ognuno di essi richiederà l'attuazione di specifiche procedure di verifica. Il \gl{team} ha deciso di adottare un ciclo di vita incrementale per lo sviluppo del progetto. Di conseguenza il processo di verifica adottato per ogni fase del progetto opererà nel modo seguente:
		\begin{itemize}
			\item \textbf{individuazione degli strumenti, analisi dei requisiti e di dettaglio:} in questa fase verranno redatti i documenti che riporteranno i requisiti individuati, le strategie e le norme adottate.
			\begin{itemize}
				\item Verrà controllata la correttezza ortografica con il correttore automatico di \gl{TeXstudio}.
				\item Verrà controllata la correttezza lessicale con un'attenta ed accurata rilettura.
				\item Verrà controllata la correttezza dei contenuti rispetto alle aspettative del documento attraverso una rilettura accurata.
				\item Verrà verificato che ogni requisito abbia una corrispondenza in un caso d'uso; per farlo si controlleranno le apposite tabelle di tracciamento, con l'ausilio di \gl{Trender}.
				\item Ogni documento dovrà rispettare le \NPdoc; per verificarlo verranno adoperati gli strumenti più appropriati.
				\item Verrà verificato che sia presente una didascalia per ogni rappresentazione grafica e il contenuto di ogni figura e tabella.
			\end{itemize}
			\item \textbf{progettazione architetturale:} verrà garantito che ogni requisito sarà rintracciabile, attraverso il processo di verifica. Ogni requisito sarà rintracciabile nei componenti individuati in questa fase, si veda la \hyperref[sec:Definizione]{Sezione 2}\ per ulteriori approfondimenti.
			\item \textbf{progettazione di dettaglio dei requisiti obbligatori, desiderabili, opzionali e della codifica:} i Programmatori svolgeranno le attività di codifica e di esecuzione dei test di unità per la verifica del codice. Queste attività avverranno nel modo più automatizzato possibile, rispettando anche i vincoli statici. I Verificatori controlleranno parallelamente la presenza di eventuali anomalie, definite nella \hyperref[sec:Gestione]{Sezione 4}.
			\item \textbf{validazione:} alla Revisione di Accettazione (RA) il gruppo \AUTORE\ garantisce il funzionamento corretto del prodotto realizzato. Le eventuali modifiche per eliminare le possibili diversità rispetto al prodotto atteso saranno a carico del \gl{fornitore}.
		\end{itemize}
		In ogni documento viene inoltre incluso il diario delle modifiche, in modo da mantenere uno storico delle attività svolte e delle relative responsabilità.
	\subsection{Pianificazione strategica e temporale}
		Per impedire una rapida diffusione degli errori è necessario che la verifica della documentazione sia sistematica ed organizzata. In questo modo inoltre l'individuazione e la correzione degli errori avverrà il prima possibile. \\
		Nel \PPdoc verranno pianificate le attività svolte per migliorare la qualità dei processi, le quali stabiliranno delle nuove norme di progetto. \\
		Per ridurre la possibilità di commettere errori e/o imprecisioni di natura tecnica/concettuale ogni attività di redazione o di codifica dovrà essere preceduta da uno studio preliminare. In questo modo viene alleggerita l'attività di verifica perché richiederà a posteriori meno interventi correttivi. \\
		Il \gl{team} ha come obiettivo il rispetto delle scadenze fissate nel \PPdoc, riportate di seguito:
		\begin{itemize}
			\item\ revisioni formali:
			\begin{itemize}
				\item\ Revisione dei Requisiti: 2016/04/18.
				\item\ Revisione di Accettazione: 2016/09/12.
			\end{itemize}
			\item revisioni di progresso:
			\begin{itemize}
				\item\ Revisione di Progettazione: 2016/06/17.
				\item\ Revisione di Qualifica: 2016/08/24.
			\end{itemize}
		\end{itemize}
	\subsection{Responsabilità}
		La responsabilità dell'assegnazione degli incarichi è a carico del \RES. L'\AM\ avrà come responsabilità l'adeguamento dell'ambiente di lavoro per lo svolgimento di tutti i compiti necessari alla realizzazione del progetto. Ogni componente del \gl{team} è responsabile del proprio materiale prodotto.
	\subsection{Risorse}
		\subsubsection{Necessarie}
			Per la realizzazione del \gl{prodotto} sono necessarie risorse sia tecnologiche che umane:
			\begin{itemize}
				\item \textbf{risorse umane:}\ vengono descritte dettagliatamente nel \PPdoc.
				\begin{itemize}
					\item\ Amministratore;
					\item\ Responsabile;
					\item\ Analista;
					\item\ Progettista;
					\item\ Programmatore;
					\item\ Verificatore.
				\end{itemize}
				\item \textbf{risorse software:} sono necessari strumenti software utili:
				\begin{itemize}
					\item\ alla stesura di documentazione in \gl{\LaTeX};
					\item\ alla creazione di diagrammi in \gl{UML};
					\item\ allo sviluppo nei linguaggi di programmazione scelti;
					\item\ a semplificare ed automatizzare la verifica;
					\item\ all'analisi statica del codice;
					\item\ alla gestione dei test sul codice.
				\end{itemize}
				\item \textbf{risorse hardware:} sono necessari computer con tutti gli strumenti software descritti nelle \NPdoc. È necessario avere a disposizione uno o più luoghi dove poter effettuare le riunioni interne del gruppo \AUTORE. Sono necessari i \gl{beacon} in un numero sufficiente per i test del progetto.
			\end{itemize}
		\subsubsection{Disponibili}
			Ogni membro di \AUTORE\ ha a disposizione almeno un computer personale dotato di tutti gli strumenti necessari con cui poter svolgere tutti i propri compiti. \\
			A scopo di test e di supporto degli strumenti scelti per l'ambiente di sviluppo vengono messi a disposizione uno spazio web e un server privato da \PROPONENTE. \\
			Per lo svolgimento delle riunioni interne il \gl{team} usa le aule del Dipartimento di Matematica dell'Università degli Studi di Padova; inoltre usa \gl{Google Hangouts} come strumento di discussione tramite videochiamate.
	\subsection{Strumenti}
		Nella sezione ‘‘Ambiente di sviluppo’’ del documento \NPdoc, sono descritti gli strumenti software impiegati dl \gl{team} per:
		\begin{itemize}
			\item\ il versionamento;
			\item\ la redazione e la verifica di documenti;
			\item\ la stesura e la verifica del codice;
			\item\ la semplificazione del lavoro tramite la stesura di \gl{script}; %visto che la Vivi ha fatto lo script per trovare le parole con \gl penso si possa lasciare quest'affermazione
			\item\ l'organizzazione e la condivisione dei file.
		\end{itemize}
	\subsection{Tecniche di analisi}
		\label{sec:TecnicheAnalisi} %3.8
		\subsubsection{Analisi statica}
			L'analisi statica è una tecnica di verifica applicabile sia ai documenti che al codice. Verrà effettuata per tutta la fase di sviluppo del progetto e serve per trovare anomalie, che verranno trattate come descritto nelle \NPdoc. Può essere applicata nei due modi seguenti.
			\paragraph{Walkthrough}
				Consiste in una lettura del documento/codice cercando anomalie ed errori ad ampio spettro, cioè senza avere un'idea precisa di quali possibili errori si potranno trovare. Per evitare incomprensioni ogni modifica verrà discussa con gli autori e concordata tra le due parti. \\
				Il \gl{walkthrough}\ è una tecnica indispensabile nelle prime fasi di sviluppo, in cui non si ha un'idea precisa dei possibili errori. Usando questa tecnica sarà poi possibile stilare una ‘‘lista di controllo’’, dove verranno archiviati gli errori più frequenti. Tale lista sarà salvata nell'Appendice delle \NPdoc. \\ %quest'ultima frase sarà da decidere se tenere o no e quindi agire di conseguenza
				Inizialmente le fasi di verifica prevederanno perlopiù l'uso della tecnica \gl{walkthrough}. Appena la lista di controllo sarà sufficientemente ampia si userà sempre di più la tecnica di \gl{Inspection}.
			\paragraph{Inspection}
				L'\gl{inspection}\ si basa sulla lettura mirata dei documenti/codice, cercando gli errori segnalati nella lista di controllo. Man mano che le verifiche procedono la lista verrà estesa e l'\gl{inspection}\ sarà più efficace.
		\subsubsection{Analisi dinamica}
			L'analisi dinamica verifica il funzionamento delle sole componenti software e aiuta ad identificarne le anomalie. \\
			Per garantire un risultato attendibile il test dev'essere ripetibile, ovvero devo ottenere sempre lo stesso output dallo stesso input. Solo così il test garantisce di trovare eventuali problemi e quindi verificare la correttezza del prodotto software. \\
			Per ogni test dev'essere definito:
			\begin{itemize}
				\item \textbf{ambiente:} è il sistema hardware e software sui quali viene pianificata l'esecuzione del test sul prodotto. Va inoltre specificato uno stato iniziale da cui il test deve partire;
				\item \textbf{specifica:} consiste nella specifica degli input e degli output attesi;
				\item \textbf{procedure:} è la specifica di ulteriori istruzioni sull'esecuzione del test e sull'interpretazione dei risultati.
			\end{itemize}
			Nelle prossime sezioni sono definiti i tipi di test effettuati sul prodotto software in sviluppo dal \gl{team}.
			\paragraph{Test di unità}
				Il test di unità verifica che ogni singola di unità di prodotto funzioni correttamente. Per unità si intende una porzione di software abbastanza piccola da poterne assegnare lo sviluppo ad un singolo \PR. \\
				Attraverso questo test verrà verificata la correttezza di ogni modulo base che compone il software, limitando gli errori di implementazione.
			\paragraph{Test di integrazione}
				Il test di integrazione verifica che due o più moduli già verificati funzionino come previsto quando vengono assemblati insieme. Inoltre aiuta a scovare i difetti residui sfuggiti dalla fase di test precedente. \\
				In più con questo test viene verificata la collaborazione tra i moduli prodotti e le componenti esterne usate come \gl{framework} o \gl{librerie}. \\
				È possibile che vengano usate delle componenti ‘‘fittizie’’ al posto dei moduli non ancora completati per poter eseguire comunque dei test.
			\paragraph{Test di sistema}
		
		\label{sec:MisureGenerale} %3.9
		\label{sec:Misure} %3.9.3