% !TEX encoding = UTF-8 Unicode
% !TEX TS-program = pdflatex
% !TEX spellcheck = it-IT

\section{Visione generale della strategia}

\subsection{Procedure di controllo di qualità di processo}
Per garantire la qualità dei processi e quindi un loro miglioramento continuo verrà usato il principio \gl{PDCA}. Di conseguenza migliorerà la qualità del \gl{prodotto}. \\
Per avere il controllo dei processi, e di conseguenza qualità, è necessario che:
\begin{itemize}
\item\ i processi siano pianificati dettagliatamente;
\item\ vi sia un controllo sul lavoro di ogni membro del \gl{team};
\item\ nella pianificazione siano ripartite chiaramente le risorse.
\end{itemize}
L'attuazione di questi punto è approfondita nel \PPdoc. \\
Analizzando la qualità del prodotto si controlla anche la qualità dei processi. Un prodotto scadente indica che i processi sono migliorabili. \\
Inoltre si possono usare delle metriche per quantificare la qualità dei processi, tali metriche sono descritte nella sezione \hyperref[sec:MisureGenerale]{3.9}.

\label{sec:MisureGenerale}
\label{sec:Misure}