% !TEX encoding = UTF-8 Unicode
% !TEX TS-program = pdflatex
% !TEX spellcheck = it-IT

\section{Definizione obiettivi di qualità}
\label{sec:2}
	Basandosi sullo standard \iso{ISO/IEC 9126} il \gl{team} si impegna a garantire al prodotto \PROGETTO\ le seguenti qualità:
	\subsection{Funzionalità}
		Il \gl{prodotto} deve garantire tutti i requisiti stabiliti nel documento \ARdoc\ e implementarli nel modo più completo ed economico possibile.
		\begin{itemize}
			\item \textbf{Misura:}\ l'unità di misura adottata sarà la quantità di requisiti presenti e funzionanti nel prodotto.
			\item \textbf{Metrica:}\ la sufficienza è stabilita nel soddisfacimento dei requisiti obbligatori.
			\item \textbf{Strumenti:}\ ogni requisito dovrà superare tutti i test previsti in modo da garantire il loro funzionamento. Per avere informazioni dettagliate sugli strumenti si veda il documento \NPdoc. 
		\end{itemize}
	\subsection{Affidabilità}
		Il \gl{prodotto} deve essere il più robusto possibile e facilmente ripristinabile in caso di errori.
		\begin{itemize}
			\item \textbf{Misura:}\ l'unità di misura adottata sarà il numero di esecuzioni che hanno successo.
			\item \textbf{Metrica:}\ le esecuzioni dovranno coinvolgere tutte le parti possibili del \gl{prodotto} ed esaminare il maggior numero possibile di casi. Non si può definire una soglia di sufficienza perché è impossibile determinare ogni situazione d'utilizzo possibile.
			\item \textbf{Strumenti:}\ da definire.
		\end{itemize}
	\subsection{Usabilità}
		Il \gl{prodotto} deve essere di facile utilizzo per la classe di utenti designata. Inoltre deve soddisfare ogni necessità dell'utilizzatore.
		\begin{itemize}
			\item \textbf{Misura:}\ verrà usata come unità di misura la valutazione soggettiva del prodotto. Questo perché non esiste uno strumento adatto ad eseguire una misurazione oggettiva dell'usabilità.
			\item \textbf{Metrica:}\ purtroppo non esiste una metrica adeguata che possa determinare una soglia di sufficienza. Il \gl{team} si impegna comunque a fornire la miglior qualità d'uso possibile. Per ottenere un risultato più soddisfacente verranno consultate delle persone esterne al gruppo per verificare l'usabilità del \gl{prodotto}. 
			\item \textbf{Strumenti:}\ si vedano le \NPdoc.
		\end{itemize}
	\subsection{Efficienza}
		Il \gl{prodotto} deve fornire tutte le funzionalità nel minore tempo possibile e minimizzando l'utilizzo di risorse.
		\begin{itemize}
			\item \textbf{Misura:}\ il tempo di latenza per ottenere una risposta in ogni pagina del \gl{prodotto}.
			\item \textbf{Metrica:}\ la sufficienza è raggiunta con un tempo di latenza minore di 5 secondi, ponendo che non ci siano problemi di connessione.
			\item \textbf{Strumenti:}\ si vedano le \NPdoc.
		\end{itemize}
	\subsection{Manutenibilità}
		Il \gl{prodotto} dev'essere comprensibile ed estensibile in modo facile e verificabile.
		\begin{itemize}
			\item \textbf{Misura:}\ l'unità di misura utilizzata saranno le metriche sul codice stabilite nella sezione \hyperref[sec:3.9.3]{3.9.3}.
			\item \textbf{Metrica:}\ il \gl{prodotto} deve raggiungere la sufficienza in tutte le metriche descritte nella sezione \hyperref[sec:3.9.3]{3.9.3}.
			\item \textbf{Strumenti:}\ si vedano le \NPdoc.
		\end{itemize}
%La sezione 2.6 è stata un po' personalizzata rispetto al documento di riferimento (quello dei ProTech), di conseguenza è più probabile che ci siano errori, questa nota è destinata soprattutto al verificatore.
	\subsection{Portabilità}
		Il \gl{prodotto} deve essere il più portabile possibile. Il \gl{front end} dev'essere utilizzabile da più dispositivi possibili. Il \gl{back end} deve poter girare su ogni sistema operativo \gl{Android} a partire dalla versione .
		\begin{itemize}
			\item \textbf{Misura:}\ il \gl{back end} dev'essere affidabile per ogni versione di \gl{Android} a partire dalla . Ogni dispositivo con questa versione del sistema operativo o una maggiore deve avere un \gl{front end} usabile, a prescindere dalle specifiche hardware o dalla risoluzione dello schermo.
			\item \textbf{Metrica:}\ il \gl{prodotto} dovrà raggiungere la sufficienza in tutte le metriche della sezione \hyperref[sec:3.9.3]{3.9.3}. Il \gl{back end} dovrà raggiungere la sufficienza in affidabilità ed efficienza in ogni dispositivo testato. Il \gl{front end} dovrà raggiungere la sufficienza in usabilità in ogni dispositivo testato.
			\item \textbf{Strumenti:}\ si vedano le \NPdoc.
		\end{itemize}
	\subsection{Altre qualità}
		Saranno inoltre garantite le seguenti caratteristiche:
		\begin{itemize}
			\item \textbf{incapsulamento:}\ per aumentare la manutenibilità e il riuso di codice verrà applicata la tecnica dell'incapsulamento. Questo implica che dove sarà possibile verrà favorito l'uso delle interfacce.
			\item \textbf{coesione:}\ per rendere il \gl{prodotto} più manutenibile, più semplice e con un indice di dipendenze minore verrà usata la tecnica della coesione. Questo significa che le funzionalità con il medesimo scopo risiederanno nello stesso componente.
		\end{itemize}
