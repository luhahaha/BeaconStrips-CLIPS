% !TEX encoding = UTF-8 Unicode
% !TEX TS-program = pdflatex
% !TEX spellcheck = it-IT

\documentclass[a4paper]{article}
\usepackage[T1]{fontenc}
\usepackage[utf8x]{inputenc}
\usepackage[italian]{babel}

%Nome del documento
\def\NOME{Piano di Qualifica}
%Versione del documento
\def\VERSIONE{1.00}
%Data del documento
\def\DATA{15/03/2016}
%Redattore/i del documento
\def\REDATTORE{Grendene Andrea}
%Verificatore del documento
\def\VERIFICATORE{Unknown}
%Responsabile del progetto
\def\RESPONSABILE{Alessi Viviana (credo)}
%Uso del documento
\def\USO{Esterno}
%Destinatari del documento
\def\DESTINATARI{\COMMITTENTE\
				 \CARDIN\
				 \PROPONENTE}
%Sommario del documento
\def\SOMMARIO{Questo documento ha lo scopo di fissare le norme necessarie ad assicurare i requisiti qualitativi del progetto \PROGETTO, regolamentando le operazioni di pianificazione e di verifica attuate per rispettare tali norme.}

\makeatletter
\def\input@path{{../../../template/}}
\makeatother
\usepackage{Riferimenti,Comandi,Stile} %Carico il template
\hypersetup{colorlinks=true,linkcolor=black,urlcolor=blue} %Modifico il colore degli url rispetto a quello standard, imposto a nero quello dei link interni (in pratica l'indice)

\newcommand{\evidenzia}[1]{\emph{‘‘#1’’}} %Evidenzia i riferimenti importanti (per ora sono riuscito ad usare altro quando serviva)
\newcommand{\importante}[1]{\textbf{#1}} %Evidenzia i titoletti per i Riferimenti

\includeonly{Parte_2}

\begin{document}

\maketitle

\begin{diario}
\modifica{Grendene Andrea}{\AN}{Impostazione della struttura e dei dettagli del documento}{15/03/2016}{0.01}
\modifica{Grendene Andrea}{\AN}{Stesura dell'Introduzione}{16/03/2016}{0.02}
\end{diario}
\tableofcontents

\section{Introduzione}
\subsection{Scopo del documento}

Questo documento fissa le regole decise dal \gl{team} per garantire una buona qualità del \gl{prodotto} e dei processi attuati durante l'intera durata del progetto. Per assicurarne il rispetto verranno svolte costantemente delle attività di verifica dei processi. In questo modo verrà garantita la qualità del prodotto e si minimizzano le risorse impiegate.

\subsection{Scopo del prodotto}

Lo scopo del prodotto è di sfruttare la tecnologia dei \gl{beacon} per (informazione mancante).

\subsection{Glossario}

Per facilitare la lettura ed evitare ogni tipo di ambiguità relativa ai termini e al linguaggio usato si farà riferimento al \Gldoc. In questo documento verrà descritto ogni termine con un significato particolare. Ognuno di questi termini verrà segnalato alla fine della parola con una g a pedice (ad esempio: \gl{Glossario}).

\subsection{Riferimenti} 
\subsubsection{Normativi}
\begin{itemize}
\item \importante{Norme di Progetto:}\ \NPdoc;
\item \importante{Capitolato d'appalto C2:}\ CLIPS: Communication \& Localisation with Indoor Positioning Systems, reperibile all'indirizzo \url{http://www.math.unipd.it/~tullio/IS-1/2015/Progetto/C2.pdf}.
\end{itemize}

\subsubsection{Informativi}
\begin{itemize}
\item \importante{Analisi dei Requisiti:}\ \ARdoc;
\item \importante{Piano di progetto:}\ \PPdoc;
\item \importante{Indice Gulpease:}\ \url{http://it.wikipedia.org/wiki/Indice_Gulpease}.
\end{itemize}

\include{Parte_2}

\end{document}