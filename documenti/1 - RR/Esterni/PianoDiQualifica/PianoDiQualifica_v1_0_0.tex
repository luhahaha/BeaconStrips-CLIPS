% !TEX encoding = UTF-8 Unicode
% !TEX TS-program = pdflatex
% !TEX spellcheck = it-IT
\documentclass[a4paper,titlepage]{article}

\usepackage[utf8x]{inputenc}

\makeatletter
\def\input@path{{../../../template/}}
\makeatother
\usepackage{Riferimenti,Comandi,Stile} %Carico il template

%Nome del documento
\def\NOME{Piano di Qualifica}
%Versione del documento
\def\VERSIONE{1.00}
%Data del documento
\def\DATA{\today}
%Redattore/i del documento. Va scritto prima il nome, poi il cognome
\def\REDATTORE{Andrea Grendene \\ & Viviana Alessio}
%Verificatore del documento
\def\VERIFICATORE{Tommaso Panozzo}
%Responsabile del progetto
\def\RESPONSABILE{Viviana Alessio}
%Uso del documento
\def\USO{Esterno}
%Destinatari del documento
\def\DESTINATARI{\COMMITTENTE \\ &
				 \CARDIN\ \\ &
				 \PROPONENTE}
%Sommario del documento
\def\SOMMARIO{Questo documento ha lo scopo di fissare le norme necessarie ad assicurare i requisiti qualitativi del progetto \PROGETTO, regolamentando le operazioni di pianificazione e di verifica attuate per rispettare tali norme.}

\begin{document}

\maketitle

	% le ultime modifiche vanno messe in testa alla tabella
\begin{diario}
	\modifica{Andrea Grendene}{\AN}{Stesura della Definizione degli obiettivi di qualità}{2016/03/21}{0.03}
	\modifica{Andrea Grendene}{\AN}{Stesura dell'Introduzione}{2016/03/16}{0.02}
	\modifica{Andrea Grendene}{\AN}{Impostazione della struttura e dei dettagli del documento}{2016/03/15}{0.01}
\end{diario}

\newpage
\tableofcontents
\newpage

\section{Introduzione}
	\subsection{Scopo del documento} 
	Questo documento ha lo scopo di spiegare dettagliatamente le strategie secondo cui il gruppo \AUTORE{} intende condurre il \gl{progetto} didattico. 
	\subsection{Scopo del \gl{prodotto}}
	\SCOPO
	\subsection{Glossario}
	\GLOSSARIO
	\subsection{Riferimenti}
		\subsubsection{Normativi}
			\begin{itemize}
				\item \textbf{Capitolato d'appalto C2 - CLIPS:} Communication \& Localisation with Indoor Positioning Systems. \\
				\url{http://www.math.unipd.it/~tullio/IS-1/2015/Progetto/C2.pdf}
				\item \textbf{Vincoli e dettagli tecnico-economici} \\
				\url{http://www.math.unipd.it/~tullio/IS-1/2015/Dispense/PD01.pdf}
				\item \textbf{Norme di Progetto} \\ \NPdoc
				\item \textbf{Regolamento di Progetto} \\
				\url{http://www.math.unipd.it/~tullio/IS-1/2015/Progetto/}
				\item \textbf{Regolamento organigramma} \\
				\url{http://www.math.unipd.it/~tullio/IS-1/2015/Progetto/PD01b.html}
			\end{itemize}	
			
		\subsubsection{Informativi}
			\begin{itemize}
				\item \textbf{Software Engineering (10th edition}) \\
				Ian Sommerville \\
				Pearson Education | Addison-Wesley
				\item \textbf{Guide to the Software Engineering Body of Knowledge}
				IEEE Computer Society. Software Engineering Coordinating Committee
				\item \textbf{Slides del \COMMITTENTE} \\ riguardo i  \href{http://www.math.unipd.it/~tullio/IS-1/2015/Dispense/L02.pdf}{processi \gl{software}}, il \href{http://www.math.unipd.it/~tullio/IS-1/2015/Dispense/L03.pdf}{ciclo di vita del \gl{software}} e \href{http://www.math.unipd.it/~tullio/IS-1/2015/Dispense/L04.pdf}{la gestione di \gl{progetto}}	
			\end{itemize}
	\subsection{Modello di ciclo di vita scelto}
	È stato scelto come ciclo di vita il modello \gl{incrementale}. Le motivazioni che ci hanno spinto verso questa direzione sono il modo in cui è strutturato il \gl{progetto} didattico e la quasi totale inesperienza dei componenti del gruppo nello sviluppare progetti \gl{software} di grandi dimensioni. Di seguito una lista di caratteristiche del metodo \gl{incrementale}:
	\begin{itemize}
		\item si può produrre valore ad ogni incremento;
		\item ogni incremento riduce il rischio di fallimento;
		\item prevede rilasci multipli;
		\item i requisiti utente sono classificati e trattati in base alla loro importanza strategica. I requisiti più importanti sono già stabili all'inizio dello sviluppo del \gl{progetto};
		\item l'analisi dei requisiti e la progettazione architetturale non vengono ripetute;
		\item prima si pensa allo sviluppo dei requisiti essenziali, poi a quelli desiderabili;
		\item Sono presenti delle iterazioni del tipo Prototipo $\rightarrow$ Validazione $\rightarrow$ Prototipo $\rightarrow$ Validazione $\rightarrow$ ecc..
	\end{itemize}
	\subsection{Scadenze}
	Il gruppo Beacon Strips ha deciso di rispettare le seguenti scadenze:
	\begin{itemize} 
		\item \textbf{Revisione dei Requisiti}: 2016-04-18
		\item \textbf{Revisione di Progettazione}: 2016-06-17
		\item \textbf{Revisione di Qualifica}: 2016-08-24
		\item \textbf{Revisione di Accettazione}: 2016-09-12
	\end{itemize}
	In base a queste scadenze e a fronte dell'analisi dei rischi verranno decise le fasi in cui suddividere il lavoro di sviluppo del \gl{progetto}.
	\subsubsection{Scelta Revisione di Progettazione}
	Si è deciso di affrontare la RP$_{\mbox{\textit{min}}}$. Il gruppo si impegna quindi per il 2016-06-17 di presentare nel documento ``Specifica Tecnica'' la progettazione ad alto livello del \gl{prodotto}.
	
% fa già \newpage in automatico

% !TEX encoding = UTF-8 Unicode
% !TEX TS-program = pdflatex
% !TEX spellcheck = it-IT

\section{Definizione obiettivi di qualità}
\label{sec:Definizione}
	Basandosi sullo standard \iso{ISO/IEC 9126} il \gl{team} si impegna a garantire al prodotto \PROGETTO\ le seguenti qualità:
	\subsection{Funzionalità}
		Il \gl{prodotto} deve garantire tutti i requisiti stabiliti nel documento \ARdoc\ e implementarli nel modo più completo ed economico possibile.
		\begin{itemize}
			\item \textbf{Misura:}\ l'unità di misura adottata sarà la quantità di requisiti presenti e funzionanti nel prodotto.
			\item \textbf{Metrica:}\ la sufficienza è stabilita nel soddisfacimento dei requisiti obbligatori.
			\item \textbf{Strumenti:}\ ogni requisito dovrà superare tutti i test previsti in modo da garantire il loro funzionamento. Per avere informazioni dettagliate sugli strumenti si veda il documento \NPdoc. 
		\end{itemize}
	\subsection{Affidabilità}
		Il \gl{prodotto} deve essere il più robusto possibile e facilmente ripristinabile in caso di errori.
		\begin{itemize}
			\item \textbf{Misura:}\ l'unità di misura adottata sarà il numero di esecuzioni che hanno successo.
			\item \textbf{Metrica:}\ le esecuzioni dovranno coinvolgere tutte le parti possibili del \gl{prodotto} ed esaminare il maggior numero possibile di casi. Non si può definire una soglia di sufficienza perché è impossibile determinare ogni situazione d'utilizzo possibile.
			\item \textbf{Strumenti:}\ da definire.
		\end{itemize}
	\subsection{Usabilità}
		Il \gl{prodotto} deve essere di facile utilizzo per la classe di utenti designata. Inoltre deve soddisfare ogni necessità dell'utilizzatore.
		\begin{itemize}
			\item \textbf{Misura:}\ verrà usata come unità di misura la valutazione soggettiva del prodotto. Questo perché non esiste uno strumento adatto ad eseguire una misurazione oggettiva dell'usabilità.
			\item \textbf{Metrica:}\ purtroppo non esiste una metrica adeguata che possa determinare una soglia di sufficienza. Il \gl{team} si impegna comunque a fornire la miglior qualità d'uso possibile. Per ottenere un risultato più soddisfacente verranno consultate delle persone esterne al gruppo per verificare l'usabilità del \gl{prodotto}. 
			\item \textbf{Strumenti:}\ si vedano le \NPdoc.
		\end{itemize}
	\subsection{Efficienza}
		Il \gl{prodotto} deve fornire tutte le funzionalità nel minore tempo possibile e minimizzando l'utilizzo di risorse.
		\begin{itemize}
			\item \textbf{Misura:}\ il tempo di latenza per ottenere una risposta in ogni pagina del \gl{prodotto}.
			\item \textbf{Metrica:}\ la sufficienza è raggiunta con un tempo di latenza minore di 5 secondi, ponendo che non ci siano problemi di connessione.
			\item \textbf{Strumenti:}\ si vedano le \NPdoc.
		\end{itemize}
	\subsection{Manutenibilità}
		Il \gl{prodotto} dev'essere comprensibile ed estensibile in modo facile e verificabile.
		\begin{itemize}
			\item \textbf{Misura:}\ l'unità di misura utilizzata saranno le metriche sul codice stabilite nella sezione \hyperref[sec:Misure]{3.9.3}.
			\item \textbf{Metrica:}\ il \gl{prodotto} deve raggiungere la sufficienza in tutte le metriche descritte nella sezione \hyperref[sec:Misure]{3.9.3}.
			\item \textbf{Strumenti:}\ si vedano le \NPdoc.
		\end{itemize}
%La sezione 2.6 è stata un po' personalizzata rispetto al documento di riferimento (quello dei ProTech), di conseguenza è più probabile che ci siano errori, questa nota è destinata soprattutto al verificatore.
	\subsection{Portabilità}
		Il \gl{prodotto} deve essere il più portabile possibile. Il \gl{front end} dev'essere utilizzabile da più dispositivi possibili. Il \gl{back end} deve poter girare su ogni sistema operativo \gl{Android} a partire dalla versione .
		\begin{itemize}
			\item \textbf{Misura:}\ il \gl{back end} dev'essere affidabile per ogni versione di \gl{Android} a partire dalla . Ogni dispositivo con questa versione del sistema operativo o una maggiore deve avere un \gl{front end} usabile, a prescindere dalle specifiche hardware o dalla risoluzione dello schermo.
			\item \textbf{Metrica:}\ il \gl{prodotto} dovrà raggiungere la sufficienza in tutte le metriche della sezione \hyperref[sec:Misure]{3.9.3}. Il \gl{back end} dovrà raggiungere la sufficienza in affidabilità ed efficienza in ogni dispositivo testato. Il \gl{front end} dovrà raggiungere la sufficienza in usabilità in ogni dispositivo testato.
			\item \textbf{Strumenti:}\ si vedano le \NPdoc.
		\end{itemize}
	\subsection{Altre qualità}
		Saranno inoltre garantite le seguenti caratteristiche:
		\begin{itemize}
			\item \textbf{incapsulamento:}\ per aumentare la manutenibilità e il riuso di codice verrà applicata la tecnica dell'incapsulamento. Questo implica che dove sarà possibile verrà favorito l'uso delle interfacce.
			\item \textbf{coesione:}\ per rendere il \gl{prodotto} più manutenibile, più semplice e con un indice di dipendenze minore verrà usata la tecnica della coesione. Questo significa che le funzionalità con il medesimo scopo risiederanno nello stesso componente.
		\end{itemize}


% !TEX encoding = UTF-8 Unicode
% !TEX TS-program = pdflatex
% !TEX spellcheck = it-IT

\section{Visione generale della strategia}

\subsection{Procedure di controllo di qualità di processo}
Per garantire la qualità dei processi e quindi un loro miglioramento continuo verrà usato il principio \gl{PDCA}.

\label{sec:Misure}

% !TEX encoding = UTF-8 Unicode
% !TEX TS-program = pdflatex
% !TEX spellcheck = it-IT

\section{Gestione amministrativa della revisione}
	\label{sec:4}
	\subsection{Comunicazione e risoluzione delle anomalie}
		Un'anomalia corrisponde a:
		\begin{itemize}
			\item\ un errore ortografico;
			\item\ la violazione delle norme tipografiche del documento;
			\item\ l'uscita dal range di accettazione degli indici di misurazione, descritti nella \hyperref[sec:3.9]{sottosezione 3.9};
			\item\ un'incongruenza del prodotto rispetto a determinate funzionalità. Tali funzionalità sono state indicate nel documento \ARdoc;
			\item\ un'incongruenza del codice con il design prodotto.
		\end{itemize}
		Nel caso in cui un \VER\ individui un'anomalia, dovrà aprire un \gl{ticket} seguendo la procedura indicata nelle \NPdoc.

\end{document}