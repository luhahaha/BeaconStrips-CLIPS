% !TEX encoding = UTF-8 Unicode
% !TEX TS-program = pdflatex
% !TEX spellcheck = it-IT
\documentclass[a4paper,titlepage]{article}

\usepackage[utf8x]{inputenc}

\makeatletter
\def\input@path{{../../../template/}}
\makeatother
\usepackage{Riferimenti,Comandi,Stile} %Carico il template

%Nome del documento
\def\NOME{Piano di Qualifica}
%Versione del documento
\def\VERSIONE{1.00}
%Data del documento
\def\DATA{\today}
%Redattore/i del documento. Va scritto prima il nome, poi il cognome
\def\REDATTORE{Andrea Grendene \\ & Viviana Alessio}
%Verificatore del documento
\def\VERIFICATORE{Tommaso Panozzo}
%Responsabile del progetto
\def\RESPONSABILE{Viviana Alessio}
%Uso del documento
\def\USO{Esterno}
%Destinatari del documento
\def\DESTINATARI{\COMMITTENTE \\ &
				 \CARDIN\ \\ &
				 \PROPONENTE}
%Sommario del documento
\def\SOMMARIO{Questo documento ha lo scopo di fissare le norme necessarie ad assicurare i requisiti qualitativi del progetto \PROGETTO, regolamentando le operazioni di pianificazione e di verifica attuate per rispettare tali norme.}

\begin{document}

\maketitle

	% le ultime modifiche vanno messe in testa alla tabella
\begin{diario}
	\modifica{Andrea Grendene}{\AN}{Stesura della Definizione degli obiettivi di qualità}{2016/03/21}{0.03}
	\modifica{Andrea Grendene}{\AN}{Stesura dell'Introduzione}{2016/03/16}{0.02}
	\modifica{Andrea Grendene}{\AN}{Impostazione della struttura e dei dettagli del documento}{2016/03/15}{0.01}
\end{diario}

\newpage
\tableofcontents
\newpage

\section{Introduzione}
	\subsection{Scopo del documento} 
	Questo documento ha lo scopo di spiegare dettagliatamente le strategie secondo cui il gruppo \AUTORE{} intende condurre il \gl{progetto} didattico. 
	\subsection{Scopo del \gl{prodotto}}
	\SCOPO
	\subsection{Glossario}
	\GLOSSARIO
	\subsection{Riferimenti}
		\subsubsection{Normativi}
			\begin{itemize}
				\item \textbf{Capitolato d'appalto C2 - CLIPS:} Communication \& Localisation with Indoor Positioning Systems. \\
				\url{http://www.math.unipd.it/~tullio/IS-1/2015/Progetto/C2.pdf}
				\item \textbf{Vincoli e dettagli tecnico-economici} \\
				\url{http://www.math.unipd.it/~tullio/IS-1/2015/Dispense/PD01.pdf}
				\item \textbf{Norme di Progetto} \\ \NPdoc
				\item \textbf{Regolamento di Progetto} \\
				\url{http://www.math.unipd.it/~tullio/IS-1/2015/Progetto/}
				\item \textbf{Regolamento organigramma} \\
				\url{http://www.math.unipd.it/~tullio/IS-1/2015/Progetto/PD01b.html}
			\end{itemize}	
			
		\subsubsection{Informativi}
			\begin{itemize}
				\item \textbf{Software Engineering (10th edition}) \\
				Ian Sommerville \\
				Pearson Education | Addison-Wesley
				\item \textbf{Guide to the Software Engineering Body of Knowledge}
				IEEE Computer Society. Software Engineering Coordinating Committee
				\item \textbf{Slides del \COMMITTENTE} \\ riguardo i  \href{http://www.math.unipd.it/~tullio/IS-1/2015/Dispense/L02.pdf}{processi \gl{software}}, il \href{http://www.math.unipd.it/~tullio/IS-1/2015/Dispense/L03.pdf}{ciclo di vita del \gl{software}} e \href{http://www.math.unipd.it/~tullio/IS-1/2015/Dispense/L04.pdf}{la gestione di \gl{progetto}}	
			\end{itemize}
	\subsection{Modello di ciclo di vita scelto}
	È stato scelto come ciclo di vita il modello \gl{incrementale}. Le motivazioni che ci hanno spinto verso questa direzione sono il modo in cui è strutturato il \gl{progetto} didattico e la quasi totale inesperienza dei componenti del gruppo nello sviluppare progetti \gl{software} di grandi dimensioni. Di seguito una lista di caratteristiche del metodo \gl{incrementale}:
	\begin{itemize}
		\item si può produrre valore ad ogni incremento;
		\item ogni incremento riduce il rischio di fallimento;
		\item prevede rilasci multipli;
		\item i requisiti utente sono classificati e trattati in base alla loro importanza strategica. I requisiti più importanti sono già stabili all'inizio dello sviluppo del \gl{progetto};
		\item l'analisi dei requisiti e la progettazione architetturale non vengono ripetute;
		\item prima si pensa allo sviluppo dei requisiti essenziali, poi a quelli desiderabili;
		\item Sono presenti delle iterazioni del tipo Prototipo $\rightarrow$ Validazione $\rightarrow$ Prototipo $\rightarrow$ Validazione $\rightarrow$ ecc..
	\end{itemize}
	\subsection{Scadenze}
	Il gruppo Beacon Strips ha deciso di rispettare le seguenti scadenze:
	\begin{itemize} 
		\item \textbf{Revisione dei Requisiti}: 2016-04-18
		\item \textbf{Revisione di Progettazione}: 2016-06-17
		\item \textbf{Revisione di Qualifica}: 2016-08-24
		\item \textbf{Revisione di Accettazione}: 2016-09-12
	\end{itemize}
	In base a queste scadenze e a fronte dell'analisi dei rischi verranno decise le fasi in cui suddividere il lavoro di sviluppo del \gl{progetto}.
	\subsubsection{Scelta Revisione di Progettazione}
	Si è deciso di affrontare la RP$_{\mbox{\textit{min}}}$. Il gruppo si impegna quindi per il 2016-06-17 di presentare nel documento ``Specifica Tecnica'' la progettazione ad alto livello del \gl{prodotto}.
	
% fa già \newpage in automatico

% !TEX encoding = UTF-8 Unicode
% !TEX TS-program = pdflatex
% !TEX spellcheck = it-IT

\section{Parte 2}


\section{Visione generale della strategia}

\subsection{Procedure di controllo di qualit� di processo}
Per garantire la qualit� dei processi e quindi un loro miglioramento continuo verr� usato il principio gl{PDCA}.

\end{document}