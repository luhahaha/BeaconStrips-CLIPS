% !TEX encoding = UTF-8 Unicode
% !TEX TS-program = pdflatex
% !TEX spellcheck = it-IT
\documentclass[a4paper,titlepage]{article}

\makeatletter
\def\input@path{{../../../template/}}
\makeatother
\usepackage{Riferimenti,Comandi,Stile} %Carico il template

%Nome del documento
\def\NOME{Piano di Qualifica}
%Versione del documento
\def\VERSIONE{1.00}
%Data del documento
\def\DATA{\today}
%Redattore/i del documento. Va scritto prima il nome, poi il cognome
\def\REDATTORE{Andrea Grendene \\ & Viviana Alessio}
%Verificatore del documento
\def\VERIFICATORE{Tommaso Panozzo}
%Responsabile del progetto
\def\RESPONSABILE{Viviana Alessio}
%Uso del documento
\def\USO{Esterno}
%Destinatari del documento
\def\DESTINATARI{\COMMITTENTE \\ &
				 \CARDIN\ \\ &
				 \PROPONENTE}
%Sommario del documento
\def\SOMMARIO{Questo documento ha lo scopo di fissare le norme necessarie ad assicurare i requisiti qualitativi del progetto \PROGETTO, regolamentando le operazioni di pianificazione e di verifica attuate per rispettare tali norme.}

% se proprio ti seervono ok. Ma andrebbero messi nel file comandi.sty
\newcommand{\evidenzia}[1]{\emph{‘‘#1’’}} %Evidenzia i riferimenti importanti (per ora sono riuscito ad usare altro quando serviva)
\newcommand{\importante}[1]{\textbf{#1}} %Evidenzia i titoletti per i Riferimenti

\begin{document}

\maketitle

	% le ultime modifiche vanno messe in testa alla tabella
\begin{diario}
	\modifica{Andrea Grendene}{\AN}{Stesura dell'Introduzione}{2016/03/16}{0.02}
	\modifica{Andrea Grendene}{\AN}{Impostazione della struttura e dei dettagli del documento}{2016/03/15}{0.01}
\end{diario}

\newpage
\tableofcontents
\newpage

\section{Introduzione}
	\subsection{Scopo del documento} 
	Questo documento ha lo scopo di spiegare dettagliatamente le strategie secondo cui il gruppo \gl{Beacon Strips} intende condurre il progetto didattico.
	\subsection{Riferimenti}
	\subsection{Ciclo di vita}
	\subsection{Scadenze}
	
% fa già \newpage in automatico

\include{Parte_2}

\end{document}