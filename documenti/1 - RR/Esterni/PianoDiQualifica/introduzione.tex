% !TEX encoding = UTF-8 Unicode
% !TEX TS-program = pdflatex
% !TEX spellcheck = it-IT

\section{Introduzione}
	\subsection{Scopo del documento}
	Questo documento fissa le regole decise dal \gl{team} per garantire una buona qualità del \gl{prodotto} e dei processi attuati durante l'intera durata del \gl{progetto}. Per assicurarne il rispetto verranno svolte costantemente delle attività di verifica dei processi. In questo modo verrà garantita la qualità del \gl{prodotto} e si minimizzano le risorse impiegate.
	
	\subsection{Scopo del \gl{prodotto}}
	% il comando deve ancora essere utimato
	\SCOPO
	
	\subsection{Glossario}
	% il comando deve ancora essere utimato
	\GLOSSARIO
	
	\subsection{Riferimenti} 
	\subsubsection{Normativi}
	\begin{itemize}
		\item \textbf{Norme di Progetto:}\ \NPdoc;
		\item \textbf{Capitolato d'appalto C2:}\ CLIPS: Communication \& Localisation with Indoor Positioning Systems, reperibile all'indirizzo \url{http://www.math.unipd.it/~tullio/IS-1/2015/Progetto/C2.pdf}.
	\end{itemize}
	
	\subsubsection{Informativi}
	\begin{itemize}
		\item \textbf{Analisi dei Requisiti:}\ \ARdoc;
		\item \textbf{Piano di \gl{progetto}:}\ \PPdoc;
		\item \textbf{Slide dell’insegnamento Ingegneria del Software modulo A:}\ \url{http://www.math.unipd.it/~tullio/IS-1/2015/};
		\item \textbf{Capacity Maturity Model (\gl{CMM}):}\ \url{http://en.wikipedia.org/wiki/Capability_Maturity_Model};
		\item \textbf{Capacity Maturity Model Integration (CMMI):}\ \url{http://en.wikipedia.org/wiki/Capability_Maturity_Model_Integration};
		\item \textbf{\gl{ISO} 9001:}\ \url{http://en.wikipedia.org/wiki/ISO_9001};
		\item \textbf{\gl{ISO}/IEC 9126:2001:}\ \url{http://en.wikipedia.org/wiki/\gl{ISO}/IEC_9126};
		\item \textbf{\gl{ISO}/IEC 15504:}\ \url{http://en.wikipedia.org/wiki/\gl{ISO}/IEC_15504};
		\item \textbf{Indice \gl{Gulpease}:}\ \url{http://it.wikipedia.org/wiki/Indice_Gulpease}.
	\end{itemize}
	


			
			