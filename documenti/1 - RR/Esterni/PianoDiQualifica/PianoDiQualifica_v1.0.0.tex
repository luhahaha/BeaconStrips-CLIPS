% !TEX encoding = UTF-8 Unicode
% !TEX TS-program = pdflatex
% !TEX spellcheck = it-IT
\documentclass[a4paper,titlepage]{article}

\usepackage[utf8x]{inputenc}

\makeatletter
\def\input@path{{../../../template/}}
\makeatother
\usepackage{Riferimenti,Comandi,Stile} %Carico il template

%Nome del documento
\def\NOME{Piano di Qualifica}
%Versione del documento
\def\VERSIONE{1.0.0}
%Data del documento
\def\DATA{\today}
%Redattore/i del documento. Va scritto prima il nome, poi il cognome
\def\REDATTORE{Andrea Grendene \\ & Viviana Alessio}
%Verificatore del documento
\def\VERIFICATORE{Tommaso Panozzo}
%Responsabile del progetto
\def\RESPONSABILE{Viviana Alessio}
%Uso del documento
\def\USO{Esterno}
%Destinatari del documento
\def\DESTINATARI{\COMMITTENTE \\ &
				 \CARDIN\ \\ &
				 \PROPONENTE}
%Sommario del documento
\def\SOMMARIO{Questo documento ha lo scopo di fissare le norme necessarie ad assicurare i requisiti qualitativi del progetto \PROGETTO, regolamentando le operazioni di pianificazione e di verifica attuate per rispettare tali norme.}

\begin{document}

\maketitle

	% le ultime modifiche vanno messe in testa alla tabella
\begin{diario}
	\modifica{Tommaso Panozzo}{\VER}{Terminata la verifica delle modifiche}{2016-04-05}{0.8.0}
	\modifica{Andrea Grendene}{\PRJ}{Aggiunti i valori dell'indice Gulpease dei documenti e il loro esito}{2016-04-05}{0.7.1}
	\modifica{Andrea Grendene}{\PRJ}{Applicate la modifiche proposte dal \VER}{2016-04-05}{0.7.0}
	\modifica{Tommaso Panozzo}{\VER}{Terminata la verifica del documento}{2016-04-04}{0.6.0}
	\modifica{Andrea Grendene}{\PRJ}{Terminata stesura della struttura del Resoconto dell'attività di verifica (senza i valori dell'indice Gulpease nella tabella)}{2016-03-25}{0.5.0}
	\modifica{Andrea Grendene}{\PRJ}{Terminata stesura della Gestione amministrativa della revisione}{2016-03-25}{0.4.0}
	\modifica{Andrea Grendene}{\PRJ}{Terminata stesura della Visione generale della strategia}{2016-03-25}{0.3.0}
	\modifica{Andrea Grendene}{\PRJ}{Terminata stesura della Definizione degli obiettivi di qualità}{2016-03-21}{0.2.0}
	\modifica{Andrea Grendene}{\PRJ}{Terminata stesura dell'Introduzione}{2016-03-20}{0.1.0}
	\modifica{Andrea Grendene}{\PRJ}{Stesura della prima parte dei Riferimenti}{2016-03-19}{0.0.3}
	\modifica{Andrea Grendene}{\PRJ}{Stesura dell'introduzione fino ai Riferimenti}{2016-03-18}{0.0.2}
	\modifica{Andrea Grendene}{\PRJ}{Impostazione della struttura e dei dettagli del documento}{2016-03-15}{0.0.1}
\end{diario}

\newpage
\tableofcontents
\newpage

\section{Introduzione}
	\subsection{Scopo del documento} 
	Questo documento ha lo scopo di spiegare dettagliatamente le strategie secondo cui il gruppo \AUTORE{} intende condurre il \gl{progetto} didattico. 
	\subsection{Scopo del \gl{prodotto}}
	\SCOPO
	\subsection{Glossario}
	\GLOSSARIO
	\subsection{Riferimenti}
		\subsubsection{Normativi}
			\begin{itemize}
				\item \textbf{Capitolato d'appalto C2 - CLIPS:} Communication \& Localisation with Indoor Positioning Systems. \\
				\url{http://www.math.unipd.it/~tullio/IS-1/2015/Progetto/C2.pdf}
				\item \textbf{Vincoli e dettagli tecnico-economici} \\
				\url{http://www.math.unipd.it/~tullio/IS-1/2015/Dispense/PD01.pdf}
				\item \textbf{Norme di Progetto} \\ \NPdoc
				\item \textbf{Regolamento di Progetto} \\
				\url{http://www.math.unipd.it/~tullio/IS-1/2015/Progetto/}
				\item \textbf{Regolamento organigramma} \\
				\url{http://www.math.unipd.it/~tullio/IS-1/2015/Progetto/PD01b.html}
			\end{itemize}	
			
		\subsubsection{Informativi}
			\begin{itemize}
				\item \textbf{Software Engineering (10th edition}) \\
				Ian Sommerville \\
				Pearson Education | Addison-Wesley
				\item \textbf{Guide to the Software Engineering Body of Knowledge}
				IEEE Computer Society. Software Engineering Coordinating Committee
				\item \textbf{Slides del \COMMITTENTE} \\ riguardo i  \href{http://www.math.unipd.it/~tullio/IS-1/2015/Dispense/L02.pdf}{processi \gl{software}}, il \href{http://www.math.unipd.it/~tullio/IS-1/2015/Dispense/L03.pdf}{ciclo di vita del \gl{software}} e \href{http://www.math.unipd.it/~tullio/IS-1/2015/Dispense/L04.pdf}{la gestione di \gl{progetto}}	
			\end{itemize}
	\subsection{Modello di ciclo di vita scelto}
	È stato scelto come ciclo di vita il modello \gl{incrementale}. Le motivazioni che ci hanno spinto verso questa direzione sono il modo in cui è strutturato il \gl{progetto} didattico e la quasi totale inesperienza dei componenti del gruppo nello sviluppare progetti \gl{software} di grandi dimensioni. Di seguito una lista di caratteristiche del metodo \gl{incrementale}:
	\begin{itemize}
		\item si può produrre valore ad ogni incremento;
		\item ogni incremento riduce il rischio di fallimento;
		\item prevede rilasci multipli;
		\item i requisiti utente sono classificati e trattati in base alla loro importanza strategica. I requisiti più importanti sono già stabili all'inizio dello sviluppo del \gl{progetto};
		\item l'analisi dei requisiti e la progettazione architetturale non vengono ripetute;
		\item prima si pensa allo sviluppo dei requisiti essenziali, poi a quelli desiderabili;
		\item Sono presenti delle iterazioni del tipo Prototipo $\rightarrow$ Validazione $\rightarrow$ Prototipo $\rightarrow$ Validazione $\rightarrow$ ecc..
	\end{itemize}
	\subsection{Scadenze}
	Il gruppo Beacon Strips ha deciso di rispettare le seguenti scadenze:
	\begin{itemize} 
		\item \textbf{Revisione dei Requisiti}: 2016-04-18
		\item \textbf{Revisione di Progettazione}: 2016-06-17
		\item \textbf{Revisione di Qualifica}: 2016-08-24
		\item \textbf{Revisione di Accettazione}: 2016-09-12
	\end{itemize}
	In base a queste scadenze e a fronte dell'analisi dei rischi verranno decise le fasi in cui suddividere il lavoro di sviluppo del \gl{progetto}.
	\subsubsection{Scelta Revisione di Progettazione}
	Si è deciso di affrontare la RP$_{\mbox{\textit{min}}}$. Il gruppo si impegna quindi per il 2016-06-17 di presentare nel documento ``Specifica Tecnica'' la progettazione ad alto livello del \gl{prodotto}.
	
% fa già \newpage in automatico

% !TEX encoding = UTF-8 Unicode
% !TEX TS-program = pdflatex
% !TEX spellcheck = it-IT

\section{Definizione obiettivi di qualità}
\label{sec:Definizione}
	Basandosi sullo standard \iso{ISO/IEC 9126} il \gl{team} si impegna a garantire al prodotto \PROGETTO\ le seguenti qualità:
	\subsection{Funzionalità}
		Il \gl{prodotto} deve garantire tutti i requisiti stabiliti nel documento \ARdoc\ e implementarli nel modo più completo ed economico possibile.
		\begin{itemize}
			\item \textbf{Misura:}\ l'unità di misura adottata sarà la quantità di requisiti presenti e funzionanti nel prodotto.
			\item \textbf{Metrica:}\ la sufficienza è stabilita nel soddisfacimento dei requisiti obbligatori.
			\item \textbf{Strumenti:}\ ogni requisito dovrà superare tutti i test previsti in modo da garantire il loro funzionamento. Per avere informazioni dettagliate sugli strumenti si veda il documento \NPdoc. 
		\end{itemize}
	\subsection{Affidabilità}
		Il \gl{prodotto} deve essere il più robusto possibile e facilmente ripristinabile in caso di errori.
		\begin{itemize}
			\item \textbf{Misura:}\ l'unità di misura adottata sarà il numero di esecuzioni che hanno successo.
			\item \textbf{Metrica:}\ le esecuzioni dovranno coinvolgere tutte le parti possibili del \gl{prodotto} ed esaminare il maggior numero possibile di casi. Non si può definire una soglia di sufficienza perché è impossibile determinare ogni situazione d'utilizzo possibile.
			\item \textbf{Strumenti:}\ da definire.
		\end{itemize}
	\subsection{Usabilità}
		Il \gl{prodotto} deve essere di facile utilizzo per la classe di utenti designata. Inoltre deve soddisfare ogni necessità dell'utilizzatore.
		\begin{itemize}
			\item \textbf{Misura:}\ verrà usata come unità di misura la valutazione soggettiva del prodotto. Questo perché non esiste uno strumento adatto ad eseguire una misurazione oggettiva dell'usabilità.
			\item \textbf{Metrica:}\ purtroppo non esiste una metrica adeguata che possa determinare una soglia di sufficienza. Il \gl{team} si impegna comunque a fornire la miglior qualità d'uso possibile. Per ottenere un risultato più soddisfacente verranno consultate delle persone esterne al gruppo per verificare l'usabilità del \gl{prodotto}. 
			\item \textbf{Strumenti:}\ si vedano le \NPdoc.
		\end{itemize}
	\subsection{Efficienza}
		Il \gl{prodotto} deve fornire tutte le funzionalità nel minore tempo possibile e minimizzando l'utilizzo di risorse.
		\begin{itemize}
			\item \textbf{Misura:}\ il tempo di latenza per ottenere una risposta in ogni pagina del \gl{prodotto}.
			\item \textbf{Metrica:}\ la sufficienza è raggiunta con un tempo di latenza minore di 5 secondi, ponendo che non ci siano problemi di connessione.
			\item \textbf{Strumenti:}\ si vedano le \NPdoc.
		\end{itemize}
	\subsection{Manutenibilità}
		Il \gl{prodotto} dev'essere comprensibile ed estensibile in modo facile e verificabile.
		\begin{itemize}
			\item \textbf{Misura:}\ l'unità di misura utilizzata saranno le metriche sul codice stabilite nella sezione \hyperref[sec:Misure]{3.9.3}.
			\item \textbf{Metrica:}\ il \gl{prodotto} deve raggiungere la sufficienza in tutte le metriche descritte nella sezione \hyperref[sec:Misure]{3.9.3}.
			\item \textbf{Strumenti:}\ si vedano le \NPdoc.
		\end{itemize}
%La sezione 2.6 è stata un po' personalizzata rispetto al documento di riferimento (quello dei ProTech), di conseguenza è più probabile che ci siano errori, questa nota è destinata soprattutto al verificatore.
	\subsection{Portabilità}
		Il \gl{prodotto} deve essere il più portabile possibile. Il \gl{front end} dev'essere utilizzabile da più dispositivi possibili. Il \gl{back end} deve poter girare su ogni sistema operativo \gl{Android} a partire dalla versione .
		\begin{itemize}
			\item \textbf{Misura:}\ il \gl{back end} dev'essere affidabile per ogni versione di \gl{Android} a partire dalla . Ogni dispositivo con questa versione del sistema operativo o una maggiore deve avere un \gl{front end} usabile, a prescindere dalle specifiche hardware o dalla risoluzione dello schermo.
			\item \textbf{Metrica:}\ il \gl{prodotto} dovrà raggiungere la sufficienza in tutte le metriche della sezione \hyperref[sec:Misure]{3.9.3}. Il \gl{back end} dovrà raggiungere la sufficienza in affidabilità ed efficienza in ogni dispositivo testato. Il \gl{front end} dovrà raggiungere la sufficienza in usabilità in ogni dispositivo testato.
			\item \textbf{Strumenti:}\ si vedano le \NPdoc.
		\end{itemize}
	\subsection{Altre qualità}
		Saranno inoltre garantite le seguenti caratteristiche:
		\begin{itemize}
			\item \textbf{incapsulamento:}\ per aumentare la manutenibilità e il riuso di codice verrà applicata la tecnica dell'incapsulamento. Questo implica che dove sarà possibile verrà favorito l'uso delle interfacce.
			\item \textbf{coesione:}\ per rendere il \gl{prodotto} più manutenibile, più semplice e con un indice di dipendenze minore verrà usata la tecnica della coesione. Questo significa che le funzionalità con il medesimo scopo risiederanno nello stesso componente.
		\end{itemize}


% !TEX encoding = UTF-8 Unicode
% !TEX TS-program = pdflatex
% !TEX spellcheck = it-IT

\section{Visione generale della strategia}

\subsection{Procedure di controllo di qualità di processo}
Per garantire la qualità dei processi e quindi un loro miglioramento continuo verrà usato il principio \gl{PDCA}.

\label{sec:Misure}

% !TEX encoding = UTF-8 Unicode
% !TEX TS-program = pdflatex
% !TEX spellcheck = it-IT

\section{Gestione amministrativa della revisione}
	\label{sec:4}
	\subsection{Comunicazione e risoluzione delle anomalie}
		Un'anomalia corrisponde a:
		\begin{itemize}
			\item\ un errore ortografico;
			\item\ la violazione delle norme tipografiche del documento;
			\item\ l'uscita dal range di accettazione degli indici di misurazione, descritti nella \hyperref[sec:3.9]{sottosezione 3.9};
			\item\ un'incongruenza del prodotto rispetto a determinate funzionalità. Tali funzionalità sono state indicate nel documento \ARdoc;
			\item\ un'incongruenza del codice con il design prodotto.
		\end{itemize}
		Nel caso in cui un \VER\ individui un'anomalia, dovrà aprire un \gl{ticket} seguendo la procedura indicata nelle \NPdoc.

\appendix
\section{Resoconto dell’attività di verifica}
\label{resocontoDellAttivitaDiVerifica}
	\subsection{Periodo di Analisi e Management}
	\label{periodoDiAnalisiEManagement}
		\subsubsection{Processi}
		\label{processiAM}
			Sono riportati ora i valori di Schedule Variance e Budget Variance per le attività del Periodo di Analisi e Management.
			\begin{tabella}{!{\VRule}l!{\VRule}l!{\VRule}l!{\VRule}}
				\intestazionethreecol{Attività}{Schedule Variance}{Budget Variance}
				\ARdoc & \euro\ -10 & \euro\ +15 \\
				\Gldoc & \euro\ 0 & \euro\ 0 \\
				\NPdoc & \euro\ -5 & \euro\ +10 \\
				\PPdoc & \euro\ 0 & \euro\ +20 \\
				\PQdoc & \euro\ +5 & \euro\ -10 \\
				\SFdoc & \euro\ -10 & \euro\ 0 \\
				
				\hiderowcolors
				\caption{Esiti verifica sui processi - Periodo di Analisi e Management}
			\end{tabella}
			In totale sono stati registrati:
			\begin{itemize}
				\item \textbf{Schedule Variance:}\ \euro\ -20;
				\item \textbf{Budget Variance:}\ \euro\ +35.
			\end{itemize}
			Dai valori ottenuti si nota subito che in alcuni casi è stata prevista qualche ora di attività in più rispetto al necessario. Questo ha portato ad avere una Budget Variance positiva, mentre la causa di questo fatto è da ricercare nell'inesperienza del gruppo, che ha portato ad una valutazione leggermente errata del carico di lavoro necessario. Sempre a causa della poca esperienza del gruppo la Schedule Variance è risultata negativa, perché alcune attività si sono concluse leggermente in ritardo rispetto alle aspettative. La causa di questo fatto è da ricercare nell'organizzazione da parte dei membri del gruppo dei propri impegni, da cui sono derivati leggeri ritardi che si sono sommati. Il risultato comunque rientra nel limite ottimale, che sarebbe di \euro\ -144.
			%Nota sull'interpretazione della giustificazione sopra: quello che volevo indicare sopra per la Budget Variance è che inizialmente ci sono stati leggeri ritardi, che comunque poi sono stati compensati alla fine (vista anche la consegna anticipata rispetto al previsto)
		\subsubsection{Documenti}
		\label{documentiAM}
			Sono riportati qui i valori dell'indice Gulpease per ogni documento presente durante l'attività di analisi nel Periodo di Analisi e Management. Un documento è considerato valido soltanto se rispetta le metriche descritte secondo la sezione \hyperref[indiceGulpease]{Indice Gulpease}.
			\begin{tabella}{!{\VRule}l!{\VRule}c!{\VRule}c!{\VRule}}
				\intestazionethreecol{Documento}{Valore}{Esito}
				\ARdoc & 66 & Superato\\
				\Gldoc & 53 & Superato\\
				\NPdoc & 56 & Superato\\
				\PPdoc & 53 & Superato\\
				\PQdoc & 60 & Superato\\
				\SFdoc & 61 & Superato\\
				
				\hiderowcolors
				\caption{Esiti verifica documenti - Periodo di Analisi e Management}
			\end{tabella}
	\subsection{Periodo di Analisi di Dettaglio}
	\label{periodoDiAnalisiDiDettaglio}
		\subsubsection{Processi}
		\label{processiAD}
			Sono riportati ora i valori di Schedule Variance e Budget Variance per le attività del Periodo di Analisi di Dettaglio.
				\begin{tabella}{!{\VRule}l!{\VRule}c!{\VRule}c!{\VRule}}
				\intestazionethreecol{Attività}{Schedule Variance}{Budget Variance}
				\ARdoc & \euro\ -20 & \euro\ -25 \\
				\Gldoc & \euro\ 0 & \euro\ 0 \\
				\NPdoc & \euro\ +5 & \euro\ +5 \\
				\PPdoc & \euro\ -10 & \euro\ -10 \\
				\PQdoc & \euro\ 0 & \euro\ -5 \\
				\SFdoc & \euro\ 0 & \euro\ 0 \\
				
				\hiderowcolors
				\caption{Esiti verifica sui processi - Periodo di Analisi di Dettaglio}
			\end{tabella}
			In totale sono stati registrati:
			\begin{itemize}
				\item \textbf{Schedule Variance:}\ \euro\ -25;
				\item \textbf{Budget Variance:}\ \euro\ -35.
			\end{itemize}
			Dai valori ottenuti si nota subito che il lavoro richiesto per eseguire le attività è stato maggiore di quello pianificato. Questo perché le modifiche necessarie si sono rivelate essere più del previsto, soprattutto per quanto riguarda l'Analisi dei Requisiti. Di conseguenza la Budget Variance e la Schedule Variance sono risultate negative, ma comunque al di sotto del limite ottimale di \euro\ -65.
		\subsubsection{Documenti}
		\label{documentiAD}
			Sono riportati qui i valori dell'indice Gulpease per ogni documento presente durante l'attività di analisi nel Periodo di Analisi di Dettaglio. Un documento è considerato valido soltanto se rispetta le metriche descritte secondo la sezione \hyperref[indiceGulpease]{Indice Gulpease}.
			\begin{tabella}{!{\VRule}l!{\VRule}c!{\VRule}c!{\VRule}}
				\intestazionethreecol{Documento}{Valore}{Esito}
				\ARdoc & 85 & Superato\\
				\Gldoc & 54 & Superato\\
				\NPdoc & 51 & Superato\\
				\PPdoc & 62 & Superato\\
				\PQdoc & 61 & Superato\\
				\SFdoc & 61 & Superato\\
				
				\hiderowcolors
				\caption{Esiti verifica documenti - Periodo di Analisi di Dettaglio}
			\end{tabella}
	\subsection{Periodo di Progettazione Architetturale}
	\label{periodoDiProgettazioneArchitetturale}
		\subsubsection{Processi}
		\label{processiPA}
			Sono riportati ora i valori di Schedule Variance e Budget Variance per le attività del Periodo di Progettazione Architetturale.
				\begin{tabella}{!{\VRule}l!{\VRule}c!{\VRule}c!{\VRule}}
				\intestazionethreecol{Attività}{Schedule Variance}{Budget Variance}
				\ARdoc & \euro\ +20 & \euro\ +20 \\
				\Gldoc & \euro\ +5 & \euro\ 0 \\
				\NPdoc & \euro\ +10 & \euro\ +15 \\
				\PPdoc & \euro\ -5 & \euro\ 0 \\
				\PQdoc & \euro\ -5 & \euro\ 0 \\
				\STdoc & \euro\ -20 & \euro\ -50 \\
				
				\hiderowcolors
				\caption{Esiti verifica sui processi - Periodo di Progettazione Architetturale}
			\end{tabella}
			In totale sono stati registrati:
			\begin{itemize}
				\item \textbf{Schedule Variance:}\ \euro\ +5;
				\item \textbf{Budget Variance:}\ \euro\ -15.
			\end{itemize}
			Dai valori ottenuti si nota subito che le attività si sono concluse al massimo leggermente dopo i tempi previsti, ad eccezione della Specifica Tecnica che ha subito notevoli ritardi.\\
			Questo ha portato ad avere una Schedule Variance leggermente positiva perché il tempo risparmiato ha compensato quello perso per la Specifica Tecnica, in particolare sono risultate importanti le ore dedicate nella fase precedente perchè ha permesso di avere dei documenti praticamente già terminati.\\
			Al contrario la budget Variance è risultata negativa ma comunque al di sotto del limite ottimale di \euro\ -204. Questo perché le ore di lavoro previste per tutti i documenti sono risultate sufficienti o addirittura eccessive, ma la Specifica Tecnica ha richiesto molte più ore del previsto.\\
			I dati negativi relativi a questo documento sono da ricercare sempre nell'inesperienza del gruppo, che ha portato a sopravvalutare non di poco i tempi e le ore richiesti per portarlo a termine, in particolare per quanto riguarda la progettazione delle classi.
		\subsubsection{Documenti}
		\label{documentiPA}
			Sono riportati qui i valori dell'indice Gulpease per ogni documento presente durante l'attività di analisi nel Periodo di Progettazione Architetturale. Un documento è considerato valido soltanto se rispetta le metriche descritte secondo la sezione \hyperref[indiceGulpease]{Indice Gulpease}.
			\begin{tabella}{!{\VRule}l!{\VRule}c!{\VRule}c!{\VRule}}
				\intestazionethreecol{Documento}{Valore}{Esito}
				\ARdoc & 84  & Superato\\
				\Gldoc & 54 & Superato\\
				\NPdoc & 71 & Superato\\
				\PPdoc & 51 & Superato\\
				\PQdoc & 61 & Superato\\
				\STdoc & 86 & Superato\\
				
				\hiderowcolors
				\caption{Esiti verifica documenti - Periodo di Progettazione Architetturale}
			\end{tabella}
		\subsection{Periodo di Progettazione di Dettaglio e Codifica}
			\label{periodoDiProgettazioneDiDettaglioECodifica}
				\subsubsection{Processi}
				\label{processiPDC}
					Sono riportati ora i valori di Schedule Variance e Budget Variance per le attività del Periodo di Progettazione di Dettaglio e Codifica.
						\begin{tabella}{!{\VRule}l!{\VRule}c!{\VRule}c!{\VRule}}
						\intestazionethreecol{Attività}{Schedule Variance}{Budget Variance}
						\ARdoc & \euro\ +10 & \euro\ +10 \\
						\DPdoc & \euro\ -20 & \euro\ -25 \\
						\Gldoc & \euro\ +5 & \euro\ +10 \\
						\MUdoc & \euro\ +20 & \euro\ +15 \\
						\NPdoc & \euro\ -5 & \euro\ 0 \\
						\PPdoc & \euro\ +10 & \euro\ +20 \\
						\PQdoc & \euro\ +5 & \euro\ 0 \\
						\STdoc & \euro\ -10 & \euro\ -15 \\
						
						\hiderowcolors
						\caption{Esiti verifica sui processi - Periodo di Progettazione di Dettaglio e Codifica}
					\end{tabella}
					In totale sono stati registrati:
					\begin{itemize}
						\item \textbf{Schedule Variance:}\ \euro\ +15;
						\item \textbf{Budget Variance:}\ \euro\ +10.
					\end{itemize}
					Dai valori ottenuti si nota subito che le attività sono state concluse in anticipo rispetto alle previsioni, e con una quantità di lavoro minore di quella prevista. I valori di Schedule Variance e di Budget Variance sono risultati quindi positivi, perché la correzione di alcuni documenti ha richiesto meno tempo del previsto. Questo risultato è da cercarsi anche nel maggior carico di lavoro richiesto dalla parte di codifica, che quindi ha portato il gruppo a sistemare presto i documenti. 
				\subsubsection{Documenti}
				\label{documentiPDC}
					Sono riportati qui i valori dell'indice Gulpease per ogni documento presente durante l'attività di analisi nel Periodo di Progettazione di Dettaglio e Codifica. Un documento è considerato valido soltanto se rispetta le metriche descritte secondo la sezione \hyperref[indiceGulpease]{Indice Gulpease}.
					\begin{tabella}{!{\VRule}l!{\VRule}c!{\VRule}c!{\VRule}}
						\intestazionethreecol{Documento}{Valore}{Esito}
						\ARdoc & 84  & Superato\\
						\DPdoc & & Superato\\
						\Gldoc & 54 & Superato\\
						\MUdoc & & Superato\\
						\NPdoc & 71 & Superato\\
						\PPdoc & 51 & Superato\\
						\PQdoc & 61 & Superato\\
						\STdoc & 86 & Superato\\
						
						\hiderowcolors
						\caption{Esiti verifica documenti - Periodo di Progettazione di Dettaglio e Codifica}
					\end{tabella}

\end{document}
