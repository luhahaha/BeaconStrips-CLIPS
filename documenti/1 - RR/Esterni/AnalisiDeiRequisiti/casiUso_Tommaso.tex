
\casoduso{XXX.1}{Utente gioca il percorso}
\desc l'utente decide di utilizzare l'app per il suo scopo principale.
\pre l'utente si trova in un ambiente abilitato, l'utente ha attivato i servizi di localizzazione e il bluetooth nel dispositivo, un percorso è stato selezionato e scaricato nel dispositivo.
\post l'utente ha completato il percorso.
\scen
  \begin{enumerate}
    \item \textbf{\UC{XXX.1.1}} l'utente si reca sul punto di partenza del percorso;
    \item \textbf{\UC{XXX.1.2}} l'utente svolge la prova relativa alla stazione in cui si trova;
    \item \textbf{\UC{XXX.1.3}} l'utente cercare la stazione successiva;
    \item \textbf{\UC{XXX.1.4}} l'utente trova la stazione;
    \item l'utente ripete i punti dal 2 al 4 fino alla conclusione del percorso;
    \item \textbf{\UC{XXX.1.5}} se l'utente è autenticato può ricevere qualche bonus/gratificazione.
  \end{enumerate}
\att l'utente.

\casoduso{XXX.1.1}{Utente si reca al punto di partenza}
\desc l'utente deve recarsi alla prima stazione del percorso.
\pre l'utente può compiere il percorso.
\post l'utente si trova alla prima base del percorso.
\att l'utente.

\casoduso{XXX.1.2}{Utente svolge una prova}
\desc l'utente svolge nell'App la prova corrispondente alla stazione del percorso in cui si trova.
\pre l'utente si trova nei pressi di una stazione del percorso e la stazione in cui si trova è la prossima (in ordine) in cui deve svolgere la prova.
\post l'utente supera la prova nella stazione.
\att l'utente.

\casoduso{XXX.1.3}{Utente supera la prova}
\desc l'utente ha superato la prova e viene visualizzata una schermata con alcune informazioni circa l'enigma appena risolto e che invita l'utente alla ricerca della prossima stazione.
\pre l'utente ha concluso la prova correttamente.
\post l'utente inizia la ricerca della prossima stazione.
\att l'utente.

\casoduso{XXX.1.4}{Utente cerca la prossima stazione}
\desc l'utente vaga per il luogo alla ricerca della prossima stazione
\pre la stazione cercata è la prossima secondo l'ordine del percorso oppure è la prima stazione se l'utente non ha ancora iniziato il percorso.
\post l'utente si trova presso una stazione.
\scen
\begin{enumerate}
  \item \textbf{\UC{XXX.1.4.1}} l'utente si avvicina alla prossima stazione;
  \item \textbf{\UC{XXX.1.4.2}} l'utente arriva alla prossima stazione.
\end{enumerate}
\att l'utente.

\casoduso{XXX.1.4.1}{Utente si avvicina alla prossima stazione}
\desc un utente alla ricerca della prossima stazione si avvicina ad essa.
\pre l'utente si avvicina alla prossima stazione.
\post l'utente vede sullo schermo l'avanzamento della sua posizione tra la base appena conclusa e la prossima.
\att l'utente.

\casoduso{XXX.1.4.2}{Utente si allontana dalla prossima stazione}
\desc un utente alla ricerca della prossima stazione si allontana da essa.
\pre l'utente si allontana dalla prossima stazione.
\post l'utente vede sullo schermo la regressione della sua posizione rispetto alla prossima stazione.
\att l'utente.

\casoduso{XXX.1.5}{Utente trova la stazione successiva}
\desc l'utente arriva alla stazione che stava cercando.
\pre l'utente era alla ricerca della stazione.
\post dipende dallo stato in cui si trova l'utente.
\att l'utente.

\casoduso{XXX.1.6}{Utente termina il percorso}
\desc l'utente completa il percorso selezionato e visualizza una schermata di congratulazioni: se l'utente è autenticato avrà la possibilità di caricare nel server i dati del percorso svolto, altrimenti avrà la possibilità di registrarsi, per caricare il percorso.
\pre l'utente ha completato tutte prove del percorso.
\post l'utente ha terminato il percorso
\scen
  \begin{enumerate}
    \item \textbf{\UC{XXX.1.6.1}} se l'utente è autenticato vengono registrati i progressi;
    \item \textbf{\UC{XXX.1.6.2}} se l'utente non è autenticato visualizza una schermata di congratulazioni.
  \end{enumerate}
\att l'utente

\casoduso{XXX.1.6.1}{Utente autenticato carica il percorso appena completato}
\desc l'utente carica nel server i dati circa il percorso appena concluso.
\pre l'utente è autenticato e ha completato tutte le prove del percorso.
% TODO: completare con il UC della visualizzazione dei percorsi conclusi
\post l'utente potrà visualizzare il percorso appena concluso nell'elenco di tutti i percorsi (cfr. \UC{YYYYYY})
\att l'utente autenticato

\casoduso{XXX.1.6.2}{Utente non autenticato termina il percorso}
%TODO: inserisci i link al UC di autenticazione
\desc se l'utente che termina il percorso non è autenticato gli viene data l'opportunità di farlo ora, seguendo il workflow dell'\UC{YYYYYY} di autenticazione.
\pre l'utente non autenticato completa il percorso.
\post l'utente è nel \UC{XXX.1.6} e può decidere se caricare il percorso appena concluso.
\scen
  \begin{itemize}
    \item l'utente si registra/autentica e può attuare il \UC{XXX.1.6.1};
    \item l'utente rifiuta di registrarsi.
  \end{itemize}
% TODO: inserire il corretto UC del workflow di autenticazione
\ext autenticazione utente (\UC{YYYYYYY}).
