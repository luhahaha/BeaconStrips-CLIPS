\section{Requisiti}
\label{sec:Requisiti}
Di seguito verranno elencati i requisiti individuati grazie all'analisi del \gl{capitolato}, la produzione dei casi d'uso, dagli incontri con il Proponente o a necessità interne. Verranno catalogati in tabelle contenenti il codice del requisito, la tipologia, la descrizione e la fonte.\\
I codici dei requisiti seguiranno la forma:
\begin{center}
	R[importanza][tipo][identificativo]
\end{center}
\begin{itemize}
	\item \textbf{Importanza}:indica se il requisito è:
	\begin{enumerate}
		\item \textbf{0}: obbligatorio;
		\item \textbf{1}: desiderabile;
		\item \textbf{2}: opzionale.
	\end{enumerate}
	\item \textbf{Tipo}: indica se è di tipo:
	\begin{enumerate}
		\item \textbf{F}: funzionale;
		\item \textbf{Q}: di qualità;
		\item \textbf{P}: prestazionale;
		\item \textbf{V}: di vincolo.
	\end{enumerate}
	\item \textbf{Identificativo}: è il codice univoco e gerarchico che automaticamente il \gl{software} assegna al requisito(esempio: 4.2.1);
\end{itemize}
