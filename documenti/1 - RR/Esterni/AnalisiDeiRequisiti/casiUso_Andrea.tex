\casoduso{1.1}{L'utente scarica e installa l'app}
\desc L'utente decide di scaricare l'app dal relativo \gl{store}\ e di installarla.
\pre L'utente deve poter accedere al proprio \gl{store}\ e trovare l'\gl{app}, inoltre deve riuscire ad installarla senza problemi.
\post L'utente ha l'\gl{app} installata e pu� ora usarla.
\scen
	\begin{enumerate}
		\item \textbf{\UC{1.1.1}}\ L'utente trova l'app nello store. %Nota: qui elenco solo i requisiti a cui noi possiamo e dobbiamo rispondere, qualsiasi problema relativo all'accesso allo store o alla possibilit� di scaricare l'app non � affar nostro. Il caso in cui l'utente rifiuta l'autorizzazione per usare GPS, ecc. non ci riguarda.
		\item \textbf{\UC{1.1.2}}\ L'utente riesce ad installare l'app. %E in caso contrario non possiamo farci niente
	\end{enumerate}
\att L'utente.

\casoduso{1.1.1}{L'utente trova l'app nello store}
\desc L'utente cerca il nome dell'\gl{app}\ nel proprio \gl{store}\ e la trova. Il nome dell'applicazione gli sar� noto tramite altri mezzi, ad esempio leggendolo da cartelli che il cliente provveder� ad appendere nelle aree dove sono presenti i \gl{beacon}.
\pre L'\gl{app}\ dev'essere gi� caricata nello \gl{store}\ e visibile da qualsiasi dispositivo.
\post L'utente ha trovato l'app e pu� scaricarla.
\att L'utente.

\casoduso{1.1.2}{L'utente riesce ad installare l'app}
\desc L'\gl{app}\ riesce ad installarsi nel dispositivo senza segnalare problemi.
\pre L'installazione comincia automaticamente quando il download � terminato.
\post L'\gl{app}\ � installata nel dispositivo e pu� essere avviata.
\att L'utente.

\casoduso{1.X}{Registrazione utente}
\desc L'\gl{app}\ permette all'utente di registrarsi inserendo lo username e la password.
\pre L'utente ha deciso di registrarsi.
\post L'utente si � registrato e risulta ora autenticato.
\scen
	\begin{enumerate}
	\end{enumerate}
\att L'utente.