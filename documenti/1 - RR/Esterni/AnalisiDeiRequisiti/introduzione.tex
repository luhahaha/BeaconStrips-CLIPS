\section{Introduzione}
\label{sec:Introduzione}

	\subsection{Scopo del Documento}
	\label{sub:ScopoDocumento}
	
	Lo scopo del documento è quello di presentare le funzionalità del prodotto che verrà sviluppato dal gruppo \AUTORE{} nell'ambito del capitolato C2 presentato da \PROPONENTE. Questo documento presenterà dettagliatamente i requisiti emersi dall'analisi del capitolato e dagli incontri con il proponente.
	
	\subsection{Scopo del Prodotto}
	\label{sub:ScopoProdotto}
		\SCOPO
	
	\subsection{Glossario}
	\label{sub:Glossario}
		\GLOSSARIO
		
	
	\subsection{Riferimenti}
	\label{sub:Riferimenti}
		\subsubsection{Normativi}
			\NORMATIVI
		\subsubsection{Informativi}
			\begin{itemize}
				\item \textbf{Software Engineering (10th edition}) \\
				Ian Sommerville \\
				Pearson Education | Addison-Wesley
				\item \textbf{Guide to the Software Engineering Body of Knowledge}
				IEEE Computer Society. Software Engineering Coordinating Committee
				\item \textbf{Slides del \COMMITTENTE} \\ riguardo l'\href{http://www.math.unipd.it/~tullio/IS-1/2015/Dispense/L06.pdf}{analisi dei requisiti} e i \href{http://www.math.unipd.it/~tullio/IS-1/2015/Dispense/E02.pdf}{diagrammi dei casi d'uso}.
			\end{itemize}
		