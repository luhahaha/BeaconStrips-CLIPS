
\lettera{U} 

\parola{UML}{Acronimo per ‘‘Unified Modeling Language’’. È un linguaggio di modellazione e specifica basato sul paradigma object-oriented. È nato per unificare i numerosi linguaggi che trattavano i medesimi argomenti per poterne sfruttare le migliori caratteristiche. In questo modo si è cercato con successo di renderlo uno degli standard più diffusi al mondo. UML 2.0 riorganizza molti degli elementi della versione precedente (1.5) in un quadro di riferimento ampliato e introduce molti strumenti, inclusi alcuni nuovi tipi di diagrammi (riferimento: \url{http://it.wikipedia.org/wiki/Unified_Modeling_Language}).}

\parola{UUID}{Acronimo dell'identificativo univoco universale, un identificativo standard usato nelle infrastrutture software, standardizzato come parte di un ambiente distribuito di computazione. (riferimento: \url{https://it.wikipedia.org/wiki/UUID}).}