
\lettera{P}



\parola{PDCA}{Acronimo di ‘‘Plan-Do-Check-Act’’ o Ciclo di Daming, è un modello per il miglioramento continuo della qualità dei processi. Esso prevede in ordine cronologico la pianificazione del processo, la sua applicazione, la sua verifica e l'applicazione delle modifiche ritenute necessarie dal verificatore.}

\parola{Portable Network Graphics}{Il Portable Network Graphics (abbreviato PNG) è un formato di file per memorizzare immagini.
Il formato PNG è superficialmente simile al GIF, in quanto è capace di immagazzinare immagini in modo lossless, ossia senza perdere alcuna informazione.
Può memorizzare immagini a 24 bit ed ha un canale dedicato per la trasparenza (canale alfa)(riferimento: \url{https://it.wikipedia.org/wiki/Portable_Network_Graphics}).}


\parola{Prodotto}{Il prodotto è il risultato di un insieme di attività. In questo caso il termine è da intendersi come un sinonimo di \PROGETTO.} 

\parola{Progetto}{Il progetto è un insieme di azioni organizzate atte a perseguire uno scopo specifico. Nel nostro caso indica tutta l'attività di progettazione di codice e di documenti e della loro verifica, quindi il prodotto finale sarà \PROGETTO. Di conseguenza questo termine verrà usato spesso come sinonimo di \gl{prodotto}.}

\parola{Proponente}{Il proponente è la persona che ha proposto al \gl{committente}\ un \gl{capitolato}\ d'appalto.}