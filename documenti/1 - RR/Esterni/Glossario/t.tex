
\lettera{T} 

\parola{Task}{In italiano ‘‘compito’’, viene utilizzato in informatica per una gruppo di compiti piccoli o un'attività da portare a termine.}

\parola{TCP}{Acronimo di Transmission Control Protocol, anche chiamato Transfer Control Protocol, è un protocollo di rete a pacchetto di livello di trasporto, appartenente alla suite di protocolli Internet, che si occupa di controllo di trasmissione ovvero rendere affidabile la comunicazione dati in rete tra mittente e destinatario (riferimento: \url{https://it.wikipedia.org/wiki/Transmission_Control_Protocol}).}

\parola{TCP/IP}{È una suite di protocolli Internet in cui i due più importanti in essa definiti sono: il Transmission Control Protocol (\gl{TCP}) e l'Internet Protocol (\gl{IP}).}

\parola{Team}{Team è un sinonimo della parola ``gruppo'' usata per indicare il gruppo \AUTORE.}

\parola{Template}{Template in informatica indica un documento o programma nel quale, come in un foglio semicompilato cartaceo, su una struttura generica o standard esistono spazi temporaneamente "bianchi" da riempire successivamente. In questo ambito, la parola è traducibile in italiano come "modello", "semicompilato", "schema", "struttura base", "ossatura generale" o "scheletro", o più correntemente "modulo", anche se di solito non così elaborato e sofisticato (riferimento: \url{https://it.wikipedia.org/wiki/Template}).}

\parola{Telegram}{Telegram è un servizio di messaggistica istantanea. I client ufficiali di Telegram sono distribuiti come \gl{software} libero per diverse piattaforme.
Le caratteristiche principali di Telegram sono la possibilità di stabilire conversazioni cifrate punto-punto, scambiare messaggi vocali, fotografie, video, stickers e file di qualsiasi tipo di dimensione fino ad 1,5 GB
(riferimento: \url{https://it.wikipedia.org/wiki/Telegram_(software)}).}

\parola{Ticket}{Insieme di informazioni che possono essere usate per definire un'attività, per permettere una migliore organizzazione del lavoro.}

\parola{Tomcat}{Apache Tomcat (o semplicemente Tomcat) è un \gl{application server} che offre una piattaforma software per l'esecuzione di applicazioni Web sviluppate in linguaggio \gl{Java} (riferimento: \url{https://it.wikipedia.org/wiki/Apache_Tomcat}).}

\parola{Toolkit}{Il termine Toolkit in informatica è utilizzato per riferirsi ad un insieme di strumenti software di base per facilitare e uniformare lo sviluppo di applicazioni derivate più complesse (riferimento: \url{https://it.wikipedia.org/wiki/Toolkit}).}

\parola{Trender}{Trender è un sistema di tracciamento dei Requisiti, Casi d'uso, Test e Verbali sviluppato da Simone Campagna con licenza MIT e disponibile su GitHub. È intenzione del gruppo contribuire al progetto dopo aver preso dimestichezza con il sistema. Maggiori informazioni su \url{http://campagna91.github.io/Trender/}}

\parola{TTS}{Acronimo di Text-To-Speech, sistemi di sintesi vocale capaci di convertire il testo in parlato.}


