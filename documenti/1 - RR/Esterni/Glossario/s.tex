
\lettera{S} 

\parola{Scalable Vector Graphics}{Scalable Vector Graphics abbreviato in SVG, indica una tecnologia in grado di visualizzare oggetti di grafica vettoriale e, pertanto, di gestire immagini scalabili dimensionalmente.
Più specificamente si tratta di un linguaggio derivato dall'XML, cioè di un'applicazione del metalinguaggio posto a base degli sviluppi del Web da parte del consorzio W3C, che si pone l'obiettivo di descrivere figure bidimensionali statiche e animate (riferimento: \url{https://it.wikipedia.org/wiki/Scalable_Vector_Graphics}).
}

\parola{Script}{Parola inglese che significa ‘‘copione’’. In informatica rappresenta un piccolo programma, solitamente sequenziale e scritto in linguaggio interpretato. Spesso ha complessità bassa e realizza un singolo \gl{task}.}

\parola{Smartphones}{in italiano ‘‘telefono intelligente’’, è un telefono cellulare con capacità di calcolo, di memoria e di connessione dati molto più avanzate rispetto ai normali telefoni cellulari, basato su un sistema operativo per dispositivi mobili.}

\parola{Stack}{Struttura dati astratta, utilizzata in diversi linguaggi di programmazione, in cui i dati possono essere inseriti e acceduti seguendo regole ben definite.}