
\lettera{L} 

\parola{\LaTeX}{\gl{Linguaggio di markup} usato per la preparazione di testi. Si basa sul principio WYSIWYM (What You See Is What You Mean), contrapposto al WYSIWYG (What You See Is What You Get) tipico dei più comuni programmi di videoscrittura. Permette di generare un file in formato .pdf dai file di \LaTeX tramite un apposito compilatore. Maggiori informazioni al sito \url{http://www.latex-project.org}.}

\parola{Linguaggio di markup}{In generale un linguaggio di markup, o linguaggio a marcatori, è un insieme di regole che descrivono i meccanismi di rappresentazione di un testo che, utilizzando convenzioni standardizzate, sono utilizzabili su più supporti. La tecnica di composizione di un testo con l'uso di marcatori (o espressioni codificate) richiede quindi una serie di convenzioni, ovvero di un linguaggio a marcatori di documenti.}