
\lettera{D} 

\parola{Dashboard}{Dashboard, in italiano ‘‘cruscotto’’, è un’interfaccia grafica che organizza e presenta le informazioni in modo semplice, intuitivo ed immediato.}

\parola{Design Pattern}{In informatica, nell'ambito dell'\gl{ingegneria del software}, design pattern è un concetto che può essere definito "una soluzione progettuale generale ad un problema ricorrente". Si tratta di una descrizione o modello logico da applicare per la risoluzione di un problema che può presentarsi in diverse situazioni durante le fasi di progettazione e sviluppo del \gl{software} (riferimento: \url{https://it.wikipedia.org/wiki/Design_pattern}).}

\parola{DSL}{Il domain-specific language o in italiano linguaggio specifico di dominio, nello sviluppo \gl{software} e nell'ingegneria di dominio è un linguaggio di programmazione o un linguaggio di specifica dedicato a particolari problemi di un dominio, a una particolare tecnica di rappresentazione e/o a una particolare soluzione tecnica (riferimento: \url{https://it.wikipedia.org/wiki/Domain-specific_language}).}