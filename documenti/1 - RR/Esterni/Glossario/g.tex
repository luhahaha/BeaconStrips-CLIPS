
\lettera{G} 

\parola{Gantt}{Con Gantt ci si riferisce ai diagrammi di Gantt, uno strumento di supporto alla gestione dei progetti,
così chiamato in ricordo dell’omonimo ingegnere statunitense. Il diagramma di Gantt è costruito partendo da un asse orizzontale - a rappresentazione dell’arco temporale totale del progetto, suddiviso in fasi incrementali (ad esempio, giorni, settimane, mesi) - e da un asse verticale - a rappresentazione delle mansioni o attività che costituiscono il progetto.
}

\parola{GanttProject}{GanttProject è un software gestionale con licenza \gl{GPL} scritto in java. È stato utilizzato dal \gl{team} per la creazione di diagrammi di \gl{Gantt}.}

\parola{Git}{Git è un sistema software di controllo di versione distribuito, creato da Linus Torvalds nel 2005 (riferimento: \url{https://it.wikipedia.org/wiki/Git_(software)}).
}

\parola{GitHub}{GitHub è un servizio web di hosting per lo sviluppo di progetti software, che usa il sistema di controllo di versione \gl{Git}. Può essere utilizzato anche per la condivisione e la modifica di file di testo e documenti revisionabili (riferimento: \url{https://it.wikipedia.org/wiki/GitHub}).}

\parola{Google}{Google Inc. è un'azienda statunitense che offre servizi online, tra cui l'omonimo motore di ricerca, Youtube, Gmail e così via. Inoltre offre anche prodotti come il browser Google Chrome e il sistema operativo \gl{Android} per dispositivi mobile (riferimento: \url{https://it.wikipedia.org/wiki/Google_Inc.}).}

\parola{Google Calendar}{Google Calendar è un sistema di calendari concepito da Google.
Offre la possibilità di creare più calendari, di condividerli e importarli da altri servizi online (Yahoo! Calendar, MSN Calendar, ecc.) o sul computer (iCal, Outlook, ecc.). Google Calendar è parte integrante dell'account Google (riferimento: \url{https://it.wikipedia.org/wiki/Google_Calendar}).}

\parola{Google Drive}{Google Drive è un servizio, in ambiente \gl{cloud computing}, di memorizzazione e sincronizzazione online. Il servizio comprende il file \gl{hosting}, il \gl{file sharing} e la modifica collaborativa di documenti fino a 15 GB gratuiti estendibili fino a 30 TB in totale (riferimento: \url{https://it.wikipedia.org/wiki/Google_Drive}).}

\parola{Google Hangouts}{È un software di messaggistica istantanea e di \gl{VoIP} sviluppato da \gl{Google}, ovvero permette di scambiare messaggi e di effettuare chiamate e videochiamate tra gli utenti. Sfrutta gli account \gl{Google}, che quindi possono usufruire di tutti gli altri servizi offerti dall'azienda (riferimento: \url{https://it.wikipedia.org/wiki/Google_Hangouts}).}

\parola{GPL}{Acronimo di GNU General Public License, è una licenza per il \gl{free software} (riferimento: \url{https://it.wikipedia.org/wiki/GNU_General_Public_License}).}

\parola{GPS}{Acronimo di sistema di posizionamento globale, è un sistema di posizionamento e navigazione satellitare civile che, attraverso una rete dedicata di satelliti artificiali in orbita, fornisce ad un terminale mobile o ricevitore GPS informazioni sulle sue coordinate geografiche ed orario (riferimento: \url{https://it.wikipedia.org/wiki/Sistema_di_posizionamento_globale}). }

\parola{Gulpease}{Gulpease o ‘‘Indice Gulpease’’ è un indice di leggibilità di testo tarato sulla lingua italiana. Rispetto ad altri ha il vantaggio di utilizzare la lunghezza delle parole anziché in sillabe, semplificandone il calcolo numerico.}