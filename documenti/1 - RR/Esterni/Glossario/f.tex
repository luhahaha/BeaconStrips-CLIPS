
\lettera{F} 

\parola{File hosting}{Il file hosting è un servizio di archiviazione su Internet appositamente progettato per ospitare i file degli utenti, permettendo loro di caricare file che possono poi essere scaricati da altri utenti (riferimento: \url{https://it.wikipedia.org/wiki/File_hosting}).}

\parola{File sharing}{Il file sharing è la condivisione di file all'interno di una rete di calcolatori (riferimento: \url{https://it.wikipedia.org/wiki/File_sharing}).}

\parola{Framework}{Un framework, in informatica e specificatamente nello sviluppo software, è un'architettura logica di supporto su cui un software può essere progettato e realizzato, spesso facilitandone lo sviluppo da parte del programmatore (riferimento: \url{https://it.wikipedia.org/wiki/Framework}).}

\parola{Free software}{In italiano ‘‘software libero’’, è un software pubblicato sotto i termini di una licenza libera, ovvero che ne incoraggia l'utilizzo, lo studio, la modifica e la redistribuzione (riferimento: \url{https://it.wikipedia.org/wiki/Software_libero}).}

\parola{Front end}{Secondo il significato più generale il front end è responsabile per l'acquisizione dei dati di ingresso e per la loro elaborazione con modalità conformi a specifiche predefinite e invarianti, in modo da renderli utilizzabili dal \gl{back end}. Nel nostro caso il front end rappresenta sia l'interfaccia grafica sia il sistema di elaborazione degli input dell'utente e degli output per l'utente.}