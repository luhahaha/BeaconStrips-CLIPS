\section{Introduzione}
	\subsection{Scopo del documento} 
	Questo documento ha lo scopo di spiegare dettagliatamente le strategie secondo cui il gruppo Beacon Strips intende condurre il progetto didattico.
	\subsection{Scopo del prodotto}
	\SCOPO
	\subsection{Glossario}
	\GLOSSARIO
	\subsection{Riferimenti}
		\subsubsection{Normativi}
			\NORMATIVI		
			
		\subsubsection{Informativi}
			\begin{itemize}
				\item \textbf{Software Engineering (10th edition}) \\
				Ian Sommerville \\
				Pearson Education | Addison-Wesley
				\item \textbf{Guide to the Software Engineering Body of Knowledge}
				IEEE Computer Society. Software Engineering Coordinating Committee
				\item \textbf{Slides del \COMMITTENTE} \\ riguardo i  \href{http://www.math.unipd.it/~tullio/IS-1/2015/Dispense/L02.pdf}{processi software}, il \href{http://www.math.unipd.it/~tullio/IS-1/2015/Dispense/L03.pdf}{ciclo di vita del software} e \href{http://www.math.unipd.it/~tullio/IS-1/2015/Dispense/L04.pdf}{la gestione di progetto}	
			\end{itemize}
	\subsection{Modello di ciclo di vita scelto}
	È stato scelto come ciclo di vita il modello \gl{incrementale}. Le motivazioni che ci hanno spinto verso questa direzione sono il modo in cui è strutturato il progetto didattico e la quasi totale inesperienza dei componenti del gruppo nello sviluppare progetti software di grandi dimensioni. Di seguito una lista di caratteristiche del metodo incrementale:
	\begin{itemize}
		\item Si può produrre valore ad ogni incremento.
		\item Ogni incremento riduce il rischio di fallimento.
		\item Prevede rilasci multipli.
		\item I requisiti utente sono classificati e trattati in base alla loro importanza strategica. I requisiti più importanti sono già stabili all'inizio dello sviluppo del progetto.
		\item L'analisi dei requisiti e la progettazione architetturale non vengono ripetute.
		\item Prima si pensa allo sviluppo dei requisiti essenziali, poi a quelli desiderabili.
		\item Sono presenti delle iterazioni del tipo Prototipo $\rightarrow$ Validazione $\rightarrow$ Prototipo $\rightarrow$ Validazione $\rightarrow$ ecc..
	\end{itemize}
	\subsection{Scadenze}
	Il gruppo Beacon Strips ha deciso di rispettare le seguenti scadenze:
	\begin{itemize}
		\item \textbf{Revisione dei Requisiti}: 18/04/2016
		\item \textbf{Revisione di Progettazione}: 17/06/2016
		\item \textbf{Revisione di Qualifica}: 24/08/2016
		\item \textbf{Revisione di Accettazione}: 12/09/2016
	\end{itemize}
	In base a queste scadenze e a fronte dell'analisi dei rischi verranno decise le fasi in cui suddividere il lavoro di sviluppo del progetto.
	
	