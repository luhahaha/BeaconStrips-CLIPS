\section{Introduzione}
	\subsection{Scopo del documento} 
	Questo documento ha lo scopo di spiegare dettagliatamente le strategie secondo cui il gruppo \AUTORE{} intende condurre il progetto didattico. 
	\subsection{Scopo del prodotto}
	\SCOPO
	\subsection{Glossario}
	\GLOSSARIO
	\subsection{Riferimenti}
		\subsubsection{Normativi}
			\begin{itemize}
				\item \textbf{Capitolato d'appalto C2 - CLIPS:} Communication \& Localisation with Indoor Positioning Systems. \\
				\url{http://www.math.unipd.it/~tullio/IS-1/2015/Progetto/C2.pdf}
				\item \textbf{Vincoli e dettagli tecnico-economici} \\
				\url{http://www.math.unipd.it/~tullio/IS-1/2015/Dispense/PD01.pdf}
				\item \textbf{Norme di Progetto} \\ \NPfile
				\item \textbf{Regolamento di Progetto} \\
				\url{http://www.math.unipd.it/~tullio/IS-1/2015/Progetto/}
				\item \textbf{Regolamento organigramma} \\
				\url{http://www.math.unipd.it/~tullio/IS-1/2015/Progetto/PD01b.html}
			\end{itemize}	
			
		\subsubsection{Informativi}
			\begin{itemize}
				\item \textbf{Software Engineering (10th edition}) \\
				Ian Sommerville \\
				Pearson Education | Addison-Wesley
				\item \textbf{Guide to the Software Engineering Body of Knowledge}
				IEEE Computer Society. Software Engineering Coordinating Committee
				\item \textbf{Slides del \COMMITTENTE} \\ riguardo i  \href{http://www.math.unipd.it/~tullio/IS-1/2015/Dispense/L02.pdf}{processi software}, il \href{http://www.math.unipd.it/~tullio/IS-1/2015/Dispense/L03.pdf}{ciclo di vita del software} e \href{http://www.math.unipd.it/~tullio/IS-1/2015/Dispense/L04.pdf}{la gestione di progetto}	
			\end{itemize}
	\subsection{Modello di ciclo di vita scelto}
	È stato scelto come ciclo di vita il modello \gl{incrementale}. Le motivazioni che ci hanno spinto verso questa direzione sono il modo in cui è strutturato il progetto didattico e la quasi totale inesperienza dei componenti del gruppo nello sviluppare progetti software di grandi dimensioni. Di seguito una lista di caratteristiche del metodo incrementale:
	\begin{itemize}
		\item si può produrre valore ad ogni incremento;
		\item ogni incremento riduce il rischio di fallimento;
		\item prevede rilasci multipli;
		\item i requisiti utente sono classificati e trattati in base alla loro importanza strategica. I requisiti più importanti sono già stabili all'inizio dello sviluppo del progetto;
		\item l'analisi dei requisiti e la progettazione architetturale non vengono ripetute;
		\item prima si pensa allo sviluppo dei requisiti essenziali, poi a quelli desiderabili;
		\item Sono presenti delle iterazioni del tipo Prototipo $\rightarrow$ Validazione $\rightarrow$ Prototipo $\rightarrow$ Validazione $\rightarrow$ ecc..
	\end{itemize}
	\subsection{Scadenze}
	Il gruppo Beacon Strips ha deciso di rispettare le seguenti scadenze:
	\begin{itemize} 
		\item \textbf{Revisione dei Requisiti}: 2016-04-18
		\item \textbf{Revisione di Progettazione}: 2016-06-17
		\item \textbf{Revisione di Qualifica}: 2016-08-24
		\item \textbf{Revisione di Accettazione}: 2016-09-12
	\end{itemize}
	In base a queste scadenze e a fronte dell'analisi dei rischi verranno decise le fasi in cui suddividere il lavoro di sviluppo del progetto.
	\subsubsection{Scelta Revisione di Progettazione}
	Si è deciso di affrontare la RP$_{\mbox{\textit{min}}}$. Il gruppo si impegna quindi per il 2016-06-17 di presentare nel documento ``Specifica Tecnica'' la progettazione ad alto livello del prodotto.
	