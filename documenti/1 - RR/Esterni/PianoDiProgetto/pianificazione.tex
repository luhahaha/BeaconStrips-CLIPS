\section{Pianificazione} 
	\subsection{Introduzione}
	A fronte dell'analisi dei rischi e della scadenza delle revisioni di avanzamento vi saranno tre periodi durante lo svolgimento del progetto: uno di \textbf{analisi}, uno di \textbf{progettazione e codifica} ed uno di \textbf{incremento e validazione}.
	Per rendere più controllabile lo sviluppo del progetto si è deciso di dividere il lavoro in sei fasi dettagliate, le quali vengono riportate nella seguente tabella con le relative date di inizio e di fine.
		
		\begin{tabella}{!{\VRule}c!{\VRule}c!{\VRule}c!{\VRule}c!{\VRule}}
				
			\intestazionefourcol{Fase}{Abbreviazione}{Data di inizio}{Data di fine}
			
			Analisi & A & 2016/03/01 & 2016/04/18  \\
			Analisi di dettaglio & AD & 2016/04/19 & 2016/04/28  \\
			Progettazione Architetturale & PA & 2016/04/29 & 2016/06/17 \\
			Progettazione di dettaglio e codifica & PDC & 2016/06/18 & 2016/08/24 \\
			Requisiti desiderabili e opzionali & RD & 2016/08/25 & 2016/08/30 \\
			Validazione e verifica & V & 2016/08/31 & 2016/09/12 \\ 
			
			\hiderowcolors
			\caption{Fasi di sviluppo con relative abbreviazioni e date di inizio e fine.}
			
		\end{tabella}
		
	Ogni fase contiene diverse attività che verranno riportate e descritte in un elenco puntato. \\ Successivamente nei diagrammi di \gl{Gantt} si potrà notare come le attività siano state suddivise temporalmente. In questi saranno inoltre presenti delle \gl{milestones} che indicheranno i giorni in cui dovranno essere consegnati i documenti in entrata alle revisioni e quelli in cui si svolgeranno le revisioni di avanzamento. \\
	Per facilitare la lettura sono stati riportati in questo documento dei diagrammi di Gantt che evidenziano, talvolta, solo le attività principali o i processi.
	
	\subsection{Fase A}
	\begin{center}
		\textbf{Data di inizio}: 2016/03/01 \\
		\textbf{Data di fine}: 2016/04/18 \\
	\end{center}

	Questa fase inizia con la formazione del gruppo e termina il giorno della Revisione dei Requisiti. \\
	I processi principali di questa fase sono: 
	\begin{itemize}
		\item \textbf{Individuazione strumenti da utilizzare}: il gruppo deve trovare degli strumenti che aiutino ad automatizzare e rendere più facile lo sviluppo del progetto.
		\item \textbf{Creazione documentazione}: Viene creata la documentazione da consegnare in ingresso alla RR.
		\att
		\begin{itemize}
			\item \textbf{Norme di Progetto}: viene steso il documento \NPdoc in cui saranno elencate e descritte le norme da seguire durante tutto lo svolgimento del progetto indipendentemente dal capitolato scelto;
			\item \textbf{Piano di Progetto}: viene steso il documento \PPdoc per pianificare dettagliatamente i tempi e i costi del progetto;
			\item \textbf{Studio di Fattibilità}: viene steso il documento \SFdoc che riporta l'analisi che ha portato il gruppo a scegliere il capitolato C2;
			\item \textbf{Analisi dei Requisiti}: viene steso il documento \ARdoc in cui viene svolta un'analisi molto più approfondita di quella svolta in \SFdoc. Vengono elencati e descritti i casi d'uso e i requisiti del prodotto che si andrà a sviluppare;
			\item \textbf{Piano di Qualifica}: viene steso il documento \PQdoc che riporta che obiettivi di qualità si è prefissato il gruppo;
			\item \textbf{Glossario}: viene steso il \Gldoc il quale riporta la descrizione dei termini presenti nei vari documenti che potrebbero causare ambiguità nel lettore.
		\end{itemize}
	\end{itemize}
	
		
		\subsubsection{Diagramma di Gantt delle attività}
		
		
	\subsection{Fase AD}
	\begin{center}
		\textbf{Data di inizio}: 2016/04/19 \\
		\textbf{Data di fine}: 2016/04/28 \\
	\end{center}
	Questa fase inizia al termine della fase A, ovvero dopo la Revisione dei Requisiti, e termina esattamente dieci giorni dopo \\ 
	Il processo principale di questa fase è:
	\begin{itemize}
		\item \textbf{Miglioramento e incremento della documentazione}
		\att
		\begin{itemize}
			\item \textbf{Miglioramento di tutti i documenti}: seguendo le indicazioni del Committente verranno attuate le modifiche necessarie a migliorare tutti i documenti stesi nella fase A;
			\item \textbf{Analisi dei Requisiti}: Questo documento oltre ad essere corretto verrà anche arricchito con nuovi requisiti.
		\end{itemize}
	\end{itemize}
		
		
		\subsubsection{Diagramma di Gantt delle attività}
		
		
	\subsection{Fase PA}
	\begin{center}
		\textbf{Data di inizio}: 2016/04/29 \\
		\textbf{Data di fine}: 2016/06/17 \\
	\end{center}
	Questa fase inizia subito dopo il termine della fase AD e termina con la data della Revisione di Progettazione. \\
	Il processo principale di questa fase è:
		\begin{itemize}
			\item \textbf{Miglioramento e incremento della documentazione}
			\att
			\begin{itemize}
				\item \textbf{Specifica Tecnica}: viene creato il documento \STdoc che conterrà le scelte progettuali decise dai progettisti;
				\item \textbf{Norme di Progetto}: viene incrementato questo documento in modo da normare anche la stesura del documento \STdoc.
				\item \textbf{Piano di Progetto}: viene aggiunto il consuntivi del periodo e preventivo a finire. Vengono inoltre riportati i rischi che si sono verificati nelle fasi precedenti;
				\item \textbf{Piano di Qualifica}: viene aggiunta la parte di pianificazione dei test;
				\item \textbf{Glossario}: viene incrementato con i nuovi termini presenti nella \STdoc.
			\end{itemize}
		\end{itemize}
		\subsubsection{Diagramma di Gantt delle attività}
		
	\subsection{Fase PDC}
	\begin{center}
		\textbf{Data di inizio}: 2016/06/18 \\
		\textbf{Data di fine}: 2016/08/24 \\
	\end{center}
	Questa fase inizia subito dopo la fine della fase PA, ovvero dopo la Revisione di Progettazione, e termina con la data della Revisione di Qualifica. \\
	I processi principali di questa fase sono: 
		\begin{itemize}
			\item \textbf{Miglioramento e incremento della documentazione}
			\att
			\begin{itemize}
				\item \textbf{Definizione di Prodotto}: viene steso il documento \DPdoc il quale definisce la struttura e la relazione tra le componenti del prodotto. È basato sul documento \STdoc;
				\item \textbf{Manuale utente}: viene redatta la versione preliminare del \MUdoc il quale fornirà agli utenti le indicazioni per l'utilizzo del prodotto;
				\item \textbf{Incremento altri documenti}: come nella fase precedente anche in questa vi sarà il miglioramento dei documenti che necessitano tale trattamento.
			\end{itemize}
			\item \textbf{Sviluppo del prodotto}
			\att
			\begin{itemize}
				\item \textbf{Codifica}: avviene la scrittura del codice del prodotto;
				\item \textbf{Test}: vengono eseguiti i test di unità e di integrazione e ne vengono osservati i risultati. 
			\end{itemize}
		\end{itemize}
		\subsubsection{Diagramma di Gantt delle attività}
		
	\subsection{Fase RD}
	\begin{center}
		\textbf{Data di inizio}: 2016/08/25 \\
		\textbf{Data di fine}: 2016/08/30 \\
	\end{center}
	Questa fase inizia subito dopo la fine della fase PDC, ovvero dopo la Revisione di Qualifica, e termina sei giorni dopo. \\
	I processi principali di questa fase sono: 
		\begin{itemize}
			\item \textbf{Miglioramento e incremento della documentazione}:
			\att
			\item \textbf{Sviluppo del prodotto}:
			\att
		\end{itemize}
		\subsubsection{Diagramma di Gantt delle attività}
		
	\subsection{Fase V}
	\begin{center}
		\textbf{Data di inizio}: 2016/08/31 \\
		\textbf{Data di fine}: 2016/09/12 \\
	\end{center}
	Questa fase inizia subito dopo la fine della fase RD e termina con la data della Revisione di Accettazione. \\
	I processi principali di questa fase sono:
		\begin{itemize}
			\item \textbf{Miglioramento e incremento della documentazione}
			\att
			\begin{itemize}
				\item \textbf{Correzioni e aggiornamenti}: Verranno corretti e aggiornati tutti i documenti che lo necessitano. Si otterrà la versione finale della documentazione. 
			\end{itemize}
			\item \textbf{Sviluppo del prodotto}
			\att
				\begin{itemize}
					\item \textbf{Test}: vengono eseguiti i test di sistema previsti e ne vengono osservati e monitorati i risultati. 
				\end{itemize}
			\item \textbf{Verifica e validazione}
			\att
			\begin{itemize}
				\item \textbf{Collaudo}: il prodotto viene collaudato sulle funzionalità previste;
				\item \textbf{Verifica}: tramite tracciamento si verifica di aver soddisfatto i requisiti presenti nel documento \ARdoc. Si verificheranno inoltre i canoni di qualità previsti nel \PQdoc;
				\item \textbf{Validazione}: una volta svolte tutte le verifiche il prodotto può considerarsi validato.
			\end{itemize}
		\end{itemize}
		\subsubsection{Diagramma di Gantt delle attività}
	
