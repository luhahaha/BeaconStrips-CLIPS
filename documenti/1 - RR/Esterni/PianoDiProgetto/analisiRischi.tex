\section{Analisi dei rischi} 
	È stata attuata una profonda analisi dei rischi in modo tale di essere pronti ad affrontarli in caso si presentassero.
	Ogni rischio è stato analizzato seguendo questa scaletta:
	\begin{enumerate}
		\item \textbf{Identificazione}: individuazione dei possibili rischi che si potranno riscontrare durante lo sviluppo del progetto.
		\item \textbf{Analisi}: verrà analizzata la probabilità che i rischi si verifichino e come questi potrebbero influire sul lavoro;
		\item \textbf{Pianificazione di controllo}: verranno delineati i metodi grazie ai quali si cercherà di evitare che il rischio si verifichi.
		\item \textbf{Tecniche di mitigazione}: verranno delineate i metodi grazie ai quali verranno mitigati i rischi, nel caso si presentassero.
	\end{enumerate}
	Per ogni rischio verranno riportate le seguenti informazioni:
	\begin{itemize}
		\item \textbf{Descrizione};
		\item \textbf{Metodi di identificazione};
		\item \textbf{Possibilità che si verifichi};
		\item \textbf{Pericolosità};
		\item \textbf{Conseguenze};
		\item \textbf{Contromisure};
	\end{itemize}
	
	\subsection{Livello tecnologico}
		\subsubsection{Uso di tecnologie e strumenti}
			\begin{itemize}
				\item \textbf{Descrizione}: alcune tecnologie e alcuni strumenti che verranno utilizzati sono sconosciuti ad alcuni membri del gruppo, altri sono sconosciuti a tutti i membri del gruppo; 
				\item \textbf{Metodi di identificazione}: ogni componente del gruppo sarà consapevole delle proprie conoscenze e dei propri limiti in fase di apprendimento;
				\item \textbf{Possibilità che si verifichi}: alta;
				\item \textbf{Pericolosità}: alta;
				\item \textbf{Conseguenze}: rallentamento generale nell'avanzamento del progetto;
				\item \textbf{Contromisure}: qualora un membro riscontrasse difficoltà con una tecnologia o uno strumento dovrà chiedere aiuto al \RES{} o ad uno degli Amministratori i quali gli forniranno quanto richiesto in forma scritta o verbale;
			\end{itemize}	
		
		\subsubsection{Danneggiamento strumentazione hardware}
			\begin{itemize}
				\item \textbf{Descrizione}: è possibile che i personal computer o altri strumenti in uso dal team subiscano danneggiamenti accidentali.
				\item \textbf{Metodi di identificazione}: ogni membro del gruppo dovrà essere consapevole del funzionamento o meno della strumentazione che possiede e$/$o che ha in uso;
				\item \textbf{Possibilità che si verifichi}: bassa;
				\item \textbf{Pericolosità}: alta;
				\item \textbf{Conseguenze}: rallentamento del lavoro che il proprietario$/$usufruente dello strumento danneggiato dovrebbe svolgere;
				\item \textbf{Contromisure}: il proprietario$/$usufruente dovrà, se possibile, preoccuparsi di aggiustare lo strumento danneggiato o di procurarne uno sostitutivo. Se necessario dovrà chiedere aiuto ad uno degli Amministratori.
			\end{itemize}
		
		\subsubsection{Problemi software strumenti utilizzati}
		\begin{itemize}
			\item \textbf{Descrizione}: è possibile che gli strumenti scelti per agevolare i processi abbiano problemi di varia natura;
			\item \textbf{Metodi di identificazione}: chi utilizza uno strumento che sembra causare problemi lo farà presente ad un \AM che si occuperà di verificare la reale esistenza del problema;
			\item \textbf{Possibilità che si verifichi}: media;
			\item \textbf{Pericolosità}: alta;
			\item \textbf{Conseguenze}: forte rallentamento del lavoro;
			\item \textbf{Contromisure}: gli Amministratori dovranno estinguere il problema se il SW che causa problemi è stato creato dal team, altrimenti provvederà a trovare uno strumento alternativo che faccia un lavoro migliore di quello che causa problemi.
		\end{itemize}
		
		
	\subsection{Livello personale}
	
	\subsection{Livello organizzativo}
