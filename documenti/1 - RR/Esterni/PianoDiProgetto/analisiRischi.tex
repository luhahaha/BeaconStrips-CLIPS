\section{Analisi dei rischi} 
	È stata attuata una profonda analisi dei rischi in modo tale di essere pronti ad affrontarli in caso si presentassero.
	Ogni rischio è stato analizzato seguendo questa scaletta:
	\begin{enumerate}
		\item \textbf{Identificazione}
		\item \textbf{Analisi}
		\item \textbf{Pianificazione di controllo}
		\item \textbf{Tecniche di mitigazione}
	\end{enumerate}
	Per ogni rischio verranno riportate le seguenti informazioni:
	\begin{itemize}
		\item \textbf{Descrizione}:
		\item \textbf{Metodi di identificaione}:
		\item \textbf{Possibilità che si verifichi}:
		\item \textbf{Pericolosità}:
		\item \textbf{Conseguenze}:
		\item \textbf{Contromisure}:
	\end{itemize}
	
	\subsection{Livello tecnologico}
		\subsubsection{Uso di tecnologie e strumenti}
		\begin{itemize}
			\item \textbf{Descrizione}: alcune tecnologie e alcuni strumenti che verranno utilizzati sono sconosciuti ad alcuni membri del gruppo, altri sono sconosciuti a tutti i membri del gruppo. 
			\item \textbf{Metodi di identificaione}: ogni componente del gruppo sarà consapevole delle proprie conoscenze e dei propri limiti in fase di apprendimento.
			\item \textbf{Possibilità che si verifichi}: alta
			\item \textbf{Pericolosità}: alta
			\item \textbf{Conseguenze}: rallentamento generale nell'avanzamento del progetto.
			\item \textbf{Contromisure}: qualora un membro riscontrasse difficoltà con una tecnologia o uno strumento dovrà chiedere aiuto al \RES{} o ad uno degli Amministratori i quali gli forniranno quanto richiesto in forma scritta o verbale.
		\end{itemize}	
		
		\subsubsection{Danneggiamento strumenti di sviluppo}
		
		
	\subsection{Livello personale}
	
	\subsection{Livello organizzativo}
