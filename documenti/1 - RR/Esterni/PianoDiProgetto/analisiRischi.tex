\section{Analisi dei rischi} 
	È stata attuata una profonda analisi dei rischi in modo tale di essere pronti ad affrontarli in caso si presentassero.
	Ogni rischio è stato analizzato seguendo questa scaletta:
	\begin{enumerate}
		\item \textbf{Identificazione}: individuazione dei possibili rischi che si potranno riscontrare durante lo sviluppo del progetto.
		\item \textbf{Analisi}: verrà analizzata la probabilità che i rischi si verifichino e come questi potrebbero influire sul lavoro;
		\item \textbf{Pianificazione di controllo}: verranno delineati i metodi grazie ai quali si cercherà di evitare che il rischio si verifichi.
		\item \textbf{Tecniche di mitigazione}: verranno delineate i metodi grazie ai quali verranno mitigati i rischi, nel caso si presentassero.
	\end{enumerate}
	Per ogni rischio verranno riportate le seguenti informazioni:
	\begin{itemize}
		\item \textbf{Descrizione};
		\item \textbf{Metodi di identificazione};
		\item \textbf{Possibilità che si verifichi};
		\item \textbf{Pericolosità};
		\item \textbf{Conseguenze};
		\item \textbf{Contromisure};
	\end{itemize}
	
	\subsection{Livello tecnologico}
		\subsubsection{Uso di tecnologie e strumenti}
			\begin{itemize}
				\item \textbf{Descrizione}: alcune tecnologie e alcuni strumenti che verranno utilizzati sono sconosciuti ad alcuni membri del gruppo, altri sono sconosciuti a tutti i membri del gruppo; 
				\item \textbf{Metodi di identificazione}: ogni componente del gruppo sarà consapevole delle proprie conoscenze e dei propri limiti in fase di apprendimento;
				\item \textbf{Possibilità che si verifichi}: alta;
				\item \textbf{Pericolosità}: alta;
				\item \textbf{Conseguenze}: rallentamento generale nell'avanzamento del progetto;
				\item \textbf{Contromisure}:  per evitare che il rischio si presenti ognuno si occuperà di studiare la tecnologia o lo strumento che ha intenzione di usare. Qualora un membro riscontrasse difficoltà con una tecnologia o uno strumento dovrà chiedere aiuto al \RES{} o ad uno degli Amministratori i quali gli forniranno quanto richiesto in forma scritta o verbale;
			\end{itemize}	
		
		\subsubsection{Danneggiamento strumentazione hardware}
			\begin{itemize}
				\item \textbf{Descrizione}: è possibile che i personal computer o altri strumenti in uso dal team subiscano danneggiamenti accidentali.
				\item \textbf{Metodi di identificazione}: ogni membro del gruppo dovrà essere consapevole del funzionamento o meno della strumentazione che possiede e$/$o che ha in uso;
				\item \textbf{Possibilità che si verifichi}: bassa;
				\item \textbf{Pericolosità}: alta;
				\item \textbf{Conseguenze}: rallentamento del lavoro che il proprietario$/$fruitore dello strumento danneggiato dovrebbe svolgere;
				\item \textbf{Contromisure}: non è possibile prevedere danneggiamenti hardware, ma per evitare perdite di lavoro ogni componente del gruppo al termine di una sessione di lavoro si occuperà di fare una commit sul \gl{repository}. Nel caso si verifichino danni hardware il proprietario$/$fruiutore dovrà, se possibile, preoccuparsi di aggiustare lo strumento danneggiato o di procurarne uno sostitutivo. Se lo considererà necessario potrà chiedere aiuto ad uno degli Amministratori.
			\end{itemize}
		
		\subsubsection{Problemi software strumenti utilizzati}
		\begin{itemize}
			\item \textbf{Descrizione}: è possibile che gli strumenti scelti per agevolare i processi abbiano problemi di varia natura;
			\item \textbf{Metodi di identificazione}: chi utilizza uno strumento che sembra causare problemi lo farà presente ad un \AM{} che si occuperà di verificare la reale esistenza del problema;
			\item \textbf{Possibilità che si verifichi}: media;
			\item \textbf{Pericolosità}: alta;
			\item \textbf{Conseguenze}: forte rallentamento del lavoro;
			\item \textbf{Contromisure}: non è possibile evitare a priori che si verifichino problemi con il SW. Nel caso questi problemi si presentassero gli Amministratori dovranno estinguere il problema se il SW che causa problemi è stato creato dal team, altrimenti provvederà a trovare uno strumento alternativo che faccia un lavoro migliore di quello che causa problemi.
		\end{itemize}
		
		
	\subsection{Livello personale}
		\subsubsection{Problemi personali dei componenti}
		\begin{itemize}
			\item \textbf{Descrizione}: ogni membro del gruppo ha impegni relativi alla propria vita privata che potrebbero incidere sulla pianificazione delle attività;
			\item \textbf{Metodi di identificazione}: il \RES{} verrà informato tempestivamente se si presenteranno impegni personali non precedentemente comunicati;
			\item \textbf{Possibilità che si verifichi}: media;
			\item \textbf{Pericolosità}: media;
			\item \textbf{Conseguenze}: rallentamento del lavoro individuale o, in casi più gravi, rallentamento del lavoro dell'intero gruppo;
			\item \textbf{Contromisure}: ogni membro del gruppo dichiarerà all'inizio del progetto i propri impegni personali al \RES{} tenendo conto anche dei possibili impegni extra che riesce a prevedere. Nel caso in cui un impegno personale rallenti un membro del gruppo per molto tempo il \RES{} sposterà in avanti le scadenze prefissate, se questo sarà possibile. In alternativa il \RES{} si occuperà di incaricare un altro membro del gruppo a svolgere il lavoro di chi non può farlo. 
		\end{itemize}
		
		
		\subsubsection{Problemi tra i componenti}
		\begin{itemize}
			\item \textbf{Descrizione}: il gruppo è formato da sei persone ed è possibile che in certi momenti del progetto sorgano disaccordi e discussioni;
			\item \textbf{Metodi di identificazione}: chi ha problemi con uno o più membri del gruppo deve comunicarlo tempestivamente al \RES{};
			\item \textbf{Possibilità che si verifichi}: bassa;
			\item \textbf{Pericolosità}: media;
			\item \textbf{Conseguenze}: svogliatezza nello svolgere i compiti assegnati, ritardo nel portare a termine tali compiti;
			\item \textbf{Contromisure}: ogni membro del gruppo si impegnerà ad essere disponibile a discussioni costruttive e cercherà di non generare litigi insensati. Qualora questi ultimi sorgessero il \RES{} si occuperà di placare gli animi.
		\end{itemize}
	
	\subsection{Livello organizzativo}
		\subsubsection{Errori nella valutazione dei tempi e dei costi}
		\begin{itemize}
			\item \textbf{Descrizione}: è possibile che durante la stesura della pianificazione venga commesso qualche errore riguardo i tempi e i costi dovuto all'inesperienza;
			\item \textbf{Metodi di identificazione}: qualora qualcuno si accorgesse di discrepanze rispetto alla pianificazione deve prontamente notificarlo al \RES{}. Quest'ultimo dovrà assicurarsi, in ogni caso, che lo svolgimento delle attività prosegua secondo i piani; 
			\item \textbf{Possibilità che si verifichi}: alta;
			\item \textbf{Pericolosità}: alta;
			\item \textbf{Conseguenze}: rallentamento nell'ultimazione delle attività pianificate;
			\item \textbf{Contromisure}: ogni membro del gruppo si impegnerà per rispettare le consegne assegnategli. Colui che assegna i tasks ai membri del gruppo dovrà tenere conto di possibili ritardi e calcolare che questi non compromettano il lavoro del resto del gruppo.
		\end{itemize}
		
		
		\subsubsection{Problemi di comprensione dei requisiti}
		\begin{itemize}
			\item \textbf{Descrizione}: è possibile che durante l'analisi dei requisiti alcuni aspetti vengano compresi in modo incompleto o, addirittura, errato. Questo sempre a causa dell'inesperienza del gruppo;
			\item \textbf{Metodi di identificazione}: se sorgeranno dubbi riguardo i requisiti ci si accorderà con il Proponente sulla strada giusta da seguire;
			\item \textbf{Possibilità che si verifichi}: media;
			\item \textbf{Pericolosità}: bassa;
			\item \textbf{Conseguenze}: rallentamento del lavoro, possibilità di mancanza di requisiti fondamentali. 
			\item \textbf{Contromisure}: non sono state identificate contromisure efficaci se non il prestare particolare attenzione a quanto detto dal Proponente durante le riunioni sostenute.
		\end{itemize}
