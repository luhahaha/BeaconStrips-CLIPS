\documentclass[a4paper,titlepage]{article}

\makeatletter
\def\input@path{{../../../template/}}
\makeatother

\usepackage{Comandi}
\usepackage{Riferimenti}
\usepackage{Stile}

\def\NOME{Verbale del Giorno 2016-03-21}
\def\VERSIONE{1.0.0}
\def\DATA{\today}
\def\REDATTORE{Luca Soldera}
\def\VERIFICATORE{Matteo Franco}
\def\RESPONSABILE{Viviana Alessio}
\def\USO{Esterno}
\def\DESTINATARI{\COMMITTENTE \\ & \CARDIN \\ & \PROPONENTE}
\def\SOMMARIO{Decisioni sul tema del progetto}


\begin{document}

\maketitle

\newpage
\tableofcontents

\newpage
\section{Informazioni}
\label{sec:Informazioni}

\begin{itemize}
  \item \textbf{Luogo}: Via Giacinto Andrea Longhin, 53/4, 35129, Padova(PD);
  \item \textbf{Data}: 2016-03-21;
  \item \textbf{Ora}: 16:30;
  \item \textbf{Durata}: 1 ora e 30 minuti;
  \item \textbf{Partecipanti}: Viviana Alessio, Luca Soldera, Matteo Franco, Andrea Grendene, Enrico Bellio, Tommaso Panozzo, Emanuele Righetto, Carlo Scattolin.
\end{itemize}

\newpage
\section{Premessa}

La riunione è stata indetta per proporre a \PROPONENTE{} idee sul progetto da sviluppare.

\section{Ordine del giorno}
\label{sec:OrdineDelGiorno}

\begin{enumerate}
  \item Prima proposta di progetto;
  \item seconda proposta di progetto;
  \item domande di chiarimento;
  \item richiesta di un server per installare Trender ed eventuale database per il progetto.
\end{enumerate}

\newpage
\section{Verbale}
\label{sec:Verbale}

\begin{enumerate}
  \item \textbf{Proposta}: \\
  Una piattaforma da installare, dove siano già presenti dei beacons installati, per il tracciamento dei movimenti e i flussi di utenti. L'idea è di creare un'applicazione molto semplice per simulare la navigazione tra i beacon ed un servizio web in cui l'amministratore possa visualizzare in modo semplice e rapido statistiche in base ai movimenti degli utenti e i flussi di spostamento più frequenti. \\
  \textbf{Risposta}:
  Una bella idea che però si allontana dalle linee proposte dal capitolato; 
  %da sistemare
  può essere poco interessante per un utente avere un'applicazione semplice e poco interessante, ma un sito web che tenga traccia degli spostamenti; inoltre \PROPONENTE{} ha già sperimentato questo tipo di tracciamento dei flussi e il calcolo delle statistiche su di essi risulta poco interessante in quanto abbastanza semplice da implementare.
  \item \textbf{Proposta}: \\
  Idea basata sul modello delle Escape Room nel quale l'utente per proseguire o uscire dal luogo in cui si trova deve cercare e risolvere indovinelli e rompicapo; la proposta è di creare un'applicazione che intrattenga l'utente grazie alla ricerca dei beacons all'interno dell'ambiente in cui si trova e alla somministrazione di giochi, domande e indovinelli in base al beacon da lui trovato. L'applicazione può essere pensata sia come gioco organizzato, quindi installato per l'occasione, sia per l'intrattenimento in un ambito in cui i beacons siano già installati per qualche altro scopo (ad esempio musei o centri commerciali).\\
  \textbf{Risposta}: \\
  Idea interessante che si potrebbe sviluppare in molti modi diversi. Entrambe le modalità, in cui l'utente si installa il proprio gioco o in uno spazio con beacons già installati, sono molto interessanti e c'è la possibilità di scegliere di svilupparne solo una oppure riuscire a creare un sistema che possa rendere entrambe realizzabili. Altro punto di forza è utilizzare la mentalità "Mondo aperto" in cui l'applicazione possa essere in seguito ampliata con nuovi giochi e rompicapo o con l'aggiunta di nuovi beacons. Importante che si riesca a progettare qualcosa di portatile e non di ancorato al luogo di prima installazione.
  \item \textbf{Domanda}:\\
  È possibile decidere di far vedere all'utente un solo beacon alla volta anche se in realtà il telefono sappiamo che capta tutti quelli che sono captabili nel punto in cui si trova?\\
  \textbf{Risposta}: \\ 
  certamente. Tramite l'applicazione si può fare questa cosa e tutte le altre cose simili che vi vengono in mente.
  \item \textbf{Domanda}:\\
  potete fornirci un server web per dove installare il sito che useremo per il tracciamento dei requisiti e per altre eventuali cose che ci serviranno?\\
  \textbf{Risposta}: \\
  noi dell'azienda usiamo un server acquistato su amazon, se ce lo chiedete ve ne attiviamo uno senza nessun problema, tramite mail diteci che tipo di servizio vi interessa e provvederemo in massimo due giorni.Lo stesso vale per il servizio di storage offerto da amazon (Amazon s3) che se vi serve non abbiamo problemi a fornirvelo.
\end{enumerate}

\subsection{Altri Argomenti}
\label{sub:AltriArgomenti}

Sono state inoltre toccati i seguenti punti:

\begin{itemize}
  \item per qualsiasi aiuto o chiarimento \PROPONENTE è disponibile a fare videoconferenze o a sentirsi tramite mail;
  \item è stato ribadito che il nostro scopo è di creare un prodotto che possa vendere ed essere interessante sia per l'acquirente sia per l'utente.
\end{itemize}

\end{document}