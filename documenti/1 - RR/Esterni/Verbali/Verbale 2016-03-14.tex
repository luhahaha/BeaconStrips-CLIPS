\documentclass[a4paper,titlepage]{article}

\makeatletter
\def\input@path{{../../../template/}}
\makeatother

\usepackage{Comandi}
\usepackage{Riferimenti}
\usepackage{Stile}

\def\NOME{Verbale del Giorno 2016-03-14}
\def\VERSIONE{1}
\def\DATA{\today}
\def\REDATTORE{Luca Soldera}
\def\VERIFICATORE{Nome Cognome}
\def\RESPONSABILE{Viviana Alessio}
\def\USO{Esterno}
\def\DESTINATARI{\COMMITTENTE \\ & \CARDIN \\ & \PROPONENTE}
\def\SOMMARIO{Presentazione con \PROPONENTE}


\begin{document}

\maketitle

\newpage
\tableofcontents

\newpage
\section{Informazioni}
\label{sec:Informazioni}

\begin{itemize}
  \item \textbf{via Castelletto, 11
  	36016 Thiene (VI)}
  \item \textbf{Data: 2016-03-14}
  \item \textbf{Ora: 15:00}
  \item \textbf{Durata: 1 ora}
  \item \textbf{Partecipanti: Viviana Alessio, Luca Soldera, Matteo Franco, Andrea Grendene, Enrico Bellio, Tommaso Panozzo, Emanuele Righetto, Carlo Scattolin.}
\end{itemize}

\newpage
\section{Premessa}

La riunione è stata indetta a scopo conoscitivo tra il team e il proponente.

\section{Ordine Del Giorno}
\label{sec:OrdineDelGiorno}

\begin{enumerate}
  \item Presentazione tra \AUTORE{} e \PROPONENTE;
  \item domande di chiarimento riguardanti il capitolato;
  \item domanda di chiarimento sugli strumenti da utilizzare;
\end{enumerate}

\newpage
\section{Verbale}
\label{sec:Verbale}

\begin{enumerate}
  \item \textbf{Domanda}: \\
  vorremmo avere qualche delucidazione su cosa sia Ubiika e se sia utile per noi utilizzarla nel nostro progetto. \\
  \textbf{Risposta}:
  Ubiika è una pittaforma accessibile tramite un'app creata da Miriade che viene utilizzata dal utente per accedere ai contenuti che un beacon offre; non crediamo sia interessante che il progetto utilizzi questa piattaforma ma piuttosto che ne venga sviluppata una simile per gli scopi desiderati;\\
  \textbf{Domanda}: \\
  cosa si intende per sviluppo di un prototipo di software di navigazione e comunicazione all’interno dell’area definita?\\
  \textbf{Risposta}: \\
  la proposta del capitolato si può dividere in due idee:
  \begin{itemize}
  	\item quella basata sulla navigazione e quindi la ricerca di una nuova idea in cui i beacons vengano utilizzati per guidare l'utente in uno spazio chiuso; per questo si chiede di sviluppare un software, che tipicamente sarà un app, che possa aiutare l'utente medio a muoversi in determinati spazi, naturalmente si chiede un prototipo poichè non sarà necessario produrre un app completa e finita dato il tempo ridotto a vostra disposizione;
  	\item quella basata su una nuova idea commerciale, che può allontanarsi dalla navigazione, ma che sfrutti la possibilità dei beacons di dare contenuti all'utente.
  \end{itemize}
  A noi va bene ugualmente una delle due proposte.
  \item \textbf{Domanda}:\\
  Nella sezione 2.1 del capitolato riguardante l'indicazione dei risultati da conseguire, sono richiesti dei documenti aggiuntivi a quelli che noi dobbiamo produrre a fini del progetto?\\
  \textbf{Risposta}: \\ 
  Quello che interessa a noi è che, nel caso soprattutto della navigazione, venga fornita una documentazione dei test che sono stati fatti sul posizionamento dei beacons e un documento che spieghi come possano essere installati in un nuovo contesto, in modo da poter utilizzare al meglio ciò che voi avete prodotto;
  \item \textbf{Domanda}:\\
  Nel capitolato tra gli strumenti c'è scritto che potreste offrirci un repository, come potremmo accedervi?se usassimo GitHub dato che alcuni di noi già lo conoscono?\\
  \textbf{Risposta}: \\
  Non c'è nessun problema se volete utilizzare uno strumento che preferite,semplicemente noi utilizziamo una repository per il versionamento di tipo SVN e se volete possiamo darvi una parte di questa; naturalmente se utilizzerete una vostra repository vi chiederemo di darci modo di accedervi;
\end{enumerate}

\subsection{Altri Argomenti}
\label{sub:AltriArgomenti}

Sono state inoltre toccati i seguenti punti:

\begin{itemize}
  \item è stato proposto da \PROPONENTE, per i futuri incontri, di trovarsi nella loro sede di Padova;
\end{itemize}
\end{document}