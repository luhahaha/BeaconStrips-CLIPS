\begin{tabella}{!{\VRule}l!{\VRule}X[1,b,l]!{\VRule}l!{\VRule}}
	\intestazionethreecol{Test}{Descrizione}{Stato}
	TU1 & Si verifica che vengano restituiti i dati esatti & N.I. \\ % parse from local e from remote
	TU2 & Si verifica che vengano aggiornati i dati nel modo corretto & N.I.\\ % updatelocal
	TU3 & Si verifica che venga creata una richiesta idonea di dati & N.I. \\ % DataRequestMaker
	TU4 & Si verifica che venga creata una query adatta per la creazione della tabella desiderata & N.I.\\
	TU5 & Si verifica che vengano scritti in modo corretto i dati nel database & N.I.\\
	TU6 & Si verifica che vengano letti in modo corretto i dati dal database & N.I. \\
	TU7 & Si verifica che vengano cancellati in modo corretto i dati dal database & N.I. \\ 
	TU8 & Si verifica che vengano aggiornati in modo corretto i dati dal database & N.I. \\
	TU9 & Si verifica che venga impostato nel modo giusto il body dell'URLRequest & N.I.
	TU10 & Si verifica che il bottone possa essere cliccato e porti all'azione desiderata & N.I. \\ %setButtonInfoProfilo() in Authentication
	TU11 & Si verifica che vengano scritti in modo corretto i campi nell'interfaccia & N.I.\\
	TU12 & Si verifica che venga calcolato in modo giusto il punteggio totale & N.I\\
	TU13 & Si verifica che venga calcolata la distanza da un edificio data una posizione & N.I. \\ 
	TU14 & Si verifica che venga inserito l'edificio nell'array degli edifici & N.I.\\
	TU15 & Si verifica che venga prodotto l'array di edifici corretti dati latitudine e longitudine & N.I.\\
	TU16 & Si verifica che venga creato il giusto dato di tipo GregorianCalendar a partire dalla stringa data & N.I.\\
	TU17 & Si verifica che il controllo del risultato di una prova sia corretto & N.I.\\
	\rowcolor{white}
	\caption{Riepilogo test di sistema}
\end{tabella}
	
	
	
	