\subsubsection{Procedure}
\label{sec:3.1.1}
	\paragraph{Ciclo di vita}
	\label{sec:3.1.1.1}
		Ogni documento può trovarsi in tre fasi differenti:
		\begin{itemize}
			\item \textbf{In lavorazione}: un documento è in lavorazione quando vengono aggiunti, modificati o rimossi dei contenuti;
			\item \textbf{Da verificare}: quando un documento è completo dovrà essere preso in consegna dai verificatori, che si occuperanno di rilevare e/o correggere errori sintattici e semantici;
			\item \textbf{Approvato}: un documento, ultimata la verifica, deve essere approvato dal \RES. L'approvazione determina lo stato finale della versione del documento.
		\end{itemize}
		Le varie fasi vengono scandite dal sistema di ticketing (vedi sezione 4.1.1.3).
\subsubsection{Norme}
\label{sec:3.1.2}
	\paragraph{\gl{Template}}
	\label{sec:3.1.2.1}
		È presente una cartella nel \gl{repository} in \file{/documenti/template}. \\
		La cartella contiene i seguenti files:
		\begin{itemize}
			\item \textit{Comandi.sty}: contiene i comandi personalizzati (ad es. \texttt {\textbackslash gl});
			\item \textit{Riferimenti.sty}: all'interno del file sono presenti vari comandi da utilizzare come scorciatoie (ad es. \texttt{\textbackslash RES)};
			\item \textit{Stile.sty}: contiene gli stili grafici da applicare al \gl{template};
			\item \textit{TemplateDoc.tex}: è il \gl{template} da utilizzare per realizzare i documenti. Ogni documento dovrà essere realizzato a partire da questo file;
			\item \textit{TemplateVerbale.tex}: questo \gl{template} va impiegato per realizzare i verbali. Ogni verbale dovrà essere realizzato a partire da questo file;
			\item \textit{Glossario.sty}: questo file serve per impostare il \textit{Glossario}, che avrà una configurazione diversa dal \gl{template} standard. %controllare
		\end{itemize}
		È presente inoltre una cartella \textit{img} dove si trova il logo del team e dove potranno essere caricate tutte le altre immagini necessarie al \gl{template}. 
	\paragraph{Norme tipografiche}
	\label{sec:3.1.2.2}
		In questa sezione vengono definite le norme riguardanti l'ortografia e la tipografia, al fine di avere uno stile uniforme per tutti i documenti prodotti.
		\subparagraph{Stile del testo}
		\label{sec:3.1.2.2.1}
			\begin{itemize}
				\item \textbf{Grassetto}: il grassetto va utilizzato nei seguenti casi:
				\begin{itemize}
					\item titoli;
					\item elenchi puntati: può essere utilizzato il grassetto nel caso sia necessario evidenziare il concetto da sviluppare;
					\item altri casi: per evidenziare parole chiave o contenuti importanti.
				\end{itemize}
				\item \textbf{Corsivo}: lo stile corsivo si applica a:
				\begin{itemize}
					\item documenti;
					\item abbreviazioni.
				\end{itemize}
				\item \gl{\LaTeX}: ogni occorrenza di \gl{\LaTeX}{} va scritta con il comando \texttt{\textbackslash LaTeX}
				\item Maiuscolo: è possibile utilizzare lo stile maiuscolo \textbf{solo} per gli acronimi;
				\item \gl{Monospace}: le porzioni di testo scritte in \gl{monospace} definiscono:
				\begin{itemize}
					\item frammenti di codice;
					\item comandi;
					\item URL.
				\end{itemize} 
				\item Glossario: le parole che hanno un riferimento nel glossario sono in corsivo e hanno una 'g' a pedice.
			\end{itemize}
		\subparagraph{Punteggiatura}
		\label{sec:3.1.2.2.2}
			\begin{itemize}
				\item \textbf{Spaziature}: ogni simbolo di punteggiatura deve essere seguito da uno spazio, e mai preceduto;
				\item \textbf{Parentesi}: le parentesi costituiscono un'eccezione riguardo le spaziature, in quanto devono essere precedute da uno spazio, ma non seguite;
				\item \textbf{Virgolette singole}: le virgolette singole indicano un singolo carattere;
				\item \textbf{Virgolette doppie}: le virgolette doppie vanno utilizzate per indicare citazioni e documenti.
			\end{itemize}
		\subparagraph{Composizione del testo}
		\label{sec:3.1.2.2.3}
			\begin{itemize}
				\item \textbf{Elenchi puntati}: ogni punto dell'elenco termina con un punto e virgola, ad eccezione dell'ultimo che deve terminare con un punto fermo. \\
				Ogni punto inizia con la minuscola, tranne nel caso in cui necessiti di una spiegazione: allora si utilizzerà la maiuscola;
				\item \textbf{Nota a piè di pagina}: ogni nota inizia con l'iniziale della prima parola maiuscola e termina con il punto. Non ci sono spaziature tra il numero della nota e il testo.
			\end{itemize}
		\subparagraph{Formati}
		\label{sec:3.1.2.2.4}
			\begin{itemize}
				\item \textbf{Date}: per le date va utilizzata la notazione definita dallo standard \gl{ISO} 8601:2004:
				\highlight{\textit{AAAA -- MM -- GG}}
				dove:
				\begin{itemize}
					\item AAAA: rappresenta l'anno utilizzando quattro cifre;
					\item MM: rappresenta il mese utilizzando due cifre;
					\item GG: rappresenta il giorno utilizzando quattro cifre.
				\end{itemize}
				\item \textbf{Orari}: gli orari rappresentati dovranno essere in linea con lo standard \gl{ISO} 8601:2004:
				\highlight{\textit{HH:MM}}
				dove:
				\begin{itemize}
					\item HH: rappresenta l'ora utilizzando due cifre;
					\item MM: rappresenta i minuti utilizzando due cifre.
				\end{itemize}
				\item \textbf{Indirizzi}: per gli indirizzi verrà utilizzata la rappresentazione:
				\highlight{\textit{Destinatario o luogo - Via e numero, CAP, Comune(Provincia)}}
				Nel caso l'indirizzo si riferisca ad uno Stato estero sarà necessario aggiungere il nome dello Stato dopo la provincia.	
				\item \textbf{Sigle}: le sigle dei documenti vanno utilizzate solo nei diagrammi o nelle tabelle, con lo scopo di risparmiare spazio. Tali sigle sono:
				\begin{itemize}
					\item \textbf{AdR} per \ARdoc;
					\item \textbf{Gl} per \Gldoc;
					\item \textbf{NdP} per \NPdoc;
					\item \textbf{PdP} per \PPdoc;
					\item \textbf{PdQ} per \PQdoc;
					\item \textbf{SdF} per \SFdoc;
					\item \textbf{ST} per \STdoc.
				\end{itemize}	
				\item \textbf{Ruoli di \gl{progetto}}: per indicare i vari ruoli vanno utilizzati i seguenti comandi:
				\begin{itemize}
					\item \file{\textbackslash AM} per \AM;
					\item \file{\textbackslash AN} per \AN;
					\item \file{\textbackslash PR} per \PR;
					\item \file{\textbackslash PRJ} per \PRJ;
					\item \file{\textbackslash RES} per \RES;
					\item \file{\textbackslash VER} per \VER.
				\end{itemize}
				\item \textbf{Nomi propri}: per indicare nomi propri va utilizzata la forma \textit{Nome Cognome};
				\item \textbf{Nome del \gl{proponente}}: tramite il comando \texttt{\textbackslash PROPONENTE} ci si riferisce al \gl{proponente} "Miriade SpA";
				\item \textbf{Nome del \gl{progetto}}: con \texttt{\textbackslash PROGETTO} verrà citato il nome del \gl{progetto}, ovvero CLIPS;
				\item \textbf{Nome del \gl{committente}}: \texttt{\textbackslash COMMITTENTE} viene impiegato per riferirsi al "Prof. Tullio Vardanega".
			\end{itemize}
			
	\paragraph{Componenti grafiche}	
	\label{sec:3.1.2.3}	
		\subparagraph{Immagini}
		\label{sec:3.1.2.3.1}		
			Il formato preferito per le immagini è SVG (\gl{Scalable Vector Graphics}), in quanto è possibile garantire una qualità maggiore e sono ridimensionabili senza perdere qualità. Nel caso non sia possibile integrare questo formato, si raccomanda l'utilizzo di immagini in formato PNG (\gl{Portable Network Graphics}).
		\subparagraph{Tabelle}
		\label{sec:3.1.2.3.2}
			Le tabelle devono essere accompagnate da una didascalia e da un numero \gl{incrementale} per garantirne la tracciabilità.
		
	\paragraph{Struttura del documento}
	\label{sec:3.1.2.4}
		\subparagraph{Frontespizio}
		\label{sec:3.1.2.4.1}
			La prima pagina di ogni documento dovrà contenere le seguenti informazioni:
			\begin{itemize}
				\item nome del gruppo;
				\item nome del \gl{progetto};
				\item nome del documento e la sua versione;
				\item sommario;
				\item data di redazione;
				\item nome e cognome dei redattori del documento;
				\item nome e cognome dei verificatori del documento;
				\item nome e cognome del responsabile per l'approvazione del documento;
				\item destinazione d'uso del documento;
				\item lista di distribuzione del documento.
			\end{itemize}
		\subparagraph{Diario delle modifiche}
		\label{sec:3.1.2.4.2}
			La seconda pagina di ogni documento dovrà contenere il diario delle modifiche.\\
			Il diario consiste in una tabella ordinata in modo decrescente secondo la data di modifica e, conseguentemente, al numero di versione. È particolarmente utile per tenere traccia delle varie modifiche effettuate nel documento. \\
			Ogni riga del diario conterrà:
			\begin{itemize}
				\item numero di versione;
				\item breve riepilogo delle modifiche effettuate;
				\item autore delle modifiche;
				\item ruolo ricoperto dall'autore;
				\item data di modifica.
			\end{itemize}		
		\subparagraph{Indici}
		\label{sec:3.1.2.4.3}
			In ogni documento si trova un indice delle sezioni, il quale fornisce una visione macroscopica della struttura del documento. Nel caso comparissero tabelle e figure nel documento, potranno essere presenti i rispettivi indici, che dovranno apparire nel seguente ordine: indice delle sezioni, indice delle tabelle ed indice delle figure.
		\subparagraph{Intestazione e piè di pagina}
		\label{sec:3.1.2.4.4}
			L'intestazione di ogni pagina contiene i seguenti elementi:
			\begin{itemize}
				\item numero della sezione;
				\item titolo della sezione.
			\end{itemize}
			A piè di pagina invece si trovano:
			\begin{itemize}
				\item nome del documento e numero di versione, allineato a sinistra;
				\item nome del team, allineato al centro;
				\item pagina X di Y, dove X è la pagina corrente e Y è il numero di pagine totali, allineato a destra.
			\end{itemize}
	
	\paragraph{Verbali}
	\label{sec:3.1.2.5}
		I verbali verranno utilizzati per tenere traccia di informazioni importanti emerse durante alcune riunioni interne ed esterne. \\
		Nella cartella \file{template} è presente un \gl{template} da utilizzare per redigere i verbali e sono presenti le seguenti sezioni:
		\begin{itemize}
			\item \textbf{Informazioni}: vanno indicate \textbf{sempre} le seguenti informazioni:
			\begin{itemize}
				\item luogo;
				\item data;
				\item ora;
				\item durata;
				\item partecipanti.
			\end{itemize}
			\item \textbf{Premessa}: la premessa consiste in una breve descrizione del motivo della riunione;
			\item \textbf{Ordine del giorno}: in questa sezione vanno indicati i vari punti da trattare durante la riunione;
			\item \textbf{Verbale}: nella sezione \textit{Verbale} verranno trascritte tutte le informazioni rilevanti trattate durante la riunione.
		\end{itemize}	
	\paragraph{Versionamento}
	\label{sec:3.1.2.6}
		I documenti prodotti sono soggetti al seguente versionamento attraverso la seguente codifica:
		\highlight{vX.Y.Z}
		dove:
		\begin{itemize}
			\item \textbf{X}: indica il numero crescente di uscite formali del documento;
			\item \textbf{Y}: indica il numero crescente di modifiche sostanziali al documento quali stesura, verifica, approvazione;
			\item \textbf{Z}: indica il numero crescente di modifiche minori effettuate sul documento.
		\end{itemize}
		I documenti vanno citati sempre con una versione specifica del documento, utilizzando i comandi:
		\begin{itemize}
			\item \texttt{\textbackslash ARdoc} per il documento \ARdoc;
			\item \texttt{\textbackslash Gldoc} per il documento \Gldoc;
			\item \texttt{\textbackslash NPdoc} per il documento \NPdoc;
			\item \texttt{\textbackslash PPdoc} per il documento \PPdoc;
			\item \texttt{\textbackslash PQdoc} per il documento \PQdoc;
			\item \texttt{\textbackslash SFdoc} per il documento \SFdoc;
			\item \texttt{\textbackslash STdoc} per il documento \STdoc.
		\end{itemize}
		Il formato da utilizzare per la creazione di file è:
		\highlight{\file{NomeDocumento\_vX.Y.Z.pdf}}

	\paragraph{Strumenti}
	\label{sec:3.1.2.7}
		\subparagraph{\gl{\LaTeX}}
		\label{sec:3.1.2.7.1}
			Per redigere la documentazione si è scelto di utilizzare il linguaggio di markup \gl{\LaTeX}, in quanto: \\
			\begin{itemize} 
				\item permette di produrre documenti di alta qualità rispetto ai word processor tradizionali; 
				\item rende possibile creare dei \gl{template} comuni per i documenti;
				\item permette una separazione tra contenuto e formattazione;
				\item è estendibile ed altamente personalizzabile tramite l'utilizzo di pacchetti specifici;
				\item è \gl{multipiattaforma}, essendo un file sorgente di \gl{\LaTeX} una semplice codifica \gl{ASCII};
				\end{itemize}
				Come editor di file \gl{LaTeX} è consigliato l'uso di \textit{TeXstudio}, anch'esso \gl{multipiattaforma}.	
		\subparagraph{Correttore ortografico}
		\label{sec:3.1.2.7.2}
			In TeXstudio è presente un correttore ortografico automatico che permette la correzione in tempo reale. I file da utilizzare sono presenti nella cartella condivisa in \gl{Google Drive} all'indirizzo \file{/dizionari}.
		\subparagraph{Script}
		\label{sec:3.1.2.7.3}
			Su Github è presente una cartella \texttt{/\gl{script}} dove sono presenti degli \gl{script} utilizzabili dal gruppo per semplificare alcune operazioni.