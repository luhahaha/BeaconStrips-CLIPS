\section{Diagrammi di Sequenza}
\label{sec:Diagrammi di Sequenza}

In questa sezione saranno riportati i diagrammi di sequenza riguardo alle operazioni dell'applicazione ritenute significative che descrivono le interazioni tra attori ed oggetti o entità di sistema.

\subsection{Richiesta lista edifici abilitati}
\label{sub:Richiesta lista edifici abilitati}

\includegraphics[scale=0.28]{img/diagrammiSequenza/ricercaEdifici.png}
\caption{Diagramma di Sequenza della richiesta di edifici abilitati ai percorsi}.
\end{figure}

Il diagramma in figura illustra come viene gestita la visualizzazione nel client della lista di edifici abilitati nelle vicinanze.
La sequenza ha inizio ogni volta che l'utente accede alla schermata di ricerca degli edifici abilitati dal proprio terminale.

L'activity che si occupa di gestire la lista degli edifici (\texttt{BuildingsActivity}) richiede i dati attraverso la classe \texttt{DataManager}, situata nel package \texttt{CLIPS::data::datamanager}.
A questo punto il \texttt{DataManager} interrogherà sia il database locale alla ricerca di dati in cache (per fornire nel più breve tempo possibile i dati all'utente), sia il server (per ottenere i dati più aggiornati).
Solitamente il database locale risponderà in tempi brevissimi, ritornando quindi l'informazione richiesta al \texttt{DataManager} che provvederà a inviarla alla activity che l'ha richiesta.
Intanto la richiesta inviata al server viene instradata dal \texttt{Server} alla corretta classe che si occuperà di interrogare il DB remoto per ottenere i dati più aggiornati.
Tali dati vengono poi inviati dal server fino al \texttt{DataManager} e poi di nuovo all'activity, che avrà ora i dati più aggiornati da mostrare all'utente.
