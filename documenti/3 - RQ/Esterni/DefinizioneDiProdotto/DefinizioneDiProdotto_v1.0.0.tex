\documentclass[a4paper,titlepage]{article}

\makeatletter
\def\input@path{{../../../template/}{./img}}
\makeatother

\usepackage{Comandi}
\usepackage{Riferimenti}
\usepackage{Stile}

\usepackage{eurosym}
\usepackage{comment}
\usepackage{hyperref}

% -- COMANDI PER LE COMPONENTI -- %
\newenvironment{componente}[1]
{
	\subsubsection{Componente \texttt{#1}}
	\label{#1}
	Informazioni sul package
	\begin{itemize}
}{
	\end{itemize}
}
\newcommand{\compDescrizione}[1] {
	\item \textbf{Descrizione}: {#1}
}
\newcommand{\compPadre}[1] {
	\item \textbf{Padre}: {#1}
}
\newcommand{\compUtilizzo}[1] {
	\item \textbf{Utilizzo}: {#1}
}
\newenvironment{compPackageContenuti}[1] {
	\item \textbf{Package contenuti}: {#1}
	\begin{itemize}
} {
	\end{itemize}
}
\newenvironment{compClassi} [1] {
	\item \textbf{Classi}: {#1}
	\begin{itemize}
} {
	\end{itemize}
}
\newenvironment{classe} [1] {
	\item[] \texttt{#1} \\
	\begin{itemize}
} {
	\end{itemize}
}
\newcommand{\classeDescrizione} [1] { \\
	\item \textbf{Descrizione}: {#1}
}
\newcommand{\classeUtilizzo} [1] { \\
	\item \textbf{Utilizzo}: {#1}
}
\newenvironment{classeAttributi} [1] { \\
	\item \textbf{Attributi}: {#1}
	\begin{itemize}
} {
	\end{itemize}
}
\newcommand{\classeAttributo}[3] { \item \texttt{{#1}: {#2}} {#3}}
\newenvironment{classeMetodi} [1] {
	\item \textbf{Metodi}: {#1}
	\begin{itemize}
} {
	\end{itemize}
}
\newcommand{\classeMetodo}[4]{
	\item \texttt{{#1}}%({#2}) : {#3}} \\ {#4}
}
\newenvironment{classeMetodoArgomenti} {
	\item \textbf{Argomenti}:
	\begin{itemize}
}{
	\end{itemize}
}
\newcommand{\classeMetodoArgomento}[3]{\item \texttt{{#1} : {#2}} \\ {#3} }
\newenvironment{classeRelazioni} { \\
	\item \textbf{Relazioni con altre classi}:
	\begin{itemize}
} {
	\end{itemize}
}
\newcommand{\classeRelazione}[3]{\item \textt{#1::#2}: {#3}}

\def\NOME{Definizione di Prodotto}
\def\VERSIONE{1.0.0}
\def\DATA{\TODO}
\def\REDATTORE{Viviana Alessio \\ & Matteo Franco \\ & Andrea Grandene \\ & Luca Soldera \\ & Tommaso Panozzo \\ & Enrico Bellio}
\def\VERIFICATORE{Matteo Franco \\ & Luca Soldera}
\def\RESPONSABILE{Tommaso Panozzo}
\def\USO{Esterno}
\def\DESTINATARI{\COMMITTENTE \\ & \CARDIN \\ & \PROPONENTE}
\def\SOMMARIO{Descrizione approfondita dell'architettura del \gl{progetto} \PROGETTO.}

\begin{document}


	\maketitle

	\begin{diario}
		\modifica{Luca Soldera}{\PRJ}{Aggiunta sezione \hyperref[strutturaDelDatabase]{Struttura del database}}{2016-07-15}{0.0.2}
		\modifica{Tommaso Panozzo}{\RES}{Creazione documento}{2016-07-10}{0.0.1}
	\end{diario}

	\newpage
	\tableofcontents
	\newpage
	\listoftables
	\newpage
	\listoffigures
	\newpage

	\section{Introduzione}
	\subsection{Scopo del documento} 
	Questo documento ha lo scopo di spiegare dettagliatamente le strategie secondo cui il gruppo \gl{Beacon Strips} intende condurre il progetto didattico.
	\subsection{Riferimenti}
	\subsection{Ciclo di vita}
	\subsection{Scadenze}
	

   \section{Standard di Progetto}
\label{sec:standardProgetto}

\subsection{Standard di progettazione architetturale}
\label{sub:Standard di progettazione architetturale}

\subsection{Standard di documentazione del codice}
\label{sub:Standard di documentazione del codice}


\subsection{Standard di denominazione di entità e relazioni}
\label{sub:Standard di denominazione di entità e relazioni}


\subsection{Standard di programmazione}
\label{sub:Standard di programmazione}


\subsection{Strumenti di lavoro}
\label{sub:Strumenti di lavoro}


   %% qui vanno le stampe da trender

	\section{Struttura del Database}
\label{strutturaDelDatabase}
Viene illustrato di seguito lo schema del database \gl{MySQL} presente nella parte server dell'applicazione.\\
Essendo un database relazionale le tabelle sono collegate da frecce che rappresentano le varie relazioni tra di esse; sono inoltre presentati i vari campi della tabella con relativo tipo.

\begin{figure}[!h]
	\centering
	\includegraphics[scale=0.33]{img/Database_Schema}  
	\caption{Struttura del database dell'applicazione}
\end{figure}

   %% qui vanno i diagrammi di sequenza

   %% qui vanno le tabelle requisito - package

   %% qui quelle package - requisito

\end{document}
