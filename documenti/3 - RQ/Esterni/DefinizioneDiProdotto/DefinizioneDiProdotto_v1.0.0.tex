\documentclass[a4paper,titlepage]{article}

\makeatletter
\def\input@path{{../../../template/}{./img}}
\makeatother

\usepackage{Comandi}
\usepackage{Riferimenti}
\usepackage{Stile}

\usepackage{eurosym}
\usepackage{comment}
\usepackage{hyperref}

\usepackage{enumitem}
\setlist[itemize,1]{label=\textbullet}
\setlist[itemize,1]{label=\normalfont\bfseries \textendash}
\setlist[itemize,1]{label=\textasteriskcentered}
\setlist[itemize,1]{label=\textperiodcentered}
\setlist[itemize,5]{label=$ $}
\setlist[itemize,6]{label=$\bullet$}
\setlist[itemize,7]{label=$\bullet$}
\setlist[itemize,8]{label=$\bullet$}
\setlist[itemize,9]{label=$\bullet$}

\renewlist{itemize}{itemize}{9}

% -- COMANDI PER LE COMPONENTI -- %
\newenvironment{componente}[1]
{
	\subsubsection{Componente \texttt{#1}}
	\label{#1}
	\textbf{Informazioni sul package}
	\begin{itemize}
}{
	\end{itemize}
}
\newcommand{\compDescrizione}[1] {
	\item \textbf{Descrizione}: {#1}
}
\newcommand{\compPadre}[1] {
	\item \textbf{Padre}: {#1}
}
\newcommand{\compUtilizzo}[1] {
	\item \textbf{Utilizzo}: {#1}
}
\newenvironment{compPackageContenuti} {
	\item \textbf{Package contenuti}:% {#1}
	\begin{itemize}
} {
	\end{itemize}
}
\newenvironment{compClassi} [1] {
	\item \textbf{Classi}: {#1}
	\begin{itemize}
} {
	\end{itemize}
}
\newenvironment{classe} [1] {
	\item[] \texttt{#1} \\
	\begin{itemize}
} {
	\end{itemize}
}
\newcommand{\classeDescrizione} [1] { \\
	\item \textbf{Descrizione}: {#1}
}
\newcommand{\classeUtilizzo} [1] { \\
	\item \textbf{Utilizzo}: {#1}
}
\newenvironment{classeAttributi} { \\
	\item \textbf{Attributi}:
	\begin{itemize}
} {
	\end{itemize}
}
\newcommand{\classeAttributo}[3] { \item \texttt{{#1}: {#2}} {#3}}
\newenvironment{classeMetodi} {
	\item \textbf{Metodi}:
	\begin{itemize}
} {
	\end{itemize}
}
\newcommand{\classeMetodo}[4]{
	\item \texttt{{#1}({#2}) : {#3}} \\ {#4}
}
\newenvironment{classeMetodoArgomenti} {
   \begin{itemize}
	\item \textit{Argomenti}:
   \begin{itemize}
}{
   \end{itemize}
	\end{itemize}
}
\newcommand{\classeMetodoArgomento}[3]{\item \texttt{{#1} : {#2}} {#3} }
\newenvironment{classeRelazioni} { \\
	\item \textbf{Relazioni con altre classi}:
	\begin{itemize}
} {
	\end{itemize}
}
\newcommand{\classeRelazione}[3]{\item \texttt{#1::#2}: {#3}}

\def\NOME{Definizione di Prodotto}
\def\VERSIONE{1.0.0}
\def\DATA{\TODO}
\def\REDATTORE{Viviana Alessio \\ & Matteo Franco \\ & Andrea Grandene \\ & Luca Soldera \\ & Tommaso Panozzo \\ & Enrico Bellio}
\def\VERIFICATORE{Matteo Franco \\ & Luca Soldera}
\def\RESPONSABILE{Tommaso Panozzo}
\def\USO{Esterno}
\def\DESTINATARI{\COMMITTENTE \\ & \CARDIN \\ & \PROPONENTE}
\def\SOMMARIO{Descrizione approfondita dell'architettura del \gl{progetto} \PROGETTO.}

\begin{document}


	\maketitle

	\begin{diario}
      \modifica{Tommaso Panozzo}{\RES}{Approvazione documento}{2016-08-17}{1.0.0}    
      \modifica{Matteo Franco}{\PRJ}{Verifica documento}{2016-08-12}{0.7.0}
      \modifica{Enrico Bellio}{\PRJ}{Inserimento immagini per le varie componenti}{2016-08-10}{0.6.1}
      \modifica{Luca Soldera}{\PRJ}{Verifica documento}{2016-08-03}{0.6.0}
      \modifica{Matteo Franco}{\PRJ}{Correzioni sezione ``Componente CLIPS::server::dataserver''}{2016-08-02}{0.5.4}
      \modifica{Enrico Bellio}{\PRJ}{Correzioni sezione ``Componente CLIPS::client::datamanager''}{2016-08-01}{0.5.3}
      \modifica{Andrea Grandene}{\PRJ}{Correzioni sezione ``Componente CLIPS::client::data''}{2016-08-01}{0.5.2}
      \modifica{Matteo Franco}{\PRJ}{Correzioni sezione ``Componente CLIPS::client::viewcontroller''}{2016-08-01}{0.5.1}
      \modifica{Luca Soldera}{\PRJ}{Verifica documento}{2016-07-30}{0.5.0}
      \modifica{Tommaso Panozzo}{\PRJ}{Stesura sezione ``Componente CLIPS::server::urlrequesthandler''}{2016-07-29}{0.4.4}
      \modifica{Luca Soldera}{\PRJ}{Stesura sezione ``Componente CLIPS::server::dataserver''}{2016-07-29}{0.4.3}
      \modifica{Luca Soldera}{\PRJ}{Stesura sezione ``Componente CLIPS::server''}{2016-07-28}{0.4.2}
      \modifica{Tommaso Panozzo}{\PRJ}{Stesura sezione ``Componente CLIPS::client::data::urlrequest''}{2016-07-27}{0.4.1}
      \modifica{Luca Soldera}{\PRJ}{Verifica documento}{2016-07-26}{0.4.0}
      \modifica{Tommaso Panozzo}{\PRJ}{Stesura sezione ``Componente CLIPS::client::pathprogress''}{2016-07-25}{0.3.2}
      \modifica{Andrea Grandene}{\PRJ}{Stesura sezione ``Componente CLIPS::client::data'' e ``Componente CLIPS::client::data::datamanager''}{2016-07-25}{0.3.1}
      \modifica{Matteo Franco}{\PRJ}{Verifica documento}{2016-07-24}{0.3.0}
      \modifica{Luca Soldera}{\PRJ}{Stesura sezione ``Componente CLIPS::client::authentication''}{2016-07-23}{0.2.4}
      \modifica{Matteo Franco}{\PRJ}{Ampliamento sezione ``Componente CLIPS::client::viewcontroller''}{2016-07-23}{0.2.3}
      \modifica{Viviana Alessio}{\PRJ}{Ampliamento sezione ``Componente CLIPS::client::viewcontroller''}{2016-07-22}{0.2.2}
      \modifica{Viviana Alessio}{\PRJ}{Stesura sezione ``Componente CLIPS::client::viewcontroller''}{2016-07-20}{0.2.1}
      \modifica{Matteo Franco}{\PRJ}{Verifica sezione ``Standard di progetto'' }{2016-07-19}{0.2.0}
      \modifica{Enrico Bellio}{\PRJ}{Ampliata sezione ``Standard di progetto'' }{2016-07-18}{0.1.1}
      \modifica{Luca Soldera}{\PRJ}{Aggiunta sezione ``Standard di progetto'' }{2016-07-17}{0.1.0}
		\modifica{Luca Soldera}{\PRJ}{Aggiunta sezione \hyperref[sub:Struttura del database]{Struttura del database}}{2016-07-15}{0.0.2}
		\modifica{Tommaso Panozzo}{\RES}{Creazione documento}{2016-07-10}{0.0.1}
	\end{diario}

	\newpage
	\tableofcontents
	\newpage
	\listoftables
	\newpage
	\listoffigures
	\newpage

	\section{Introduzione}
	\subsection{Scopo del documento} 
	Questo documento ha lo scopo di spiegare dettagliatamente le strategie secondo cui il gruppo \gl{Beacon Strips} intende condurre il progetto didattico.
	\subsection{Riferimenti}
	\subsection{Ciclo di vita}
	\subsection{Scadenze}
	

   %% qui va indicato dove stanno gli standard del progetto
   \section{Standard di Progetto}
\label{sec:standardProgetto}

\subsection{Standard di progettazione architetturale}
\label{sub:Standard di progettazione architetturale}

\subsection{Standard di documentazione del codice}
\label{sub:Standard di documentazione del codice}


\subsection{Standard di denominazione di entità e relazioni}
\label{sub:Standard di denominazione di entità e relazioni}


\subsection{Standard di programmazione}
\label{sub:Standard di programmazione}


\subsection{Strumenti di lavoro}
\label{sub:Strumenti di lavoro}


   %% qui vanno le componenti (descrizioni delle classi e della struttura del DB)
   \section{Specifica delle componenti}
\label{sec:Specifica delle componenti}

\subsection{Specifica dei componenti}
\label{sub:Specifica dei componenti}

\componente{CLIPS}
\compDescrizione{package generale contenente il prodotto del progetto}
\begin{compPackageContenuti}
	\item CLIPS::client: componente globale per il front end del prodotto che utilizza il design pattern \gl{MVC}. Si occupa di fornire un'interfaccia grafica dell'applicazione e di interagire con il lato server.
	\item CLIPS::server: componente globale per il back end del prodotto
	\end{compPackageContenuti}
	\end{componente}
	\componente{CLIPS::client}
	\compDescrizione{componente globale per il front end del prodotto che utilizza il design pattern \gl{MVC}. Si occupa di fornire un'interfaccia grafica dell'applicazione e di interagire con il lato server.}
	\compPadre{CLIPS}
	\begin{compPackageContenuti}
		\item CLIPS::client::authentication: componente che si occupa di gestire l'autenticazione dell'utente
		\end{compPackageContenuti}
		\end{componente}
		\componente{CLIPS::client::authentication}
		\compDescrizione{componente che si occupa di gestire l'autenticazione dell'utente}
		\compPadre{client}
		\begin{compClassi}
			\begin{classe}{CLIPS::client::authentication::ForgotPasswordView}
				\classeDescrizione{classe che si occupa della visualizzazione della schermata per la richiesta di una nuova password}
				\classeUtilizzo{consente all'utente di inserire la mail per ricevere una nuova password}
				\end{classe}\begin{classe}{CLIPS::client::authentication::LoggedUser}
				\classeDescrizione{classe che si occupa di memorizzare in locale i dati dell'utente loggato}
				\classeUtilizzo{permette il salvataggio in locale dei dati di un utente loggato}
				\end{classe}\begin{classe}{CLIPS::client::authentication::LoginView}
				\classeDescrizione{classe che si occupa della visualizzazione della schermata per il login}
				\classeUtilizzo{consente all'utente di inserire i propri dati per effettuare il login}
				\end{classe}\begin{classe}{CLIPS::client::authentication::RegistrationView}
				\classeDescrizione{classe che si occupa della visualizzazione della schermata per la registrazione}
				\classeUtilizzo{consente all'utente di inserire i propri dati per effettuare la registrazione}
				\end{classe}\begin{classe}{CLIPS::client::authentication::UpdateUserInfoView}
				\classeDescrizione{classe che si occupa della visualizzazione della schermata per il cambio delle credenziali}
				\classeUtilizzo{consente all'utente di inserire i nuovi dati per cambiare le sue credenziali}
				\end{classe}\end{compClassi}
				\end{componente}
				\componente{CLIPS::server}
				\compDescrizione{componente globale per il back end del prodotto}
				\compPadre{CLIPS}
				\end{componente}
				\componente{gamelogic}
				\compDescrizione{componente che si occupa di tutte le componenti riguardanti il gioco.}
				\begin{compPackageContenuti}
					\item gamelogic::buildings: componete che gestisce le informazioni e le interazioni dell'utente con gli edifici abilitati
					\item gamelogic::paths: componente che gestisce i dati dei percorsi giocabili dall'utente
					\end{compPackageContenuti}
					\end{componente}
					\componente{gamelogic::buildings}
					\compDescrizione{componete che gestisce le informazioni e le interazioni dell'utente con gli edifici abilitati}
					\compPadre{gamelogic}
					\end{componente}
					\componente{gamelogic::paths}
					\compDescrizione{componente che gestisce i dati dei percorsi giocabili dall'utente}
					\compPadre{gamelogic}
					\begin{compClassi}
						\begin{classe}{gamelogic::paths::Path}
							\classeDescrizione{classe che si occupa di salvare in locale i dati riguardanti un percorso}
							\classeUtilizzo{permette di salvare i dati di un percorso in locale}
							\end{classe}\begin{classe}{gamelogic::paths::PathInfo}
							\classeDescrizione{classe che si occupa di salvare in locale le informazioni generali di un percorso}
							\classeUtilizzo{consente di salvare in locale le informazioni generali di un percorso}
							\end{classe}\begin{classe}{gamelogic::paths::PathView}
							\classeDescrizione{classe che si occupa della visualizzazione della schermata riguardante un percorso}
							\classeUtilizzo{consente all'utente di visualizzare le informazioni riguardanti un percorso e se l'utente si trova nell'edificio del percorso consente di iniziarlo}
							\end{classe}\end{compClassi}
							\end{componente}
							\componente{games}
							\compDescrizione{componente che gestisce le prove che l'utente deve completare all'interno di un percorso}
							\begin{compClassi}
								\begin{classe}{games::MultipleChoiceQuiz}
									\classeDescrizione{classe per il quiz a risposta multipla}
									\classeUtilizzo{si occupa di fornire un'interfaccia per il quiz a risposta multipla}
									\begin{classeAttributi}
										\classeAttributo{answerButtons}{void}{una lista di buttons per visualizzare le possibili risposte}
										\end{classeAttributi}
										\begin{classeMetodi}
											\classeMetodo{buttonPressed}{atIndex}{void}{segnala al controller il button premuto}
											\begin{classeMetodoArgomenti}
												\classeMetodoArgomento{atIndex}{int}{indica l'indice della risposta selezionata}
												\end{classeMetodoArgomenti}
												\end{classeMetodi}
												\end{classe}\begin{classe}{games::QuizResultView}
												\classeDescrizione{classe per la visualizzazione del risultato di un quiz}
												\classeUtilizzo{fornisce all'utente un'interfaccia affinché visualizzi il risultato del quiz}
												\begin{classeAttributi}
													\classeAttributo{continueButton}{void}{button per chiudere la schermata e continuare il percorso}
													\classeAttributo{feedbackLabel}{string}{mostra la frase di successo/fallimento del quiz}
													\end{classeAttributi}
													\begin{classeMetodi}
														\classeMetodo{continueButtonPressed}{}{void}{notifica il controller che il button per continuare è stato premuto}
														\classeMetodo{showFailureResult}{correctAnswer}{void}{mostra la risposta corretta se il quiz è stato fallito}
														\begin{classeMetodoArgomenti}
															\classeMetodoArgomento{correctAnswer}{string}{indica la risposta corretta}
															\end{classeMetodoArgomenti}
															\classeMetodo{showSuccessfulResult}{score}{void}{mostra il risultato ottenuto se il quiz è stato superato}
															\begin{classeMetodoArgomenti}
																\classeMetodoArgomento{score}{int}{indica il punteggio ottenuto}
																\end{classeMetodoArgomenti}
																\end{classeMetodi}
																\end{classe}\begin{classe}{games::QuizView}
																\classeDescrizione{classe base per i quiz}
																\classeUtilizzo{fornisce una base per i vari tipi di test da istanziare}
																\begin{classeAttributi}
																	\classeAttributo{questionLabel}{string}{rappresenta la domanda da porre nel quiz}
																	\end{classeAttributi}
																	\end{classe}\begin{classe}{games::StrangersQuiz}
																	\classeDescrizione{classe per il quiz strana coppia}
																	\classeUtilizzo{si occupa di fornire un'interfaccia per la prova strana coppia}
																	\begin{classeAttributi}
																		\classeAttributo{answerCheckBox}{void}{fornisce una lista di risposte da selezionare tramite checkbox}
																		\classeAttributo{confirmButton}{void}{button grafico per confermare la selezione delle risposte}
																		\end{classeAttributi}
																		\begin{classeMetodi}
																			\classeMetodo{confirmButtonPressed}{}{void}{notifica il controller che è stato premuto il tasto conferma e quindi si può procedere alla valutazione delle risposte}
																			\end{classeMetodi}
																			\end{classe}\begin{classe}{games::TestResultView}
																			\classeDescrizione{classe che fornisce una base per la visualizzazione del risultato della prova}
																			\classeUtilizzo{permette all'utente di visualizzare il risultato della prova}
																			\begin{classeMetodi}
																				\classeMetodo{showResult()}{}{void}{restituisce la view con il risultato}
																				\end{classeMetodi}
																				\end{classe}\begin{classe}{games::TestView}
																				\classeDescrizione{questa classe fornisce una base dalla quale è possibile creare vari tipi di giochi }
																				\classeUtilizzo{viene utilizzata per visualizzare un'interfaccia di gioco all'utente}
																				\begin{classeMetodi}
																					\classeMetodo{showTest}{}{TestView}{restituisce l'interfaccia grafica del test}
																					\end{classeMetodi}
																					\end{classe}\begin{classe}{games::TrueFalseQuiz}
																					\classeDescrizione{classe per il quiz vero/falso}
																					\classeUtilizzo{si occupa di fornire un'interfaccia per la prova di tipo vero/falso}
																					\begin{classeAttributi}
																						\classeAttributo{falseButton}{void}{button grafico per rispondere falso al quiz}
																						\classeAttributo{trueButton}{void}{button grafico per rispondere vero al quiz}
																						\end{classeAttributi}
																						\begin{classeMetodi}
																							\classeMetodo{falseButtonPressed}{}{void}{questo metodo si occupa di notificare al controller che è stato premuto falseButton}
																							\classeMetodo{trueButtonPressed}{}{void}{questo metodo si occupa di notificare al controller che è stato premuto trueButton}
																							\end{classeMetodi}
																							\end{classe}\end{compClassi}
																							\end{componente}
																							



   %% qui vanno i diagrammi di sequenza
   \section{Diagrammi di Sequenza}
\label{sec:Diagrammi di Sequenza}

In questa sezione saranno riportati i diagrammi di sequenza riguardo alle operazioni dell'applicazione ritenute significative che descrivono le interazioni tra attori ed oggetti o entità di sistema.

\subsection{Richiesta lista edifici abilitati}
\label{sub:Richiesta lista edifici abilitati}

\includegraphics[scale=0.28]{img/diagrammiSequenza/ricercaEdifici.png}
\caption{Diagramma di Sequenza della richiesta di edifici abilitati ai percorsi}.
\end{figure}

Il diagramma in figura illustra come viene gestita la visualizzazione nel client della lista di edifici abilitati nelle vicinanze.
La sequenza ha inizio ogni volta che l'utente accede alla schermata di ricerca degli edifici abilitati dal proprio terminale.

L'activity che si occupa di gestire la lista degli edifici (\texttt{BuildingsActivity}) richiede i dati attraverso la classe \texttt{DataManager}, situata nel package \texttt{CLIPS::data::datamanager}.
A questo punto il \texttt{DataManager} interrogherà sia il database locale alla ricerca di dati in cache (per fornire nel più breve tempo possibile i dati all'utente), sia il server (per ottenere i dati più aggiornati).
Solitamente il database locale risponderà in tempi brevissimi, ritornando quindi l'informazione richiesta al \texttt{DataManager} che provvederà a inviarla alla activity che l'ha richiesta.
Intanto la richiesta inviata al server viene instradata dal \texttt{Server} alla corretta classe che si occuperà di interrogare il DB remoto per ottenere i dati più aggiornati.
Tali dati vengono poi inviati dal server fino al \texttt{DataManager} e poi di nuovo all'activity, che avrà ora i dati più aggiornati da mostrare all'utente.


   %% qui vanno le tabelle package - requisito
   \section{Tabella tracciamento Requisiti-Componenti}

\begin{tabella}{!{\VRule}c!{\VRule}X[c,b,c]!{\VRule}c!{\VRule}}
	\intestazionethreecol{Requisiti}{Descrizione}{Componenti}
	
	R0F1  & L'utente deve poter possedere un account registrato nel database &    \cellacaporiga{dataserver \\ server \\} \\
	
	R0F1.1 & Il sistema deve permettere all'utente di autenticarsi  & \cellacaporiga{authentication \\
	client \\
	data \\
	datamanager \\
	server \\
	urlrequest \\
	urlrequesthandler \\} \\

	R0F1.1.1  &  Il sistema deve chiedere all'utente l'email per la procedura di autenticazione & \cellacaporiga{authentication \\
	client \\ } \\
	
	R0F1.1.2  &  Il sistema deve chiedere all'utente la password per la procedura di autenticazione & \cellacaporiga{authentication \\
		client \\ } \\
	
	R0F1.1.3  &  Il sistema deve permettere all'utente di confermare i dati inseriti per la procedura di autenticazione & \cellacaporiga{authentication \\
		client \\ } \\
	
	R0F1.1.4  &  Se i dati confermati dall'utente sono validi, egli viene autenticato  &  \cellacaporiga{authentication \\
		client \\ data \\ datamanager \\ server \\ urlrequest \\ urlrequesthandler } \\
	
	R0F1.1.5  &  Il sistema deve segnalare all'utente ogni eventuale errore durante la procedura di autenticazione  & \cellacaporiga{authentication \\
		client \\ data \\ datamanager \\ server \\ urlrequest \\ urlrequesthandler } \\
	
	R0F1.1.5.1 & Il sistema interrompe la procedura di autenticazione se l'email non corrisponde a nessuna di quelle registrate & \cellacaporiga{authentication \\
		client \\ data \\ datamanager \\ server \\ urlrequest \\ urlrequesthandler } \\
	
	R0F1.1.5.2 & Il sistema interrompe la procedura di autenticazione se la password non corrisponde a quella associata all'email  &  \cellacaporiga{authentication \\
		client \\ data \\ datamanager \\ server \\ urlrequest \\ urlrequesthandler } \\
	
	R0F1.1.5.3 & L'applicazione informa l'utente dell'errore nell'inserimento dei dati & \cellacaporiga{authentication \\
		client \\} \\
	
	R0F1.1.5.4  & L'utente può chiedere una nuova password inserendo l'email & authentication \\
	
	R0F1.1.5.4.1 & L'utente deve inserire l'email del proprio account per proseguire al recupero delle credenziali & \cellacaporiga{authentication \\
		client \\} \\
	
	R0F1.1.5.4.2  &  L'utente riceve un'email all'indirizzo inserito con una password casuale. &  \cellacaporiga{authentication \\
		client \\} \\
	
	R0F1.1.5.4.3 & La password dell'utente viene sostituita con la password inviata  & \cellacaporiga{dataserver \\
	server \\ } \\

	R0F1.2 & L'utente deve poter creare un proprio account & 
	\cellacaporiga{authentication \\
	client \\
	data \\
	dataserver \\
	server \\
	urlrequest \\
	urlrequesthandler \\} \\

	R0F1.2.1  &  La registrazione richiede l'email all'utente  &  \cellacaporiga{authentication \\
		client \\} \\
	R0F1.2.2  &  La registrazione richiede l'username all'utente & \cellacaporiga{authentication \\
		client \\} \\
	R0F1.2.3  &  La registrazione richiede la password all'utente   & \cellacaporiga{authentication \\
		client \\} \\
	R0F1.2.4  &  L'utente reinserisce la nuova password & \cellacaporiga{authentication \\
		client \\} \\
	R0F1.2.5  &  Il sistema chiede all'utente di confermare i dati della propria registrazione  & \cellacaporiga{authentication \\
		client \\} \\
	R0F1.2.6  &  Se i dati confermati dall'utente sono validi, il suo account viene registrato  & \cellacaporiga{authentication \\
		client \\ data \\ dataserver \\ server \\ urlrequest \\ urlrequesthandler} \\
	
	R0F1.2.6.1 & L'utente viene automaticamente autenticato & \cellacaporiga{authentication \\
		client \\ data \\ dataserver \\ server \\ urlrequest \\ urlrequesthandler} \\
	
	R0F1.2.7 & Il sistema deve segnalare all'utente i vari problemi di registrazione  & \cellacaporiga{authentication \\
		client \\ data \\ dataserver \\ server \\ urlrequest \\ urlrequesthandler} \\
	
	R0F1.2.7.1 & L'utente ha inserito un'email non valida   & \cellacaporiga{authentication \\
		client \\ dataserver \\ server \\ urlrequest \\ urlrequesthandler} \\
	
	R0F1.2.7.1.1  &  L'email deve avere un formato valido: deve contenere una @ preceduto da altri caratteri seguito da un dominio valido & \cellacaporiga{dataserver \\
	server \\ } \\

	R0F1.2.7.1.2   & L'email non deve essere già  in uso  & \cellacaporiga{dataserver \\
		server \\ } \\
	
	R0F1.2.7.2 & L'utente ha inserito un username non valido & \cellacaporiga{authentication \\
		client \\ data \\ dataserver \\ server \\ urlrequest \\ urlrequesthandler} \\
	R0F1.2.7.2.1 &   L'username deve avere un formato valido: deve contenere solo caratteri alfanumerici & \cellacaporiga {dataserver \\
	server } \\
	R0F1.2.7.2.2 &   L'username non deve essere già in uso da un altro utente & \cellacaporiga {dataserver \\
	server} \\
	R0F1.2.7.3 & La password deve contenere un minimo di 6 carattere ed un massimo di 16 & \cellacaporiga{dataserver \\
	server} \\
	R0F1.2.7.4 & L'utente ha reinserito una password che non corrisponde a quella inserita in precedenza &  \cellacaporiga {dataserver \\
	server} \\
	R0F2  &  L'utente deve poter fare il logout & \cellacaporiga {authentication \\
	client \\
	data \\
	datamanager \\
	server \\
	urlrequest \\
	urlrequesthandler} \\
	R0F3  &  L'utente deve poter effettuare un percorso tra quelli disponibili nel luogo in cui si trova & \cellacaporiga{ building \\
	client \\
	data \\
	datamanager \\
	dataserver \\
	games \\
	location \\
	pathprogress \\
	urlrequest \\
	urlrequesthandler \\
	viewcontroller} \\
	R0F3.1 & L'utente gioca il percorso selezionato fino alla sua conclusione & \cellacaporiga{ building \\
	client \\
	data \\
	datamanager \\
	games \\
	location \\
	pathprogress \\
	viewcontroller }\\
	R0F3.1.1  &  L'utente cerca il beacon corrispondente alla prima stazione &  \cellacaporiga{building \\
	client \\
	games \\
	location \\
	pathprogress \\
	viewcontroller} \\
	R0F3.1.1.1 & Se il dispositivo dell'utente trova il beacon corrispondente alla stazione corretta l'utente può cominciare a giocare & \cellacaporiga{ building \\
	client \\
	games \\
	pathprogress \\
	viewcontroller}\\
	R0F3.1.1.2 & I beacon che il dispositivo rileva non inerenti con la stazione cercata devono essere ignorati dall'app & \cellacaporiga {building \\
	client \\
	games \\
	viewcontroller} \\
	R0F3.1.2  &  L'utente gioca la prova relativa a quella stazione & \cellacaporiga{client \\
	games \\
	viewcontroller} \\
	R0F3.1.2.1 & La prova proposta all'utente è una fra quelle disponibili & \cellacaporiga {dataserver \\
	server} \\
	R0F3.1.2.2 & L'app mostra le istruzioni della prova & \cellacaporiga {client \\
	games \\
	viewcontroller }\\
	R0F3.1.2.3 & L'utente deve poter svolgere la prova & \cellacaporiga {client \\
	games \\
	viewcontroller}\\
	R0F3.1.2.4 & Finita la prova il dispositivo deve mostrare il risultato ottenuto & \cellacaporiga {client \\
	games \\
	viewcontroller } \\
	R0F3.1.2.5 & Finita la prova l'utente può proseguire il percorso o averlo terminato & \cellacaporiga {building \\
	client\\
	games \\
	pathprogress \\
	viewcontroller} \\
	R0F3.1.2.5.1  &  Se la prova non è l'ultima l'app indica qual'è la prossima stazione & \cellacaporiga{client \\
	games\\
	viewcontroller}\\
	R0F3.1.2.5.1.1 & L'app mostra all'utente le informazioni per trovare la prossima stazione & \cellacaporiga{client \\
	games \\
	viewcontroller}\\
	R0F3.1.2.5.2  &  Se la prova è l'ultima l'utente ha finito il percorso & \cellacaporiga {building \\
	client \\
	pathprogress \\
	viewcontroller}\\
	R0F3.2 & Quando il percorso è finito il dispositivo mostra all'utente il riepilogo dei dati suoi e generali &   \cellacaporiga{client \\
	savedresults \\
	viewcontroller} \\
	R0F3.2.1 & Il dispositivo mostra il tempo totale del percorso  & \cellacaporiga {client \\
	savedresults \\ 
	viewcontroller} \\
	R0F3.2.2  & Il dispositivo mostra il tempo totale impiegato per eseguire le prove \cellacaporiga {client \\
	savedresults\\
	viewcontroller}\\
	R0F3.2.2.1 & Il tempo totale per eseguire le prove viene calcolato sommando i tempi impiegati per eseguire ogni prova (è il tempo totale senza considerare gli spostamenti effettuati per cambiare stazione) & \cellacaporiga {client \\
	data}\\
	R0F3.2.2.1.1  &  La durata della prova viene misurata da quando si accetta di affrontarla fino alla conferma della soluzione & \cellacaporiga {client \\
	data}\\
	R0F3.2.3 &  Il dispositivo mostra il punteggio totale ottenuto & \cellacaporiga {client \\
	savedresults \\
	viewcontroller} \\
	R0F3.2.3.1 & Il punteggio totale viene calcolato sommando il numero di punti ottenuto in ogni singola prova & \cellacaporiga {client\\
	data}\\
	R0F3.2.4  &  Il dispositivo mostra la posizione in classifica che l'utente ha raggiunto con il punteggio ottenuto & \cellacaporiga  {client \\
	savedresults\\
	viewcontroller}\\
	R0F3.2.4.1 & L'utente deve poter visualizzare la classifica generale client & \cellacaporiga{
	savedresults\\
	viewcontroller}\\
	R0F3.2.5  &  Il dispositivo visualizza la prova con il maggior numero di punti realizzati & \cellacaporiga {client\\
	dataserver\\
	viewcontroller}\\
	R0F3.2.6 & Il dispositivo visualizza la prova con il minor numero di punti realizzati & \cellacaporiga {client \\
	savedresults\\
	viewcontroller}\\
	R0F3.3 & L'utente autenticato può salvare il risultato ottenuto & \cellacaporiga{client\\
	datamanager\\
	dataserver\\
	games\\
	server\\
	urlrequest\\
	urlrequesthandler\\
	viewcontroller}\\
	R0F4 &   L'utente deve poter visualizzare alcune informazioni sull'app & \cellacaporiga {client\\
	data\\
	datamanager\\
	dataserver\\
	server\\
	urlrequest\\
	urlrequesthandler\\
	utility}\\
	R0F4.1 & L'utente deve poter visualizzare una schermata con una descrizione generale dell'app & \cellacaporiga {client\\
	data\\
	datamanager\\
	dataserver\\
	server\\
	urlrequest\\
	urlrequesthandler\\
	utility}\\
	R0F4.2 & L'utente deve poter accedere alla pagina web relativa all'app tramite un link presente all'interno dell'app & \cellacaporiga {client\\
	data\\
	datamanager\\
	dataserver\\
	server\\
	urlrequest\\
	urlrequesthandler \\
	utility}\\
	R0F4.3 & L'utente deve poter inviare una segnalazione tramite mail in caso di errore &  \cellacaporiga {client \\
	data \\
	datamanager\\
	dataserver\\
	server\\
	urlrequest\\
	urlrequesthandler\\
	utility}\\
	R0F4.3.1 & Quando l'utente clicca il pulsante per inviare la segnalazione viene aperta una nuova email da scrivere tramite il gestore di email predefinito. Nell'email il destinatario viene impostato con l'email destinata alle segnalazioni &  \cellacaporiga {client\\
	utility}\\
	R0F5 & L'utente autenticato deve poter visualizzare i risultati dei percorsi effettuati precedentemente  & \cellacaporiga {building\\
	client\\
	data\\
	dataserver\\
	server\\
	urlrequest\\
	urlrequesthandler\\
	viewcontroller}\\
	R0F5.1 & Se l'utente è autenticato e ha percorsi salvati deve poterne vedere l'elenco & \cellacaporiga {building\\
	client\\
	data\\
	datamanager\\
	dataserver\\
	server\\
	urlrequest\\
	urlrequesthandler\\
	viewcontroller}\\
	R0F5.1.1 & Se l'utente clicca su un percorso l'app deve fornire tutte le informazioni salvate & \cellacaporiga {building\\
	client\\
	data\\
	datamanager\\
	dataserver\\
	server\\
	urlrequest\\
	urlrequesthandler\\
	viewcontroller}\\
	R0F5.1.1.1 & Il dispositivo deve visualizzare il nome del percorso & \cellacaporiga {building\\
	client\\
	data\\
	datamanager\\
	dataserver\\
	server\\
	urlrequest\\
	urlrequesthandler\\
	viewcontroller}\\
	R0F5.1.1.2 & Il dispositivo deve visualizzare il nome del luogo dove si è svolto il percorso & \cellacaporiga {building\\
	client\\
	data\\
	datamanager\\
	dataserver\\
	server\\
	urlrequest\\
	urlrequesthandler\\
	viewcontroller}\\
	R0F5.1.1.3 & Il dispositivo deve visualizzare la data in cui si è svolto il percorso & \cellacaporiga {building\\
	client\\
	data\\
	datamanager\\
	dataserver\\
	server\\
	urlrequest\\
	urlrequesthandler}\\
	R0F5.1.1.4 & Il dispositivo deve visualizzare il punteggio totale del percorso & \cellacaporiga {building\\
	client\\
	data\\
	datamanager\\
	dataserver\\
	server\\
	urlrequest\\
	urlrequesthandler\\
	viewcontroller}\\
	R0F5.1.1.5 & Il dispositivo deve visualizzare il tempo totale per lo svolgimento del percorso & \cellacaporiga {building\\
	client\\
	data\\
	datamanager\\
	dataserver\\
	server\\
	urlrequest\\
	urlrequesthandler\\
	viewcontroller}\\
	R0F5.1.1.6 & Il dispositivo può visualizzare la posizione attuale nella classifica generale del risultato ottenuto & \cellacaporiga {client\\
	savedresults\\
	server\\
	urlrequest\\
	urlrequesthandler\\
	viewcontroller}\\

	
	\rowcolor{white}
	\caption{Tracciamento requisiti-componenti}
\end{longtable}

   %% qui quelle requisito - package
   \section{Tabella tracciamento Componenti-Requisiti}

\begin{tabella}{!{\VRule}c!{\VRule}c!{\VRule}}
	\intestazionetwocol{Componenti}{Requisiti}

	CLIPS::client::authentication & \cellacaporiga{
							R0F1.1 \\
							R0F1.1.1 \\
							R0F1.1.2 \\
							R0F1.1.3 \\
							R0F1.1.4 \\
							R0F1.1.5 \\
							R0F1.1.5.1 \\
							R0F1.1.5.2 \\
							R0F1.1.5.3 \\
							R0F1.1.5.4 \\
							R0F1.1.5.4.1 \\
							R0F1.1.5.4.2 \\
							R0F1.2 \\
							R0F1.2.1 \\ 
							R0F1.2.2 \\
							R0F1.2.3 \\
							R0F1.2.4 \\
							R0F1.2.5 \\
							R0F1.2.6 \\
							R0F1.2.7 \\
							R0F1.2.7.1 \\
							R0F1.2.7.2 \\
							R0F2 \\
							R0F5.3 \\
							R0F7 \\
							R0F7.1 \\
							R0F7.2 \\
							R0F7.3 
						} \\
	CLIPS::client::viewcontroller::building & \cellacaporiga{
					R0F3 \\
					R0F3.1 \\
					R0F3.1.1 \\
					R0F3.1.1.1 \\
					R0F3.1.1.2 \\
					R0F3.1.2.5 \\
					R0F3.1.2.5.2 \\
					R0F5 \\
					R0F5.1 \\
					R0F5.1.1 \\
					R0F5.1.1.1 \\
					R0F5.1.1.2 \\
					R0F5.1.1.3 \\
					R0F5.1.1.4 \\
					R0F5.1.1.5 \\
					R0F5.2.1 \\
					R0F5.3.1 \\
					R0F6 \\
					R0F6.1 \\
					R0F6.1.1 \\
					R0F6.1.1.1 \\
					R0F6.2 \\
					R0F6.2.1 \\
					R0F6.2.1.1 \\ 
					R0F6.2.1.2 \\
					R0F6.2.1.3 \\ 
					R0F6.2.2 \\
					R0F6.2.3 \\
					R0F6.2.3.1 \\
					R0F6.2.3.2 \\
					R0F6.2.3.3 \\
					R0F6.2.3.4 \\
					R0F6.2.3.5 \\ 
					R0F6.2.3.6 \\
				} \\
				
	CLIPS::client & \cellacaporiga{
		
					R0F1.1 \\
					R0F1.1.1 \\
					R0F1.1.2 \\
					R0F1.1.3 \\
					R0F1.1.4 \\
					R0F1.1.5 \\ 
					R0F1.1.5.1 \\
					R0F1.1.5.2 \\
					R0F1.1.5.3 \\
					R0F1.1.5.4.1 \\
					R0F1.1.5.4.2 \\
					R0F1.2 \\
					R0F1.2.1 \\
					R0F1.2.2 \\
					R0F1.2.3 \\
					R0F1.2.4 \\ 
					R0F1.2.5 \\
					R0F1.2.6 \\
					R0F1.2.6.1 \\
					R0F1.2.7 \\
					R0F1.2.7.1 \\ 
					R0F1.2.7.2 \\
					R0F2 \\
					R0F3 \\
					R0F3.1 \\
					R0F3.1.1 \\
					R0F3.1.1.1 \\
					R0F3.1.1.2 \\
					R0F3.1.2 \\
					R0F3.1.2.2 \\
					R0F3.1.2.3 \\
					R0F3.1.2.4 \\
					R0F3.1.2.5 \\
					R0F3.1.2.5.1 \\
					R0F3.1.2.5.1.1 \\
					
				} \\
				
	CLIPS::client & \cellacaporiga{
					
					R0F3.1.2.5.2 \\
					R0F3.2 \\
					R0F3.2.1 \\
					R0F3.2.2 \\
					R0F3.2.2.1 \\
					R0F3.2.2.1.1 \\
					R0F3.2.3 \\
					R0F3.2.3.1 \\
					R0F3.2.4 \\
					R0F3.2.4.1 \\
					R0F3.2.5 \\
					R0F3.2.6 \\
					R0F3.3 \\
					R0F4 \\
					R0F4.1 \\
					R0F4.2 \\
					R0F4.3 \\
					R0F4.3.1 \\
					R0F5 \\
					R0F5.1 \\
					R0F5.1.1 \\
					R0F5.1.1.1 \\
					R0F5.1.1.2 \\
					R0F5.1.1.3 \\
					R0F5.1.1.4 \\
					R0F5.1.1.5 \\
					R0F5.1.1.6 \\
					R0F5.2 \\
					R0F5.2.1 \\
					R0F5.3 \\
					R0F5.3.1 \\
					R0F6 \\
					R0F6.1 \\
					R0F6.1.1 \\
					R0F6.1.1.1 \\
				} \\ 
					
	CLIPS::client & \cellacaporiga{
					R0F6.2 \\
					R0F6.2.1 \\
					R0F6.2.1.1 \\
					R0F6.2.1.2 \\
					R0F6.2.1.3 \\
					R0F6.2.2 \\
					R0F6.2.3 \\
					R0F6.2.3.1 \\
					R0F6.2.3.2 \\
					R0F6.2.3.3 \\
					R0F6.2.3.4 \\
					R0F6.2.3.5 \\
					R0F6.2.3.6 \\
					R0F7 \\
					R0F7.1 \\
					R0F7.2 \\
					R0F7.3 \\
					R1F3.2.3.2 \\
				} \\ 
	CLIPS::client::data & \cellacaporiga{
					R0F1.1 \\		
					R0F1.1.4 \\ 
					R0F1.1.5 \\
					R0F1.1.5.1 \\
					R0F1.1.5.2 \\
					R0F1.2 \\
					R0F1.2.6 \\
					R0F1.2.6.1 \\
					R0F1.2.7.2 \\
					R0F2 \\
					R0F3 \\
					R0F3.1 \\
					R0F3.2.2.1 \\
					R0F3.2.2.1.1 \\
					R0F3.2.3.1 \\
					R0F4 \\
					R0F4.1 \\
					R0F4.2 \\
					R0F4.3 \\
					R0F5 \\
					R0F5.1 \\
					R0F5.1.1 \\
					R0F5.1.1.1 \\
					R0F5.1.1.2 \\
					R0F5.1.1.3 \\
					R0F5.1.1.4 \\
					R0F5.1.1.5 \\
					R0F6.1 \\
					R0F6.2 \\
					R0F6.2.1 \\
					R0F6.2.1.1 \\
					R0F6.2.1.2 \\
					R0F6.2.1.3 \\
					R0F6.2.2 \\
					R0F6.2.3 \\
				} \\
					
		CLIPS::client::data & \cellacaporiga{
					R0F6.2.3.1 \\
					R0F6.2.3.2 \\
					R0F6.2.3.3 \\
					R0F6.2.3.4 \\
					R0F6.2.3.5 \\
					R0F6.2.3.6 \\
					R0F7 \\
					R0F7.1 \\
					R0F7.2 \\
			} \\
	  CLIPS::client::data::datamanager  & \cellacaporiga{
					R0F1.1 \\
					R0F1.1.4 \\
					R0F1.2.6.1 \\
					R0F2 \\
					R0F3 \\
					R0F3.1 \\
					R0F3.3 \\
					R0F4 \\
					R0F4.1 \\
					R0F4.2 \\
					R0F4.3 \\
					R0F5.1 \\
					R0F5.1.1 \\
					R0F5.1.1.1 \\
					R0F5.1.1.2 \\
					R0F5.1.1.3 \\
					R0F5.1.1.4 \\
					R0F5.1.1.5 \\
					R0F6 \\
					R0F6.1 \\
					R0F6.1.1.1 \\
					R0F6.2 \\
					R0F6.2.1 \\
					R0F6.2.1.1 \\
					R0F1 \\
					R0F1.1.5 \\
					R0F1.1.5.1 \\
					R0F1.1.5.4.3 \\
					R0F1.2 \\
					R0F1.2.6 \\
					R0F1.2.6.1 \\
					R0F1.2.7 \\
					R0F1.2.7.1 \\
					R0F1.2.7.1.1 \\
				} \\
				
	CLIPS::client::data::datamanager  & \cellacaporiga{	
					R0F1.2.7.1.2 \\
					R0F1.2.7.2 \\
					R0F1.2.7.2.1 \\
					R0F1.2.7.2.2 \\
					R0F1.2.7.3 \\
					R0F1.2.7.4 \\
					R0F3 \\
					R0F3.1.2.1 \\
					R0F3.2.5 \\
					R0F3.3 \\
					R0F4 \\
					R0F4.1 \\
					R0F4.2 \\
					R0F4.3 \\
					R0F5 \\
					R0F5.1 \\
					R0F5.1.1 \\
					R0F5.1.1.1 \\
					R0F5.1.1.2 \\
					R0F5.1.1.3 \\
					R0F5.1.1.4 \\
					R0F5.1.1.5 \\
					R0F6 \\
					R0F6.1 \\
					R0F6.1.1.1 \\
					R0F6.2.1 \\
					R0F6.2.1.1 \\
					R0F7 \\
					R0F7.1 \\
					R0F7.1.1 \\
					R0F7.2 \\
					R0F7.2.1 \\
					R0F7.3 \\
		} \\
		
		CLIPS::server::dataserver & \cellacaporiga{R0F1 \\
		R0F1.1.5\\
		R0F1.1.5.1\\
		R0F1.1.5.4.3\\
		R0F1.2\\
		R0F1.2.6\\
		R0F1.2.6.1\\
		R0F1.2.7\\
		R0F1.2.7.1\\
		R0F1.2.7.1.1\\
		R0F1.2.7.1.2\\
		R0F1.2.7.2\\
		R0F1.2.7.2.1\\
		R0F1.2.7.2.2\\
		R0F1.2.7.3\\
		R0F1.2.7.4\\
		R0F3\\
		R0F3.1.2.1\\
		R0F3.2.5\\
		R0F3.3\\
		R0F4\\
		R0F4.1\\
		R0F4.2\\
		R0F4.3\\
		R0F5\\
		R0F5.1\\
		R0F5.1.1\\
		R0F5.1.1.1\\
		R0F5.1.1.2\\
		R0F5.1.1.3\\
		R0F5.1.1.4\\
		R0F5.1.1.5\\
		R0F6\\
		R0F6.1\\
		R0F6.1.1.1\\
		R0F6.2.1\\
		R0F6.2.1.1\\
		R0F7\\
		R0F7.1\\
		R0F7.1.1\\
		R0F7.2\\
		R0F7.2.1\\
		R0F7.3\\
	}\\
		CLIPS::client::viewcontroller::games   & \cellacaporiga{
				R0F3 \\
				R0F3.1 \\
				R0F3.1.1 \\
				R0F3.1.1.1 \\
				R0F3.1.1.2 \\
				R0F3.1.2 \\
				R0F3.1.2.2 \\
				R0F3.1.2.3 \\
				R0F3.1.2.4 \\
				R0F3.1.2.5 \\
				R0F3.1.2.5.1 \\
				R0F3.1.2.5.1.1 \\
				R0F3.3 \\
			} \\ 
		
	CLIPS::client::location  & \cellacaporiga{  
				R0F3
				R0F3.1 \\
				R0F3.1.1 \\
				R0F6 \\
				R0F6.1.1.1 \\
			} \\
	CLIPS::client::pathprogress  & \cellacaporiga{ 
				R0F3 \\ 
				R0F3.1 \\
				R0F3.1.1 \\
				R0F3.1.1.1 \\
				R0F3.1.2.5 \\
				R0F3.1.2.5.2 \\
			} \\
	CLIPS::viewcontroller::savedresults & \cellacaporiga {
			    R0F3.2 \\
 				R0F3.2.1 \\
				R0F3.2.2 \\
				R0F3.2.3 \\
				R0F3.2.4 \\
				R0F3.2.4.1 \\
				R0F3.2.6 \\
				R0F5.1.1.6 \\
				R0F5.2 \\
				R1F3.2.3.2 \\
			} \\
			
	CLIPS::server  & \cellacaporiga{
				R0F1 \\
				R0F1.1 \\
				R0F1.1.4 \\ 
				R0F1.1.5 \\
				R0F1.1.5.1 \\
				R0F1.1.5.2 \\
				R0F1.1.5.4.3 \\
				R0F1.2 \\
				R0F1.2.6 \\
				R0F1.2.6.1 \\
				R0F1.2.7 \\
				R0F1.2.7.1 \\
				R0F1.2.7.1.1 \\
				R0F1.2.7.1.2 \\
				R0F1.2.7.2 \\
				R0F1.2.7.2.1 \\
				R0F1.2.7.2.2 \\
				R0F1.2.7.3 \\
				R0F1.2.7.4 \\
				R0F2 \\
				R0F3.1.2.1 \\
				R0F3.3 \\
				R0F4 \\
				R0F4.1 \\
				R0F4.2 \\
				R0F4.3 \\
				R0F5 \\
				R0F5.1 \\
				R0F5.1.1 \\
				R0F5.1.1.1 \\
				R0F5.1.1.2 \\
				R0F5.1.1.3 \\
				R0F5.1.1.4 \\
				R0F5.1.1.5 \\
				R0F5.1.1.6 \\
			} \\ 
	CLIPS::server  & \cellacaporiga{				
				R0F6 \\
				R0F6.1 \\
				R0F6.1.1.1 \\
				R0F6.2 \\
				R0F6.2.1 \\
				R0F6.2.1.1 \\
				R0F7 \\
				R0F7.1 \\
				R0F7.1.1 \\
				R0F7.2 \\
				R0F7.2.1 \\
				R0F7.3 \\
	} \\ 	
	CLIPS::client::data:urlrequest  & \cellacaporiga{
					R0F1.1
				R0F1.1.4 \\
 				R0F1.1.5 \\
				R0F1.1.5.1 \\
				R0F1.1.5.2 \\
				R0F1.2 \\
				R0F1.2.6 \\
				R0F1.2.6.1 \\
				R0F1.2.7 \\
				R0F1.2.7.1 \\
				R0F1.2.7.2 \\
				R0F2 \\
				R0F3 \\
				R0F3.3 \\
				R0F4 \\
				R0F4.1 \\
				R0F4.2 \\
				R0F4.3 \\
				R0F5 \\
				R0F5.1 \\
				R0F5.1.1 \\
				R0F5.1.1.1 \\
				R0F5.1.1.2 \\
				R0F5.1.1.3 \\
				R0F5.1.1.4 \\
				R0F5.1.1.5 \\
				R0F5.1.1.6 \\
				R0F6 \\
				R0F6.1 \\
				R0F6.1.1.1 \\
				R0F6.2 \\
				R0F6.2.1 \\
				R0F6.2.1.1 \\
				R0F7 \\
				R0F7.1 \\
				R0F7.2 \\
				R0F7.3  \\
	} \\
	CLIPS::server::urlrequesthandler   & \cellacaporiga{  
				R0F1.1 \\
				R0F1.1.4 \\
				R0F1.1.5 \\
				R0F1.1.5.1 \\
				R0F1.1.5.2 \\
				R0F1.2 \\
				R0F1.2.6 \\
				R0F1.2.6.1 \\
				R0F1.2.7 \\
				R0F1.2.7.1 \\
				R0F1.2.7.2 \\
				R0F2 \\
				R0F3 \\
				R0F3.3 \\
				R0F4 \\
				R0F4.1 \\
				R0F4.2 \\
				R0F4.3 \\
				R0F5 \\
				R0F5.1 \\
				R0F5.1.1 \\
				R0F5.1.1.1 \\
				R0F5.1.1.2 \\
				R0F5.1.1.3 \\
				R0F5.1.1.4 \\
				R0F5.1.1.5 \\
				R0F5.1.1.6 \\
				R0F6 \\
				R0F6.1 \\
				R0F6.1.1.1 \\
				R0F6.2 \\
				R0F6.2.1 \\
				R0F6.2.1.1 \\
				R0F7 \\
				R0F7.1 \\
				R0F7.2 \\
				R0F7.3 \\
			}\\ 
			
			CLIPS::client::viewcontroller::utility & \cellacaporiga{
				R0F4 \\
				R0F4.1 \\
				R0F4.2 \\
				R0F4.3 \\
				R0F4.3.1 \\
			} \\
			CLIPS::client::viewcontroller  & \cellacaporiga{
				R0F3 \\
				R0F3.1 \\
				R0F3.1.1 \\
				R0F3.1.1.1 \\
				R0F3.1.1.2 \\
				R0F3.1.2 \\
				R0F3.1.2.2 \\
				R0F3.1.2.3 \\
				R0F3.1.2.4 \\
				R0F3.1.2.5 \\
				R0F3.1.2.5.1 \\
				R0F3.1.2.5.1.1 \\
				R0F3.1.2.5.2 \\
				R0F3.2 \\
				R0F3.2.1 \\
				R0F3.2.2 \\
				R0F3.2.3 \\
				R0F3.2.4 \\
				R0F3.2.4.1 \\
				R0F3.2.5 \\
				R0F3.2.6 \\
				R0F3.3 \\
				R0F5 \\
				R0F5.1 \\
				R0F5.1.1 \\
				R0F5.1.1.1 \\
				R0F5.1.1.2 \\
				R0F5.1.1.4 \\
				R0F5.1.1.5 \\
				R0F5.1.1.6 \\
				R0F5.2 \\
				R0F5.2.1 \\
				R0F5.3.1 \\
				R0F6 \\ 
			} \\
			
			CLIPS::client::viewcontroller  & \cellacaporiga{
				R0F6.1.1 \\
				R0F6.1.1.1 \\
				R0F6.2 \\
				R0F6.2.1 \\
				R0F6.2.1.1 \\
				R1F3.2.3.2 \\
			} \\
\rowcolor{white}
\caption{Tracciamento componenti-requisiti}
\end{longtable}

\end{document}
