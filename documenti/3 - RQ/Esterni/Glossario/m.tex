\invisiblesection{M}
\lettera{M} 

\parola{MaaP}{Acronimo di MongoDB as an Admin Platform progetto per larealizzazione di un \gl{framework} sviluppato dal gruppo SteakHolders durante l'anno accademico 2013/2014} 

\parola{Milestone}{Milestone è il termine inglese per riferirsi alla pietra miliare. Come è facilmente intuibile dal nome, la milestone determina importanti traguardi in termini di tempo o di attività, che sanciscono una metrica di avanzamento del progetto.}

\parola{MongoDB}{MongoDB (da "humongous", enorme) è un \gl{DBMS} non relazionale, orientato ai documenti. Si allontana dalla struttura tradizionale basata su tabelle dei \gl{database relazionali} in favore di documenti in stile \gl{JSON} con schema dinamico, rendendo l'integrazione di dati di alcuni tipi di applicazioni più facile e veloce (riferimento: \url{https://it.wikipedia.org/wiki/MongoDB}).}

\parola{Monospace}{Il Monospace o monospazio in italiano è un tipo di carattere (font) con glifi a larghezza fissa, è molto comune in informatica e ricorda i caratteri della macchina da scrivere.}

\parola{MQTT}{Acronimo di MQ Telemetry Transport è un protocollo di messaggistica leggero posizionato in cima a \gl{TCP/IP}. È stato disegnato per le situazioni in cui è richiesto un basso impatto e dove la banda è limitata (riferimento: \url{https://it.wikipedia.org/wiki/MQTT}).}

\parola{Multipiattaforma}{In informatica il termine multipiattaforma può essere riferito ad un linguaggio di programmazione, ad un'applicazione \gl{software} o ad un dispositivo \gl{hardware} che funziona su più di un sistema o appunto, piattaforma (riferimento: \url{https://it.wikipedia.org/wiki/Multipiattaforma}).}

\parola{MVC}{Il Model-View-Controller, in informatica, è un pattern architetturale molto diffuso nello sviluppo di sistemi software, in particolare nell'ambito della programmazione orientata agli oggetti, in grado di separare la logica di presentazione dei dati dalla logica di business (riferimento: \url{https://it.wikipedia.org/wiki/Model-View-Controller}).}

\parola{MVP}{Il Model-View-Presenter è un pattern architetturale derivato dall'\gl{MVC}\ e molto usato per le applicazioni \gl{Android}. È simile ad esso ma presenta delle differenze importanti: ad esempio le View sono completamente passive, quindi ogni evento viene gestito dal Presenter, e quest'ultimo ha il compito di aggiornare la View quando occorre; inoltre le View non possono comunicare direttamente con il Model, bensì devono fare richiesta al Presenter, l'unico abilitato ad effettuare le richieste al Model (riferimento: \url{https://en.wikipedia.org/wiki/Model-view-presenter}).}

\parola{MySQL}{MySQL o Oracle MySQL è un Relational database management system (RDBMS) composto da un client a riga di comando e un server (riferimento: \url{https://it.wikipedia.org/wiki/MySQL}).}
