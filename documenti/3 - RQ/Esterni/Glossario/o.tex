
\lettera{O} 

\parola{Objective-C}{Objective-C è un linguaggio di programmazione riflessivo orientato agli oggetti ed è un'estensione a oggetti del linguaggio C con cui tiene completa compatibilità (riferimento: \url{https://it.wikipedia.org/wiki/Objective-C}).}

\parola{On demand}{Per on demand, nel campo dell'informatica, si intende l'accesso alle risorse informatiche solo quando necessario, eventualmente pagando le stesse in base all'utilizzo e non in base a un canone fisso o acquistando una licenza (riferimento: \url{https://it.wikipedia.org/wiki/On_demand_(informatica)}).}

\parola{Open-source}{Open source (termine inglese che significa sorgente aperta), in informatica, indica un software di cui gli autori (più precisamente i detentori dei diritti) rendono pubblico il codice sorgente, favorendone il libero studio e permettendo a programmatori indipendenti di apportarvi modifiche ed estensioni. Questa possibilità è regolata tramite l'applicazione di apposite licenze d'uso (riferimento: \url{https://it.wikipedia.org/wiki/Open_source}).}

\parola{OrientDB}{In informatica OrientDB è un Document-Graph Database scritto in Java, una base di dati orientata al documento con relazioni gestite come in un database a grafo con connessioni dirette tra i record (riferimento: \url{https://it.wikipedia.org/wiki/Orient_ODBMS}).}