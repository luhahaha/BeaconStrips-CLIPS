\documentclass[a4paper,titlepage]{article}

\makeatletter
\def\input@path{{../../../template/}}
\makeatother

\usepackage{Comandi}
\usepackage{Riferimenti}
\usepackage{Stile}
\usepackage{glossario}


\def\NOME{Glossario}
\def\VERSIONE{3.0.0}
\def\DATA{2016-08-16}
\def\REDATTORE{Luca Soldera \\ & Andrea Grendene}
\def\VERIFICATORE{Enrico Bellio \\ & Viviana Alessio}
\def\RESPONSABILE{Tommaso Panozzo}
\def\USO{Esterno}
\def\DESTINATARI{\COMMITTENTE \\ & \CARDIN \\ & \PROPONENTE}
\def\SOMMARIO{Questo documento ha lo scopo di chiarire le possibili ambiguità causate dal gergo tecnico usato dal gruppo Beacon Strips all'interno della documentazione prodotta.}


\begin{document}
	
\maketitle

\begin{diario}
	\modifica{Tommaso Panozzo}{\RES}{Approvazione documento}{2016-08-16}{3.0.0}
	\modifica{Luca Soldera}{\PRJ}{Aggiunte parole ‘‘MySQL’’ ‘‘phpMyAdmin’’}{2016-07-11}{2.0.2}
	\modifica{Luca Soldera}{\PRJ}{Aggiunto indice}{2016-06-20}{2.0.1}
	\modifica{Matteo Franco}{\RES}{Approvazione documento}{2016-06-09}{2.0.0}
	\modifica{Viviana Alessio}{\VER}{Verifica parole aggiunte}{2016-06-06}{1.1.0}
	\modifica{Luca Soldera}{\PRJ}{Aggiunte le parole ‘‘Client’’, ‘‘Server’’, ‘‘Training Process’’, ‘‘MVC’’ e ‘‘Android Studio’’}{2016-06-06}{1.0.2}
	\modifica{Andrea Grendene}{\AN}{Aggiunte le parole ‘‘Package’’, ‘‘Statement’’, ‘‘Libreria’’ e ‘‘API’’}{2016-06-03}{1.0.1}
	\modifica{Viviana Alessio}{\RES}{Approvazione documento}{2016-04-06}{1.0.0}
	\modifica{Enrico Bellio}{\VER}{Verifica documento}{2016-04-06}{0.6.0}
	\modifica{Luca Soldera}{\AM}{Correzione errori}{2016-04-06}{0.5.1}
	\modifica{Enrico Bellio}{\AN}{Verifica documento}{2016-04-06}{0.5.0}
	\modifica{Luca Soldera}{\AM}{Aggiunte definizioni segnalate nel \NPdoc}{2016-4-5}{0.4.0}
	\modifica{Luca Soldera}{\AM}{Aggiunte definizioni segnalate nel \SFdoc}{2016-4-4}{0.3.0}
	\modifica{Andrea Grendene}{\AN}{Aggiunte al \Gldoc\ le parole segnalate nel\PQdoc}{2016-04-01}{0.2.0}
	\modifica{Viviana Alessio}{\RES}{Stesura intestazione e indice documento}{2016-03-16}{0.1.0}

\end{diario}

\newpage
\tableofcontents
\newpage


\lettera{A}

\parola{Albero}
{Esempio di descrizione Esempio di descrizione Esempio di descrizione Esempio di descrizione Esempio di descrizione Esempio di descrizione Esempio di descrizione Esempio di descrizione Esempio di descrizione Esempio di descrizione Esempio di descrizione Esempio di descrizione Esempio di descrizione Esempio di descrizione } 

\parola{Abete}
{Altro parola Esempio di descrizione Esempio di descrizione Esempio di descrizione Esempio di descrizione Esempio di descrizione Esempio di descrizione Esempio di descrizione Esempio di descrizione Esempio di descrizione Esempio di descrizione Esempio di descrizione Esempio di descrizione Esempio di descrizione Esempio di descrizione Esempio di descrizione  }
 
\lettera{B}

\parola{Back end}{Il back end è l'insieme di interfacce che interagiscono con un programma. In altri termini è un programma con cui l'utente interagisce indirettamente, in generale attraverso l'applicazione di un \gl{front end}. Nel nostro caso il back end è rappresentato dall'interazione con il server web.}
 
\lettera{C}

\parola{CamelCase}{La Notazione a Cammello o in inglese CamelCase è la pratica nata durante gli anni settanta di scrivere parole composte o frasi unendo tutte le parole tra loro, ma lasciando le loro iniziali maiuscole (riferimento: \url{https://it.wikipedia.org/wiki/Notazione_a_cammello}).}

\parola{Capitolato}{Il capitolato è un documento tecnico, in genere allegato ad un contratto di appalto, che vi fa riferimento per definire in quella sede le specifiche tecniche delle opere che andranno ad eseguirsi per effetto del contratto stesso.}

\parola{Client}{Un client, in informatica, indica una componente che accede ai servizi o alle risorse di un'altra componente detta server. Esso fa parte dunque dell'architettura logica di rete detta client-server (riferimento: \url{https://it.wikipedia.org/wiki/Client}).}

\parola{Cloud computing}{In informatica con il termine inglese cloud computing si indica un paradigma di erogazione di risorse informatiche, come l'archiviazione, l'elaborazione o la trasmissione di dati, caratterizzato dalla disponibilità \gl{on demand} attraverso Internet a partire da un insieme di risorse preesistenti e configurabili (riferimento: \url{https://it.wikipedia.org/wiki/Cloud_computing}).}

\parola{CMM}{Acronimo di ‘‘Capability Maturity Model’’, è un approccio al miglioramento dei processi il cui obiettivo è aiutare un'organizzazione a migliorare le proprie prestazioni. A partire dal 1997, è stato sostituito dal CMMI, di cui è erede.}

\parola{Commit}{in informatica è il tentativo di apportare una serie di cambiamenti permanenti. Nell'ambito di progetto è utilizzato come comando di \gl{git} per apportare modifiche permanenti alla \gl{repository} remota. }

\parola{Committente}{Committente è una parola italiana derivante dalla parola latina ‘‘committo’’. essa indica la persona che consegna/assegna un oggetto. In questo caso si riferisce al \COMMITTENTE\ che ha assegnato al gruppo \AUTORE\ il compito di sviluppare il \gl{capitolato}\ C2.}

\parola{CSS}{Acronimo di Cascading Style Sheets, in italiano fogli di stile a cascata, è un linguaggio usato per definire la formattazione di documenti \gl{HTML} e \gl{XML}, ad esempio i siti web e relative pagine web (riferimento: \url{https://it.wikipedia.org/wiki/CSS}).}

\parola{CSS3}{CSS3 è la terza e più attuale delle specifiche dell linguaggio \gl{CSS}}

\parola{Cvs}{Il comma-separated values è un formato di file basato su file di testo utilizzato per l'importazione ed esportazione (ad esempio da fogli elettronici o database) di una tabella di dati (riferimento: \url{https://it.wikipedia.org/wiki/Comma-separated_values}).}

\lettera{D} 

 
\lettera{E}

\lettera{F} 

\parola{Front end}{Secondo il significato più generale il front end è responsabile per l'acquisizione dei dati di ingresso e per la loror elaborazione con modalità conformi a specifiche predefinite e invarianti, in modo da renderli utilizzabili dal \gl{back end}. Nel nostro caso il front end rappresenta sia l'interfaccia grafica sia il sistema di elaborazione degli input dell'utente e degli output per l'utente.}

\lettera{G} 


\parola{Google}{Google Inc. è un'azienda statunitense che offre servizi online, tra cui l'omonimo motore di ricerca, Youtube, Gmail e così via. Inoltre offre anche prodotti come il browser Google Chrome e il sistema operativo Android per dispositivi mobile (riferimento: \url{https://it.wikipedia.org/wiki/Google_Inc.}).}

\parola{Google Drive}{}

\parola{Google Hangouts}{È un software di messaggistica istantanea e di VoIP sviluppato da \gl{Google}, ovvero permette di scambiare messaggi e di effettuare chiamate e videochiamate tra gli utenti. Sfrutta gli account \gl{Google}, che quindi possono usufruire di tutti gli altri servizi offerti dall'azienda (riferimento: \url{https://it.wikipedia.org/wiki/Google_Hangouts}).}

\parola{Gulpease}{Gulpease o ‘‘Indice Gulpease’’ è un indice di leggibilità di testo tarato sulla lingua italiana. Rispetto ad altri ha il vantaggio di utilizzare la lunghezza delle parole anziché in sillabe, semplificandone il calcolo numerico.}

\lettera{H} 


\lettera{I} 

\parola{Incrementale}
{Per modello incrementale o modello iterativo si intende, nell'ambito dell'ingegneria informatica, un modello di sviluppo di un progetto software basato sulla successione dei seguenti passi principali:
	pianificazione,
	analisi dei requisiti,
	progetto,
	implementazione,
	prove,
	valutazione.
	Questo ciclo può essere ripetuto diverse volte, denominate ``iterazioni'', fino a che la valutazione del prodotto diviene soddisfacente rispetto ai requisiti richiesti.}
	
\parola{Inspection}{L'inspection si basa sulla lettura mirata dei documenti/codice in cerca di errori specifici.}

\parola{ISO}{L'Organizzazione internazionale per la normazione (in inglese International Organization for Standardization), abbreviazione ISO, è la più importante organizzazione a livello mondiale per la definizione di norme tecniche.}


\lettera{J} 


\lettera{K} 


\lettera{L} 

\parola{\LaTeX}{\gl{Linguaggio di markup} usato per la preparazione di testi. Si basa sul principio WYSIWYM (What You See Is What You Mean), contrapposto al WYSIWYG (What You See Is What You Get) tipico dei più comuni programmi di videoscrittura. Permette di generare un file in formato .pdf dai file di \LaTeX tramite un apposito compilatore. Maggiori informazioni al sito \url{http://www.latex-project.org}.}

\parola{Linguaggio di markup}{In generale un linguaggio di markup, o linguaggio a marcatori, è un insieme di regole che descrivono i meccanismi di rappresentazione di un testo che, utilizzando convenzioni standardizzate, sono utilizzabili su più supporti. La tecnica di composizione di un testo con l'uso di marcatori (o espressioni codificate) richiede quindi una serie di convenzioni, ovvero di un linguaggio a marcatori di documenti.}

\lettera{M} 

\parola{MaaP}{Acronimo di MongoDB as an Admin Platform progetto per larealizzazione di un \gl{framework} sviluppato dal gruppo SteakHolders durante l'anno accademico 2013/2014} 

\parola{Milestone}{Milestone è il termine inglese per riferirsi alla pietra miliare. Come è facilmente intuibile dal nome, la milestone determina importanti traguardi in termini di tempo o di attività, che sanciscono una metrica di avanzamento del progetto.}

\parola{MongoDB}{MongoDB (da "humongous", enorme) è un \gl{DBMS} non relazionale, orientato ai documenti. Si allontana dalla struttura tradizionale basata su tabelle dei \gl{database relazionali} in favore di documenti in stile \gl{JSON} con schema dinamico, rendendo l'integrazione di dati di alcuni tipi di applicazioni più facile e veloce (riferimento: \url{https://it.wikipedia.org/wiki/MongoDB}).}

\parola{Monospace}{Il Monospace o monospazio in italiano è un tipo di carattere (font) con glifi a larghezza fissa, è molto comune in informatica e ricorda i caratteri della macchina da scrivere.}

\parola{MQTT}{Acronimo di MQ Telemetry Transport è un protocollo di messaggistica leggero posizionato in cima a \gl{TCP/IP}. È stato disegnato per le situazioni in cui è richiesto un basso impatto e dove la banda è limitata (riferimento: \url{https://it.wikipedia.org/wiki/MQTT}).}

\parola{Multipiattaforma}{In informatica il termine multipiattaforma può essere riferito ad un linguaggio di programmazione, ad un'applicazione \gl{software} o ad un dispositivo \gl{hardware} che funziona su più di un sistema o appunto, piattaforma (riferimento: \url{https://it.wikipedia.org/wiki/Multipiattaforma}).}

\parola{MVC}{Il Model-View-Controller, in informatica, è un pattern architetturale molto diffuso nello sviluppo di sistemi software, in particolare nell'ambito della programmazione orientata agli oggetti, in grado di separare la logica di presentazione dei dati dalla logica di business (riferimento: \url{https://it.wikipedia.org/wiki/Model-View-Controller}).}


\lettera{N} 

\parola{Node.js}{Node.js è un \gl{framework} guidato dagli eventi per il motore \gl{JavaScript}. Si tratta quindi di un \gl{framework} relativo all'utilizzo di \gl{Javascript} lato server. Maggiori informazioni su: \url{nodejs.org}.}
\invisiblesection{O}
\lettera{O} 

\parola{Objective-C}{Objective-C è un linguaggio di programmazione riflessivo orientato agli oggetti ed è un'estensione a oggetti del linguaggio C con cui tiene completa compatibilità (riferimento: \url{https://it.wikipedia.org/wiki/Objective-C}).}

\parola{On demand}{Per on demand, nel campo dell'informatica, si intende l'accesso alle risorse informatiche solo quando necessario, eventualmente pagando le stesse in base all'utilizzo e non in base a un canone fisso o acquistando una licenza (riferimento: \url{https://it.wikipedia.org/wiki/On_demand_(informatica)}).}

\parola{Open-source}{Open source (termine inglese che significa sorgente aperta), in informatica, indica un software di cui gli autori (più precisamente i detentori dei diritti) rendono pubblico il codice sorgente, favorendone il libero studio e permettendo a programmatori indipendenti di apportarvi modifiche ed estensioni. Questa possibilità è regolata tramite l'applicazione di apposite licenze d'uso (riferimento: \url{https://it.wikipedia.org/wiki/Open_source}).}

\parola{OrientDB}{In informatica OrientDB è un Document-Graph Database scritto in Java, una base di dati orientata al documento con relazioni gestite come in un database a grafo con connessioni dirette tra i record (riferimento: \url{https://it.wikipedia.org/wiki/Orient_ODBMS}).}
\invisiblesection{P}
\lettera{P}

\parola{Package}{Un package è una struttura per l'organizzazione delle classi in sottogruppi ordinati. Esso viene usato spesso dai programmatori per riunire classi logicamente correlate o con uno scopo simile, anche perché crea un unico spazio dei nomi per le classi contenute in esso. Un esempio di struttura a package con classi gerarchizzate sono le \gl{librerie}\ \gl{API}\ di \gl{Java}.}

\parola{Pattern}{Pattern è un termine inglese, che può essere tradotto, a seconda del contesto, con "disegno, modello, schema, schema ricorrente, struttura ripetitiva" e può essere utilizzato per indicare una regolarità che si riscontra all'interno di un insieme di oggetti osservati. Nell'\gl{ingegneria del software} è spesso utilizzata per parlare di \gl{design pattern}.}

\parola{PDCA}{Acronimo di ‘‘Plan-Do-Check-Act’’ o Ciclo di Daming, è un modello per il miglioramento continuo della qualità dei processi. Esso prevede in ordine cronologico la pianificazione del processo, la sua applicazione, la sua verifica e l'applicazione delle modifiche ritenute necessarie dal verificatore.}

\parola{Phonegap}{PhoneGap è un framework \gl{multipiattaforma} mobile che consente di sviluppare delle applicazioni native attraverso l'utilizzo di tecnologie web quali \gl{HTML}, \gl{CSS} e \gl{JavaScript} (riferimento: \url{https://it.wikipedia.org/wiki/PhoneGap}).}

\parola{PHP}{Acronimo ricorsivo di "PHP: Hypertext Preprocessor", è un preprocessore di ipertesti principalmente utilizzato per sviluppare applicazioni web lato server (riferimento: \url{https://it.wikipedia.org/wiki/PHP}).}

\parola{Play Framework}{Play è un \gl{framework} \gl{open source}, scritto in \gl{Java} e \gl{Scala}, che implementa il \gl{pattern} \gl{model-view-controller}. Il suo scopo è quello di migliorare la produttività degli sviluppatori grazie al caricamento del codice a caldo e alla visualizzazione degli errori nel browser (riferimento: \url{https://it.wikipedia.org/wiki/Play_framework}).}

\parola{Portable Network Graphics}{Il Portable Network Graphics (abbreviato PNG) è un formato di file per memorizzare immagini.
Il formato PNG è superficialmente simile al GIF, in quanto è capace di immagazzinare immagini in modo lossless, ossia senza perdere alcuna informazione.
Può memorizzare immagini a 24 bit ed ha un canale dedicato per la trasparenza (canale alfa) (riferimento: \url{https://it.wikipedia.org/wiki/Portable_Network_Graphics}).}


\parola{Prodotto}{Il prodotto è il risultato di un insieme di attività. In questo caso il termine è da intendersi come un sinonimo di \PROGETTO.} 

\parola{Progetto}{Il progetto è un insieme di azioni organizzate atte a perseguire uno scopo specifico. Nel nostro caso indica tutta l'attività di progettazione di codice e di documenti e della loro verifica, quindi il prodotto finale sarà \PROGETTO. Di conseguenza questo termine verrà usato spesso come sinonimo di \gl{prodotto}.}

\parola{Proponente}{Il proponente è la persona che ha proposto al \gl{committente}\ un \gl{capitolato}\ d'appalto.}

\parola{Prototipo}{Il prototipo è il modello originale o il primo esemplare di un manufatto, rispetto a una sequenza di eguali o similari realizzazioni successive. Sul prototipo verranno effettuati collaudi, modifiche e perfezionamenti, fino al prototipo definitivo, da avviare alla produzione in serie.}

\parola{Push}{Comando di \gl{git} per rendere effettivi sulla \gl{repository} remota una serie di cambiamenti apportati con una o più \gl{commit}.}
\invisiblesection{Q}
\lettera{Q} 


\lettera{R} 

\parola{Repository}
{Una repository (in italiano deposito o ripostiglio) è un ambiente di un sistema informativo in cui vengono gestiti i metadati, attraverso tabelle relazionali. I file presenti in un repository sono sottoposti a versionamento. Per il progetto è stato scelto Git per gestire il sistema informativo.}

\parola{Range}{Parola inglese che indica un intervallo continuo in cui si può trovare un valore ricercato.}

\lettera{S} 

parola{Scala}{Scala (da Scalable Language) è un linguaggio studiato per integrare le caratteristiche e funzionalità dei \gl{linguaggi orientati agli oggetti} e dei \gl{linguaggi funzionali}. La compilazione di codice sorgente Scala produce codice per l'esecuzione su una \gl{JVM}.}

\parola{Scalable Vector Graphics}{Scalable Vector Graphics abbreviato in SVG, indica una tecnologia in grado di visualizzare oggetti di grafica vettoriale e, pertanto, di gestire immagini scalabili dimensionalmente.
Più specificamente si tratta di un linguaggio derivato dall'XML, cioè di un'applicazione del metalinguaggio posto a base degli sviluppi del Web da parte del consorzio W3C, che si pone l'obiettivo di descrivere figure bidimensionali statiche e animate (riferimento: \url{https://it.wikipedia.org/wiki/Scalable_Vector_Graphics}).
}

\parola{Script}{Parola inglese che significa ‘‘copione’’. In informatica rappresenta un piccolo programma, solitamente sequenziale e scritto in linguaggio interpretato. Spesso ha complessità bassa e realizza un singolo \gl{task}.}

\parola{Server}{Nell'architettura client-server il server è preposto alla gestione della logica del sistema e all'implementazione di tutte le tecniche di gestione degli accessi, allocazione e rilascio delle risorse, condivisione e sicurezza dei dati o delle risorse (riferimento: \url{https://it.wikipedia.org/wiki/Sistema_client/server}).}

\parola{Software}{In informatica, l'insieme delle procedure e delle istruzioni in un sistema di elaborazione dati. Si identifica con un insieme di programmi o applicazioni.}

\parola{Software as a Service}{Software as a service (SaaS) è un modello di distribuzione del software applicativo dove un produttore di software sviluppa, opera e gestisce un'applicazione web che mette a disposizione dei propri clienti via internet (riferimento: \url{https://it.wikipedia.org/wiki/Software_as_a_service}).}

\parola{Slack}{Slack è uno strumento basato su cloud ideato per la comunicazione aziendale e di gruppi lavorativi, utilizzabile sia tramite web che tramite applicazione poichè \gl{multipiattaforma}.}

\parola{Smartphones}{In italiano ‘‘telefono intelligente’’, è un telefono cellulare con capacità di calcolo, di memoria e di connessione dati molto più avanzate rispetto ai normali telefoni cellulari, basato su un sistema operativo per dispositivi mobili.}

\parola{Stack}{Struttura dati astratta, utilizzata in diversi linguaggi di programmazione, in cui i dati possono essere inseriti e acceduti seguendo regole ben definite.}

\parola{StarUML}{StarUML è uno software \gl{UML} per la creazione di diagrammi con linguaggio \gl{UML}. Per maggiori informazioni \url{http://staruml.io/}.}

\parola{Statement}{È un termine che, nel contesto dei linguaggi di programmazione, indica un unico enunciato o comando.}

\parola{Subtask}{In italiano ‘‘sottoincarico’’, compito che deve essere portato a termine per completare un \gl{task} più complesso.}

\parola{SVN}{Acronimo di subversion, è un sistema di controllo versione per software.}

\lettera{T} 

\parola{Team}{Team è un sinonimo della parola ``gruppo'' usata per indicare il gruppo \AUTORE.}

\parola{Trender}{}


\lettera{U} 

\parola{UML}{Acronimo per ‘‘Unified Modeling Language’’. È un linguaggio di modellazione e specifica basato sul paradigma object-oriented. È nato per unificare i numerosi linguaggi che trattavano i medesimi argomenti per poterne sfruttare le migliori caratteristiche. In questo modo si è cercato con successo di renderlo uno degli standard più diffusi al mondo. UML 2.0 riorganizza molti degli elementi della versione precedente (1.5) in un quadro di riferimento ampliato e introduce molti strumenti, inclusi alcuni nuovi tipi di diagrammi (riferimento: \url{http://it.wikipedia.org/wiki/Unified_Modeling_Language}).}

\parola{UUID}{Acronimo dell'identificativo univoco universale, un identificativo standard usato nelle infrastrutture software, standardizzato come parte di un ambiente distribuito di computazione. (riferimento: \url{https://it.wikipedia.org/wiki/UUID}).}

\lettera{V} 

\parola{VoIP}{In telecomunicazioni e informatica con Voice over \gl{IP} si intende una tecnologia che rende possibile effettuare una conversazione telefonica sfruttando una connessione Internet o una qualsiasi altra rete dedicata a commutazione di pacchetto che utilizzi il protocollo \gl{IP} senza connessione per il trasporto dati (riferimento: \url{https://it.wikipedia.org/wiki/Voice_over_IP}).}

\lettera{W} 

\invisiblesection{X}
\lettera{X} 

\invisiblesection{Y}
\lettera{Y} 


\lettera{Z} 


\end{document}
