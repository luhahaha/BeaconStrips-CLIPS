\invisiblesection{A}
\lettera{A}

\parola{Akka}{Akka è un \gl{toolkit} \gl{open-source} per semplificare la costruzione di applicazioni concorrenti e distribuiti sulla \gl{JVM}. Akka si specializza soprattutto sul modello basato sugli attori.}

\parola{AltBeacon}{
	AltBeacon è la specifica open source che definisce il formato del messaggio di avvertimento che Bluetooth Low Energy trasmette quando si è nella vicinanza di beacons (riferimento: \url{http://altbeacon.org/}). }

\parola{Amazon Web Services}{Amazon Web Services (AWS) è una collezione di servizi di \gl{cloud computing} che compongono la piattaforma \gl{on demand} offerta da dall'azienda Amazon (riferimento: \url{https://it.wikipedia.org/wiki/Amazon_Web_Services}).}

\parola{Android}{Android è un sistema operativo per dispositivi mobili sviluppato da \gl{Google} Inc..}

\parola{Android Studio}{Android Studio è un ambiente di sviluppo integrato (IDE) per lo sviluppo per la piattaforma \gl{Android} (riferimeto: \url{https://it.wikipedia.org/wiki/Android_Studio}).}

\parola{API}{Acronimo di ‘‘Application Programming Interface’’, indica un insieme di procedure disponibili al programmatore, che lo aiutano a svolgere un certo compito all'interno del programma.}

\parola{Application server}{In informatica un application server (a volte abbreviato con la sigla AS) è una tipologia di server che fornisce l'infrastruttura e le funzionalità di supporto, sviluppo ed esecuzione di applicazioni nonché altri componenti server in un contesto distribuito (riferimento: \url{https://it.wikipedia.org/wiki/Application_server}).}

\parola{Asana}{Asana è un \gl{Software as a Service} ideato per favorire la collaborazione in un gruppo di lavoro. Permette ad ogni utente la gestione online di progetti e task senza l'utilizzo di mail. Per maggiori informazioni \url{https://asana.com/}.}