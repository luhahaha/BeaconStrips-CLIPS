\invisiblesection{B} 
\lettera{B}

\parola{Back end}{Il back end è l'insieme di interfacce che interagiscono con un programma. In altri termini è un programma con cui l'utente interagisce indirettamente, in generale attraverso l'applicazione di un \gl{front end}. Nel nostro caso il back end è rappresentato dall'interazione con il server web.}

\parola{Beacon}{Nuova classe di trasmettitori, a bassa potenza e a basso costo, che possono notificare la propria presenza a dispositivi vicini. La tecnologia consente ad uno smartphone o ad un altro dispositivo di effettuare delle azioni quando sono nelle vicinanza di un beacon. Sfrutta la tecnologia Bluetooth Low Energy (\gl{BLE}), conosciuta anche come Bluetooth Smart. I beacon usano la percezione di prossimità del Bluetooth Low Energy per trasmettere un identificativo unico universale (\gl{UUID}), che sarà poi letto da una specifica app o sistema operativo.}

\parola{Bitbucket}{Bitbucket è un servizio web di hosting per lo sviluppo di progetti \gl{software}, che usa il sistema di controllo di versione Git.}

\parola{BLE}{Acronimo di Bluetooth Low Energy, è una nuova tecnologia del nuovo standard Bluetooth 4. La principale caratteristica è un sostanziale risparmio energetico a discapito della velocità di trasmissione, infatti si passa da 24 Mbit/s a 1 Mbit/s.}

\parola{Bluetooth}{È uno standard tecnico-industriale di trasmissione dati per reti personali senza fili. Fornisce un metodo standard, economico e sicuro per scambiare informazioni tra dispositivi diversi attraverso una frequenza radio sicura a corto raggio.}

\parola{BOM}{Il Byte Order Mark (BOM) è una piccola sequenza di byte che viene posizionata all'inizio di un flusso di dati di puro testo (tipicamente un file) per indicarne il tipo di codifica Unicode (riferimento: \url{https://it.wikipedia.org/wiki/Byte_Order_Mark}).}

\parola{Bootstrap}{Bootstrap è una raccolta di strumenti liberi per la creazione di siti e applicazioni per il Web. Essa contiene modelli di progettazione basati su \gl{HTML} e \gl{CSS}, sia per la tipografia, che per le varie componenti dell'interfaccia, come moduli, bottoni e navigazione, e altri componenti dell'interfaccia, così come alcune estensioni opzionali di \gl{JavaScript} (riferimento: \url{https://it.wikipedia.org/wiki/Bootstrap_(informatica)}).}