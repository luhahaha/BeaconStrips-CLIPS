\section{Tecnologie} 
\label{tecnologie}
Per lo sviluppo del progetto abbiamo la necessità di utlizzare delle tecnologie specifiche che sono state scelte dopo attenta analisi. 
In questa sezione del documento vengono riportate quelle che verranno utilizzate principalmente. 

\subsection{Java}
	
	Java è virtualmente alla base di qualsiasi tipo di applicazione in rete ed è lo standard globale per lo sviluppo e la distribuzione di applicazioni incorporate e per sistemi portatili, giochi, contenuto basato su Web e software aziendale. Con oltre 9 milioni di sviluppatori in tutto il mondo, Java consente di sviluppare, distribuire e utilizzare applicazioni e servizi entusiasmanti in modo efficiente. \\ 
	In particolare Java viene utilizzato come linguaggio di programmazione per creare applicazioni mobile per il sistema operativo Android.
	
\subsection{XML}
	
	XML è un metalinguaggio per la definizione di linguaggi di markup, ovvero un linguaggio marcatore basato su un meccanismo sintattico che consente di definire e controllare il significato degli elementi contenuti in un documento o in un testo.
	In particolare nello sviluppo di applicazioni Android la parte grafica viene scritta proprio attraverso file XML, quindi questo linguaggio è essenziale per lo sviluppo del nostro progetto.
	

\subsection{JSON}
	JSON (JavaScript Object Notation) è un semplice formato per lo scambio di dati. Per le persone è facile da leggere e scrivere, mentre per le macchine risulta facile da generare e analizzarne la sintassi.

\subsection{JavaScript}

		

\subsection{PHP}
	PHP è un linguaggio di scripting interpretato, originariamente concepito per la programmazione di pagine web dinamiche. 
	Attualmente è principalmente utilizzato per sviluppare applicazioni web lato server, ma può essere usato anche per scrivere script a riga di comando o applicazioni desktop con interfaccia grafica.
	
\subsection{SQL}
	SQL (Structured Query Language) è un linguaggio standardizzato per database basati sul modello relazionale (RDBMS) progettato per:
	\begin{itemize}
		\item 	creare e modificare schemi di database;
		\item 	inserire, modificare e gestire dati memorizzati;
		\item 	interrogare i dati memorizzati;
		\item 	creare e gestire strumenti di controllo ed accesso ai dati.
	\end{itemize}
	Useremo SQL per gestire i dati dell'applicazione.
	
\subsection{GraphQL} 
	GraphQL è un query language GraphQL progettato per creare applicazioi client fornendo una sintassi intuitiva e flessibile per descrivere le loro interazioni con i dati. 
	Una query scritta in GraphQL è una stringa interpretata da un server che ritorna dati in un formato specificato. 
	Questo linguaggio ci permetterà di interrogare il database che scriveremo in SQL.

\subsection{Android Beacon Library}
	Android Beacon Library è una libreria android che fornisce APIs per interagirre con i beacon. Permette ai dispositivi Android di usare i beacon similmente a quanto viene fatto dai dispositivi iOS. Un'applicazione può richiedere di ricevere notifiche quanto uno o più beacon appaiono o spariscono. Può richiedere inoltre di ricevere degli aggiornamenti da uno o più beacon alla frequenza di 1Hz.
	Android Beacon Library usa di default la specifica AltBeacon.
	
\subsubsection{AltBeacon}
	AltBeacon è una specifica che definisce il formato dei messaggi che i dispositivi BLE trasmettono. Permette di gestire l'interazione con i beacon come funziona per iBeacon: trasmettono un UUID dal beacon stesso e usano un database esterno per dare un significato ai beacon.
	Al contrario di iBeacon, AltBeacon è open source e offre una quantità di dati più grande da utilizzare.
	
	
	
	
	