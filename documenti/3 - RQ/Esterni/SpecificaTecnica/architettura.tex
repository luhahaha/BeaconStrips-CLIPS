\section{Descrizione architettura} 
\label{architettura}
	In questa sezione verrà descritta lo schema generale dell'architettura, nella sezione successiva % aggiungere label
	verranno riportate tutti i packages e le classi dettagliatamente.
	 Per i diagrammi delle classi e di attività è stato utilizzato il formalismo UML 2.0. \\
	 L'architettura descritta in questo documento è ad alto livello. Classi, sottoclassi e attributi verranno descritti più dettagliatamente nel periodo in cui attueremo la Progettazione di dettaglio. \\
	 Si procederà nella descrizione dell'architettura con un approccio top-down. Saranno descritte prima quindi le parti più generali per poi andare sempre più nello specifico. Si procederà quindi prima con la descizione dei packages e delle componenti, per poi passare alle singole classi. Successivamente verranno descritti i design pattern utilizzati.
	
	\subsection{Architettura generale}
	L'architettura generale dell'applicazione segue il modello client - server. \\
	In particolare si utilizzerà lo stile architetturale REST (representational state transfer) per coordinare compomenti, connettori e dati attraverso un sistema ipermediale distribuito dove l'attenzione è data al ruolo delle componenti ed ai vincoli imposti dalle loro interazioni tra di essi. \\
	Alcune caratteristiche e vincoli di REST sono i seguenti:	
	\begin{itemize}
		\item REST offre una interfaccia che separa il client dal server. Questa separazione permette di avere ben chiaro quali siano i ruoli delle due componenti in modo che non ci siano conflitti tra le due. Questo porta beneficio anche alla portabilità e alla scalabilità del prodotto;
		\item la comunicazione tra client e server è stateless, il che significa che ogni richiesta del client sarà interpretabile dal server senza che esso utilizzi alcun contesto presente nel server;
		\item i risultati ottenuti da una richiesta client-server possono essere salvati in cache così che se il client avesse la necessità di riutilizzare dati già richiesti lo possa fare senza effettuare nuove richieste.	
	\end{itemize}
	
	\subsection{Architettura del database}
	\label{architetturaDelDatabase}
	Nella parte server è presente il database dell'applicazione.\\
	In particolare verrà utilizzato un database relazionale \gl{MySQL} con la possibilità di interagire con esso tramite l'applicazione web \gl{phpMyAdmin} in modo da semplificarne l'amministrazione.
	Le rappresentazione del database verrà illustrata nel documento \DPdoc.
		