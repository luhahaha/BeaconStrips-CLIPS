\section{Resoconto dell’attività di verifica}
\label{resocontoDellAttivitaDiVerifica}
	\subsection{Periodo di Analisi e Management}
	\label{periodoDiAnalisiEManagement}
		\subsubsection{Processi}
		\label{processiAM}
			Sono riportati ora i valori di Schedule Variance e Budget Variance per le attività del Periodo di Analisi e Management.
			\begin{tabella}{!{\VRule}l!{\VRule}l!{\VRule}l!{\VRule}}
				\intestazionethreecol{Attività}{Schedule Variance}{Budget Variance}
				\ARdoc & \euro\ -10 & \euro\ +15 \\
				\Gldoc & \euro\ 0 & \euro\ 0 \\
				\NPdoc & \euro\ -5 & \euro\ +10 \\
				\PPdoc & \euro\ 0 & \euro\ +20 \\
				\PQdoc & \euro\ +5 & \euro\ -10 \\
				\SFdoc & \euro\ -10 & \euro\ 0 \\
				
				\hiderowcolors
				\caption{Esiti verifica sui processi - Periodo di Analisi e Management}
			\end{tabella}
			In totale sono stati registrati:
			\begin{itemize}
				\item \textbf{Schedule Variance:}\ \euro\ -20;
				\item \textbf{Budget Variance:}\ \euro\ +35.
			\end{itemize}
			Dai valori ottenuti si nota subito che in alcuni casi è stata prevista qualche ora di attività in più rispetto al necessario. Questo ha portato ad avere una Budget Variance positiva, mentre la causa di questo fatto è da ricercare nell'inesperienza del gruppo, che ha portato ad una valutazione leggermente errata del carico di lavoro necessario. Sempre a causa della poca esperienza del gruppo la Schedule Variance è risultata negativa, perché alcune attività si sono concluse leggermente in ritardo rispetto alle aspettative. La causa di questo fatto è da ricercare nell'organizzazione da parte dei membri del gruppo dei propri impegni, da cui sono derivati leggeri ritardi che si sono sommati. Il risultato comunque rientra nel limite ottimale, che sarebbe di \euro\ -144.
			%Nota sull'interpretazione della giustificazione sopra: quello che volevo indicare sopra per la Budget Variance è che inizialmente ci sono stati leggeri ritardi, che comunque poi sono stati compensati alla fine (vista anche la consegna anticipata rispetto al previsto)
		\subsubsection{Documenti}
		\label{documentiAM}
			Sono riportati qui i valori dell'indice Gulpease per ogni documento presente durante l'attività di analisi nel Periodo di Analisi e Management. Un documento è considerato valido soltanto se rispetta le metriche descritte secondo la sezione \hyperref[indiceGulpease]{Indice Gulpease}.
			\begin{tabella}{!{\VRule}l!{\VRule}c!{\VRule}c!{\VRule}}
				\intestazionethreecol{Documento}{Valore}{Esito}
				\ARdoc & 66 & Superato\\
				\Gldoc & 53 & Superato\\
				\NPdoc & 56 & Superato\\
				\PPdoc & 53 & Superato\\
				\PQdoc & 60 & Superato\\
				\SFdoc & 61 & Superato\\
				
				\hiderowcolors
				\caption{Esiti verifica documenti - Periodo di Analisi e Management}
			\end{tabella}
	\subsection{Periodo di Analisi di Dettaglio}
	\label{periodoDiAnalisiDiDettaglio}
		\subsubsection{Processi}
		\label{processiAD}
			Sono riportati ora i valori di Schedule Variance e Budget Variance per le attività del Periodo di Analisi di Dettaglio.
				\begin{tabella}{!{\VRule}l!{\VRule}c!{\VRule}c!{\VRule}}
				\intestazionethreecol{Attività}{Schedule Variance}{Budget Variance}
				\ARdoc & \euro\ -20 & \euro\ -25 \\
				\Gldoc & \euro\ 0 & \euro\ 0 \\
				\NPdoc & \euro\ +5 & \euro\ +5 \\
				\PPdoc & \euro\ -10 & \euro\ -10 \\
				\PQdoc & \euro\ 0 & \euro\ -5 \\
				\SFdoc & \euro\ 0 & \euro\ 0 \\
				
				\hiderowcolors
				\caption{Esiti verifica sui processi - Periodo di Analisi di Dettaglio}
			\end{tabella}
			In totale sono stati registrati:
			\begin{itemize}
				\item \textbf{Schedule Variance:}\ \euro\ -25;
				\item \textbf{Budget Variance:}\ \euro\ -35.
			\end{itemize}
			Dai valori ottenuti si nota subito che il lavoro richiesto per eseguire le attività è stato maggiore di quello pianificato. Questo perché le modifiche necessarie si sono rivelate essere più del previsto, soprattutto per quanto riguarda l'Analisi dei Requisiti. Di conseguenza la Budget Variance e la Schedule Variance sono risultate negative, ma comunque al di sotto del limite ottimale di \euro\ -65.
		\subsubsection{Documenti}
		\label{documentiAD}
			Sono riportati qui i valori dell'indice Gulpease per ogni documento presente durante l'attività di analisi nel Periodo di Analisi di Dettaglio. Un documento è considerato valido soltanto se rispetta le metriche descritte secondo la sezione \hyperref[indiceGulpease]{Indice Gulpease}.
			\begin{tabella}{!{\VRule}l!{\VRule}c!{\VRule}c!{\VRule}}
				\intestazionethreecol{Documento}{Valore}{Esito}
				\ARdoc & 85 & Superato\\
				\Gldoc & 54 & Superato\\
				\NPdoc & 51 & Superato\\
				\PPdoc & 62 & Superato\\
				\PQdoc & 61 & Superato\\
				\SFdoc & 61 & Superato\\
				
				\hiderowcolors
				\caption{Esiti verifica documenti - Periodo di Analisi di Dettaglio}
			\end{tabella}
	\subsection{Periodo di Progettazione Architetturale}
	\label{periodoDiProgettazioneArchitetturale}
		\subsubsection{Processi}
		\label{processiPA}
			Sono riportati ora i valori di Schedule Variance e Budget Variance per le attività del Periodo di Progettazione Architetturale.
				\begin{tabella}{!{\VRule}l!{\VRule}c!{\VRule}c!{\VRule}}
				\intestazionethreecol{Attività}{Schedule Variance}{Budget Variance}
				\ARdoc & \euro\ +20 & \euro\ +20 \\
				\Gldoc & \euro\ +5 & \euro\ 0 \\
				\NPdoc & \euro\ +10 & \euro\ +15 \\
				\PPdoc & \euro\ -5 & \euro\ 0 \\
				\PQdoc & \euro\ -5 & \euro\ 0 \\
				\STdoc & \euro\ -20 & \euro\ -50 \\
				
				\hiderowcolors
				\caption{Esiti verifica sui processi - Periodo di Progettazione Architetturale}
			\end{tabella}
			In totale sono stati registrati:
			\begin{itemize}
				\item \textbf{Schedule Variance:}\ \euro\ +5;
				\item \textbf{Budget Variance:}\ \euro\ -15.
			\end{itemize}
			Dai valori ottenuti si nota subito che le attività si sono concluse al massimo leggermente dopo i tempi previsti, ad eccezione della Specifica Tecnica che ha subito notevoli ritardi.\\
			Questo ha portato ad avere una Schedule Variance leggermente positiva perché il tempo risparmiato ha compensato quello perso per la Specifica Tecnica, in particolare sono risultate importanti le ore dedicate nella fase precedente perchè ha permesso di avere dei documenti praticamente già terminati.\\
			Al contrario la budget Variance è risultata negativa ma comunque al di sotto del limite ottimale di \euro\ -204. Questo perché le ore di lavoro previste per tutti i documenti sono risultate sufficienti o addirittura eccessive, ma la Specifica Tecnica ha richiesto molte più ore del previsto.\\
			I dati negativi relativi a questo documento sono da ricercare sempre nell'inesperienza del gruppo, che ha portato a sopravvalutare non di poco i tempi e le ore richiesti per portarlo a termine, in particolare per quanto riguarda la progettazione delle classi.
		\subsubsection{Documenti}
		\label{documentiPA}
			Sono riportati qui i valori dell'indice Gulpease per ogni documento presente durante l'attività di analisi nel Periodo di Progettazione Architetturale. Un documento è considerato valido soltanto se rispetta le metriche descritte secondo la sezione \hyperref[indiceGulpease]{Indice Gulpease}.
			\begin{tabella}{!{\VRule}l!{\VRule}c!{\VRule}c!{\VRule}}
				\intestazionethreecol{Documento}{Valore}{Esito}
				\ARdoc & 84  & Superato\\
				\Gldoc & 54 & Superato\\
				\NPdoc & 71 & Superato\\
				\PPdoc & 51 & Superato\\
				\PQdoc & 61 & Superato\\
				\STdoc & 86 & Superato\\
				
				\hiderowcolors
				\caption{Esiti verifica documenti - Periodo di Progettazione Architetturale}
			\end{tabella}
		\subsection{Periodo di Progettazione di Dettaglio e Codifica}
			\label{periodoDiProgettazioneDiDettaglioECodifica}
				\subsubsection{Processi}
				\label{processiPDC}
					Sono riportati ora i valori di Schedule Variance e Budget Variance per le attività del Periodo di Progettazione di Dettaglio e Codifica.
						\begin{tabella}{!{\VRule}l!{\VRule}c!{\VRule}c!{\VRule}}
						\intestazionethreecol{Attività}{Schedule Variance}{Budget Variance}
						\ARdoc & \euro\ +10 & \euro\ +10 \\
						\DPdoc & \euro\ -20 & \euro\ -25 \\
						\Gldoc & \euro\ +5 & \euro\ +10 \\
						\MUdoc & \euro\ +20 & \euro\ +15 \\
						\NPdoc & \euro\ -5 & \euro\ 0 \\
						\PPdoc & \euro\ +10 & \euro\ +20 \\
						\PQdoc & \euro\ +5 & \euro\ 0 \\
						\STdoc & \euro\ -10 & \euro\ -15 \\
						
						\hiderowcolors
						\caption{Esiti verifica sui processi - Periodo di Progettazione di Dettaglio e Codifica}
					\end{tabella}
					In totale sono stati registrati:
					\begin{itemize}
						\item \textbf{Schedule Variance:}\ \euro\ +15;
						\item \textbf{Budget Variance:}\ \euro\ +10.
					\end{itemize}
					Dai valori ottenuti si nota subito che le attività sono state concluse in anticipo rispetto alle previsioni, e con una quantità di lavoro minore di quella prevista. I valori di Schedule Variance e di Budget Variance sono risultati quindi positivi, perché la correzione di alcuni documenti ha richiesto meno tempo del previsto. Questo risultato è da cercarsi anche nel maggior carico di lavoro richiesto dalla parte di codifica, che quindi ha portato il gruppo a sistemare presto i documenti. 
				\subsubsection{Documenti}
				\label{documentiPDC}
					Sono riportati qui i valori dell'indice Gulpease per ogni documento presente durante l'attività di analisi nel Periodo di Progettazione di Dettaglio e Codifica. Un documento è considerato valido soltanto se rispetta le metriche descritte secondo la sezione \hyperref[indiceGulpease]{Indice Gulpease}.
					\begin{tabella}{!{\VRule}l!{\VRule}c!{\VRule}c!{\VRule}}
						\intestazionethreecol{Documento}{Valore}{Esito}
						\ARdoc & 79 & Superato\\
						\DPdoc & 79 & Superato\\
						\Gldoc & 48 & Superato\\
						\MUdoc & 50 & Superato\\
						\NPdoc & 70 & Superato\\
						\PPdoc & 51 & Superato\\
						\PQdoc & 52 & Superato\\
						\STdoc & 87 & Superato\\
						
						\hiderowcolors
						\caption{Esiti verifica documenti - Periodo di Progettazione di Dettaglio e Codifica}
					\end{tabella}
				\subsubsection{Soddisfacimento metriche}
				\label{codicePDC}
					\paragraph{Grado d'instabilità}
						Viene riportata di seguito una tabella riassuntiva che espone gli indici di instabilità per i componenti rilevati durante la progettazione.
						\begin{tabella}{!{\VRule}l!{\VRule}c!{\VRule}l!{\VRule}}
	\intestazionethreecol{Componente}{Indice d'instabilità}{Esito}
	CLIPS & 0 &  {\color[rgb]{0.44,0.74,0.48} Superato} \\
	 CLIPS::client & 0 & {\color[rgb]{0.44,0.74,0.48} Superato} \\
	 CLIPS::client::authentication & 0 & {\color[rgb]{0.44,0.74,0.48} Superato} \\
	  CLIPS::client::data & 0 & {\color[rgb]{0.44,0.74,0.48} Superato} \\
	  CLIPS::client::data::datamanager & 0.02 & {\color[rgb]{0.44,0.74,0.48} Superato} \\
	  CLIPS::client::data::urlrequest & 0.8 & {\color[rgb]{0.44,0.74,0.48} Superato} \\
	  CLIPS::client::pathprogress & 0 & {\color[rgb]{0.44,0.74,0.48} Superato} \\
	  CLIPS::client::viewcontroller & 0 & {\color[rgb]{0.44,0.74,0.48} Superato} \\
	  CLIPS::client::viewcontroller::building & 0 & {\color[rgb]{0.44,0.74,0.48} Superato} \\
	  CLIPS::client::viewcontroller::games & 0 & {\color[rgb]{0.44,0.74,0.48} Superato} \\
	  CLIPS::client::viewcontroller::savedresults & 0 & {\color[rgb]{0.44,0.74,0.48} Superato} \\
	  CLIPS::client::viewcontroller::utility & 0 & {\color[rgb]{0.44,0.74,0.48} Superato} \\
	  CLIPS::server & 0 & {\color[rgb]{0.44,0.74,0.48} Superato} \\
	  CLIPS::server::data & 0 & {\color[rgb]{0.44,0.74,0.48} Superato} \\
	  CLIPS::server::urlrequesthandler & 0 & {\color[rgb]{0.44,0.74,0.48} Superato}
	  \end{tabella}

					\paragraph{Classi}
						Viene riportata di seguito una tabella riassuntiva che espone gli esiti delle metriche sulle classi.\\
						Per ragioni di spazio la Complessità Ciclomatica è stata abbreviata in CC.\\
						Inoltre si è deciso di riportare, per quanto riguarda il numero di parametri dei metodi, il numero del metodo della classe con numero massimo di parametri, escludendo dal calcolo i costruttori.
						\paragraph{Client}
\begin{tabella}{!{\VRule}l!{\VRule}X[c,b,c]!{\VRule}c!{\VRule}X[c,b,c]!{\VRule}c!{\VRule}}
    \intestazionefivecol{Classe}{Numero attributi}{CC}{Numero massimo parametri}{Esito}
	AccountActivity & 0 & 2 & 1 & {\color[rgb]{0,1,0} Superato} \\
	AccountView & 4 & 1 & 0 & {\color[rgb]{0,1,0} Superato} \\
	ConfirmRegistration & 0 & 2 & 1 & {\color[rgb]{0,1,0} Superato} \\
	ForgotPasswordActivity & 0 & 2 & 1 & {\color[rgb]{0,1,0} Superato} \\
	ForgotPasswordView & 2 & 1 & 0 & {\color[rgb]{0,1,0} Superato} \\
	LoginActivity & 0 & 2 & 1 & {\color[rgb]{0,1,0} Superato} \\
	LoginView & 4 & 1 & 0 & {\color[rgb]{0,1,0} Superato} \\
	RecoverPassword & 0 & 1 & 1 & {\color[rgb]{0,1,0} Superato} \\
	RegistrationActivity & 0 & 9 & 1 & {\color[rgb]{0,1,0} Superato} \\
	RegistrationView & 9 & 1 & 0 & {\color[rgb]{0,1,0} Superato} \\
	UpdateUserInfoActivity & 0 & 2 & 1 & {\color[rgb]{0,1,0} Superato} \\
	UpdateUserInfoView & 6 & 1 & 0 & {\color[rgb]{0,1,0} Superato} \\
	AbstractQuiz & 2 & 1 & 0 & {\color[rgb]{0,1,0} Superato} \\
	AppInfo & 3 & 1 & 0 & {\color[rgb]{0,1,0} Superato} \\
	Beacon & 4 & 1 & 0 & {\color[rgb]{0,1,0} Superato} \\
	BiggerShape & 4 & 1 & 0 & {\color[rgb]{0,1,0} Superato} \\
	BiggerShapeTest & 3 & 1 & 0 & {\color[rgb]{0,1,0} Superato} \\
	Building & 16 & 1 & 0 & {\color[rgb]{0,1,0} Superato} \\
	GameCollection & 2 & 1 & 0 & {\color[rgb]{0,1,0} Superato} \\
	LeaderBoard & 5 & 1 & 0 & {\color[rgb]{0,1,0} Superato} \\
	LinearScoringAlgorithm & 6 & 1 & 3 & {\color[rgb]{0,1,0} Superato} \\
	LoggedUser & 3 & 1 & 0 & {\color[rgb]{0,1,0} Superato} \\
	MultipleChoiceImageQuiz & 1 & 1 & 0 & {\color[rgb]{0,1,0} Superato} \\
	MultipleChoiceTest & 3 & 1 & 0 & {\color[rgb]{0,1,0} Superato} \\
	MultipleChoiceTextQuiz & 2 & 1 & 1 & {\color[rgb]{0,1,0} Superato} \\
	Path & 4 & 4 & 2 & {\color[rgb]{0,1,0} Superato} \\
	PathInfo & 5 & 1 & 0 & {\color[rgb]{0,1,0} Superato} \\
	PathProgress & 4 & 2 & 1 & {\color[rgb]{0,1,0} Superato} \\
	PathResult & 7 & 1 & 0 & {\color[rgb]{0,1,0} Superato} \\
	Proof & 5 & 1 & 0 & {\color[rgb]{0,1,0} Superato} \\
	ProofResult & 4 & 1 & 1 & {\color[rgb]{0,1,0} Superato} \\
	Proximity & 3 & 1 & 0 & {\color[rgb]{0,1,0} Superato} \\
	Score & 3 & 1 & 0 & {\color[rgb]{0,1,0} Superato} \\
	ScoringAlgorithmFactory & 0 & 1 & 1 & {\color[rgb]{0,1,0} Superato} \\
	Step & 3 & 1 & 0 & {\color[rgb]{0,1,0} Superato} \\
	Test & 0 & 1 & 0 & {\color[rgb]{0,1,0} Superato} \\
	TestBuilder & 0 & 1 & 8 & {\color[rgb]{0,1,0} Superato} \\
	TrueFalseImageQuiz & 1 & 1 & 0 & {\color[rgb]{0,1,0} Superato} \\
	TrueFalseTest & 2 & 1 & 0 & {\color[rgb]{0,1,0} Superato} \\
	TrueFalseTextQuiz & 1 & 1 & 1 & {\color[rgb]{0,1,0} Superato} \\
	Utility & 0 & 3 & 4 & {\color[rgb]{0,1,0} Superato} \\
	AbstractDataManagerListener & 0 & 1 & 1 & {\color[rgb]{0,1,0} Superato} \\
	AppInfoDataRequest & 3 & 2 & 1 & {\color[rgb]{0,1,0} Superato} \\
	BuildingsDataRequest & 3 & 4 & 1 & {\color[rgb]{0,1,0} Superato} \\
	CheckResult & 4 & 1 & 1 & {\color[rgb]{0,1,0} Superato} \\
	DataManager & 5 & 5 & 0 & {\color[rgb]{0,1,0} Superato} \\
	DataRequestMaker & 0 & 1 & 5 & {\color[rgb]{0,1,0} Superato} \\
	DBHandler & 1 & 4 & 3 & {\color[rgb]{0,1,0} Superato} \\
	GetRankingDataRequest & 1 & 3 & 1 & {\color[rgb]{0,1,0} Superato} \\
	GetResultsDataRequest & 0 & 6 & 1 & {\color[rgb]{0,1,0} Superato} \\
	LoginManager & 4 & 5 & 4 & {\color[rgb]{0,1,0} Superato} \\
	PathDataRequest & 1 & 4 & 1 & {\color[rgb]{0,1,0} Superato} \\
	SaveDataManager & 2 & 1 & 1 & {\color[rgb]{0,1,0} Superato}
	SaveResultDataRequest & 1 & 1 & 0 & {\color[rgb]{0,1,0} Superato} \\
	AbstractUrlRequestListener & 0 & 1 & 1 & {\color[rgb]{0,1,0} Superato} \\
	AppInfoRequest & 0 & 1 & 0 & {\color[rgb]{0,1,0} Superato} \\
	BuildingsRequest & 0 & 2 & 3 & {\color[rgb]{0,1,0} Superato} \\
	ChangeProfileDataRequest & 0 & 2 & 3 & {\color[rgb]{0,1,0} Superato} \\
	CheckRequest & 0 & 2 & 3 & {\color[rgb]{0,1,0} Superato} \\
	GetProfileDataRequest & 0 & 1 & 0 & {\color[rgb]{0,1,0} Superato} \\
	GetRankingRequest & 0 & 2 & 1 & {\color[rgb]{0,1,0} Superato} \\
	GetResultsRequest & 0 & 1 & 0 & {\color[rgb]{0,1,0} Superato} \\
	LoginRequest & 0 & 2 & 2 & {\color[rgb]{0,1,0} Superato} \\
	LogoutRequest & 0 & 1 & 0 & {\color[rgb]{0,1,0} Superato} \\
	PathRequest & 0 & 1 & 0 & {\color[rgb]{0,1,0} Superato} \\
	RegistrationRequest & 0 & 2 & 3 & {\color[rgb]{0,1,0} Superato} \\
	RequestMaker & 0 & 1 & 5 & {\color[rgb]{0,1,0} Superato} \\
	SaveResultRequest & 0 & 1 & 4 & {\color[rgb]{0,1,0} Superato} \\
	ServerError & 3 & 1 & 0 & {\color[rgb]{0,1,0} Superato} \\
	URLDataConstants & 1 & 0 & 0 & {\color[rgb]{0,1,0} Superato} \\
	URLRequest & 7 & 8 & 1 & {\color[rgb]{0,1,0} Superato} \\
	BeaconDiscover & 1 & & & {\color[rgb]{0,1,0} Superato} \\
	BeaconDiscoverDelegate & 0 & 1 & 1 & {\color[rgb]{0,1,0} Superato} \\
	PathProgressController & 4 & 3 & 1 & {\color[rgb]{0,1,0} Superato} \\
	PathProgressControllerDelegate & 0 & 1 & 1 & {\color[rgb]{0,1,0} Superato} \\
	RawBeacon & 3 & 1 & 0 & {\color[rgb]{0,1,0} Superato} \\
	BuildingActivity & 2 & 6 & 1 & {\color[rgb]{0,1,0} Superato} \\
	BuildingRow & 2 & 1 & 0 & {\color[rgb]{0,1,0} Superato} \\
	BuildingSearchActivity & 0 & 2 & 1 & {\color[rgb]{0,1,0} Superato} \\
	BuildingSearchView & 2 & 1 & 0 & {\color[rgb]{0,1,0} Superato} \\
	BuildingView & 13 & 1 & 0 & {\color[rgb]{0,1,0} Superato} \\
	PathActivity & 1 & 1 & 1 & {\color[rgb]{0,1,0} Superato} \\
	PathRow & 1 & 1 & 0 & {\color[rgb]{0,1,0} Superato} \\
	PathView & 5 & 1 & 0 & {\color[rgb]{0,1,0} Superato} \\
	SearchCriteriaRow & 4 & 1 & 0 & {\color[rgb]{0,1,0} Superato} \\
	MultipleChoiceQuizActivity & 0 & 1 & 1 & {\color[rgb]{0,1,0} Superato} \\
	MultipleChoiceQuizView & 4 & 1 & 0 & {\color[rgb]{0,1,0} Superato} \\
	ProofActivity & 0 & 1 & 1 & {\color[rgb]{0,1,0} Superato} \\
	ProofResultActivity & 0 & 1 & 1 & {\color[rgb]{0,1,0} Superato} \\
	ProofResultRow & 0 & 1 & 0 & {\color[rgb]{0,1,0} Superato} \\
	ProofResultView & 1 & 1 & 0 & {\color[rgb]{0,1,0} Superato} \\
	ProofView & 2 & 1 & 0 & {\color[rgb]{0,1,0} Superato} \\
	QuizActivity & 0 & & & {\color[rgb]{0,1,0} Superato} \\
	QuizResultView & 2 & 1 & 0 & {\color[rgb]{0,1,0} Superato} \\
	QuizView & 1 & 1 & 0 & {\color[rgb]{0,1,0} Superato} \\
	SearchNewStepActivity & 0 & 1 & 1 & {\color[rgb]{0,1,0} Superato} \\
	SearchNewStepView & 2 & 1 & 0 & {\color[rgb]{0,1,0} Superato} \\
	TestResultView & 0 & 1 & 0 & {\color[rgb]{0,1,0} Superato} \\
	TestView & 0 & 1 & 0 & {\color[rgb]{0,1,0} Superato} \\
	TrueFalseQuizActivity & 0 & 1 & 1 & {\color[rgb]{0,1,0} Superato} \\
	TrueFalseQuizView & 2 & 1 & 0 & {\color[rgb]{0,1,0} Superato} \\
	PlayerRow & 2 & 1 & 0 & {\color[rgb]{0,1,0} Superato} \\
	RankingActivity & 0 & 1 & 1 & {\color[rgb]{0,1,0} Superato} \\
	RankingView & 2 & 1 & 0 & {\color[rgb]{0,1,0} Superato} \\
	ResultActivity & 0 & 1 & 1 & {\color[rgb]{0,1,0} Superato} \\
	ResultRow & 5 & 1 & 0 & {\color[rgb]{0,1,0} Superato} \\
	ResultView & 12 & 1 & 0 & {\color[rgb]{0,1,0} Superato} \\
	SavedResultsActivity & 0 & 1 & 1 & {\color[rgb]{0,1,0} Superato} \\
	SavedResultView & 4 & 1 & 0 & {\color[rgb]{0,1,0} Superato} \\
	AppInfoActivity & 0 & 1 & 1 & {\color[rgb]{0,1,0} Superato} \\
	AppInfoView & 1 & 1 & 0 & {\color[rgb]{0,1,0} Superato} \\
	MenuActivity & 0 & 9 & 1 & {\color[rgb]{0,1,0} Superato} \\
	MenuView & 0 & 1 & 0 & {\color[rgb]{0,1,0} Superato} \\
\end{tabella}

\paragraph{Server}
\begin{tabella}{!{\VRule}l!{\VRule}c!{\VRule}c!{\VRule}c!{\VRule}l!{\VRule}}
	\intestazionefivecol{Classe}{Numero attributi}{Complessità ciclomatica}{Numero massimo parametri}{Esito}
	AppInfo & 0 & & & {\color[rgb]{0,1,0} Superato} \\
	AuthToken & 0 & & & {\color[rgb]{0,1,0} Superato} \\
	Beacon & 0 & & & {\color[rgb]{0,1,0} Superato} \\
	Building & 0 & & & {\color[rgb]{0,1,0} Superato} \\
	LeaderBoard & 0 & & & {\color[rgb]{0,1,0} Superato} \\
	PathResult & 0 & & & {\color[rgb]{0,1,0} Superato} \\
	Path & 0 & & & {\color[rgb]{0,1,0} Superato} \\
	ProofResult & 0 & & & {\color[rgb]{0,1,0} Superato} \\
	Proof & 0 & & & {\color[rgb]{0,1,0} Superato} \\
	Proximity & 0 & & & {\color[rgb]{0,1,0} Superato} \\
	ScoreManager & 0  & & & {\color[rgb]{0,1,0} Superato} \\
	Step & 0 & & & {\color[rgb]{0,1,0} Superato} \\
	User & 0 & & & {\color[rgb]{0,1,0} Superato} \\
	AppInfoProvider & 0 & & & {\color[rgb]{0,1,0} Superato} \\
	BuildingsProvider & 0 & & & {\color[rgb]{0,1,0} Superato} \\
	DBHandler & 1 & & & {\color[rgb]{0,1,0} Superato} \\
	EmailChecker & 0 & & & {\color[rgb]{0,1,0} Superato} \\
	LoginHandler & 0 & & & {\color[rgb]{0,1,0} Superato} \\
	LogoutHandler & 0 & & & {\color[rgb]{0,1,0} Superato} \\
	PasswordChecker & 1 & & & {\color[rgb]{0,1,0} Superato} \\
	PathProvider & 0 & & & {\color[rgb]{0,1,0} Superato} \\
	PathResultsProvider & 0 & & & {\color[rgb]{0,1,0} Superato} \\
	RegistrationFieldsValidator & 0 & & & {\color[rgb]{0,1,0} Superato} \\
	RegistrationHandler & 0 & & & {\color[rgb]{0,1,0} Superato} \\
	URLRequestHandler & 2 & & & {\color[rgb]{0,1,0} Superato} \\
	URLRequestRouter & 0 & & & {\color[rgb]{0,1,0} Superato} \\
	UsernameChecker & 0 & & & {\color[rgb]{0,1,0} Superato} \\
	UserResultProvider & 0 & & & {\color[rgb]{0,1,0} Superato} \\
\end{tabella}