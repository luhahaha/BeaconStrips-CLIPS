\documentclass[a4paper,titlepage]{article}

\makeatletter
\def\input@path{{../../../template/}{./img}}
\makeatother

\usepackage{Comandi}
\usepackage{Riferimenti}
\usepackage{Stile}

\usepackage{eurosym}
\usepackage{comment}
\usepackage{hyperref}

\def\NOME{Manuale Utente}
\def\VERSIONE{1.0.0}
\def\DATA{2016-08-16}
\def\REDATTORE{Viviana Alessio}
\def\VERIFICATORE{Luca Soldera}
\def\RESPONSABILE{Tommaso Panozzo}
\def\USO{Esterno}
\def\DESTINATARI{\COMMITTENTE \\ & \CARDIN \\ & \PROPONENTE}
\def\SOMMARIO{Manuale destinato agli utenti del \gl{progetto} \PROGETTO\  del gruppo \AUTORE.}


\begin{document}


\maketitle

\begin{diario}
	\modifica{Viviana Alessio}{\PRJ}{Stesura Introduzione}{2016-08-11}{0.2.0}
	\modifica{Viviana Alessio}{\PRJ}{Stesura intestazione e indice documento}{2016-08-10}{0.1.0}
\end{diario}

\newpage
\tableofcontents
\newpage
\listoftables
\newpage
\listoffigures
\newpage

\section{Introduzione}
	\subsection{Scopo del documento} 
	Questo documento ha lo scopo di spiegare dettagliatamente le strategie secondo cui il gruppo \gl{Beacon Strips} intende condurre il progetto didattico.
	\subsection{Riferimenti}
	\subsection{Ciclo di vita}
	\subsection{Scadenze}
	

\section{Requisiti di Sistema} 
L'applicazione da noi realizzata può essere eseguita solamente su dispositivi mobile con sistema operativo \gl{Android} di versione uguale o superiore alla 5.0 (Lollipop, \gl{API} 21). \\
Il dispositivo mobile in uso deve inoltre avere attivi i seguenti servizi:
\begin{itemize}
	\item \gl{GPS};
	\item connessione internet attiva;
	\item \gl{Bluetooth};
\end{itemize} 

\section{Utilizzo dell'applicazione}
\subsection{Informazioni generali}
Tutte le funzionalità dell'applicazione sono accessibili dal menu laterali apribile tramite uno \gl{swipe} da sinistra verso destra oppure cliccando sul bottone \gl{Hamburger button} presente in alto a sinistra. 
\subsection{Autenticazione} 
\subsubsection{Registrazione}

\begin{figure}[!h]
	\centering
	\includegraphics[scale=0.15]{screenshot/Registrazione}
	\caption{Schermata di registrazione}
\end{figure}

Attraverso questa schermata l'utente deve inserire nella form tutti i dati richiesti e successivamente potrà registrarsi cliccando l'apposito bottone ``REGISTRATI''. Successivamente verrà automaticamente autenticato nel sistema.
Se l'utente invece si fosse già registrato nel sistema precedentemente potrà accedere alla schermata di Login tramite il bottone ``GIÀ REGISTRATO? EFFETTUA IL LOGIN'', oppure tramite l'apposito elemento nel menu.
\newpage

\subsubsection{Login}
\begin{figure}[!h]
	\centering
	\includegraphics[scale=0.15]{screenshot/login}
	\caption{Schermata di registrazione}
\end{figure}
Qualora un utente si fosse già registrato nel sistema può accedere a questa schermata che gli consentirà, inserendo Email e Password di accedere al sistema.
\newpage


\subsubsection{Account}
\begin{figure}[!h]
	\centering
	\includegraphics[scale=0.15]{screenshot/account}
	\caption{Schermata di registrazione}
\end{figure}
In questa schermata è possibile vedere i propri dati di registrazione. Sono presenti anche due bottoni: ``CAMBIA LE TUE CREDENZIALI'' indirizza l'utente in una pagina da cui potrà cambiare il proprio username e la propria password, ``VISUALIZZA I TUOI RISULTATI'' indirizza ad una schermata in cui l'utente può visualizzare i propri risultati

\appendix
\section{Glossario}
\subsection{Android}
Android è un sistema operativo per dispositivi mobili sviluppato da Google Inc..
\subsection{API}
Acronimo di ‘‘Application Programming Interface’’. Indica un insieme di procedure disponibili al programmatore, che lo aiutano a svolgere un certo compito all'interno del programma.
\subsection{Beacon}
Nuova classe di trasmettitori, a bassa potenza e a basso costo, che possono notificare la propria presenza a dispositivi vicini. La tecnologia consente ad uno smartphone o ad un altro dispositivo di effettuare delle azioni quando sono nelle vicinanza di un beacon. Sfrutta la tecnologia Bluetooth Low Energy (\gl{BLE}), conosciuta anche come Bluetooth Smart. I beacon usano la percezione di prossimità del Bluetooth Low Energy per trasmettere un identificativo unico universale (\gl{UUID}), che sarà poi letto da una specifica app o sistema operativo.
\subsection{BLE}
Acronimo di Bluetooth Low Energy, è una nuova tecnologia del nuovo standard Bluetooth 4. La principale caratteristica è un sostanziale risparmio energetico a discapito della velocità di trasmissione, infatti si passa da 24 Mbit/s a 1 Mbit/s.
\subsection{Bluetooth}
È uno standard tecnico-industriale di trasmissione dati per reti personali senza fili. Fornisce un metodo standard, economico e sicuro per scambiare informazioni tra dispositivi diversi attraverso una frequenza radio sicura a corto raggio.
\subsection{Checkbox}
è un controllo grafico con cui l'utente può effettuare selezioni. Solitamente, i checkbox sono mostrati sullo schermo come dei quadrati che possono contenere spazio bianco (quando non sono selezionati), segno di spunta (quando sono selezionati) o un quadrato (indeterminato). Adiacente al checkbox è solitamente mostrata una breve descrizione. Per invertire lo stato (selezionato/non selezionato) del checkbox è sufficiente cliccare sul riquadro o sulla descrizione.
\subsection{GPS}
Acronimo di sistema di posizionamento globale. È un sistema di posizionamento e navigazione satellitare civile che, attraverso una rete dedicata di satelliti artificiali in orbita, fornisce ad un terminale mobile o ricevitore GPS informazioni sulle sue coordinate geografiche ed orario (riferimento: \url{https://it.wikipedia.org/wiki/Sistema_di_posizionamento_globale}).
\subsection{Hamburger button}
È un simbolo, tipico del material design, formato da tre linee orizzontali parallele. Viene usato per rappresentare il bottone di apertura menu.
\subsection{Prodotto}
Il prodotto è il risultato di un insieme di attività. In questo caso il termine è da intendersi come un sinonimo di \PROGETTO.
\subsection{Progetto}
Il progetto è un insieme di azioni organizzate atte a perseguire uno scopo specifico. Nel nostro caso indica tutta l'attività di progettazione di codice e di documenti e della loro verifica, quindi il prodotto finale sarà \PROGETTO. Di conseguenza questo termine verrà usato spesso come sinonimo di \gl{prodotto}.
\subsection{Slider}
Uno slider è un componente grafico (widget) con il quale un utente può impostare un valore muovendo un indicatore, solitamente con uno spostamento orizzontale. In alcuni casi l'utente può anche cliccare in un punto dello slider per cambiare le impostazioni.
Uno slider è solitamente rappresentato con una barra orizzontale che indica l'intervallo dei valori validi, e di un indicatore che ha la duplice funzione di indicare il valore corrente e di permettere all'utente di modificare il valore.
\subsection{Swipe}
Movimento compiuto dall'utente su dispositivi forniti di touchscreen. Si intente il movimento del dito dell'utente che dal bordo, tipicamente sinistro, va versso il centro dello schermo.
\subsection{UUID}
Acronimo dell'identificativo univoco universale, un identificativo standard usato nelle infrastrutture software, standardizzato come parte di un ambiente distribuito di computazione (riferimento: \url{https://it.wikipedia.org/wiki/UUID}).
\end{document}