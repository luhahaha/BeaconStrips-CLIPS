\documentclass[a4paper,titlepage]{article}

\makeatletter
\def\input@path{{../template/}}
\makeatother

\usepackage{Comandi}
\usepackage{Riferimenti}
\usepackage{Stile}

\def\NOME{Titolo del Documento}
\def\VERSIONE{0.1}
\def\DATA{\today}
\def\REDATTORE{Nome Cognome \\ & Tizio Rossi \\ & Caio Bianchi}
\def\VERIFICATORE{Nome Cognome}
\def\RESPONSABILE{Nome Cognome}
\def\USO{Esterno}
\def\DESTINATARI{\COMMITTENTE \\ & \CARDIN \\ & \PROPONENTE}
\def\SOMMARIO{Breve descrizione del doc.}


\begin{document}

\maketitle

\begin{diario}
  \modifica{Tommaso Panozzo}{\AM}{Stesura Preliminare}{2016-03-11}{0.1}
  \modifica{Romina Power}{\PM}{riepilogo}{2016-03-12}{0.2}
  \modifica{Loredana Lecciso}{\AN}{riepilogo}{2016-03-13}{1.0}
\end{diario}

\newpage
\tableofcontents

\newpage
\section{Perché usare il template}
\label{sec:PercheTemplate}


Per scrivere documenti in \LaTeX  è sufficiente basarsi su questo template e aggiungere il minor numero possibile di pacchetti/impostazioni personali o altro.

Il vantaggio che si ha nell'usare questo \textit{template} sta nel poter modificarealcuni elementi comuni ai vari elementi in seguito in particolare.

\section{Configurazioni necessarie}
\label{sec:ConfigurazioniNecessarie}

Per poter usare questo template è necessario compiere alcune piccole configurazioni:

\subsection{\texttt{sty} files}
\label{sub:styFiles}

È necessario includere con \texttt{usepackage} i file \file{Comandi.sty}, \file{Stile.sty} e \file{Riferimenti.sty}.
Per farlo la path alla cartella che li contiene dev'essere inserita all'inizio dei file \texttt{.tex} alle righe:

\begin{lstlisting}
  \makeatletter
  \def\input@path{{../template/}}
  \makeatother
\end{lstlisting}

\subsection{Definizioni}
\label{sub:Definizioni}

Inoltre è necessario definire alcune variabili prima dell'inizio del documento.\\
Le variabili sono:
\begin{itemize}
  \item[\textbf{Nome}] Il titolo del documento \textit{senza versione}
  \item[\textbf{Versione}] La versione attuale del documento
  \item[\textbf{Data}] La data di creazione della versione attuale
  \item[\textbf{Redattore}] Nome e Cognome del redattore, se i redattori sono più di uno vanno scritti separati da \texttt{\textbackslash \textbackslash \&}
  \item[\textbf{Verificatore}] Nome e Cognome con le stesse regole di redattore
  \item[\textbf{Responsabile}] Nome e Cognome del responsabile
  \item[\textbf{Uso}] Per indicare se l'uso è interno o esterno
  \item[\textbf{Destinatari}] La lista di distribuzione, le persone cui è destinato il documento, separati da \texttt{\textbackslash \textbackslash \&}
  \item[\textbf{Sommario}] Un breve sommario del documento.
\end{itemize}

\section{Frontespizio}
\label{sec:Frontespizio}

La creazione della copertina si ottiene semplicemente con il comando \texttt{maketitle}.

\section{Diario Modifiche}
\label{sec:DiarioModifiche}

Il \textbf{Diario delle Modifiche} si ottiene utilizzando l'environment \texttt{diario} e le entry si inseriscono con il comando \texttt{modifica\{Nome\}\{Ruolo\}\{Riepilogo\}\{Data\}\{Versione\}}.

\section{Riferimenti}
\label{sec:Riferimenti}

È importante consultare il file \file{Riferimenti.sty} ed utilizzare le costanti ivi definite per permettere riferirsi a file/documenti/ruoli in maniera consistente tra i vari documenti.

\section{Comandi}
\label{sec:Comandi}

Così come per i riferimenti, è importante anche utilizzare i comandi contenuti nel file \file{Comandi} così da uniformare la presentazione di concetti nei vari documenti. Ad esempio ogni parola che si troverà nel glossario va scritta con il comando \texttt{gl} che la farà apparire \gl{parola}.

\end{document}
