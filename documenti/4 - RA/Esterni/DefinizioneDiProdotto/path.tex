{
   id: String (l'id del percorso)
   algorithm: \hyperref[sub:Algorithm]{Algorithm} (l'algoritmo per …)
}


\begin{lstlisting}[language=json,firstnumber=1]
{"menu": {
  "id": "file",
  "value": "File",
  "popup": {
    "menuitem": [
      {"value": "New", "onclick": "CreateNewDoc()"},
      {"value": "Open", "onclick": "OpenDoc()"},
      {"value": "Close", "onclick": "CloseDoc()"}
    ]
  }
}}
0123456789
\end{lstlisting}

%Path
\begin{lstlisting}[language=json,firstnumber=1]
{
   id             : int (l'id del percorso),
   title          : String (il titolo del percorso),
   startingMessage: String (il messaggio mostrato all'inizio del percorso),
   rewardMessage  : String (il messaggio mostrato alla fine del percorso),
   steps          : [
      \hyperref[sub:Step]{Step},
      ...
   ] (l'array che contiene le tappe del percorso)
}
\end{lstlisting}

%Step
\begin{lstlisting}[language=json,firstnumber=1]
{
   stopBeacon     : \hyperref[sub:Beacon]{Beacon} (il beacon che segnala la posizione della tappa)
   proximities: [
      \hyperref[sub:Proximity]{Proximity}
      ...
   ] (l'array che contiene gli oggetti in cui sono salvati i dati per segnalare all'utente la distanza che manca per raggiungere questa stazione)
   proof      : \hyperref[sub:Proof]{Proof} (la prova che l'utente deve affrontare quando raggiunge la stazione)
}
\end{lstlisting}

%Beacon
\begin{lstlisting}[language=json,firstnumber=1]
{
   id   : int (l'id del beacon)
   UUID : String (la stringa che identifica i beacon dell'applicazione o i beacon dell'edificio, è il primo campo che permette di identificare il beacon)
   Major: int (il secondo campo che permette di identificare il beacon)
   Minor: int (il terzo campo che permette di identificare il beacon)
}
\end{lstlisting}

%Proximity
\begin{lstlisting}[language=json,firstnumber=1]
{
   beacon: \hyperref[sub:Beacon]{Beacon} (il beacon che viene usato per comunicare all'utente quanto è vicino alla stazione cercata)
   percentage: double (il valore percentuale della distanza percorsa, utile per una rappresentazione grafica della distanza che manca)
   textToDisplay: String (il testo da far visualizzare all'applicazione per informare l'utente se si sta avvicinando e quanto è distante)
}
\end{lstlisting}
