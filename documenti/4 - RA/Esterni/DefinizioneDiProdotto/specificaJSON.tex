\section{Specifica JSON client-server}
\label{sec:Specifica JSON client-server}

In questa sezione definiamo come sono strutturati i JSON utilizzati per la trasmissione di dati tra client e server.

\subsection{AppInfo}
\label{sub:AppInfo}
\begin{lstlisting}[language=json,firstnumber=1]
{
   description  : String (testo di descrizione delle funzioni dell'App)
   supportemail : String (email per il supporto)
   websiteURL   : String (url del sito di supporto)
   discoveryUUID: String (UUID dei beacon usati dall'App)
}
\end{lstlisting}



\subsection{Path}
\label{sub:Path}
{
   id: String (l'id del percorso)
   algorithm: \hyperref[sub:Algorithm]{Algorithm} (l'algoritmo per …)
}


\begin{lstlisting}[language=json,firstnumber=1]
{"menu": {
  "id": "file",
  "value": "File",
  "popup": {
    "menuitem": [
      {"value": "New", "onclick": "CreateNewDoc()"},
      {"value": "Open", "onclick": "OpenDoc()"},
      {"value": "Close", "onclick": "CloseDoc()"}
    ]
  }
}}
0123456789
\end{lstlisting}

%Path
\begin{lstlisting}[language=json,firstnumber=1]
{
   id             : int (l'id del percorso),
   title          : String (il titolo del percorso),
   startingMessage: String (il messaggio mostrato all'inizio del percorso),
   rewardMessage  : String (il messaggio mostrato alla fine del percorso),
   steps          : [
      \hyperref[sub:Step]{Step},
      ...
   ] (l'array che contiene le tappe del percorso)
}
\end{lstlisting}

%Step
\begin{lstlisting}[language=json,firstnumber=1]
{
   stopBeacon     : \hyperref[sub:Beacon]{Beacon} (il beacon che segnala la posizione della tappa)
   proximities: [
      \hyperref[sub:Proximity]{Proximity}
      ...
   ] (l'array che contiene gli oggetti in cui sono salvati i dati per segnalare all'utente la distanza che manca per raggiungere questa stazione)
   proof      : \hyperref[sub:Proof]{Proof} (la prova che l'utente deve affrontare quando raggiunge la stazione)
}
\end{lstlisting}

%Beacon
\begin{lstlisting}[language=json,firstnumber=1]
{
   id   : int (l'id del beacon)
   UUID : String (la stringa che identifica i beacon dell'applicazione o i beacon dell'edificio, è il primo campo che permette di identificare il beacon)
   Major: int (il secondo campo che permette di identificare il beacon)
   Minor: int (il terzo campo che permette di identificare il beacon)
}
\end{lstlisting}

%Proximity
\begin{lstlisting}[language=json,firstnumber=1]
{
   beacon: \hyperref[sub:Beacon]{Beacon} (il beacon che viene usato per comunicare all'utente quanto è vicino alla stazione cercata)
   percentage: double (il valore percentuale della distanza percorsa, utile per una rappresentazione grafica della distanza che manca)
   textToDisplay: String (il testo da far visualizzare all'applicazione per informare l'utente se si sta avvicinando e quanto è distante)
}
\end{lstlisting}


\subsection{Algorithm}
\label{sub:Algorithm}
\begin{lstlisting}[language=json,firstnumber=1]
{
   minScore      : int
   maxScore      : int

   minTime       : double
   maxTime       : double

   timeWeight    : double
   accuracyWeight: double
}
per la spiegazione delle chiavi si veda la \hyperref[sec:Algoritmo di assegnazione del punteggio]{sezione corrispondente}.
\end{lstlisting}

\subsection{Proof}
\label{sub:Proof}

Le prove possono essere di tipi diversi: il primo sono domande a scelta multipla, con eventuali immagini.
\textit{indichiamo le chiavi opzionali con un punto di domanda}

\subsubsection{Scelta Multipla}
\label{subs:Scelta Multipla}
\begin{lstlisting}[language=json,firstnumber=1]
{
   testType: "standardQuiz" (indica il tipo di test)
   testTypeVersion: 0.1     (indica la versione, per compatibilità con futuri aggiornamenti)
   data : (dati delle domande) {
      shuffleAnswers: Bool (se vero il client deve presentare le domande in ordine casuale)
      shuffleQuestions: Bool (se vero il client deve mescolare le risposte per ogni domanda)
      questions : (domande) [
         {
            question: String (testo della domanda)
            answers : [String] (array di risposte tra cui scegliere)
            image   : String? (url dell'immagine da mostrare, opzionale)
            correctIndex : 0 (indice della risposta corretta)
         },
         . . .
      ]
   }
}
\end{lstlisting}

\subsubsection{Vero o Falso}
\label{subs:Vero o Falso}
\begin{lstlisting}[language=json,firstnumber=1]
{
   testType: "VeroFalso"
   testTypeVersion: 0.1
   data : {
      shuffleQuestions: Bool (vero se le domande vanno mescolate)
      questions : [
         {
            question: String (domanda)
            correctAnswer: Bool (risposta)
         },
         . . .
      ]
   }
}

\end{lstlisting}
