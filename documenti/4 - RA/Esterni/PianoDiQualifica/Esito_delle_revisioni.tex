\section{Esito delle revisioni}
\label{esitoDelleRevisioni}
	Durante lo sviluppo del progetto ci saranno quattro revisioni a cui sottoporsi. Il \gl{committente} segnalerà gli errori riscontrati fornendo una valutazione generica dell'andamento del progetto ed una dettagliata per ogni documento. Si elencano di seguito le modifiche apportate in seguito alle revisioni.
	\subsection{Revisione dei Requisiti}
	\label{revisioneDeiRequisiti}
		\begin{itemize}
			\item \textbf{Studio di fattibilità:}\ sono stati corretti gli acronimi scritti in maniera errata.
			\item \textbf{Norme di progetto:}\ nel documento sono state aggiunte le sezioni che erano state impropriamente inserite nel documento \PQdocRR.
			\item \textbf{Analisi dei Requisiti:}\ sono stati modificati numerosi casi d'uso, cambiando ad esempio la descrizione, lo scenario principale e il padre. Inoltre sono stati aggiunti parecchi casi d'uso, mentre altri sono stati eliminati. Sono stati modificati anche numerosi requisiti sia come conseguenza delle modifiche dei casi d'uso sia per le segnalazioni ricevute. Sono stati aggiunti tanti requisiti e ne sono stati eliminati alcuni.
			\item \textbf{Piano di Progetto:}\ sono state aggiunte le sezioni sul'Analisi dinamica dei rischi e sono stati cambiati i nomi dei periodi.
			\item \textbf{Piano di Qualifica:}\ il documento è stato completamente rivisto a seguito delle segnalazioni. Sono state aggiunte altre metriche, soprattutto per la verifica dei processi, ed è stata ampliata l'appendice con i risultati della verifica.
			%se anche altri documenti vengono modificati, tipo il Glossario, bisogna aggiungere qua cosa è stato modificato
		\end{itemize}
		
	\subsection{Revisione di Progettazione}
	\label{revisioneDiProgettazione}
		\begin{itemize}
			\item \textbf{Norme di Progetto:}\ è stata ampliata la sezione "Attività" del "Processo di sviluppo".
			\item \textbf{Analisi dei Requisiti:}\ sono stati aggiunti alcuni diagrammi delle attività.
			\item \textbf{Specifica Tecnica:}\ sono stati aggiunti i vantaggi e gli svantaggi per le tecnologie utilizzate per lo sviluppo dell'applicazione. È stata cancellata la tecnologia \gl{PHP}, perché nei casi dove era previsto il suo uso è stato usato \gl{Javascript}, inoltre è stato aggiunto il framework \gl{Volley}\ come libreria per l'utilizzo e l'implementazione delle chiamate \gl{REST}. È stato deciso di semplificare i diagrammi dividendoli in più parti e descrivendo quindi ogni frammento, permettendo così di focalizzare la presentazione e la spiegazione su poche classi alla volta. È stato sostituito il \gl{design pattern}\ \gl{MVC}\ con il \gl{MVP}, dato che è quello più usato per \gl{Android}\ e in più ci permette di conservare l'architettura già progettata, perché di fatto i package individuati sono conformi a questo \gl{pattern}\ architetturale. È stato aggiunto LoginManager come Singleton. Sono stati corretti i metodi che risultavano barrati anziché sottolineati. È stata tolta la parola "Server" dai nomi delle classi del server. Aggunta la spiegazione per quanto riguarda il simbolo "?".
			\item \textbf{Piano di Progetto:}\ sono stati aggiornati i rischi riscontrati, eliminando quelli decaduti e aggiungendo quelli nuovi.
			\item \textbf{Piano di Qualifica:}\ è stato aggiunto il modello \gl{CMM}\ come metrica per il controllo di qualità dei processi, inoltre sono stati aggiunti e modificati i test di validazione.
			\item \textbf{Glossario:}\ è stato aggiunto l'indice.
		\end{itemize}