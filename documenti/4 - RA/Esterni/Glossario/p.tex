\invisiblesection{P}
\lettera{P}

\parola{Package}{Un package è una struttura per l'organizzazione delle classi in sottogruppi ordinati. Esso viene usato spesso dai programmatori per riunire classi logicamente correlate o con uno scopo simile, anche perché crea un unico spazio dei nomi per le classi contenute in esso. Un esempio di struttura a package con classi gerarchizzate sono le \gl{librerie}\ \gl{API}\ di \gl{Java}.}

\parola{Pattern}{Pattern è un termine inglese, che può essere tradotto, a seconda del contesto, con "disegno, modello, schema, schema ricorrente, struttura ripetitiva" e può essere utilizzato per indicare una regolarità che si riscontra all'interno di un insieme di oggetti osservati. Nell'\gl{ingegneria del software} è spesso utilizzata per parlare di \gl{design pattern}.}

\parola{PDCA}{Acronimo di ‘‘Plan-Do-Check-Act’’ o Ciclo di Daming, è un modello per il miglioramento continuo della qualità dei processi. Esso prevede in ordine cronologico la pianificazione del processo, la sua applicazione, la sua verifica e l'applicazione delle modifiche ritenute necessarie dal verificatore.}

\parola{Phonegap}{PhoneGap è un framework \gl{multipiattaforma} mobile che consente di sviluppare delle applicazioni native attraverso l'utilizzo di tecnologie web quali \gl{HTML}, \gl{CSS} e \gl{JavaScript} (riferimento: \url{https://it.wikipedia.org/wiki/PhoneGap}).}

\parola{PHP}{Acronimo ricorsivo di "PHP: Hypertext Preprocessor", è un preprocessore di ipertesti principalmente utilizzato per sviluppare applicazioni web lato server (riferimento: \url{https://it.wikipedia.org/wiki/PHP}).}

\parola{phpMyAdmin}{phpMyAdmin è un'applicazione web scritta in \gl{PHP} che consente di amministrare un database \gl{MySQL} tramite un qualsiasi browser. L'applicazione è indirizzata sia agli amministratori del database, sia agli utenti. Gestisce i permessi prelevandoli dal database \gl{MySQL} (riferimento: \url{https://it.wikipedia.org/wiki/PhpMyAdmin}).}

\parola{Play Framework}{Play è un \gl{framework} \gl{open source}, scritto in \gl{Java} e \gl{Scala}, che implementa il \gl{pattern} \gl{model-view-controller}. Il suo scopo è quello di migliorare la produttività degli sviluppatori grazie al caricamento del codice a caldo e alla visualizzazione degli errori nel browser (riferimento: \url{https://it.wikipedia.org/wiki/Play_framework}).}

\parola{Portable Network Graphics}{Il Portable Network Graphics (abbreviato PNG) è un formato di file per memorizzare immagini.
Il formato PNG è superficialmente simile al GIF, in quanto è capace di immagazzinare immagini in modo lossless, ossia senza perdere alcuna informazione.
Può memorizzare immagini a 24 bit ed ha un canale dedicato per la trasparenza (canale alfa) (riferimento: \url{https://it.wikipedia.org/wiki/Portable_Network_Graphics}).}


\parola{Prodotto}{Il prodotto è il risultato di un insieme di attività. In questo caso il termine è da intendersi come un sinonimo di \PROGETTO.} 

\parola{Progetto}{Il progetto è un insieme di azioni organizzate atte a perseguire uno scopo specifico. Nel nostro caso indica tutta l'attività di progettazione di codice e di documenti e della loro verifica, quindi il prodotto finale sarà \PROGETTO. Di conseguenza questo termine verrà usato spesso come sinonimo di \gl{prodotto}.}

\parola{Proponente}{Il proponente è la persona che ha proposto al \gl{committente}\ un \gl{capitolato}\ d'appalto.}

\parola{Prototipo}{Il prototipo è il modello originale o il primo esemplare di un manufatto, rispetto a una sequenza di eguali o similari realizzazioni successive. Sul prototipo verranno effettuati collaudi, modifiche e perfezionamenti, fino al prototipo definitivo, da avviare alla produzione in serie.}

\parola{Push}{Comando di \gl{git} per rendere effettivi sulla \gl{repository} remota una serie di cambiamenti apportati con una o più \gl{commit}.}