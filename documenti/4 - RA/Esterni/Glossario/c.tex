\invisiblesection{C} 
\lettera{C}

\parola{CamelCase}{La Notazione a Cammello o in inglese CamelCase è la pratica nata durante gli anni settanta di scrivere parole composte o frasi unendo tutte le parole tra loro, ma lasciando le loro iniziali maiuscole (riferimento: \url{https://it.wikipedia.org/wiki/Notazione_a_cammello}).}

\parola{Capitolato}{Il capitolato è un documento tecnico, in genere allegato ad un contratto di appalto, che vi fa riferimento per definire in quella sede le specifiche tecniche delle opere che andranno ad eseguirsi per effetto del contratto stesso.}

\parola{Client}{Un client, in informatica, indica una componente che accede ai servizi o alle risorse di un'altra componente detta server. Esso fa parte dunque dell'architettura logica di rete detta client-server (riferimento: \url{https://it.wikipedia.org/wiki/Client}).}

\parola{Cloud computing}{In informatica con il termine inglese cloud computing si indica un paradigma di erogazione di risorse informatiche, come l'archiviazione, l'elaborazione o la trasmissione di dati, caratterizzato dalla disponibilità \gl{on demand} attraverso Internet a partire da un insieme di risorse preesistenti e configurabili (riferimento: \url{https://it.wikipedia.org/wiki/Cloud_computing}).}

\parola{CMM}{Acronimo di ‘‘Capability Maturity Model’’, è un approccio al miglioramento dei processi il cui obiettivo è aiutare un'organizzazione a migliorare le proprie prestazioni. A partire dal 1997, è stato sostituito dal CMMI, di cui è erede.}

\parola{Commit}{in informatica è il tentativo di apportare una serie di cambiamenti permanenti. Nell'ambito di progetto è utilizzato come comando di \gl{git} per apportare modifiche permanenti alla \gl{repository} remota. }

\parola{Committente}{Committente è una parola italiana derivante dalla parola latina ‘‘committo’’. essa indica la persona che consegna/assegna un oggetto. In questo caso si riferisce al \COMMITTENTE\ che ha assegnato al gruppo \AUTORE\ il compito di sviluppare il \gl{capitolato}\ C2.}

\parola{CSS}{Acronimo di Cascading Style Sheets, in italiano fogli di stile a cascata, è un linguaggio usato per definire la formattazione di documenti \gl{HTML} e \gl{XML}, ad esempio i siti web e relative pagine web (riferimento: \url{https://it.wikipedia.org/wiki/CSS}).}

\parola{CSS3}{CSS3 è la terza e più attuale delle specifiche dell linguaggio \gl{CSS}}

\parola{Cvs}{Il comma-separated values è un formato di file basato su file di testo utilizzato per l'importazione ed esportazione (ad esempio da fogli elettronici o database) di una tabella di dati (riferimento: \url{https://it.wikipedia.org/wiki/Comma-separated_values}).}