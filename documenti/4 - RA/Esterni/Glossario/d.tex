\invisiblesection{D}
\lettera{D} 

\parola{Dashboard}{Dashboard, in italiano ‘‘cruscotto’’, è un’interfaccia grafica che organizza e presenta le informazioni in modo semplice, intuitivo ed immediato.}

\parola{Dependency Injection}{\gl{Design pattern}\ che prevede l'invio dall'esterno delle dipendenze necessarie per la costruzione della classe. Ha il vantaggio di separare il comportamento della componente dalla risoluzione delle sue dipendenze, disaccoppiando quindi le due parti. Esistono molte varianti della Dependency Injection, ognuna con i propri pro e contro. Lo svantaggio principale, comunque a tutte le varianti, è che spesso richiede una complessità di costruzione superiore rispetto ai pattern che sostituisce, come ad esempio il \gl{Singleton} (riferimento: \url{http://www.math.unipd.it/~tullio/IS-1/2014/Dispense/E9.pdf}).}

\parola{Design Pattern}{In informatica, nell'ambito dell'\gl{ingegneria del software}, design pattern è un concetto che può essere definito "una soluzione progettuale generale ad un problema ricorrente". Si tratta di una descrizione o modello logico da applicare per la risoluzione di un problema che può presentarsi in diverse situazioni durante le fasi di progettazione e sviluppo del \gl{software} (riferimento: \url{https://it.wikipedia.org/wiki/Design_pattern}).}

\parola{DSL}{Il domain-specific language o in italiano linguaggio specifico di dominio, nello sviluppo \gl{software} e nell'ingegneria di dominio è un linguaggio di programmazione o un linguaggio di specifica dedicato a particolari problemi di un dominio, a una particolare tecnica di rappresentazione e/o a una particolare soluzione tecnica (riferimento: \url{https://it.wikipedia.org/wiki/Domain-specific_language}).}