\section{Consuntivo e preventivo a finire}
\label{consuntivo}
A fronte delle spese sostenute nei vari periodi, verranno illustrate le differenze tra il preventivo e ciò che effettivamente è stato speso. \\
Infine verrà riportato il bilancio totale contenente la somma delle spese rispetto al preventivo e il preventivo a finire.

\subsection{Analisi e Management}

\subsubsection{Consuntivo parziale}
Verranno indicate, per ogni ruolo, le ore e le spese effettivamente sostenute durante il periodo di Analisi e Management (ore non rendicontate). 

\begin{tabella}{!{\VRule}c!{\VRule}c!{\VRule}c!{\VRule}}
		
	\intestazionethreecol{Ruolo}{Ore}{Costo}
		
	Responsabile& 12(-2) & \euro360,00(-\euro60,00)\\
	Amministratore & 36(-4) & \euro720,00(-\euro80,00)\\
	Analista & 50(+3) & \euro1250,00(+\euro75,00) \\
	Progettista & - & - \\
	Programmatore & - & -\\
	Verificatore & 37(+3) & \euro555,00(+\euro45,00) \\
	\hline
	\textbf{Totale consuntivo} & 135 & \euro2865,00\\
	\textbf{Totale preventivo} & 135 & \euro2885,00\\
	\textbf{Totale (differenza)} & - & -\euro20,00\\
		
	\hiderowcolors
	\caption{Ore non rendicontate - differenza preventivo/consuntivo periodo di Analisi e Management}
		
\end{tabella}
	
\subsubsection{Conclusioni}
Il team ha impiegato tre ore in più in attività di analisi, tre in più in attività di verifica, quattro in meno in attività di amministrazione e due in meno in attività da \RES.
Poichè il periodo di Analisi e Management non fa parte delle ore rendicontate i \textbf{\euro20,00} risparmiati non potranno essere riutilizzati nei periodi seguenti.

\newpage

\subsection{Analisi di Dettaglio}
\label{consuntivoAD}

\subsubsection{Consuntivo parziale}
Verranno indicate, per ogni ruolo, le ore e le spese effettivamente sostenute durante il periodo di Analisi di Dettaglio (ore rendicontate).

\begin{tabella}{!{\VRule}c!{\VRule}c!{\VRule}c!{\VRule}}
	
	\intestazionethreecol{Ruolo}{Ore}{Costo}
	
	Responsabile& 5 & \euro150,00\\
	Amministratore & 6(-2) & \euro120,00(-\euro40,00)\\
	Analista & 34(+4) & \euro850,00(+\euro100,00) \\
	Progettista & - & - \\
	Programmatore & - & -\\
	Verificatore & 16(-1) & \euro240,00(-\euro15,00) \\
	\hline
	\textbf{Totale consuntivo} & 61 & \euro1405,00\\
	\textbf{Totale preventivo} & 60 & \euro1360,00\\
	\textbf{Totale (differenza)} & +1 & +\euro45,00\\
	
	\hiderowcolors
	\caption{Ore rendicontate - differenza preventivo/consuntivo periodo di Analisi di Dettaglio}

\end{tabella}

\subsubsection{Conclusioni}
Il team ha impiegato quattro ore in più in attività di analisi, due in meno in attività di amministrazione e una in meno in attività di verifica. È stata utilizzata un' ora in più rispetto a quelle preventivate e sono stati utilizzati \textbf{\euro45,00} in più rispetto a quanto preventivato. \\
Il consuntivo supera il preventivo quindi il bilancio è in negativo di \textbf{\euro45,00} e il team si impegnerà nel periodo di Progettazione Architetturale ad ammortizzare i costi.\\
Grazie ai due periodi passati abbiamo riscontrato una notevole facilità nel gestire le attività di amministrazione e gestione del progetto per questo si cercherà di risparmiare, se possibile, nelle ore di \AM{} e \RES{}. A tal proposito è stato modificato il preventivo del periodo di Progettazione Architetturale come segue:

\begin{itemize}
	\item per far fronte all'ora in più utilizzata nel periodo di Analisi e Management, è stata tolta un ora di \RES{} al periodo di Progettazione Architetturale;
	\item per far fronte ai \textbf{\euro45,00} spesi in più nel periodo di analisi e di management, sono state tolte un' ora di \AM{} e un'ora di \AN, mentre sono state aggiunte due ore di \VER.
\end{itemize}

Grazie a questi cambiamenti il preventivo riesce ad ammortizzare i \textbf{\euro45,00} in eccesso. Il preventivo rimane quindi \textbf{11581\euro} come precedentemente stabilito.\\

È stata inoltre modificata la distribuzione delle ore per limitare i rischi illustrati nella \hyperref[sez2.2]{sezione 2.2}; infatti si è deciso di scambiare i ruoli di Tommaso e Matteo nei periodi di Progettazione Architetturale e Progettazione di Dettaglio. Queste modifiche non hanno apportato cambiamenti nel conteggio totale delle ore di ciascun componente nè del preventivo totale.

\subsection{Progettazione Architetturale}

\subsubsection{Consuntivo parziale}
Verranno indicate, per ogni ruolo, le ore e le spese effettivamente sostenute durante il periodo di Progettazione Architetturale (ore rendicontate).

\begin{tabella}{!{\VRule}c!{\VRule}c!{\VRule}c!{\VRule}}
	
	\intestazionethreecol{Ruolo}{Ore}{Costo}
	
	Responsabile& 9 & \euro270,00\\
	Amministratore & 9(-1) & \euro180,00(-\euro20,00)\\
	Analista & 19(-3) & \euro475,00(-\euro75,00) \\
	Progettista & 91(+4) &  \euro2002,00(+\euro88,00) \\
	Programmatore & - & -\\
	Verificatore & 77 & \euro1155,00 \\
	\hline
	\textbf{Totale consuntivo} & 205 & \euro4075,00\\
	\textbf{Totale preventivo} & 205 & \euro4082,00\\
	\textbf{Totale (differenza)} & 0 & -\euro7,00\\
	
	\hiderowcolors
	\caption{Ore rendicontate - differenza preventivo/consuntivo periodo di Progettazione Architetturale}
	
\end{tabella}

\subsubsection{Conclusioni}
Il team ha impiegato quattro ore in più in attività di progettazione, tre in meno in attività di analisi e una in meno in attività di amministrazione. Non sono state utilizzate ore in più rispetto a quanto preventivato, ma sono stati risparmiati \textbf{\euro7,00} grazie al cambio delle ore mostrato sopra.

\subsubsection{Preventivo a finire}
Grazie alle modifiche del preventivo effettuate al termine del periodo di Analisi di Dettaglio, si è riusciti ad ammortizzare i costi in eccesso e a risparmiare, infatti ora il preventivo supera il consuntivo: il bilancio quindi è in positivo di \textbf{\euro7,00}.
Il team potrà utilizzare questi \textbf{\euro7,00} nei periodi successivi nel caso fosse necessario un maggiore dispendio di ore.

\subsection{Progettazione di Dettaglio e Codifica}

\subsubsection{Consuntivo parziale}
Verranno indicate, per ogni ruolo, le ore e le spese effettivamente sostenute durante il periodo di Progettazione di Dettaglio e Codifica (ore rendicontate).

	\begin{tabella}{!{\VRule}c!{\VRule}c!{\VRule}c!{\VRule}}
		\intestazionethreecol{Ruolo}{Ore}{Costo}
		
		Responsabile & 10 & 300\euro \\
		Amministratore & 5(-1) & 100\euro(-\euro20,00) \\
		Analista & 9(-3) & 225\euro(-\euro75,00) \\
		Progettista & 45(+5) & 990\euro(+\euro110,00) \\
		Programmatore & 75(+3) & 1125\euro(+\euro45,00) \\
		Verificatore & 65(-5) & 975\euro(-\euro75,00) \\
		\hline
		\textbf{Totale consuntivo} & \textbf{208} & \textbf{\euro3700} \\
		\textbf{Totale preventivo} & \textbf{209} & \textbf{\euro3715} \\
		\textbf{Totale (differenza)} & 1 & -\euro15,00\\
		
		\hiderowcolors
		\caption{Ore rendicontate - differenza preventivo/consuntivo periodo di Progettazione di Dettaglio e Codifica}
	\end{tabella}
		
\subsubsection{Conclusioni}
Il team ha impiegato un'ora in meno in attività di amministrazione, tre in meno in attività di analisi e cinque in meno in attività di verifica. Sono state invece impiegate cinque ore in più per attività di progettazione e tre in più in attività di programmazione. \\
Complessivamente è stata utilizzata un'ora in meno rispetto a quanto preventivato, grazie a questo sono stati risparmiati \textbf{\euro15,00}.

\subsubsection{Preventivo a finire}
Il bilancio resta in positivo di \textbf{\euro22} grazie all'ora in meno utilizzata in questo periodo che ci ha permesso di risparmiare \textbf{\euro15,00} e a quanto era stato risparmiato nella fasi precedenti, ovvero \textbf{\euro7,00}. \\ Il team potrà utilizzare i \textbf{\euro22} risparmiati nell'ultima parte del progetto.

\subsection{Requisiti Opzionali e Desiderabili}

\subsubsection{Consuntivo parziale}
Verranno indicate, per ogni ruolo, le ore e le spese effettivamente sostenute durante il periodo di Requisiti Opzionali e Desiderabili (ore rendicontate).

\begin{tabella}{!{\VRule}c!{\VRule}c!{\VRule}c!{\VRule}}
	\intestazionethreecol{Ruolo}{Ore}{Costo}
	
	Responsabile & 3 & 90\euro \\
	Amministratore & 3 & 60\euro \\
	Analista & 2 & 50\euro \\
	Progettista & 14 & 308\euro \\
	Programmatore & 26 & 390\euro \\
	Verificatore & 21 & 315\\
	\hline
	\textbf{Totale consuntivo} & \textbf{69} & \textbf{1213\euro} \\
	\textbf{Totale preventivo} & \textbf{69} & \textbf{1213\euro} \\
	\textbf{Totale (differenza)} & 2 & +\euro15,00\\
	
	\hiderowcolors
	\caption{Ore rendicontate - differenza preventivo/consuntivo periodo di Requisiti Opzionali e Desiderabili}
\end{tabella}

\subsubsection{Conclusioni}
Il team ha impiegato le stesse ore preventivate per questo breve periodo. \\
Non sono stati quindi né spesi né soldi rispetto il preventivo.

\subsubsection{Preventivo a finire}
Il bilancio resta in positivo di \textbf{\euro22} come il periodo precedente in quanto non ci sono stati né esuberi né guadagni.

\subsection{Validazione e Collaudo}

\subsubsection{Consuntivo parziale}
Verranno indicate, per ogni ruolo, le ore e le spese effettivamente sostenute durante il periodo di Validazione e Collaudo (ore rendicontate).

	\begin{tabella}{!{\VRule}c!{\VRule}c!{\VRule}c!{\VRule}}
		\intestazionethreecol{Ruolo}{Ore}{Costo}
		
		Responsabile & 5(-1) & 150\euro(-\euro30,00) \\
		Amministratore & 8 & 160\euro \\
		Analista & 2 & 50\euro \\
		Progettista & 8 & 176\euro \\
		Programmatore & 16(+3) & 240\euro(+\euro45,00) \\
		Verificatore & 29 & \\
		\hline
		\textbf{Totale consuntivo} & \textbf{70} & \textbf{1226\euro} \\
		\textbf{Totale preventivo} & \textbf{68} & \textbf{1211\euro} \\
		\textbf{Totale (differenza)} & 2 & +\euro15,00\\
		
		\hiderowcolors
		\caption{Ore rendicontate - differenza preventivo/consuntivo periodo di Validazione e Collaudo}
	\end{tabella}
	
	\subsubsection{Conclusioni}
	Il team ha impiegato un'ora in meno in attività di amministrazione e tre in più in attività di programmazione. \\
	Complessivamente sono state utilizzate due ore in più rispetto a quanto preventivato, spendendo \textbf{\euro15,00} in più.
	Grazie ai \textbf{\euro22} risparmiati nei periodi precedenti non ci sono state conseguenze negative sul preventivo,risparmiando totalmente \textbf{\euro7} .
	
\subsection{Consuntivo e ore rendicontate finali}
	
	\subsubsection{Ore rendicontate per componente}
	
\begin{tabella}{!{\VRule}c!{\VRule}c!{\VRule}}
	\intestazionetwocol{Nome}{Ore finali}
	
	Viviana Alessio & 102 \\
	Enrico Bellio & 102 \\
	Matteo Franco & 102 \\
	Andrea Grendene & 102 \\
	Tommaso Panozzo & 102 \\
	Luca Soldera & 102 \\
	
	\hiderowcolors
	\caption{Ore finali rendicontate per componente}
\end{tabella}

\subsubsection{Consuntivo finale}
 Da quanto si evince dai consuntivi parziali dei vari periodi rendicontati, si riportano di seguito i costi sostenuti
 da BeaconStrips durante l’intero progetto per l’elaborazione del consuntivo finale.
 
 \begin{tabella}{!{\VRule}c!{\VRule}c!{\VRule}}
 	\intestazionetwocol{Periodo}{Consuntivo}
 	
 	Analisi di Dettaglio & \euro1405,00 \\
 	Progettazione Architetturale & \euro4075,00 \\
 	Progettazione di Dettaglio e Codifica & \euro3700 \\
 	Requisiti Opzionali e Desiderabili & \euro1213 \\
 	Validazione e Collaudo & 1226\euro \\
 	\hline
 	\textbf{Totale} & \textbf{\euro11619(-45)}
 	
 	\hiderowcolors
 	\caption{Ore finali rendicontate per componente}
 \end{tabella}

\subsubsection{Conclusioni}
Il consuntivo finale risulta di \textbf{\euro11619}, a cui però bisogna sottrarre \textbf{\euro45} poichè in seguito al periodo di Analisi di Dettaglio sono stati fatti dei cambiamenti di ore come segnalato nella \hyperref[consuntivoAD]{sezione apposita}, per cui il consuntivo finale è di \textbf{\euro11574}.
Considerando quindi che il preventivo era di \textbf{\euro11581} abbiamo risparmiato in totale \textbf{\euro7}.\\
Pertanto Beaconstrips comunica che la cifra finale per la realizzazione di CLIPS è di \textbf{\euro11574}.
