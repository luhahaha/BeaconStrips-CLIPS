\section{Descrizione architettura} 
\label{architettura}
	In questa sezione verrà descritta lo schema generale dell'architettura, nella sezione successiva % aggiungere label
	verranno riportate tutti i packages e le classi dettagliatamente.
	 Per i diagrammi delle classi e di attività è stato utilizzato il formalismo UML 2.0. \\
	 L'architettura descritta in questo documento è ad alto livello. Classi, sottoclassi e attributi verranno descritti più dettagliatamente nel periodo in cui attueremo la Progettazione di dettaglio. \\
	 Si procederà nella descrizione dell'architettura con un approccio top-down. Saranno descritte prima quindi le parti più generali per poi andare sempre più nello specifico. Si procederà quindi prima con la descizione dei packages e delle componenti, per poi passare alle singole classi. Successivamente verranno descritti i design pattern utilizzati.
	
	\subsection{Architettura generale}
	L'architettura generale dell'applicazione segue il modello client - server. \\
	In particolare si utilizzerà lo stile architetturale REST (representational state transfer) per coordinare compomenti, connettori e dati attraverso un sistema ipermediale distribuito dove l'attenzione è data al ruolo delle componenti ed ai vincoli imposti dalle loro interazioni tra di essi. \\
	Alcune caratteristiche e vincoli di REST sono i seguenti:	
	\begin{itemize}
		\item REST offre una interfaccia che separa il client dal server. Questa separazione permette di avere ben chiaro quali siano i ruoli delle due componenti in modo che non ci siano conflitti tra le due. Questo porta beneficio anche alla portabilità e alla scalabilità del prodotto;
		\item la comunicazione tra client e server è stateless, il che significa che ogni richiesta del client sarà interpretabile dal server senza che esso utilizzi alcun contesto presente nel server;
		\item i risultati ottenuti da una richiesta client-server possono essere salvati in cache così che se il client avesse la necessità di riutilizzare dati già richiesti lo possa fare senza effettuare nuove richieste.	
	\end{itemize}
	
	\subsubsection{Client}
	Per la parte Client si è deciso di utilizzare il linguaggio Android per la creazione dell'applicazione con cui gli utenti interagiranno;
	verrà utilizzato il pattern architetturale Model View Presenter, spiegato in dettaglio nell'\hyperref[pattern]{Appendice}, per consentire la separazione logica delle varie componenti, permettendo di non dover modificare una componente logica qualora ne venga modificata un'altra, e facilitando lo sviluppo di unit test.\\
	Il Client inoltre avrà un proprio database locale nel quale momorizzare temporaneamente i dati dell'utente e i dati richiesti al server, consentendo una maggiore fluidità limitando le chiamate al server.
	\subsection{Server}
	La parte server sarà costituita da un database in linguanggio SQL, per il salvataggio dei dati riguardanti l'utente, come salvataggi e dati per l'autenticazione, e tutte le informazioni sugli edifici che includono percorsi giocabili dall'utente con i relativi percorsi e classifiche. \\
	Le interazioni tra Client e database verranno gestite da alcune classi progettate in Javascript, che gestiranno le richieste di dati e il loro aggiornamento.\\
	Per il trasferimento di dati tra Server e Client verrà utilizzato il formato JSON.
	\subsection{Architettura del database}
	\label{architetturaDelDatabase}
	In particolare verrà utilizzato un database relazionale \gl{MySQL} con la possibilità di interagire con esso tramite l'applicazione web \gl{phpMyAdmin} in modo da semplificarne l'amministrazione.
	Le rappresentazione del database verrà illustrata nel documento \DPdoc.
		