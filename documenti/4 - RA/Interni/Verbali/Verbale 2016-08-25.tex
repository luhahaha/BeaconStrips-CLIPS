% !TEX encoding = UTF-8 Unicode
% !TEX TS-program = pdflatex
% !TEX spellcheck = it-IT
\documentclass[a4paper,titlepage]{article}

\usepackage[utf8x]{inputenc}

\makeatletter
\def\input@path{{../../../template/}}
\makeatother

\usepackage{Comandi}
\usepackage{Riferimenti}
\usepackage{Stile}

\def\NOME{Verbale del Giorno 2016-08-25}
\def\VERSIONE{1.0.0}
\def\DATA{2016-06-24}
\def\REDATTORE{Viviana Alessio}
\def\VERIFICATORE{Luca Soldera}
\def\RESPONSABILE{Andrea Grendene}
\def\USO{Interno}
\def\DESTINATARI{\COMMITTENTE \\ & \CARDIN \\ & \PROPONENTE}
\def\SOMMARIO{In questa riunione si è deciso come organizzare il periodo ``Requisiti desiderabili e opzionali'' e ``Validazione e collaudo''.}


\begin{document}

\maketitle

\begin{diario}
	\modifica{Andrea Grendene}{\RES}{Approvazione del verbale}{2016-09-1}{1.0.0}
	\modifica{Luca Soldera}{\VER}{Verifica del verbale}{2016-08-28}{0.2.0}
	\modifica{Viviana Alessio}{\PRJ}{Stese le Sezioni 1, 2 e 3}{2016-08-27}{0.1.0}
\end{diario}

\newpage
\tableofcontents

\newpage
\section{Informazioni}
\label{sec:Informazioni}

\begin{itemize}
 \item \textbf{Modalità}: Videochiamata di gruppo su Google Hangouts.
 \item \textbf{Data}: 2016-08-25;
 \item \textbf{Ora}: 9:30;
 \item \textbf{Durata}: 1 ora;
 \item \textbf{Partecipanti}: Viviana Alessio, Luca Soldera, Matteo Franco, Andrea Grendene, Enrico Bellio, Tommaso Panozzo.
\end{itemize}

\section{Ordine del giorno}
\label{sec:Ordine del giorno}
Di seguito sono riportati i punti affrontati durante la riunione.

\begin{enumerate}
	\item Decidere come organizzarsi per soddisfare i requisiti desiderabili e opzionali.
	\item Decidere come organizzarsi per i test.
	\item Decidere come organizzarsi per ultimare la produzione dei documenti.
	\item Decidere come mantenere i contatti in entrambi i periodi.
\end{enumerate}

\section{Decisioni}
Vengono riportate ora le decisioni prese durante la riunione. \\
Per rendere più facile il tracciamento e il riferimento dentro e fuori questo documento ad ogni decisione viene assegnato un codice identificativo secondo la seguente codifica:
\begin{center}
D[Codice Identificativo]
\end{center}
Vengono riportate anche le fonti che hanno portato a queste decisioni, sia interne che esterne. Per fonti interne si intendono le domande a cui è stata trovata risposta durante la riunione.

\begin{tabella}{!{\VRule}c!{\VRule}X[l,b,l]!{\VRule}c!{\VRule}}
	\intestazionethreecol{Decisione}{Descrizione}{Fonti}
		D1 & 
		Per concludere il lavoro di codifica è deciso di mantenere i tre sottogruppi formati nel periodo precedente: 
		\begin{enumerate}
			\item\ Viviana e Matteo si occuperanno del viewcontroller e dell'integrazione di esso con il datamanager
			\item\ Enrico e Andrea si occuperanno del model
			\item\ Tommaso e Luca si occuperanno di server e database
		\end{enumerate}
		& Punto 1 \\
		D2 & Si è deciso di dividere il lavoro relativo ai test in questo modo:
			\begin{enumerate}
				\item\ Matteo si occuperà dei test del viewcontroller
				\item\ Andrea si occuperà dei test del model
				\item\ Tommaso si occuperà dei test del server
				\item\ Enrico si occuperà dei test del database
			\end{enumerate}
		& Punto 2 \\
		D3 & 
		L'ultimazione dei documenti è stata divisa in questo modo: 
			\begin{enumerate}
				\item\ Andrea si occuperà del Piano di progetto
				\item\ Luca si occuperà del Piano di qualifica
				\item\ Matteo, Viviana, Tommaso ed Enrico si occuperanno di tutti gli altri documenti
			\end{enumerate}
		& Punto 3 \\
		D4 & Si è deciso di rimanere in contatto mediante slack e chiamate di gruppo su Google Hangouts a discrezione del responsabile. Inoltre si è deciso che ognuno garantirà la propria presenza in Torre Archimede a Padova nella settimana 5-9 Settembre 2016 per le prove pre-collaudo.  & Punto 4 \\
	
	\hiderowcolors
	\caption{Tabella delle decisioni prese}
\end{tabella}

\end{document}