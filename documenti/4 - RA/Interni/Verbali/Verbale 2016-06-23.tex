% !TEX encoding = UTF-8 Unicode
% !TEX TS-program = pdflatex
% !TEX spellcheck = it-IT
\documentclass[a4paper,titlepage]{article}

\usepackage[utf8x]{inputenc}

\makeatletter
\def\input@path{{../../../template/}}
\makeatother

\usepackage{Comandi}
\usepackage{Riferimenti}
\usepackage{Stile}

\def\NOME{Verbale del Giorno 2016-06-23}
\def\VERSIONE{1.0.0}
\def\DATA{2016-06-24}
\def\REDATTORE{Viviana Alessio}
\def\VERIFICATORE{Luca Soldera}
\def\RESPONSABILE{Tommaso Panozzo}
\def\USO{Interno}
\def\DESTINATARI{\COMMITTENTE \\ & \CARDIN \\ & \PROPONENTE}
\def\SOMMARIO{In questa riunione si è deciso come organizzare il periodo di progettazione di dettaglio e codifica.}


\begin{document}

\maketitle

\begin{diario}
	\modifica{Tommaso Panozzo}{\RES}{Approvazione del verbale}{2016-06-27}{1.0.0}
	\modifica{Luca Soldera}{\VER}{Verifica del verbale}{2016-06-26}{0.2.0}
	\modifica{Viviana Alessio}{\PRJ}{Stese le Sezioni 1, 2 e 3}{2016-05-12}{0.1.0}
\end{diario}

\newpage
\tableofcontents

\newpage
\section{Informazioni}
\label{sec:Informazioni}

\begin{itemize}
 \item \textbf{Luogo}: LuF1, Plesso Paolotti - Via G. B. Belzoni 7, 35121, Padova (PD);
 \item \textbf{Data}: 2016-06-23;
 \item \textbf{Ora}: 9:30;
 \item \textbf{Durata}: 2 ora;
 \item \textbf{Partecipanti}: Viviana Alessio, Luca Soldera, Matteo Franco, Andrea Grendene, Enrico Bellio, Tommaso Panozzo.
\end{itemize}

\section{Ordine del giorno}
\label{sec:Ordine del giorno}
Di seguito sono riportati i punti affrontati durante la riunione.

\begin{enumerate}
	\item Decidere come organizzarsi per la progettazione di dettaglio.
	\item Decidere come organizzarsi per la codifica.
	\item Decidere come mantenere i contatti nella la fase di codifica.
\end{enumerate}

\section{Decisioni}
Vengono riportate ora le decisioni prese durante la riunione. \\
Per rendere più facile il tracciamento e il riferimento dentro e fuori questo documento ad ogni decisione viene assegnato un codice identificativo secondo la seguente codifica:
\begin{center}
D[Codice Identificativo]
\end{center}
Vengono riportate anche le fonti che hanno portato a queste decisioni, sia interne che esterne. Per fonti interne si intendono le domande a cui è stata trovata risposta durante la riunione.

\begin{tabella}{!{\VRule}c!{\VRule}X[c,b,c]!{\VRule}c!{\VRule}}
	\intestazionethreecol{Decisione}{Descrizione}{Fonti}
		D1 & Prima di procedere con la progettazione di dettaglio si è deciso di aspettare il giudizio dei professori riguardo i documenti della RP. Nel frattempo si procederà a studiare le nuove tecnologie che si è deciso di utilizzare.
		& Punto 1 \\
		D2 & Si è deciso di dividere il gruppo in tre sottogruppi formati ciascuno da due persone: 
			\begin{enumerate}
				\item\ Viviana e Matteo si occuperanno del viewcontroller
				\item\ Enrico e Andrea si occuperanno del model
				\item\ Tommaso e Luca si occuperanno di server e database
			\end{enumerate}
		& Punto 2 \\
		D3 & Si è deciso di rimanere in contatto durante il periodo di codifica mediante una chiamata settimanale di gruppo su hangout durante la quale ognuno esporrà i progressi che ha fatto durante la settimana precedente. Abbiamo dovuto adottare questa soluzione in quanto la maggior parte del gruppo in estate ha difficoltà a recarsi a Padova, inoltre per gran parte del mese di Agosto le strutture universitarie chiudono e quindi non avremo modo di trovarci fisicamente agevolmente & Punto 3 \\
	
	\hiderowcolors
	\caption{Tabella delle decisioni prese}
\end{tabella}

\end{document}